%%% License: Creative Commons Attribution Share Alike 4.0 (see https://creativecommons.org/licenses/by-sa/4.0/)

\documentclass[english,10pt
,aspectratio=169
%,handout
%%%,notes
]{beamer}
%%% License: Creative Commons Attribution Share Alike 4.0 (see https://creativecommons.org/licenses/by-sa/4.0/)

\DeclareGraphicsExtensions{.eps, .pdf,.png,.jpg,.mps,}
\usetheme{reMedian}
\usepackage{parskip}
\makeatother

\renewcommand{\baselinestretch}{1.1} 

\usepackage{amsmath, amssymb, amsfonts, amsthm}
\usepackage{enumerate}
%\usepackage{enumitem}
\usepackage{hyperref}
\usepackage{url}
\usepackage{bbm}
\usepackage{color}

\usepackage{tikz}
\usepackage{tikzscale}
\newcommand*\circled[1]{\tikz[baseline=(char.base)]{
		\node[shape=circle,draw, inner sep=-20pt] (char) {#1};}}
\usetikzlibrary{automata,positioning}
\usetikzlibrary{decorations.pathreplacing}
\usepackage{pgfplots}
\usepgfplotslibrary{fillbetween}
\usepackage{graphicx}

\usepackage{setspace}
\thinmuskip=1mu
\medmuskip=1mu 
\thickmuskip=1mu 


\usecolortheme{default}
\usepackage{verbatim}
\usepackage[normalem]{ulem}

\usepackage{apptools}
\AtAppendix{
	\setbeamertemplate{frame numbering}[none]
}
\usepackage{natbib}


% red strikeout
\newcommand\soutred{\bgroup\markoverwith
	{\textcolor{red}{\rule[0.55ex]{2pt}{0.8pt}}}\ULon}



% To use LyX frames from old version:
\def\lyxframeend{} % In case there is a superfluous frame end
\long\def\lyxframe#1{\@lyxframe#1\@lyxframestop}%
\def\@lyxframe{\@ifnextchar<{\@@lyxframe}{\@@lyxframe<*>}}%
\def\@@lyxframe<#1>{\@ifnextchar[{\@@@lyxframe<#1>}{\@@@lyxframe<#1>[]}}
\def\@@@lyxframe<#1>[{\@ifnextchar<{\@@@@@lyxframe<#1>[}{\@@@@lyxframe<#1>[<*>][}}
\def\@@@@@lyxframe<#1>[#2]{\@ifnextchar[{\@@@@lyxframe<#1>[#2]}{\@@@@lyxframe<#1>[#2][]}}
\long\def\@@@@lyxframe<#1>[#2][#3]#4\@lyxframestop#5\lyxframeend{%
	\frame<#1>[#2][#3]{\frametitle{#4}#5}}


\title{Mechanism Design}

\subtitle{6: Dynamic Mechanisms}

\author{Egor Starkov}

\date{K{\o}benhavns Unversitet \\
	Fall 2022}


\begin{document}
	\AtBeginSection[]{
		\frame{
			\frametitle{This slide deck:}
			\tableofcontents[currentsection,currentsubsection]
	}}
	\AtBeginSubsection[]{
		\frame{
			\frametitle{This slide deck:}
			\tableofcontents[currentsection,currentsubsection]
	}}
	\frame[plain]{\titlepage}


\section{Dynamic Mechanisms: Introduction}

\begin{frame}{Dynamic Problems}
\begin{itemize}
	\item Models considered so far were static: one report, one outcome.
	\begin{itemize}
		\item Although we hinted towards dynamic incentives when discussing interim vs ex post IC/IR constraints.
	\end{itemize}
	\item There are many \structure{dynamic} problems in the real world:
	\begin{itemize}
		\item Dynamic pricing when buyers' tastes evolve (e.g. experience goods) or buyers come and go over time;
		\item Procurement from firms with changing costs;
		\item Design of tax and social security systems;
		\item Dynamic labor contracts
	\end{itemize}
	\item How to develop dynamic mechanisms? Will see today.
	\item This lecture mostly follows \cite{bergemann_dynamic_2019}.
\end{itemize}
\end{frame}


\begin{frame}{What defines a dynamic problem? (1)}
\begin{itemize}
	\item Why can a dynamic problem not be seen as a sequence of independent static problems? 
	
	\item Because there can be \structure{linkages} across periods: (which ruin the independence)
	\begin{enumerate}
		\item \alert{Information} -- future info evolves from (so depends on) past info and possibly past allocations.
		\item \alert{Preferences} -- usually evolve gradually. For our purposes, can see this as persistence in nformation.
		\item \alert{Allocations} -- set of feasible allocations today may depend on past outcomes (example: sale of fixed number of items over many periods).
	\end{enumerate}

	\item The same linkages mean that if we try to see the problem as a huge static problem (with same player in different periods seen as different players), the correlations in players' info and the set of feasible allocations will look weird and complicated.
\end{itemize}
\end{frame}


\begin{frame}{Dynamic Model}
\begin{itemize}
	\item \structure{Periods} $t \in \{0,1,...,T\}$; terminal time $T \leq \infty$; all players (incl. designer) have common \structure{discount} factor $\delta$.
	\item \structure{Players} $i \in \{1,2,...,N\}$ have evolving \structure{types} $\theta_{i,t} \in \Theta_i$, \alert{indep}. across $i$.
	\begin{itemize}
		\item Common \structure{prior} $\theta_{i,0} \sim F_{i,0}$; \structure{types} are Markov processes: $$\theta_{i,t+1} \sim F_{i,t}(\theta_{i,t+1} | \theta_{i,t},k_t).$$
	\end{itemize}
	\item Every period: \structure{allocation} $k_t \in K_t$ and \structure{payments} $p_t \in \mathbb{R}^N$.
	\\ Set of \structure{feasible allocations} evolves as $K_{t+1} = g(K_t,k_t)$.
	\item Players' \structure{utilities}: $u_i((k_t,p_t),\theta_t) = v_i(k_t,\theta_{i,t}) - p_{i,t}$.
\end{itemize}
\end{frame}


\begin{frame}{Evolving Types}
Possible interpretations of \structure{evolving types}:
\begin{itemize}
	\item \alert{Exogenous evolution} ($\theta_{t+1} \perp k$);
	\begin{itemize}
		\item Example: procuring goods over time from a firm with stochastically evolving costs $\theta_{i,t+1} = \gamma \theta_{i,t} + \varepsilon_{i,t+1}$.
	\end{itemize}
	\item \alert{Endogenous evolution} (depending on $k_t$);
	\begin{itemize}
		\item Example: worker assigned to training by $k_t$ will improve their future productivity $\theta_{i,t+1}$.
	\end{itemize}
	\item \alert{Random arrival};
	\begin{itemize}
		\item Players can arrive at the mechanism at random times.
		\item Can model that by setting $\theta_{i,t} = \varnothing$ whenever $i$ is not in the market/mechanism.
	\end{itemize}
\end{itemize}
\end{frame}


\begin{frame}{Dynamic Model: Assumptions}
To fix ideas, assume the following for this class:
\begin{itemize}
	\item The designer can \structure{commit} to the whole future mechanism at $t=0$.
	\item Contracts are binding -- we ignore per-period IR constraints (except maybe IR at $t=0$).
	\begin{itemize}
		\item Justification: in quasilinear model, can ask players to put collateral at $t=0$, to be repaid at a later date -- this would eliminate incentives to quit mechanism after $t=0$.
	\end{itemize}
	\item All past reports and allocations are publicly observed.
	\item Player $i$ at time $t$ observes their type $\theta_{i,t}$ but not future types.
\end{itemize}
\end{frame}


\begin{frame}{Direct Mechanisms}
\begin{itemize}
	\item As usual, we have the \structure{revelation principle}, though there are caveats \citep{sugaya_revelation_2021}.
	\item So can focus on mechanisms which ask players to report their types every period.
	\item \structure{Reporting strategies} given by $\rho_i = \{r_{i,t}\}_{t=0}^T$, where $r_{i,t}: \Theta_i \times H_t \to \Theta_i$ and $H_t$ is the set of \structure{public histories} $h_t = \{k_s,(r_{1,s},...,r_{N,s})\}_{s < t}$.
	\item A \structure{dynamic direct mechanism} is $(\kappa,\pi) = \{k_t,p_t\}_{t=0}^T$, where $k_t: \Theta \times H_t \to K_t$ and $p_t: \Theta \times H_t \to \mathbb{R}^N$.
\end{itemize}
\end{frame}


\begin{frame}{Dynamic Implementation}
\begin{itemize}
	\item Looking for a truthful equilibrium in a direct mechanism.
	\item ``Equilibrium'' is a sketchy term in dynamic incomplete-info games.
	\begin{itemize}
		\item There is at least a dozen different equilibrium concepts and refinements in use.
		\item Main concern in general: off-equilibrium-path beliefs. \textcolor{gray}{What should a player believe after observing an event they considered impossible? Different answers can strongly affect the predicted outcome.}
		\item Not a big problem in mechdesign \textcolor{gray}{-- players do not observe any actions until it's too late to act.}
	\end{itemize}
	\item Look for \structure{Perfect Bayesian Equilibria}.
	\begin{itemize}
		\item Each player chooses report to maximize expected util, expecting others to report truthfully.
		\item Beliefs are updated using Bayes' rule whenever possible (i.e., on equilibrium path).
		\item In general in PBE: We can assume anything we want about off-path beliefs to sustain eqm. In our problem: won't need to.
	\end{itemize}
\end{itemize}
\end{frame}


\section{Efficient Dynamic Implementation}

\begin{frame}{Efficient Allocation}
\begin{itemize}
	\item Suppose we want to implement the \structure{efficient allocation} $\kappa^*$.
	\item But what is $\kappa^*$ in a dynamic problem?
	$$ \kappa^* \in \arg \max_{\{k_t\}_{t=0}^T} \mathbb{E} \left\{ \sum_{t=0}^T \delta^t \sum_{i=0}^N v_i(k_t,\theta_{i,t}) \right\}$$
	\item Must optimize over the whole path $\{k_t\}_{t=0}^T$ rather than period-by-period.
	\begin{itemize}
		\item Today's allocation $k_t$ may affect tomorrow's types $\theta_{t+1}$ and set of alternatives $K_{t+1}$.
	\end{itemize}
	\item Also remember that $k_t : \Theta \times H_t \to K_t$ is a highly-dimensional object in itself.
	\item So simply finding $\kappa^*$ is in general a difficult optimal control problem.
	\item \textbf{Remark}: \alert{ex post efficiency} is \alert{unattainable} in dynamics -- $k_t$ must be chosen before $\theta_{t+s}$ learned. \structure{Interim efficiency} is the best we can hope for.
\end{itemize}
\end{frame}


\begin{frame}{Efficient Implementation}
\begin{itemize}
	\item Ok, suppose we found $\kappa^*$, what next?
	\item In static setting we used VCG aka the pivot mechanism: each player had to pay the externality they imposed on everyone else:
	$$ p_i(\theta) = -\sum_{j \neq i} v_j \left( k^*(\theta),\theta_j \right) + \sum_{j \neq i} v_j \left( k^*_{-i}(\theta_{-i}),\theta_j \right) $$
	\item The idea translates almost verbatim to the dynamics.
	\begin{itemize}
		\item Problem: the externality that $i$ imposes on others via report $\theta_{i,t}$ may manifest in other periods -- not necessarily at $t$.
	\end{itemize}
	\item Enter \structure{dynamic pivot mechanism}! \citep{bergemann_dynamic_2010}
\end{itemize}
\end{frame}


\begin{frame}{Dynamic Pivot Mechanism}
\vspace{-2em}\begin{align*}
	&\text{Flow social surplus}	& w_t(k_t,\theta_t)	&\equiv \sum_{i=1}^N v_i(k_t,\theta_{i,t}).
	\\
	&\text{Welfare}	& W_t(\theta_t,K_t)	&\equiv \max_{k_t \in K_t} \left\{ w_t(k_t,\theta_t) + \delta \mathbb{E} W_{t+1}(\theta_{t+1},K_{t+1}) \right\}.
	\\
	&\text{\footnotesize $i$'s marginal contribution}	& M_{i,t}(\theta_t,K_t)	&\equiv W_t(\theta_t,K_t) - W_{-i,t}(\theta_t,K_t)
	\\
	&\text{\footnotesize can be written recursively as}	& M_{i,t}(\theta_t,K_t)	&= \structure{m_{i,t}(\theta_t,K_t)} + \delta \mathbb{E} M_{i,t+1}(\theta_{t+1},K_{t+1}).
	\\
	&\text{Payments}	& p_{i,t}^*	&\equiv v_i(k^*_t,\theta_{i,t}) - m_{i,t}(\theta_t,K_t).
\end{align*}\vspace{-1em}

\begin{itemize}
	\item The dynamic pivot mechanism is given by \structure{$\kappa = \kappa^*$} and \structure{$\rho = \{ p_{i,t}^* \}_{t=0}^T$}.
	\item Note that $i$ must pay his \structure{flow marginal contribution} rather than simply $w(k^*)-w(k^*_{-i})$.
	\item This is because $i$ by influencing today's allocation $k_t$, $i$ will also affect future types of other players and the set of available allocations -- have to account for that.
\end{itemize}
\end{frame}



\section{Dynamic Revenue Maximization}

\begin{frame}{Dynamic Revenue Maximization}
\begin{itemize}
	\item Second canonical question: what is the \structure{optimal mechanism}?
	\begin{itemize}
		\item Main example: dynamic pricing (there's huge literature, more or less related to DMD).
		\item With binding contracts: mobile service, loans, insurance
	\end{itemize}
	\begin{block}{Question}
		There is one buyer with time-changing valuation $\theta_t \in \Theta \subset \mathbb{R}$ for the item. \\
		What is the seller-optimal mechanism for \{repeated purchases, one-time purchase\}?
	\end{block}
	\item Again, insights from static models carry over after reasonable modifications.
	\begin{itemize}
		\item Now we want to distinguish between info that the buyer has \structure{before} signing up for a mechanism
		\item and which they acquire \alert{after} signing the contract.
	\end{itemize}
\end{itemize}
\end{frame}


\begin{frame}{Flashback: Static Model}
\begin{itemize}
	\item In the static optimal mechanism, seller's expected revenue was
	\begin{gather*}
		\mathbb{E}R = \mathbb{E}_\theta \left[ v(k(\theta),\theta) - \frac{1-F(\theta)}{\phi(\theta)} \frac{\partial v(k(\theta),\theta)}{\partial \theta} \right]
		\\
		\left(\text{we derived this for $v(k,\theta) = k\theta$:} \quad 
		\mathbb{E}R = \mathbb{E}_\theta \left[ k(\theta) VS^{static}(\theta) \right] \right)
	\end{gather*}
	\item Had to trade off max social surplus $v(k,\theta)$ (i.e., efficiency) against information rents.
	\begin{itemize}
		\item Had to leave some money to the buyer to incentivize truthful reporting.
	\end{itemize}
\end{itemize}
\end{frame}


\begin{frame}{Static Model -- Posterior Information}
\begin{example}
	\begin{itemize}
		\item Consider static optimal mechanism setting (1 period, 1 item, 1 buyer),
		\item \alert{except}: buyer only learns $\theta$ \structure{after} signing up for the mechanism. 
		\item What is the optimal contract? 
		
		\pause
		\begin{itemize}
			\item Designer's problem is
			\begin{align*}
				\max_{(k,p)} & \left\{ \mathbb{E}_\theta p(\theta) \right\}
				\\
				\text{s.t. } (IC):\enspace & v(k(\theta),\theta) - p(\theta) \geq v(k(\theta),\hat{\theta}) - p(\hat{\theta}) \quad \forall \theta, \hat{\theta},
				\\
				(eaIR):\enspace & \mathbb{E}_\theta \left[v(k,\theta) - p\right] .
			\end{align*}
			\item Only real difference from Myerson: ex ante IR instead of interim IR.
			\item \alert{Solution}: choose efficient $k^*$ and charge $p(\theta) \equiv p = \mathbb{E}_\theta \left[v(k^*(\theta),\theta)\right]$
			\item Perfect information extraction; no information rents to the buyer; full efficiency.
			\item \structure{Remark}: this solution would not work with $N>1$ bidders competing for 1 item (why?)
		\end{itemize}
	\end{itemize}
\end{example}
\end{frame}


\begin{frame}{Dynamics and Information (i)}
	\begin{block}{Statement (Future Extraction)}
		Designer can extract all of buyer's \structure{future} info at no cost.
	\end{block}
	\begin{minipage}[c][0.55\textheight][s]{\textwidth}
	\begin{itemize}
		\item Same \structure{idea}: ``sell'' the item (subscription) to the buyer at ex ante expected value.
		\item Then only buyer's \alert{initial} info $\theta_0$ matters for IC:
		\begin{itemize}
			\item in future periods use buyer-optimal allocation rule $\Rightarrow$ buyer's IC is satisfied without any extra transfers.
		\end{itemize}
		\item (FE) sounds reasonable, but it is not a formal theorem.
		\item The literature is currently at the stage ``let's hope that (FE) holds''. 
	\end{itemize}
	\end{minipage}
\end{frame}


\begin{frame}{Dynamics and Information (ii)}
	\begin{block}{Statement (Future Extraction)}
		Designer can extract all of buyer's \structure{future} info at no cost.
	\end{block}
	\begin{minipage}[c][0.55\textheight][s]{\textwidth}
		\begin{itemize}
			\item The literature is currently at the stage ``let's hope that (FE) holds''. 
			\item In particular, the protocol is:
			\begin{enumerate}
				\item Solve the dynamic problem \alert{as if} all future info is \alert{public}.
				\item Get some allocation and transfers.
				\item Check whether the resulting mechanism satisfies dynamic IC (at $t>0$).
				\item Pray that it does.
			\end{enumerate}
			\item \cite*{pavan_dynamic_2014} provide some sufficient conditions for this to work, but these are considered by some as too restrictive.
			\item We today take the ``pray that (FE) holds'' approach and only worry about extracting the buyer's initial type $\theta_0$ -- we are \structure{back} to the \structure{static problem}.
		\end{itemize}
	\end{minipage}
\end{frame}


\begin{frame}{Caveat}
	\begin{block}{Statement (Future Extraction)}
		Designer can extract all of buyer's \structure{future} info at no cost.
	\end{block}
	\begin{minipage}[c][0.55\textheight][s]{\textwidth}
		\begin{alertblock}{Caveat}
			``Ignore future information'' is not the same as ``ignore future types''!
		\end{alertblock}
		\begin{itemize}
			\item Type $\theta_0$ is (in general) correlated with future $\theta_t$,
			\item so $\theta_0$ contains some information about $\theta_t$,
			\item so we \textbf{cannot} work as if know $\theta_t$ for $t \geq 1$.
		\end{itemize}
	\end{minipage}
\end{frame}


\begin{frame}{Caveat}
	\begin{block}{Statement (Future Extraction)}
		Designer can extract all of buyer's \structure{future} info at no cost.
	\end{block}
	\begin{minipage}[c][0.55\textheight][s]{\textwidth}
		\begin{alertblock}{Caveat}
			``Ignore future information'' is not the same as ``ignore future types''!
		\end{alertblock}
		\begin{itemize}
			\item Solution: separate \alert{types} from \alert{information} through \structure{orthogonalization}.
			\begin{itemize}
				\item Suppose $\theta_{t+1} \sim F_{t+1} (\theta_{t+1} | \theta_t, k_t)$.
				\item Let $\varepsilon_{t+1} \equiv F_{t+1} (\theta_{t+1} | \theta_t, k_t)$. Then $\varepsilon_{t+1} \sim U[0,1]$ and independent of $\theta_t$.
				\item In a direct mechanism, ask player to report $\theta_0$ in period $0$ and $\varepsilon_{t}$ in period $t$, then recover $\theta_{t+1}$ from these reports.
			\end{itemize}
		\end{itemize}
	\end{minipage}
\end{frame}


\begin{frame}{Virtual Surplus}
\begin{itemize}
	\item Optimal allocation $\kappa$ maximizes \alert{virtual surplus} = \structure{real surplus} -- information rents.
	\begin{itemize}
		\item This pins down optimal mechanism $(\kappa,\pi+C)$ up to the constant $C$.
		\item $C$ is determined from IR at $t=0$ -- skip the step of finding it.
	\end{itemize}
	\item In \alert{static} model, \alert{virtual surplus} is (\emph{note inconsistency in how VS is defined here vs in past lectures!})
		$$ VS(k,\theta) = \structure{v(k(\theta),\theta)} - \frac{1-F(\theta)}{\phi(\theta)} \frac{\partial \structure{v(k(\theta),\theta)}}{\partial \theta} $$
	\item Now in \alert{dynamics}, \structure{real surplus} is 
		$$ \structure{S(\kappa,\theta)} \equiv \sum_{t\geq 0} \delta^t v(k_t(\theta_t),\theta_t).$$ 
	%$$ VS(\theta_0) = S(\kappa,\theta) - \frac{1-F(\theta)}{\phi(\theta)} \frac{\partial S(\kappa,\theta)}{\partial \theta_0} $$
	Calculating $VS(\kappa,\theta) = S(\kappa,\theta) - \frac{1-F_0(\theta_0)}{\phi_0(\theta_0)} \frac{\partial S(\kappa,\theta)}{\partial \theta_0}$ requires understanding how $S$ depends on $\theta_0$ (the only source of inforents for the buyer).
\end{itemize}
\end{frame}


\begin{frame}{Virtual Surplus}
\begin{align*}
	\frac{\partial S(\kappa,\theta)}{\partial \theta_0} &= \sum_{t\geq 0} \delta^t \frac{\partial v(k_t(\theta_t),\theta_t)}{\partial \theta_t} \frac{\partial \theta_t}{\partial \theta_0}
\end{align*}
\begin{itemize}
	\item Let $I_t(\theta_t | \theta^{t-1}, k_{t-1}) \equiv \frac{\partial \theta_t}{\partial \theta_0}$ be \structure{impulse response function}, where $\theta^t \equiv (\theta_0, \theta_1,...,\theta_t)$.
	\item $I_t$ shows the effect of $\theta_0$ on $\theta_t$ given fixed realization of uncertainty $\{\varepsilon_s\}_{s\leq t}$.
	\item Can compute that $$I_t(\theta_t | \theta^{t-1}, k_{t-1}) = -\prod_{s=1}^t \frac{\frac{\partial F_s(\theta_s | \theta^{s-1}, k_{s-1})}{\partial \theta_{s-1}}}{\phi_s(\theta_s | \theta^{s-1}, k_{s-1})}.$$
\end{itemize}
\end{frame}


\begin{frame}{Virtual Surplus}
Then
\begin{align*}
	\frac{\partial S(\kappa,\theta)}{\partial \theta_0} &= \sum_{t\geq 0} \delta^t \frac{\partial v(k_t(\theta_t),\theta_t)}{\partial \theta_t} \frac{\partial \theta_t}{\partial \theta_0}
	\\
	&= \sum_{t\geq 0} \delta^t I_t(\theta_t | \theta^{t-1}, k_{t-1}) \frac{\partial v(k_t(\theta_t),\theta_t)}{\partial \theta_t}
\end{align*}
so the whole virtual surplus as a function of the whole $\theta=(\theta_1,\theta_2,...)$ is
\begin{align*}
	VS(\kappa,\theta) &= S(\kappa,\theta) - \frac{1-F_0(\theta_0)}{\phi_0(\theta_0)} \frac{\partial S(\kappa,\theta)}{\partial \theta_0}
	\\
	&= S(\kappa,\theta) - \frac{1-F_0(\theta_0)}{\phi_0(\theta_0)} \sum_{t\geq 0} \delta^t I_t \frac{\partial v_t}{\partial \theta_t}
\end{align*}
(again, definition slightly different than in static opt.mech.; this one is more general)
\end{frame}


\begin{frame}{Optimal Mechanism}
\begin{itemize}
	\item To find optimal \structure{allocation}, take expectation of $VS(\kappa,\theta)$ over $\{\varepsilon_t\}$ to get $VS(\kappa,\alert{\theta_0})$ and maximize over $\kappa$.
	(Still a difficult problem, for the same reasons as for efficient $\kappa^*$.)
	\vspace{-0.5em}\begin{align*}
		\max_\kappa \mathbb{E}_\varepsilon \left[ S(\kappa,\theta) - \frac{1-F_0(\theta_0)}{\phi_0(\theta_0)} \sum_{t\geq 0} \delta^t I_t \frac{\partial v_t}{\partial \theta_t} \mid \theta_0 \right]
	\end{align*}\vspace{-1em}

	\item Then find expected (as of $t=0$) \structure{payments} from the envelope representation of the buyer's expected utility:
	\vspace{-0.5em}\begin{align*}
		\frac{d U_{b,0} (\theta_0)}{d \theta_0} = \mathbb{E} \left[ \sum_{t=0}^{T} \delta^t I_t(\theta_t | \theta^{t-1}, k_{t-1})\frac{\partial v(k_t,\theta_t)}{\partial \theta_t} \mid \theta_0 \right].
	\end{align*}\vspace{-1em}

	\item Note that this will pin down the ``expected-at-time-0'' payments $\mathbb{E}_\varepsilon [\sum_t \delta^t p_t(\theta^t)|\theta_0]$. These payments can be redistributed across periods and histories since both seller and buyer are risk-neutral.
	\begin{itemize}
		\item Will usually have to do this redistribution to ensure IC at $t > 0$. No good recipe here.
	\end{itemize}
\end{itemize}
\end{frame}



\begin{frame}{Dynamic Revenue Maximization: Conclusions}
\begin{align*}
	\max_\kappa \mathbb{E}_\varepsilon \left[ S(\kappa,\theta) - \frac{1-F_0(\theta_0)}{\phi_0(\theta_0)} \sum_{t\geq 0} \delta^t I_t \frac{\partial v_t}{\partial \theta_t} \mid \theta_0 \right]
\end{align*}
\begin{itemize}
	\item \structure{Insight}: if $|I_t|$ decreasing with $t$, i.e., $\theta_0$ contains little information about $\theta_t$ for large $t$ then optimal $k_t$ converges to the efficient $k^*_t$.
	\item \structure{\textbf{Distortions vanish over time.}}
	\item See Bergemann and Välimäki (2019, ch.5) for applications.
\end{itemize}
\end{frame}




\section{Three dynamic polarization results}

\begin{frame}{What now?}
\begin{itemize}
	\item Will look at dynamic mechanisms within some special settings.
	\item \structure{Beyond} the models we looked at, not \alert{within}.
	\item Will go very quickly: no solving models, just setup and results.
	\item Will see a common theme emerging.
\end{itemize}
\end{frame}


\subsection{Thomas and Worrall (1990)}

\begin{frame}{Dynamic Insurance \citep{thomas_income_1990}}
\begin{itemize}
	\item One risk-neutral lender (designer), one \alert{risk-averse} borrower (agent), common discount factor $\beta$.
	\item Time $t=0,1,...$.
	\item Agent receives random exogenous income $\theta_t \sim i.i.d. F[\underline{\theta},\bar{\theta}]$.
	\begin{itemize}
		\item \structure{Concave} utility $u(c)$, so would like to insure.
		%\item DARA: [weakly] decreasing absolute risk-aversion $-\frac{u''(c)}{u'(c)}$.
		\item Special assumption: $u(\underline{c}) = -\infty$, where $\underline{c}>0$ is subsistence level.
	\end{itemize}
	\item Principal designs insurance contract.
	\begin{itemize}
		\item Goal: minimize cost of providing (ex ante expected) util $V_0$ to agent.
		\item Agent reports $\theta_t$ in every period, mechanism pays him $b_t(\theta_t, \theta_{t-1}, ...)$
		\item Perfect commitment on both sides -- no IR.
		\item But must incentivize truthful reporting of income $\theta_t$ -- IC.
	\end{itemize}
	%\item Departure from baseline: no allocations $k$, risk-aversion (no quasilinearity in money).
\end{itemize}
\end{frame}


\begin{frame}{Agent's incentives}
\begin{itemize}
	\item At all $t$, agent maximizes lifetime utility 
	$$V_t \equiv \sum_{s=t}^\infty \beta^s u(\theta_s + b_s).$$
	\item Let $g^t = (\hat{\theta}_0, ..., \hat{\theta}_t)$ be history of past reports.
	\item Then agent's IC at $g^{t-1}$ is:
	\begin{align*}
		u(\theta_t + b_t(g^{t-1},{\theta}_t)) +\beta V_{t+1}(g^{t-1},{\theta}_t) \geq 
		\\
		\geq u(\theta_t + b_t(g^{t-1},\hat{\theta}_t)) +\beta V_{t+1}(g^{t-1},\hat{\theta}_t).
	\end{align*}
	for all $\theta_t,\hat{\theta}_t$.
\end{itemize}
\end{frame}


\begin{frame}{Relation to standard model}
\begin{itemize}
	\item Note that there are no allocations, only money across periods.
	\item One way to relate to our standard quasilinear model:
	\medskip
	\begin{center}
		\begin{tabular}{| c | c |}
			\hline
			\emph{usual model} 	& \emph{this model} \\ \hline
			allocation $k$ 	& today's transfer $b_t$\\ \hline
			transfer $t$ 	& continuation util $V_{t+1}$\\ \hline
		\end{tabular}
	\end{center}
	\medskip
	\item The main intertemporal linkage comes from the need to deliver on promised $V_{t+1}$.
\end{itemize}
\end{frame}


\begin{frame}{Efficient contract}
\begin{itemize}
	\item Moving on to the results.
	\item In the optimal contract, at every $g^{t-1}$:
	\begin{itemize}
		\item $b_t$ is decreasing in $\hat{\theta}_t$ (insurance);
		\item $V_{t+1}$ is increasing in $\hat{\theta}_t$ (IC).
		\item In particular, $b_t(\underline{\theta}) > 0 > b_t(\overline{\theta})$; $V_{t+1}(\underline{\theta}) < V_t < V_{t+1}(\overline{\theta})$.
	\end{itemize}
	\item First-best (cheapest way to deliver util $V_t$) would be to provide full insurance, but have to trade efficiency against info rents, so incomplete insurance in the optimum. (Standard opt.mech logic)
	%\item (NEW) Can trade off levels: raise all $b_t$ while lowering all $V_{t+1}$ or vice versa. 
\end{itemize}
\end{frame}


\begin{frame}{Immiseration}
\begin{theorem}[Immiseration]
	$$\lim_{t\to \infty} V_t \overset{a.s.}{=} -\infty$$
\end{theorem}
\begin{itemize}
	\item In the limit, consumption and utility converge to $\bar{c}$ and $-\infty$ resp.
	\item Neat mathematical result, but I haven't found any good intuitive explanations of where it comes from and after some thorough thinking cannot offer any correct intuition of my own.
	\item Popular paper, has quite some citations and influential follow-up papers.
\end{itemize}
\end{frame}


\subsection{Guo and H{\"o}rner (2018)}

\begin{frame}{Dynamic Allocation without Money \citep{guo_dynamic_2018}}
\begin{itemize}
	\item One principal, one agent.
	\item Time $t=0,1,...$
	\item In each period: agent's type $v \in \{L,H\}$, principal chooses $a \in \{0,1\}$. Utilities (P,A):
	\medskip
	\begin{center}
		\begin{tabular}{c | c | c |}
			$(u_P,u_A)$ & $v = H$ 	& $v=L$ \\ \hline
			$a=0$	& $(0,0)$ 	& $(0,0)$	\\ \hline
			$a=1$	& $(H-c,H)$ 	& $(L-c,L)$	\\ \hline
		\end{tabular}
	\end{center}
	\medskip
	with $H > c > L > 0$.
	\item Idea: principal can provide funding for agent's project, it is costly for the principal, but agent always wants more funding.
	\item Persistence: $\mathbb{P}(v_{t+1} = v_t) = \rho \geq 1/2$.
	\item Principal's goal: max own discounted util subject to IC.
\end{itemize}
\end{frame}


\begin{frame}{Connection}
\begin{itemize}
	\item Like Thomas and Worrall, but there had only transfers, no allocations. Here only allocations, no transfers.
	\item Same idea behind IC: induce truthtelling today by varying future utility promises.
	\medskip
	\begin{center}
		\begin{tabular}{| c | c |}
			\hline
			\emph{usual model} 	& \emph{this model} \\ \hline
			allocation $k$ 	& today's \structure{allocation $a_t$}\\ \hline
			transfer $t$ 	& continuation util $V_{t+1}$\\ \hline
		\end{tabular}
	\end{center}
	\medskip
	\item Opt. mech: if agent does not require funding today, allow to claim more funding in the future. For $v=H$ agent, funding today is more valuable than in the future (since $\mathbb{E} v_{t+s} < H$), for $v=L$ future funding is more valuable than today $\Rightarrow$ IC.
\end{itemize}
\end{frame}


\begin{frame}{Polatization}
\begin{itemize}
	\item Let $U_t \equiv (1-\delta) \mathbb{E} \left[\sum_{s \geq t} \delta^{s-t} a_t v_t\right]$ denote agent's util. 
	
	Note $U_t \in \left[0,\bar{U}\right]$ for some $\bar{U}$.
	
	\begin{theorem}[Polarisation]
		Under the optimal mechanism, $U_t \to \{0, \bar{U}\}$ as $t \to \infty$.
	\end{theorem}
	
	\item $U_t$ is (not really, but similar for our purposes to) a martingale bounded on both sides -- both boundaries are absorbing, and $U_t$ hits one of them sooner or later.
\end{itemize}
\end{frame}


\subsection{Li, Matouschek, Powell (2017)}

\begin{frame}{Power Dynamics in Organizations \citep{li_power_2017}}
\begin{itemize}
	\item One principal, one agent.
	\item Time $t=0,1,...$
	\item In each period: principal chooses $a \in \{0,1,2,3\}$. Utilities (P,A):
	\medskip
	\begin{center}
		\begin{tabular}{c c | c | c |}
			& & principal 	& agent \\ \hline
			$a=0$ & (default)	& $0$ 	& $0$	\\ \hline
			$a=1$ & (agent-preferred)	& $b$ 	& $B$	\\ \hline
			$a=2$ & (principal-preferred)	& $B$ 	& $b$	\\ \hline
			$a=3$ & (nuke humanity)	& $-\infty$	& $-\infty$ \\ \hline
		\end{tabular}
	\end{center}
	\medskip
	with $B > b > 0$.
	\item Principal-preferred project \structure{only available} at any $t$ \structure{with probability $p$}. Only the agent knows whether $a=2$ is available at a given $t$. Agent suggests a project to principal at every $t$.
	\item Principal's goal: maximize expected util subject to agent's IC.
	\item Principal \alert{cannot commit} to future contracts (so this is just a repeated game).
\end{itemize}
\end{frame}


\begin{frame}{Possible Modes}
\begin{itemize}
	\item Centralization
	\begin{itemize}
		\item Principal always chooses the default $a=0$.
	\end{itemize}
	\item Cooperative Empowerment
	\begin{itemize}
		\item Agent suggests and principal implements principal-preferred $a=2$ when available, agent-preferred $a=1$ otherwise.
		\item The ``best'' outcome.
	\end{itemize}
	\item Restricted Empowerment
	\begin{itemize}
		\item Agent suggests and principal implements principal-preferred $a=2$ when available, default $a=0$ otherwise.
	\end{itemize}
	\item Unrestricted Empowerment
	\begin{itemize}
		\item Agent suggests and principal implements agent-preferred $a=1$ always.
	\end{itemize}
	\item Total Annihilation
	\begin{itemize}
		\item Principal implements $a=3$; only used as off-path threat.
	\end{itemize}
\end{itemize}
\end{frame}


\begin{frame}{Polatization}
\begin{theorem}
	In the optimal relational contract, the principal chooses cooperative empowerment for the first $\tau$ periods, where $\tau$ is random and finite with probability one.
	
	For $t > \tau$, the relationship results in unrestricted empowerment, restricted empowerment, or centralization forever
\end{theorem}
\begin{itemize}
	\item The relationship inevitably slips out of the cooperative mode into one of the uncooperative ones:
	\begin{itemize}
		\item either the agent gets \alert{unlimited power},
		\item or the principal \structure{loses trust} in him.
		\item Although convergence to restricted empowerment (semi-cooperative outcome) is possible...
	\end{itemize}
\end{itemize}
\end{frame}


\begin{frame}{Conclusion}
\begin{itemize}
	\item Lessons from the three papers:
	\begin{itemize}
		\item relying on promises of future utility for incentive provision leads to huge asymptotic inefficiencies.
	\end{itemize}
	\item Drastically different from the quasilinear setting we considered,
	\begin{itemize}
		\item where inefficiencies vanished over time...
	\end{itemize}
\end{itemize}
\end{frame}


\appendix
\begin{frame}[allowframebreaks]{References}
\bibliography{teaching}
\bibliographystyle{abbrvnat}
\end{frame}


\end{document}