%%% License: Creative Commons Attribution Share Alike 4.0 (see https://creativecommons.org/licenses/by-sa/4.0/)

\documentclass[english,10pt
,aspectratio=169
%,handout
%%%%,notes
]{beamer}
%%% License: Creative Commons Attribution Share Alike 4.0 (see https://creativecommons.org/licenses/by-sa/4.0/)

\DeclareGraphicsExtensions{.eps, .pdf,.png,.jpg,.mps,}
\usetheme{reMedian}
\usepackage{parskip}
\makeatother

\renewcommand{\baselinestretch}{1.1} 

\usepackage{amsmath, amssymb, amsfonts, amsthm}
\usepackage{enumerate}
%\usepackage{enumitem}
\usepackage{hyperref}
\usepackage{url}
\usepackage{bbm}
\usepackage{color}

\usepackage{tikz}
\usepackage{tikzscale}
\newcommand*\circled[1]{\tikz[baseline=(char.base)]{
		\node[shape=circle,draw, inner sep=-20pt] (char) {#1};}}
\usetikzlibrary{automata,positioning}
\usetikzlibrary{decorations.pathreplacing}
\usepackage{pgfplots}
\usepgfplotslibrary{fillbetween}
\usepackage{graphicx}

\usepackage{setspace}
\thinmuskip=1mu
\medmuskip=1mu 
\thickmuskip=1mu 


\usecolortheme{default}
\usepackage{verbatim}
\usepackage[normalem]{ulem}

\usepackage{apptools}
\AtAppendix{
	\setbeamertemplate{frame numbering}[none]
}
\usepackage{natbib}


% red strikeout
\newcommand\soutred{\bgroup\markoverwith
	{\textcolor{red}{\rule[0.55ex]{2pt}{0.8pt}}}\ULon}



% To use LyX frames from old version:
\def\lyxframeend{} % In case there is a superfluous frame end
\long\def\lyxframe#1{\@lyxframe#1\@lyxframestop}%
\def\@lyxframe{\@ifnextchar<{\@@lyxframe}{\@@lyxframe<*>}}%
\def\@@lyxframe<#1>{\@ifnextchar[{\@@@lyxframe<#1>}{\@@@lyxframe<#1>[]}}
\def\@@@lyxframe<#1>[{\@ifnextchar<{\@@@@@lyxframe<#1>[}{\@@@@lyxframe<#1>[<*>][}}
\def\@@@@@lyxframe<#1>[#2]{\@ifnextchar[{\@@@@lyxframe<#1>[#2]}{\@@@@lyxframe<#1>[#2][]}}
\long\def\@@@@lyxframe<#1>[#2][#3]#4\@lyxframestop#5\lyxframeend{%
	\frame<#1>[#2][#3]{\frametitle{#4}#5}}


\title{Mechanism Design}

\subtitle{2: Mechanisms for arbitrary s.c.f.s}

\author{Egor Starkov}

\date{K{\o}benhavns Unversitet \\
	Fall 2020}


\begin{document}
	\AtBeginSection[]{
		\frame{
			\frametitle{This slide deck:}
			\tableofcontents[currentsection,currentsubsection]
	}}
	\frame[plain]{\titlepage}



\section{Social Choice Theory}


\begin{frame}{Social Choice Theory}
\begin{itemize}
	\item Sum of utilities is just one measure of welfare -- and pretty arbitrary at that.
	\begin{itemize}
		\item utilities $u_i$ are nice for exploring \alert{intra}personal trade-offs when making decisions;
		\item not so good for \alert{inter}personal comparisons -- how can we even measure relative preference intensity?
	\end{itemize}
	\item Can we have a better preference aggregator?
	\item \structure{Social Choice Theory} (\& Welfare Economics) deal with aggregating individual preferences into social preference.
\end{itemize}
\end{frame}


\begin{frame}{Social choice: Goal}
\begin{itemize}
	\item Adopt the \alert{general} setting.
	\item We do not like cardinal utilities -- can instead work with ordinal preference relations $\succsim_i$.
	\begin{exampleblock}{The goal}
		Given the environment ($N,\{\Theta_i\},X,\{\succsim_i(\theta)\}$), find a s.c.f. $f$ -- or, more generally, a social preference relation $\succsim_S(\theta)$ over $X$ -- that would be worthy of the name ``efficient''
	\end{exampleblock}
	\item But what are the requirements that a good efficient $f$ must satisfy?
	\begin{itemize}
		\item Previously, we wanted to maximize sum of utilities. What do we want now?
	\end{itemize}
	\item Since we are not talking about incentives at the moment, I will omit $\theta$s in what follows.
\end{itemize}
\end{frame}


\begin{frame}{Social choice: Axiomatic approach}
	\begin{itemize}
		\item Can impose \alert{axioms} on how a collection of $\{\succsim_i\}$ should map into $\succsim_S$ or $f$.
		\item Possible reasonable axioms:
	\end{itemize}
	\begin{description}
		\item[(A1)] \structure{Domain}: any collection of individual preferences $\left(\succsim_1, ..., \succsim_N \right)$ can be aggregated into $\succsim_S$.
		\item[(A2)] \structure{Unanimity}: if $a \succsim_i b$ for all $i$ then $a \succsim_S b$.
		\item[(A3)] \structure{Independence of Irrelevant Alternatives}: if $\succsim_i$ and $\succsim'_i$ rank alternatives $a$ and $b$ the same for all $i$ then so should $\succsim_S$ and $\succsim'_S$.
	\end{description}
\end{frame}


\begin{frame}{Social choice: Arrow's theorem}
	\begin{theorem}[Arrow]
		With more than three alternatives, if $\succsim_S$ satisfies (A1)-(A3) then it is dictatorial, i.e. $\exists i: a \succsim_S b \Leftrightarrow a \succsim_i b$.
	\end{theorem}
	\begin{itemize}
		\item The same is true for s.c.f.s if we rephrase the axioms appropriately. (See MWG Prop 21.E.1.)
		\item See \href{https://link.springer.com/article/10.1007/s00199-004-0556-7}{\uline{Geanakoplos (2005)}} for [slightly] more details on Arrow's Thm and a proof or three.
		\item Arrow's Thm is a sad result for us. Were our axioms too strong?
		\begin{itemize}
			\item We can relax requirements to $\succsim_S$ -- e.g., say it only needs to be acyclical, but not transitive. But in that case some form of dictatorship still arises (oligarchy, veto power...)
			\item Also, A1 is actually very strong once you think of it -- let's try to relax that.
		\end{itemize}
	\end{itemize}
\end{frame}


\begin{frame}{Social choice: Single-peaked preferences}
	\begin{itemize}
		\item Suppose now that the alternatives are ordered in some sense: \\
		$X = (x_1, x_2, ..., x_M)$ with $x_1 < x_2 < ... < x_M$.
		\item Suppose also that preferences are single-peaked w.r.t. that order:
		\begin{exampleblock}{}
			Preference relation $\succsim_i$ is \structure{single-peaked} if $\exists x^*$ s.t. for any $x_k < x_l\leq x^*$, $x_l \succsim_i x_k$, and for any $x^* \leq x_k < x_l$, $x_k \succsim_i x_l$.
		\end{exampleblock}
	\end{itemize}
	In this environment  we can construct both a reasonable social preference $\succsim_S$ that satisfies all the axioms, and a mechanism that implements it.
\end{frame}


\begin{frame}{Social choice: Majority voting}
	\begin{theorem}[MWG 21.D.1-2]
		In the general setting with single-peaked preferences and odd number of players $N$, \alert{pairwise majority voting} generates a well-defined social preference $\succsim_S$.
		\\
		S.c.f. $f_S$ based on such $\succsim_S$ will always select the peak of the median voter.
	\end{theorem}
	\begin{itemize}
		\item You can verify that PMV satisfies (A1)-(A3).
		
		\item This gives us a natural notion of \structure{efficiency} which says \\
		``Social preference is $x_k \succsim_S x_l$ if and only if for the \structure{majority} of $i$,  $x_k \succsim_i x_l$''.
		
		\item Note that PMV always generates some $\succsim_S$, but without single-peaked preferences it may be nonsensical. E.g., it can be intransitive: $x_1 \succ_S x_2 \succ_S x_3 \succ_S x_1$ (a \structure{Condorcet paradox}).
	\end{itemize}
\end{frame}


\begin{frame}{Social choice: conclusion}
	\begin{itemize}
		\item Choosing as a society is hard.
		\item There is no single good recipe to always make the efficient choice
		\item ...so in the past weeks we worked with a notion of efficiency, with which we can actually work.
		\begin{itemize}
			\item We selected some rule according to which $f^*(\theta)$ (or $x^*(\theta)$ or $k^*(\theta)$) must be constructed and tried to implement that
			\item instead of debating what this rule must be.
			\item But this is an open debate as well.
		\end{itemize}
	\end{itemize}
\end{frame}



\section{Testing Implementability}

\begin{frame}{Testing Implementability}
	\begin{itemize}
		\item Forget all about efficiency and social choice.
		\item Suppose we have some s.c.f. $f(\theta)$ that we want to implement. Question now is:
		\begin{exampleblock}{}
			How can we check whether a given s.c.f. $f(\theta)$ is implementable?
		\end{exampleblock} 
		\item In the course so far, we have seen two answers already.
	\end{itemize}
\end{frame}


\begin{frame}{DSIC: Weak Preference Reversal Property}
	\begin{itemize}
		\item \textbf{Answer 1}: \structure{revelation principle}. Construct a DRM and check all players' \alert{IC conditions}. I.e., that for all $i,\theta_i',\theta_i'',\theta_{-i}$ it holds that
		$$ u_{i}(f(\theta_{i}', \theta_{-i}), \theta_{i}') \geq u_{i}(f(\theta_{i}'', \theta_{-i}), \theta_{i}').$$
		
		\item Note that they imply the following \structure{preference reversal}: for all $i,\theta_i',\theta_i'',\theta_{-i}$,
		\begin{equation*}
			\begin{cases}
				u_{i}(f(\theta_{i}', \theta_{-i}), \theta_{i}') \geq u_{i}(f(\theta_{i}'', \theta_{-i}), \theta_{i}'),
				\\
				u_{i}(f(\theta_{i}', \theta_{-i}), \theta_{i}'') \leq u_{i}(f(\theta_{i}'', \theta_{-i}), \theta_{i}'').
			\end{cases}
		\end{equation*}
		\item $i$'s preference between $f(\theta_{i}', \theta_{-i})$ and $f(\theta_{i}'', \theta_{-i})$ should flip when his type changes from $\theta_i'$ to $\theta_i''$ for $f$ to be DSIC. 
		\item ``To each their own'': different types should get their most preferred option among the available one.
	\end{itemize}
\end{frame}


\begin{frame}{Monotonicity: Euclidean}
	\begin{itemize}
		\item \textbf{Answer 2}: \structure{monotonicity}.
		\item In a Euclidean setting, $k(\theta)$ is implementable only if it is weakly monotone (increasing).
		\item Turns out, this is a sharp characterization: \\
		if $k(\theta)$ is monotone, there exist transfers $t$ such that $\Gamma = (\Theta, (k,t))$ is DSIC.
		\begin{itemize}
			\item To prove: use the envelope representation of payoffs to construct transfers; can then show that the resulting mechanism is DSIC.
		\end{itemize}
		\item This characterization is extendable to more general settings, although the statements get uglier.
	\end{itemize}
\end{frame}


\begin{frame}{Monotonicity: quasilinear}
	One example of this.
	\begin{exampleblock}{Definition (weak monotonicity)}
		Allocation $k$ is \alert{weakly monotone} if for all $i,\theta_i',\theta_i'',\theta_{-i}$:
		\begin{equation*}
			v_{i}(k(\alert{\theta_{i}'}, \theta_{-i}), \alert{\theta_{i}'}) - 
			v_{i}(k(\alert{\theta_{i}''}, \theta_{-i}), \alert{\theta_{i}'}) 
			\geq 
			v_{i}(k(\alert{\theta_{i}'}, \theta_{-i}), \alert{\theta_{i}''}) - 
			v_{i}(k(\alert{\theta_{i}''}, \theta_{-i}), \alert{\theta_{i}''}) 
		\end{equation*}
	\end{exampleblock}
	\begin{theorem}[Necessity of weak monotonicity in qlin setting]
		In a quasilinear setting: if $k$ is DSIC then $k$ is weakly monotone.
	\end{theorem}
	So $k$ must be weakly monotone to be implementable, but weak monotonicity does not guarantee implementability. But we can strengthen this...
\end{frame}


\begin{frame}{Monotonicity: quasilinear}
	\begin{exampleblock}{Definition (cyclical monotonicity)}
		Allocation $k$ is \alert{cyclically monotone} if for all $i,\theta_{-i}$, and all sequences $(\theta_i^1,\theta_i^2,...,\theta_i^M)\in \Theta_i^M$ of arbitrary length $M$ s.t. $\theta_i^M=\theta_i^1$, the following holds:
		\begin{equation*}
			\sum_{m=1}^{M-1}
			\left[
			v_{i}(k(\alert{\theta_{i}^m}, \theta_{-i}), \alert{\theta_{i}^{m+1}}) - 
			v_{i}(k(\alert{\theta_{i}^{m+1}}, \theta_{-i}), \alert{\theta_{i}^m}) 
			\right] 
			\leq 0
		\end{equation*}
	\end{exampleblock}
	\begin{theorem}[Rochet, 1987]
		In a quasilinear setting: $k$ is DSIC if and only if $k$ is cyclically monotone.
	\end{theorem}
	\textbf{Note}: ``Weak monotonicity'' = ``cyclical monotonicity for $M=3$''.
	See B{\"o}rgers, ch.5.3-5.4 for proofs or references to proofs (for $N=1$). See rest of ch.5 for other kinds of monotonicity for quasilinear setting.
\end{frame}


\begin{frame}{Monotonicity: general}
	Without transfers, interesting results come up... (\textbf{Assume} $X$ is finite and $\Theta_i$ are rich enough to contain all possible preferences over $X$ for all $i$.)
	\begin{exampleblock}{Definition (outcome monotonicity)}
		In a general setting, outcome $x$ is \alert{monotone} if for all $\theta',\theta'' \in \Theta$ the following holds:
		\begin{itemize}
			\item if for all $i$ and all $x \in X$ s.t. $u_i(x(\theta'),\theta') \geq u_i(x,\theta')$, it also holds that $u_i(x(\theta'),\theta'') \geq u_i(x,\theta'')$,
			\item then $x(\theta'')=x(\theta')$
		\end{itemize}
	\end{exampleblock}
	In words, if $x(\theta')$ is ``higher up'' in preferences under $\theta''$ than it is under $\theta'$, and it was chosen under $\theta'$, then it must definitely be chosen under $\theta''$.
	\begin{theorem}[Necessity of monotonicity in general setting]
		In a general setting: if $x$ is DSIC and $x(\Theta)=X$ then $x$ is monotone.
	\end{theorem}
	(This is not THE interesting part yet. The next result is.)
\end{frame}


\begin{frame}{Monotonicity: general}
	\begin{exampleblock}{Definition (dictatorial s.c.f.)}
		S.c.f. $f$ is \alert{dictatorial} if there exists $i \in N$ s.t. for all type profiles $\theta$:
		$$ f(\theta) \in \arg \max_x u_i(x,\theta_i)$$
	\end{exampleblock}
	\begin{theorem}[Gibbard-Satterthwaite, 1973,'75]
		In a general setting with $|X|\geq 3$: if $x(\Theta)=X$, then\\
		\centering
		$x$ is \alert{DSIC} if and only if $x$ is \alert{dictatorial}.
	\end{theorem}
	\begin{itemize}
		\item \textbf{Note}: restriction $x(\Theta)=X$ is immaterial: if $x(\Theta) \subset X$ then only preferences over alternatives in $x(\Theta)$ are relevant, and the society ex ante agrees to rule out everything else. We will still have a dictatorship on $x(\Theta)$.
	\end{itemize}
\end{frame}


\begin{frame}{Monotonicity: general}
	\begin{theorem}[Monotonicity implies dictatorship]
		In a general setting with $|X|\geq 3$: if $x(\Theta)=X$, then\\
		\centering
		if $x$ is monotone then $x$ is dictatorial.
	\end{theorem}
	\begin{itemize}
		\item The missing link is the following:
		\item \textbf{Note}: all three results for the general setting hold with ordinal preferences ($\succsim_i$) too, do not rely on cardinal utilities $u_i$.
	\end{itemize}
\end{frame}


\begin{frame}{Monotonicity: general}
	\begin{itemize}
		\item Note parallels between GS theorem and Arrow theorem.
		\item GS thm is extendable to infinite $X$.
		\item The source of evil in GS thm is again the domain assumption: we often know \emph{something} about some players' preferences, so ``all possible preferences over $X$'' may be too rich for $\Theta_i$.
		\begin{itemize}
			\item We have already seen that once we restrict preferences (e.g. to quasilinear -- assuming ``everyone always likes money'') then there exist non-dictatorial DSIC implementable s.c.f.s (outcome functions) $f(\theta)$.
			\item Single-peaked preferences, in particular, also allow for non-dictatorial DSIC mechanisms. (B{\"o}rgers, Prop 8.6)
		\end{itemize}
		\item For a proof of GS thm, see B{\"o}rgers, ch.8.2, or \href{http://dx.doi.org/10.1016/j.jmateco.2014.09.007}{\uline{Svensson and Reffgen (2014)}}
	\end{itemize}
\end{frame}


\end{document}