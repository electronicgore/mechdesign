%%% License: Creative Commons Attribution Share Alike 4.0 (see https://creativecommons.org/licenses/by-sa/4.0/)

\documentclass[english,10pt
,aspectratio=169
%,handout
%%%%%,notes
]{beamer}
%%% License: Creative Commons Attribution Share Alike 4.0 (see https://creativecommons.org/licenses/by-sa/4.0/)

\DeclareGraphicsExtensions{.eps, .pdf,.png,.jpg,.mps,}
\usetheme{reMedian}
\usepackage{parskip}
\makeatother

\renewcommand{\baselinestretch}{1.1} 

\usepackage{amsmath, amssymb, amsfonts, amsthm}
\usepackage{enumerate}
%\usepackage{enumitem}
\usepackage{hyperref}
\usepackage{url}
\usepackage{bbm}
\usepackage{color}

\usepackage{tikz}
\usepackage{tikzscale}
\newcommand*\circled[1]{\tikz[baseline=(char.base)]{
		\node[shape=circle,draw, inner sep=-20pt] (char) {#1};}}
\usetikzlibrary{automata,positioning}
\usetikzlibrary{decorations.pathreplacing}
\usepackage{pgfplots}
\usepgfplotslibrary{fillbetween}
\usepackage{graphicx}

\usepackage{setspace}
\thinmuskip=1mu
\medmuskip=1mu 
\thickmuskip=1mu 


\usecolortheme{default}
\usepackage{verbatim}
\usepackage[normalem]{ulem}

\usepackage{apptools}
\AtAppendix{
	\setbeamertemplate{frame numbering}[none]
}
\usepackage{natbib}


% red strikeout
\newcommand\soutred{\bgroup\markoverwith
	{\textcolor{red}{\rule[0.55ex]{2pt}{0.8pt}}}\ULon}



% To use LyX frames from old version:
\def\lyxframeend{} % In case there is a superfluous frame end
\long\def\lyxframe#1{\@lyxframe#1\@lyxframestop}%
\def\@lyxframe{\@ifnextchar<{\@@lyxframe}{\@@lyxframe<*>}}%
\def\@@lyxframe<#1>{\@ifnextchar[{\@@@lyxframe<#1>}{\@@@lyxframe<#1>[]}}
\def\@@@lyxframe<#1>[{\@ifnextchar<{\@@@@@lyxframe<#1>[}{\@@@@lyxframe<#1>[<*>][}}
\def\@@@@@lyxframe<#1>[#2]{\@ifnextchar[{\@@@@lyxframe<#1>[#2]}{\@@@@lyxframe<#1>[#2][]}}
\long\def\@@@@lyxframe<#1>[#2][#3]#4\@lyxframestop#5\lyxframeend{%
	\frame<#1>[#2][#3]{\frametitle{#4}#5}}


\title{Mechanism Design}

\subtitle{2: Efficient Mechanisms}

\author{Egor Starkov}

\date{K{\o}benhavns Unversitet \\
	Fall 2023}


\begin{document}
	\AtBeginSection[]{
		\frame{
			\frametitle{This slide deck:}
			\tableofcontents[currentsection,currentsubsection]
	}}
	\frame[plain]{\titlepage}



\section{Problem setup}

\begin{frame}{Quasilinear Preferences}
	\begin{alertblock}{Assumption: Quasilinear setting}
		\begin{itemize}
			\item Instead of allowing all possible preferences, adopt a special structure.
			\item Instead of $x \in X$ describing everything related to outcome, split it into:
			\begin{itemize}
				\item $\alert{k(\theta) \in K}$, ``real/material outcome'' a.k.a. \structure{allocation}
				\item $\alert{t(\theta) \in \mathbb{R}^N}$, \structure{transfers/payments}
			\end{itemize}
			\item Instead of arbitrary $u_i(x,\theta)$ focus on \structure{quasilinear preferences}:
			$$\alert{ u_i(x,\theta_i) = v_i(k,\theta_i) - t_i }$$
			%NOTE: v_i can only depend on theta_i (as opposed to theta) due to Clarke payments
			\vspace{-1em}
			\item S.c.f. is $f(\theta) = \left( k(\theta), t_1(\theta), ..., t_N(\theta) \right)$
		\end{itemize}
	\end{alertblock}
\end{frame}


\begin{frame}{Quasilinear Preferences}
\begin{itemize}
	\item Common interpretation: transfers=payments. This comes with a bunch of assumptions:
	\begin{itemize}
		\item Monetary transfers always available,
		\item individual utility is linear in money (risk-neutrality),
		\item marginal social utility of money is constant across types and people and independent of allocation.
	\end{itemize}
	\item All three are sometimes restrictive, the latter two especially. 
	\item However, monetary payments are not necessary! Anything that $i$ cares about that is not directly included in the allocation $k$ can be used to adjust $i$'s utility as needed!
	\begin{itemize}
		\item $i$'s time, $i$'s effort, promises to $i$, etc
	\end{itemize}
\end{itemize}
\end{frame}


\begin{frame}{Efficient Implementation}
\begin{itemize}
	\item A frequent question: ``Dr.Professor, how can we as society implement \structure{the efficient outcome}?''
	\item Reminder: efficient outcome $x^*(\theta) = (k^*(\theta),t^*(\theta))$ is 
	\vspace{-0.5em}\begin{align*}
	x^*(\theta) &= \arg \max_x \sum_{i=0}^N u_i(x,\theta_i) 
	= \arg \max_{(k,t)} \sum_{i=0}^N \left[v_i(k,\theta_i) - t_i\right]
	\end{align*}
	\item Transfers just reallocate utility across agents, so focus on \structure{efficient allocation $k^*(\theta)$}:
	\vspace{-1em}\begin{align*}
	k^*(\theta) = \arg \max_k \sum_{i=0}^N v_i(k,\theta_i)
	\end{align*}
	\item Note that we can include $i=0$ into welfare calculations. This can capture designer's preferences or any social costs/benefits not captured by individual agents (e.g., cost of implementing a public project)
\end{itemize}
\end{frame}


\begin{frame}{Efficient Implementation}
\begin{itemize}
	\item How do we do that?
	\begin{itemize}
		\item We already know that it's enough to consider direct revelation mechanisms.
		\item We have the desired allocation rule $k^*$, but we can design the transfers $t$ -- the problem is not just ``check whether s.c.f. $x^*$ is IC'', but ``is there such $t$ that $k^*$ is IC?''
	\end{itemize}
	\item What we as designers want:
	\vspace{-1em}\begin{align*}
	\max \sum_{i=0}^N v_i(k,\theta_i)
	\end{align*}
	\item What agent $i$ wants:
	\vspace{-1em}\begin{align*}
	\max v_i(k,\theta_i) - t_i
	\end{align*}
	\item How to reconcile the two?
\end{itemize}
\end{frame}



\section{VCG}

\begin{frame}{VCG Mechanism: intro}
\begin{itemize}
	\item We now introduce the VCG mechanism that DSIC-implements the efficient allocation. We shall do it in a few steps.
	\item \textbf{NOTE}: while the broad idea behind the VCG mechanism is the same everywhere, the \alert{exact definition} of the VCG mechanism \alert{differs} in different sources (textbooks).
\end{itemize}
\end{frame}


\begin{frame}{VCG Mechanism: Groves' Transfers}
\begin{itemize}
	\item More formally, the problem of agent $i$ of type $\theta_i$ is:
	\vspace{-0.5em}\begin{align*}
	\max_{\hat{\theta}_i} \left\{  v_i(k^*(\hat{\theta}_i,\theta_{-i}),\theta_i) - t_i(\hat{\theta}_i,\theta_{-i}) \right\}
	\end{align*}\vspace{-1em}
	\item Try \alert<1>{Groves' transfers}:
	\vspace{-0.5em}\begin{align*}
	\structure<1>{ t^G_{i}(\theta) \equiv -\left(\sum_{j\neq i} v_{j}(k^*(\theta_i, \theta_{-i}), \theta_{j}) \right) + h_{i}(\theta_{-i}) }
	\end{align*}\vspace{-1em}
	\item Agent's problem is now
	\vspace{-0.5em}\begin{align*}
	\max_{\hat{\theta}_i} \left\{ \structure{v_i(k^*(\alert<3>{\hat{\theta}_i},\theta_{-i}),\theta_i) + \left( \sum_{j\neq i} v_{j}(k^*(\alert<3>{\hat{\theta}_i},\theta_{-i}), \theta_{j}) \right)} - h_{i}(\theta_{-i}) \right\}
	\end{align*}
\end{itemize}
\end{frame}
\note{
	Note on interpretation: the happier you make others, the less you have to pay.
	\\
	The next slide is a direct continuation of this one.
}


\begin{frame}{VCG Mechanism: Groves' Transfers}
\begin{itemize}
	\item Agent's problem is now
	\vspace{-0.5em}\begin{align*}
	\max_{\hat{\theta}_i} \left\{ \structure{ \sum_{j \in N} v_{j}(k^*(\alert{\hat{\theta}_i},\theta_{-i}), \theta_{j}) } - h_{i}(\theta_{-i}) \right\}
	\end{align*}
	\item Every agent $i$ chooses report $\hat{\theta}_i$ to maximize welfare!
	\begin{itemize}
		\item Optimal to report true $\hat{\theta}_i$,
		\item for any $\theta_{-i}$.
	\end{itemize}
	\item Crucial that $h_i(\theta_{-i})$ does not depend on $i$'s report.
\end{itemize}
\end{frame}
%TODO: VCG proof?


\begin{frame}{VCG Mechanism: Example}
\begin{exampleblock}{Example (Moon Base)}
	\begin{itemize}
		\item $N$ citizens decide whether to build a Moon base which costs $c$
		\item citizen $i$ has private valuation $\theta_{i}$ for the base and quasilinear utility
		
		(so if base built then $v_i = \theta_i$, otherwise $v_i = 0$)
	\end{itemize}
\end{exampleblock}
\begin{itemize}
	\item What are Groves' transfers? (Take $h_i(\theta_{-i}) \equiv 0$.)
	% $$ t_i(\theta) = - \mathbb{I} \left\{\sum_{j=1}^N \theta_j \geq c \right\} \cdot \sum_{j \neq i} \theta_j $$
	\item The incentives are there... but at what cost?
\end{itemize}
\end{frame}
\note{
	Assign as in-class problem:
	\begin{itemize}
		\item $N = 7M$ citizens decide whether to build a Moon base which costs $500M$ DKK.
		\item Every citizen has private valuation $\theta_i$ and qlin util
		\item Calculate the efficient scf $k^*(\theta)$
		\item Calculate Groves' transfers in case everyone values base at $\theta_i = 100$ DKK, taking $h_i(\theta_{-i})=0$.
		\item Answer: $699,999,900$ DKK paid to every citizen
	\end{itemize}
}


\begin{frame}{VCG Mechanism: Clarke Term}
\begin{itemize}
	\item A suggestion for $h_i(\theta_{-i})$ made by Clarke (``pivot mechanism''):
	\vspace{-0.5em}\begin{align*}
	&h_{i}(\theta_{-i})=\sum_{j\neq i} v_{j}(k^*(\theta_{-i}),\theta_{j}),
	\\ &\text{where } k^*(\theta_{-i}) \in \arg\max_{k} \sum_{j\neq i}v_{j}(k,\theta_{j}).
	%TODO: k^* or k^{-i}? I think the latter is in pset3 -- fix whichever is wrong
	\end{align*}

	\item \textbf{NOTE}: it is the \structure{default allocation rules} $k^*(\theta_{-i})$ that each textbook defines differently (or replaces with other things). My version is quite robust, but you can use other default rules if they make more sense in a given setting (so long as the rule for $i$ is independent of $\theta_i$)
	
	\item Resulting \alert{VCG transfers}:
	\vspace{-0.5em}\begin{align*}
	\structure{ t_{i}^{VCG}(\theta) } \equiv -\left(\sum_{j\neq i} v_{j}(k^*(\theta_i, \theta_{-i}), \theta_{j}) \right) + \sum_{j\neq i} v_{j}(k^*(\theta_{-i}), \theta_{j})
	\end{align*}
\end{itemize}
\end{frame}
\note{
	\begin{itemize}
		\item If you ``say nothing'', presumably some outcome is still implemented.
		\item So if you decide to ``say anything'', it is to tilt the decision in your direction at the expense of everyone else.
		\item So you must pay exactly the externality you impose on everyone else.
		\item Sometimes though your report may make others happier: e.g. if a class is only organized if at least 10 people sign up, then you signing up may make others happier -- then you get paid.
	\end{itemize}
}


\begin{frame}{VCG Mechanism: Final Transfers}
\begin{align*}
\structure{ t_{i}^{VCG}(\theta) } = -\left(\sum_{j\neq i} v_{j}(k^*(\theta_i, \theta_{-i}), \theta_{j}) \right) + \sum_{j\neq i} v_{j}(k^*(\theta_{-i}), \theta_{j})
\end{align*}
\begin{itemize}
	\item What's the big idea?
	\begin{itemize}
		\item Agent $i$ receives the externality his report imposes on others (mind the signs).
		\item $i$'s transfer is non-zero only if his presence affects the decision $k$.
		\item Note that $i$ cannot misreport $\theta_i$ and get lower transfer without also changing $k$.
	\end{itemize}
	\item What are VCG transfers in the Moon Base question?
\end{itemize}
\end{frame}
\note{
	Only read the third bullet in first group, others covered in previous note.
	
	In the Moon Base VCG mechanism, no one pays anything because no single agent is pivotal.
}

\begin{frame}{VCG Mechanism: Example}
\begin{example}[Auction]
	\begin{itemize}
		\item One indivisible item to be allocated among $N$ bidders.
		\item Bidder $i$'s valuation is \structure{$\theta_i$} (private info).
		%\item Want to allocate the item efficiently (to whoever values it most).
		\item What is the VCG mechanism?
	\end{itemize}
\end{example}
\begin{itemize}
	\item VCG mechanism is the second-price auction (efficient and DSIC).
	\item Also known as the Vickrey auction (the V in VCG).
\end{itemize}
\end{frame}
\note{
	Assign this as a problem in class:
	\begin{itemize}
		\item One item, $N$ bidders, each value the item at $\theta_i$
		\item so $v_i(k,\theta_i) = \theta_i k_i$, where $k_i$ is the prob. that $i$ gets the item ($\sum_{i=0}^{N}k_i = 1$)
		\item Calculate formally the efficient $k^*$ and the VCG transfers that support it
		\item (Even if know the answer -- forget it and apply the concepts)
	\end{itemize}
}


\begin{frame}{VCG aftermath}
\begin{itemize}
	\item We have an easy recipe to implement the \structure{efficient} outcome in \structure{dominant} strategies.
	\item Any problems?
\end{itemize}
\end{frame}
\note{
	If have time in lecture, continue straight to next slide/topic. Otherwise, what is meant is:
	\begin{itemize}
		\item It may not be budget balanced or, more generally, maximize revenue...
		\item All revelation principles apply (privacy, principal's commitment)
		\item All DSE issues apply (collusion, misspecification, (complexity))
	\end{itemize}
}


%NOTE to self: if you want to rearrange these sections (IR-BB-PE-BIC-gVCG-AGV) - DON'T, unless you have a VERY GOOD REASON TO! I spent three days reshuffling these, and this order is close enough to optimal!
\section{Individual Rationality and Budget Balance}

\begin{frame}{Feature example: bilateral trade}
\begin{example}[Bilateral Trade]
	\begin{itemize}
		\item One indivisible good.
		\item Two agents: buyer and seller. 
		\item Private valuations $\theta_b,\theta_s \in [0,1]$ resp.
		\item Find the VCG transfers (take no trade as efficient when $\theta_s = \theta_b$).
	\end{itemize}
\end{example}
\end{frame}


\begin{frame}{Feature example: bilateral trade}
\begin{itemize}
	\item If you did everything correctly, you'll get
	\begin{align*}
		t_b^{VCG}(\theta) &= \theta_s \cdot \mathbb{I} \{ \theta_s < \theta_b \} 
		\\ t_s^{VCG}(\theta) &= \theta_b \cdot \mathbb{I} \{ \theta_s \geq \theta_b \} 
	\end{align*}
	\pause
	\item The seller pays to keep the good and doesn't get anything from selling it. Good deal?
\end{itemize}
\end{frame}


\begin{frame}{Individual rationality}
\begin{itemize}
	\item In many settings can't force players to participate in mechanism:
	\begin{definition}[IR]
		A mechanism $\Gamma$ is:
		\begin{itemize}
			\item \structure{interim} \alert{individually rational} if
			$\mathbb{E}_{\theta_{-i}} \left[u_i(\theta_i,\theta_{-i})\right] \geq \underline{U}_i(\theta_i)$ for all $\theta_i$;
			\item \structure{ex post} \alert{individually rational} if
			$u_i(\theta_i,\theta_{-i}) \geq \underline{U}_i(\theta_i)$ for all $\theta$.
		\end{itemize}
		
		
	\end{definition}
	\item $\underline{U}_i(\theta_i)$ is the outside option of type $\theta_i$
	\begin{itemize}
		\item (in bilateral trade: $\underline{U}_s(\theta_s) = \theta_s$)
	\end{itemize}
	\item expectation means that distribution of $\theta$s now matters!
	\begin{itemize}
		\item (whether a mechanism is DSIC does not depend on the distr-n; but whether it is IR does)
	\end{itemize}
\end{itemize}
\end{frame}


\begin{frame}{Detour -- brief review}
\begin{itemize}
	\item \structure{ex ante} = $i$ knows nothing;
	\item \structure{ex interim} = $i$ knows $\theta_i$;
	\item \structure{ex post} = $i$ knows $\theta_i$ and $\theta_{-i}$.
	\item We'll mostly work with interim IR; 
	\item ex post IR is also sometimes used in the literature.
\end{itemize}
\end{frame}


\begin{frame}{Budget balance}
\begin{itemize}
	\item VCG for bilateral trade example is not IR for seller (outside option = keep the good).
	\pause\medskip
	\item If we want mechanism to be IR, easy solution is to decrease $t_i(\theta)$ by a lot, for all $\theta$.
	\item But that's expensive -- want mechanism to be \structure{budget balanced}:
	\pause
	\begin{definition}[BB]
		\begin{itemize}
			%NOTE: here t0 is NOT included because the designer is the sink for any transfer imbalance
			\item Mechanism $\Gamma$ is \structure{ex ante} \alert{budget balanced} if $\mathbb{E}_\theta \left[ \sum_{i=1}^N t_i (\theta) \right] \geq 0$;
			\item Mechanism $\Gamma$ is \structure{ex post} \alert{budget balanced} if $\sum_{i=1}^N t_i (\theta) \geq 0$ for all $\theta$.
		\end{itemize}
	\end{definition}
	\item Mechanism is \structure{exactly BB} if the above hold with equalities.
	\item If $\Gamma$ is ex post BB then it is ex ante BB (prove).
\end{itemize}
\end{frame}


\begin{frame}{IR vs BB}
\begin{itemize}
	\item Fundamental tension between IR and BB.
	\item We want to ask the following question (within our \structure{bilateral trade} example, in particular):
	\begin{exampleblock}{}
		Does there exist a mechanism that is: \structure{efficient, DSIC, IR, BB}?
	\end{exampleblock}
	\pause
	\item We know VCG was not IR, but that's just one mechanism. Can we say whether any other mechanisms satisfy all requirements?
	\begin{itemize}
		\item Not in most general case*, but all examples (trade, auction, pub.project) fit a much narrower model where we can.
		\item *though see Prop 23.C.5 in MWG
	\end{itemize}
\end{itemize}
\end{frame}





%NOTE to self: if you want to rearrange these sections (IR-BB-PE-BIC-gVCG-AGV) - DON'T, unless you have a VERY GOOD REASON TO! I spent three days reshuffling these, and this order is close enough to optimal!
\section{Payoff Equivalence}

\begin{frame}{The Euclidean model}
\begin{alertblock}{Assumption: Euclidean setting}
	Make the following assumptions on top of quasilinearity:
	\begin{itemize}
		\item $\theta_i \in \Theta_{i} = [\underline{\theta}_i, \bar{\theta}_i] \subseteq \mathbb{R}$, full support;
		\item $k \in K \subseteq \mathbb{R}^N$, $K$ compact, convex set;
		%TODO2021: convexity of K is never actually used, is it?
		\item $u_i(x,\theta_i) = \theta_i k_i - t_i$.
	\end{itemize}
\end{alertblock}
\begin{itemize}
	\item I'll call the above \alert{the Euclidean model} (not standard name).
	\item We'll derive \structure{Payoff-equivalence} as a necessary condition for $\Gamma$ to be \structure{DSIC} in \alert{Euclidean} model. It's a cool property on its own and will help answer the question about BB/IR mechanisms later.
	\item Given $\Gamma$, denote $U_i(\theta_i, \theta_{-i}) \equiv u_i\left(x(\theta_i, \theta_{-i}), \theta_i \right)$.
\end{itemize}
% Examples of non-Euclidean setting -- multiproduct auction, social choice between multiple alternatives.
\end{frame}


\begin{frame}{Monotonicity}
\begin{itemize}
	\item Assume $\Gamma$ is a \structure{direct} mechanism (or consider its direct equivalent).
	\item Play a bit with $i$'s IC (truthtelling constraint): for any $i,\theta_i,\hat{\theta}_i,\theta_{-i}$,
	{ \footnotesize
		\begin{align*}
			U_i(\theta_i, \theta_{-i}) &\geq u_i\left(x(\hat{\theta}_i,\theta_{-i}), \theta_i \right)
			\\ 
			\visible<2->{ &\equiv \theta_i k_i(\hat{\theta}_i,\theta_{-i}) - t_i (\hat{\theta}_i,\theta_{-i})}
			\\ 
			\visible<3->{&= \hat{\theta}_i k_i(\hat{\theta}_i,\theta_{-i}) - t_i (\hat{\theta}_i,\theta_{-i}) + \left(\theta_i - \hat{\theta}_i \right) k_i(\hat{\theta}_i,\theta_{-i})}
			\\ 
			\visible<4->{&= U_i(\hat{\theta}_i, \theta_{-i}) + \left(\theta_i - \hat{\theta}_i \right) k_i(\hat{\theta}_i,\theta_{-i})}
		\end{align*}	}
	\item \pause[5] In the end:
	\vspace{-0.5em}\begin{align*}
		U_i(\theta_i, \theta_{-i}) &\geq U_i(\hat{\theta}_i, \theta_{-i}) + \left(\theta_i - \hat{\theta}_i \right) k_i(\hat{\theta}_i,\theta_{-i}).
	\end{align*}\vspace{-1em}
	\item Similarly, type $\hat{\theta}_i$ should not want to report $\theta_i$:
	\vspace{-0.5em}\begin{align*}
		U_i(\hat{\theta}_i, \theta_{-i}) &\geq U_i(\theta_i, \theta_{-i}) + \left(\hat{\theta}_i - \theta_i \right) k_i(\theta_i,\theta_{-i}).
	\end{align*}\vspace{-1em}
\end{itemize}
\end{frame}


\begin{frame}{Monotonicity}
\begin{itemize}
	\item Combining the two \structure<2->{for $\theta_i > \hat{\theta}_i$}, we get
	{\small \vspace{-0.5em}\begin{align}
		\label{eq:k_mon}
			k_i(\theta_i,\theta_{-i}) 
			\geq 
			\frac{ U_i(\theta_i, \theta_{-i}) - U_i(\hat{\theta}_i, \theta_{-i}) }{ \theta_i - \hat{\theta}_i } 
			\geq 
			k_i(\hat{\theta}_i,\theta_{-i}),
		\end{align}\vspace{-1em}}
	\pause
	\item meaning \structure{$k_i(\theta_i,\theta_{-i}) \geq k_i(\hat{\theta}_i,\theta_{-i})$} -- allocation rule must be \alert{monotone}.
	\item DSIC: ``Those who value things more should get more things.''
	\item \alert{Monotonicity} is \structure{necessary} for $f$ to be \structure{DSIC} in \alert{Euclidean} settings.
	\item From monotonicity we can build up to \alert{payoff equivalence}, 
	the second cool result in mechanism design (after revelation principle, not monotonicity).
\end{itemize}
\end{frame}


\begin{frame}{Payoff Equivalence}
\begin{itemize}
	\item $k_i(\theta_i,\theta_{-i})$ is monotone in $\theta_i$, hence continuous a.e.: $\lim_{\hat{\theta}_i \to \theta_i} k_i(\hat{\theta}_i,\theta_{-i}) = k_i(\theta_i,\theta_{-i})$.
	\pause
	\item Together with the big inequality \eqref{eq:k_mon}	this means that a.e. we have
	\pause
	\begin{align*}
		\frac{\partial U_i(\theta_i,\theta_{-i})}{\partial \theta_i} = \lim_{\hat{\theta}_i \to \theta_i} \frac{ U_i(\theta_i, \theta_{-i}) - U_i(\hat{\theta}_i, \theta_{-i}) }{ \theta_i - \hat{\theta}_i }  = k_i(\theta_i,\theta_{-i}).
	\end{align*}
	\pause
	\item So if $k(\theta)$ is integrable in $\theta_i$ (e.g. if it's bounded) then for all $\theta_i$
	\begin{align*}
	\structure{
		U_i(\theta_i, \theta_{-i}) = U_i (\underline{\theta}_i,\theta_{-i}) + \int_{\underline{\theta}_i}^{\theta_i} k_i(s,\theta_{-i}) d s
	}
	\end{align*}
	(Note that the lower limit does not need to be $\underline{\theta}_i$ -- it can be any other type.)
	\item This is the \alert{envelope representation of payoffs} a.k.a. Mirrlees condition. From it we can immediately get revenue equivalence.
\end{itemize}
\end{frame}


\begin{frame}{Payoff Equivalence}
	\begin{theorem}[Payoff Equivalence for DSIC Euclidean mechanisms]
		In the Euclidean setting, for any two DSIC DRMs with $x = (k,t)$ and $x' = (k',t')$ respectively:
		\\ 	
		\structure{if $k(\theta)=k'(\theta)$} for all $\theta$ \alert{then $t_i(\theta) = t'_i(\theta) + c_i (\theta_{-i})$} for all $\theta$ for some $c_i(\theta_{-i})$.
	\end{theorem}
	\pause
	\textbf{Proof}. Given the envelope representation, invoke the definition of $U_i$:
	\vspace{-0.5em}\begin{align*}
		v_i(k^*(\theta_i,\theta_{-i}),\theta_i) - t_i(\theta_i,\theta_{-i}) = U_i (\underline{\theta}_i,\theta_{-i}) + \int_{\underline{\theta}_i}^{\theta_i} k_i(s,\theta_{-i}) d s.
	\end{align*}
	The above holds in any DSIC DRM. Take the two mechanisms in the statement, fix some $\theta,i$, express $t_i(\theta)$ and $t_i'(\theta)$ from the above, and you will get that
	\vspace{-0.5em}$$ t_i(\theta) = t_i'(\theta) - U_i (\underline{\theta}_i,\theta_{-i}) + U_i' (\underline{\theta}_i,\theta_{-i}),$$
	where $U_i$ and $U_i'$ denote the eqm utilities in the two mechanisms. The last two terms only depend on $i$ and $\theta_{-i}$ (but not $\theta_i$), hence denoting them as $c_i(\theta_{-i})$ concludes the proof. $\square$
\end{frame}


\begin{frame}{Payoff Equivalence: intuition}
\begin{itemize} 
	\item Given allocation $k$ (doesn't have to be efficient), utility of one type (usually ``lowest'' type) pins down utils of all types of player $i$ given fixed $\theta_{-i}$.
	\pause
	\item Equivalently, have only one degree of freedom for $i$'s transfers given $\theta_{-i}$.
	\item Reminds you of anything?
\end{itemize}
\end{frame}


\begin{frame}{Payoff Equivalence of Efficient Mechanisms}
\begin{itemize}
	\item Recall Groves' transfers: efficient $k^*$ can be impl-d in DS by
	\vspace{-0.5em}\begin{align*}
		t_{i}(\theta) &= -\left(\sum_{j\neq i} v_{j}(k^*(\theta_i, \theta_{-i}), \theta_{j}) \right) + h_{i}(\theta_{-i})
		\\ &= \structure{-\left(\sum_{j\neq i} \theta_j k_j^*(\theta_i, \theta_{-i}) \right)} + h_{i}(\theta_{-i})
	\end{align*}
	
	\structure<1>{for some $h_i(\theta_{-i})$.}
	\pause
	\item Payoff equivalence implies that \alert{efficient} $k^*$ in a \alert{Euclidean} model can \alert{ONLY} be DS-implemented by some \structure{Groves'} mechanism.
	% Similarity is that there's just one DoF for i's payoffs given theta_{-i}
\end{itemize}
\end{frame}


\begin{frame}{Payoff Equivalence: Beyond Euclidean}
	Both monotonicity and payoff equvalence hold beyond the Euclidean setting:
	\begin{itemize}
		\item various forms of \alert{monotonicity} are necessary and sufficient for DSIC in quasilinear setting;
		\item \alert{payoff equivalence} of DSIC mechanisms is generalizable beyond Euclidean (but you cannot get to general quasilinear setting);
		\item see B{\"o}rgers for details:
		\begin{itemize}
			\item ch.5: single-player problems,
			\item ch.7: DSIC results building on ch.5.
		\end{itemize}
	\end{itemize}
\end{frame}


\begin{frame}{Payoff Equivalence (DSIC): Conclusion}
	\begin{itemize}
		\item So what does payoff equivalence tell us?
		\item Efficient allocation $k^*$ can \textbf{only} be implemented using Groves' transfers...
		\begin{itemize}
			\item ...but $h_i(\theta_{-i})$ still provides a lot of flexibility!
			\item So it's hard to know whether VCG is the best mechanism or there are others.
		\end{itemize}
		\item So let us \emph{weaken} our implementation concept to obtain a \textbf{stronger} version of payoff equivalence.
	\end{itemize}
\end{frame}


\begin{frame}{Back to bilateral trade}
\begin{itemize}
	\item Remember how this detour started?
\end{itemize}
\begin{example}[Bilateral Trade]
	\begin{itemize}
		\item One indivisible good.
		\item Two agents: buyer and seller. 
		\item Private valuations $\theta_b,\theta_s \in [0,1]$ resp.
		\item Is there an \structure{efficient, DSIC, ex post IR, ex post BB} mechanism?
	\end{itemize}
\end{example}
Can we answer this question now?
%do on the board
\end{frame}
\note{ 
	Assign as in-class problem:
	\begin{itemize}
		\item setup as on the slide. Write out detailed VCG transfers as below.
		\item Give a hint: start with VCG and change $h_i(\theta_{-i})$ until the resulting mechanism is ex post IR. Will it be BB?
	\end{itemize}
	Answer: 
	\begin{itemize}
		\item In VCG:
		\begin{itemize}
			\item Groves' transfers: $t_b^G = -\theta_s \cdot \mathbb{I}\{\text{no trade}\}$; \quad $t_s^G = -\theta_b \cdot \mathbb{I}\{\text{trade}\}$ for seller; 
			\item Clarke's transfers: $h_b(\theta_s)=\theta_s$; \quad $h_s(\theta_b)=\theta_b$;
			\item $t_b(\theta)=\theta_s \cdot \mathbb{I}\{\text{trade}\}$; \quad $t_s(\theta)=\theta_b \cdot \mathbb{I}\{\text{no trade}\}$
		\end{itemize}
		\item In IR mechanism: start from same Groves' transfers, then taking $\mathbb{I}_T = \mathbb{I} \{\theta_b>\theta_s \}$:
		\begin{itemize}
			\item $u_b(\theta) = \theta_b \mathbb{I}_T + \theta_s(1-\mathbb{I}_T) - h_b(\theta_s)$ (draw to see that setting $h_b(\theta_s)=\theta_s$ sets $u_b$ to zero if no trade)
			\item $u_s(\theta) = \theta_s(1-\mathbb{I}_T) + \theta_b \mathbb{I}_T - h_s(\theta_b)$ (draw to see that setting $h_s(\theta_b)=0$ sets $u_s = \theta_s$ (IR) if no trade)
		\end{itemize}
	\end{itemize}
}



%NOTE to self: if you want to rearrange these sections (IR-BB-PE-BIC-gVCG-AGV) - DON'T, unless you have a VERY GOOD REASON TO! I spent three days reshuffling these, and this order is close enough to optimal!
%[2022.09]: I did not listen to myself and pulled out most of BIC definitions out to pair the DSIC definitions

\section{Payoff Equivalence in BIC}

%\begin{frame}{BIC: Efficient Mechanisms}
%\begin{itemize}
%	\item Now restrict ourselves to \alert{quasilinear} setting again.
%	\begin{itemize}
%		\item $u_i(\theta) = v(k(\theta),\theta_i) - t_i(\theta)$
%	\end{itemize}
%	\item Mission same: find mechanism that implements the \structure{efficient} $k^*(\theta)$. Ideally the one that is IR and/or BB.
%	\begin{itemize}
%		\item It is now more suitable to consider interim perspective rather than ex post -- so look for interim IR, ex ante BB.
%	\end{itemize}
%	\pause
%	\item We'll look at two mechanisms:
%	\begin{itemize}
%		\item \structure{generalized VCG} mechanism -- guarantees IR
%		\item \structure{AGV} mechanism (d'Aspremont and Gerard-Varet, 1979) -- for when you absolutely need BB;
%	\end{itemize}
%\end{itemize}
%\end{frame}


\begin{frame}{Payoff Equivalence in BIC}
	We now show payoff equivalence for BIC mechanisms in the Euclidean setting.
	\\
	New \alert{assumption}: types are independent across players: $\theta_i \perp \theta_{-i}$ for all $i$.
	\begin{theorem}[Payoff Equivalence for BIC mechanisms]
		For any two BIC DRMs with $x = (k,t)$ and $x' = (k',t')$ resp.:
		
		\structure{if \quad $\mathbb{E}_{\theta_{-i}} k_i(\theta_i, \theta_{-i}) = \mathbb{E}_{\theta_{-i}} k_i'(\theta_i, \theta_{-i})$} for all $i,\theta_i$,
		
		\alert{then $\mathbb{E}_{\theta_{-i}} t_i(\theta_i, \theta_{-i}) = \mathbb{E}_{\theta_{-i}} t'_i(\theta_i, \theta_{-i}) + h_i$} for all $i,\theta_i$ for some $h_i$.
	\end{theorem}
	\begin{itemize}
		\item As before, implies that for given $k$ (any, not just the efficient) we only have one degree of freedom for $t_i(\theta)$,
		\begin{itemize}
			\item now ``just one'' instead of ``just one given $\theta_{-i}$''.
		\end{itemize}
	\end{itemize}
\end{frame}


\begin{frame}{Payoff Equivalence in BIC. Proof}
	\begin{itemize}
		\item Let 
		\vspace{-0.5em}\begin{align*}
		\bar{U}_i (\hat{\theta}_i,\theta_i) &\equiv \mathbb{E}_{\theta_{-i}} \left[ u_i \left( x(\hat{\theta}_i, \theta_{-i}), \theta_i \right) | \theta_i \right]
		\\
		&\,\,= \mathbb{E}_{\theta_{-i}} \left[ \theta_i k_i(\hat{\theta}_i, \theta_{-i}) - t_i \left( \hat{\theta}_i, \theta_{-i} \right) | \theta_i \right].
		\end{align*}\vspace{-1em}
		
		(do not confuse with $U_i$ in Euclidean model for DS.)
		\pause
		\item Take full derivative w.r.t $\theta_i$ at $\hat{\theta}_i=\theta_i$:
		\vspace{-0.5em}\begin{align*}
		\frac{d}{d \theta_i} \bar{U}_i (\theta_i,\theta_i) &= \left. \frac{\partial}{\partial \hat{\theta}_i} \bar{U}_i (\hat{\theta}_i,\theta_i) \right|_{\hat{\theta}_i=\theta_i} + \left. \frac{\partial}{\partial \theta_i} \bar{U}_i (\hat{\theta}_i,\theta_i) \right|_{\hat{\theta}_i=\theta_i}
		\\ &= \hspace{1.4cm} 0 \hspace{1.4cm} + \mathbb{E}_{\theta_{-i}} \left[k_i (\theta_i, \theta_{-i}) | \theta_i \right]
		\end{align*}\vspace{-1em}
		
		The first term is zero because truthful report $\hat{\theta}_i = \theta_i$ maximizes $\bar{U}_i (\hat{\theta}_i,\theta_i)$.
	\end{itemize}
\end{frame}


\begin{frame}{Payoff Equivalence in BIC. Proof}
	\begin{itemize}
		\item Then by the Fundamental Theorem of Calculus
		\begin{align*}
		\bar{U}_i (\theta_i) &= \bar{U}_i (\underline{\theta}_i) + \int_{\underline{\theta}_i}^{\theta_i} \frac{d \bar{U}_i}{d \theta_i}(s,s) d s
		\\&= \bar{U}_i (\underline{\theta}_i) + \int_{\underline{\theta}_i}^{\theta_i} \mathbb{E}_{\theta_{-i}} \left[k_i(s,\theta_{-i}) | \theta_i \right] d s,
		\end{align*}
		meaning that $k$ and $\bar{U}_i (\underline{\theta}_i)$ pin down utilities $\bar{U}_i (\theta_i)$ for all $\theta_i$. $\square$
		\medskip
		\item Remark: here we used a different argument to get $\frac{d \bar{U}_i (\theta_i,\theta_{i})}{d \theta_i} = \mathbb{E}_{\theta_{-i}} \left[k(\theta_i,\theta_{-i}) | \theta_i \right]$ compared to DSIC proof. Either argument can be used in both proofs.
	\end{itemize}
\end{frame}


\begin{frame}{Payoff Equivalence in BIC. Generalization}
	The proof is nice and illustrative in Euclidean setting.\\
	\cite{krishna_convex_2001} present a more general result for the following setting:
	\begin{itemize}
		\item quasilinear setting;
		\item independent types;
		\item $\Theta_i \subseteq \mathbb{R}^{K_i}$ is a convex set for every $i$ 
		\item $v_i(k,\theta_i)$ is convex in $\theta_i$ for all $i$.
	\end{itemize}
	%(I believe you can relax convexity of $\Theta_i$ to connectedness, but I do not have a reference for that.)
\end{frame}


\begin{frame}{Ex ante and ex post revenue in BIC}
	One cool thing about BIC mechanisms is that ex post BB is free if you have ex ante BB:
	\begin{theorem}
		In a \alert{quasilinear} setting, for every direct mechanism $\Gamma = (\Theta, (k,t))$ that is BIC and \alert{ex ante BB}, there exists a direct mechanism $\Gamma' = (\Theta, (k',t'))$ which is:
		\begin{itemize}
			\item BIC,
			\item \alert{ex post BB},
			\item equivalent to $\Gamma$ in the sense of: $k'(\theta) = k(\theta)$ for all $\theta$ and $\mathbb{E}_{\theta_{-i}} t'_i(\theta_i,\theta_{-i}) = \mathbb{E}_{\theta_{-i}} t_i(\theta_i,\theta_{-i})$ for all $i$ and $\theta_i$.
		\end{itemize}
	\end{theorem}
	For proof, see Prop 6.3 \& Prop 3.6 in B{\"o}rgers.
\end{frame}


%NOTE to self: if you want to rearrange these sections (IR-BB-PE-BIC-gVCG-AGV) - DON'T, unless you have a VERY GOOD REASON TO! I spent three days reshuffling these, and this order is close enough to optimal!
\section{gVCG}

\begin{frame}{Generalized VCG}
	\begin{itemize}
		\item As it turns out, VCG can be (interim) IR with a slight modification... And it will be (ex ante) revenue-maximizing among all such mechanisms in that case.
		\item Enter \alert{generalized VCG} \citep{krishna_efficient_2000}.
		\pause\medskip
		\item Define \structure{least charitable type} $\tilde{\theta}_i$ as
		\begin{align*}
		\tilde{\theta}_i \in \arg \min_{\theta_i \in \Theta_i} \mathbb{E}_{\theta_{-i}} \left[ \sum_{j=0}^{N} v_j (k^*(\theta_i,\theta_{-i}),\theta_j) - \underline{U}_i (\theta_i) \right]
		\end{align*}
		(expectation taken w.r.t the common prior $\phi \in \varDelta(\Theta)$)
	\end{itemize}
	%LCT intuition will become clear later...
\end{frame}


\begin{frame}{Generalized VCG}
	GVCG mechanism is a DRM with the efficient allocation $k^*(\theta)$ and payments
	\begin{align*}
	t_i^{GVCG} (\theta) \equiv& \sum_{j \neq i} v_j (k^*(\tilde{\theta}_i,\theta_{-i}),\theta_j) + v_i (k^*(\tilde{\theta}_i,\theta_{-i}),\tilde{\theta}_i) -
	\\& - \sum_{j \neq i} v_j (k^*(\theta_i,\theta_{-i}),\theta_j) - \underline{U}_i (\tilde{\theta}_i)
	\end{align*}
	\pause
	Has the usual Groves' term (the third one); the other three guarantee IR.
	\begin{itemize}
		\item The first term is similar to Clarke's term, but with $k^*(\tilde{\theta}_i,\theta_{-i})$ instead of $k^*(\theta_{-i})$
		\item $2^{nd}$ and $4^{th}$ is the net utility that LCT $\tilde{\theta}_i$ gets from participating in the mechanism -- need to also pay it to all other types
	\end{itemize}
\end{frame}


%NOTE: the two theorems are not equivalent because QLIN vs EUCL settings!
\begin{frame}{Generalized VCG}
	\begin{theorem}[gVCG, part 1]
		In a \structure{quasilinear} model, gVCG is:
		\begin{itemize}
			\item efficient (by construction),
			\item DSIC,
			\item interim IR.
			\item %maximizes expected revenue among all mechanisms that are BIC, IIR, and implement the efficient $k^*$.
		\end{itemize}
	\end{theorem}
	Prove DSIC on your own (analogous to VCG).
\end{frame}


\begin{frame}{Generalized VCG. Proof: IIR}
	Interim expected utility for $\theta_i$ is
	\begin{align*}
		\mathbb{E}_{\theta_{-i}} \left[ \sum_{j =0}^N v_j (k^*(\theta),\theta_j) - \left. \sum_{j =0}^N v_j (k^*(\tilde{\theta}_i,\theta_{-i}),\theta_j) \right|_{\theta_i = \tilde{\theta}_i} | \theta_i \right] + \underline{U}_i (\tilde{\theta}_i)
		\visible<2>{
			\geq \underline{U}_i(\theta_i)
		}
	\end{align*}
	\pause
	The inequality above is the IIR constraint, and it holds since
	\begin{align*}
	\tilde{\theta}_i \in \arg \min_{\theta_i \in \Theta_i} \mathbb{E}_{\theta_{-i}} \left[ \sum_{j=0}^{N} v_j (k^*(\theta_i,\theta_{-i}),\theta_j) - \underline{U}_i (\theta_i) | \theta_i \right]
	\end{align*}
	%TODO: Talk about the meaning of LCT here
\end{frame}


\begin{frame}{Generalized VCG}
	\begin{theorem}[gVCG, part 2]
		In a \alert{Euclidean} model with independent players' types, gVCG is:
		\begin{itemize}{\color{gray}
				\item efficient (by construction),
				\item DSIC,
				\item interim IR;}
			\item maximizes expected revenue among all mechanisms that are BIC, IIR, and implement the efficient $k^*$.
		\end{itemize}
	\end{theorem}
	If gCVG is not ex ante budget balanced, there does not exist a \\
	\{EFF + BIC + IIR + ex ante BB\} mechanism (so no ex post BB either).
\end{frame}


\begin{frame}{Generalized VCG. Proof: revenue maximizing in Euclidean}
\begin{itemize}
	\item Given revenue equivalence, just need to show we cannot decrease $h_i$ for any player w/o violating IIR.
	\pause
	\item Decreasing $h_i$ only possible if IR slack for \emph{all} types of $i$.
	\pause
	\item But IR binds for $\tilde{\theta}_i$: $\bar{U}_i (\tilde{\theta}_i) = \underline{U}_i (\tilde{\theta}_i)$ (verify). \qed
\end{itemize}
\end{frame}


\begin{frame}{Application: Bilateral Trade}
	\begin{example}[Bilateral Trade (revisited)]
		\begin{itemize}
			\item One indivisible good.
			\item Two agents: buyer and seller. 
			\item Private valuations $\theta_b,\theta_s \sim \text{i.i.d.} U[0,1]$ resp.
			\item Is there an \soutred{efficient, DSIC, ex post IR, ex post BB}
			\structure{efficient, BIC, interim IR, ex ante BB} mechanism?
		\end{itemize}
	\end{example}
	\begin{itemize}
		\item No, because gVCG is not BB. 
		(This is the \structure{Myerson-Satterthwaite Theorem})
	\end{itemize}
\end{frame}




%NOTE to self: if you want to rearrange these sections (IR-BB-PE-BIC-gVCG-AGV) - DON'T, unless you have a VERY GOOD REASON TO! I spent three days reshuffling these, and this order is close enough to optimal!
\section{AGV}

\begin{frame}{AGV mechanism}
\begin{itemize}
	\item One last mechanism before we go -- in case you care about BB, but not IR.
	\item Let 
	\vspace{-0.5em}\begin{align*}
		\tilde{t}_i (\theta_i) \equiv \mathbb{E}_{\theta_{-i}} \left[ \sum_{j \neq i} v_j (k^*(\theta_i,\theta_{-i}), \theta_j) | \theta_i \right]
	\end{align*}
	
	be the ``expected externality'' imposed by $i$ on everyone else.
	\item \structure<1>{AGV transfers} are given by
	\vspace{-0.5em}\begin{align*}
		\structure<1>{t_i^{AGV} (\theta) \equiv \alert<2>{\frac{1}{N-1} \sum_{j \neq i} \tilde{t}_j (\theta_j)} \structure<2>{- \tilde{t}_i (\theta_i)}}.
	\end{align*}\vspace{-1em}
	\pause
	\item \structure{The second term} is the averaged version of Groves' transfer,
	\item \alert{the first term} is $h_i(\theta_{-i})$ which balances the budget.
\end{itemize}
\end{frame}


\begin{frame}{AGV mechanism}
\begin{theorem}[AGV]
	In a \alert{quasilinear} model, AGV is:
	\begin{itemize}
		\item efficient (by construction),
		\item exactly ex post BB,
		\item BIC.
	\end{itemize}
\end{theorem}
Not necessarily IR. :(
\end{frame}


\begin{frame}{AGV mechanism. Proof: budget balance.}
\begin{itemize}
	\item Observe that
	\vspace{-0.5em}\begin{align*}
		\sum_i t_i^{AGV} (\theta) = \sum_i \left[ \frac{1}{N-1} \sum_{j \neq i} \tilde{t}_j (\theta_j) - \tilde{t}_i (\theta_i) \right].
	\end{align*}
	\item For any $j$, RHS has:
	\begin{itemize}
		\item $N-1$ terms of the form $\frac{1}{N-1} \tilde{t}_j (\theta_j)$, and
		\item $1$ term of the form $-\tilde{t}_j(\theta_j)$.
	\end{itemize}
	\item These cancel out and exhaust all terms in the sum. Therefore, $\sum_i t_i^{AGV} (\theta) = 0$ for all $\theta$ = ex post exact budget balance.
	\begin{itemize}
		\item \textbf{Note:} if the mechanism needs to raise some fixed sum \emph{for any} $\theta$, it can be treated as $\tilde{t}_0$. \\
		If the mechanism needs to raise some sum that is \emph{dependent on} $\theta$ (e.g. fund a public project iff it is built), AGV \alert{cannot} handle that.
	\end{itemize}
\end{itemize}
\end{frame}


\begin{frame}{AGV mechanism. Proof: BIC.}
\begin{itemize}
	\item If $i$ reports $\hat{\theta}_i$ then receives utility
	{\footnotesize 
		\vspace{-0.5em}\begin{align*}
			\mathbb{E}_{\theta_{-i}} \left[ v_i(k^*(\hat{\theta}_i, \theta_{-i}), \theta_i) + \sum_{j \neq i} v_j(k^*(\hat{\theta}_i, \theta_{-i}), \theta_j) | \theta_i \right] - \frac{1}{N-1} \sum_{j \neq i} \tilde{t}_j (\theta_j)
		\end{align*}
	}
	\item Last term indep of $\hat{\theta}_i$; 
	
	bracket max-d by $\hat{\theta}_i = \theta_i$ for every $\theta_{-i}$ (since $k^*$ efficient), 
	
	so max-d by $\hat{\theta}_i = \theta_i$ in expectation as well.
	
	\item Reporting truth is a best response to $-i$ reporting truthfully 
	
	$\Rightarrow$ truthful reporting is a BNE of the mechanism. \qed
\end{itemize}
\end{frame}





\begin{frame}{Bottom line}
\begin{itemize}
	\item We have two mechanisms that implement the \structure{efficient} $k^*$ in \alert{quasilinear} model:
	\begin{itemize}
		\item AGV: BIC + BB,
		\item gVCG: DSIC + IIR.
	\end{itemize}
	\item In the \alert{Euclidean} model, gVCG is also [ex ante-]revenue-maximizing among BIC+IIR mechanisms.
	\item Revenue equivalence is powerful, but needs more structure than just quasilinear model.
\end{itemize}
\end{frame}


\appendix
\begin{frame}[allowframebreaks]{References}
\bibliography{teaching}
\bibliographystyle{abbrvnat}
\end{frame}




%\begin{frame}{Exercises}
%	(problem set on absalon)
%	\begin{enumerate}
%		\item Design an efficient attendance mechanism for our course.
%		\item Solve a problem on VCG.
%	\end{enumerate}
%\end{frame}



%%TODO: move this to ``arbitrary scf''
%\begin{frame}{DSIC: Weak Preference Reversal Property}
%\begin{itemize}
%	\item ``To each their own'': different types should get their most preferred option among the available ones:
%	$$ u_{i}(f(\theta_{i}', \theta_{-i}), \theta_{i}') \geq u_{i}(f(\theta_{i}'', \theta_{-i}), \theta_{i}')$$
%	$$ u_{i}(f(\theta_{i}', \theta_{-i}), \theta_{i}'') \leq u_{i}(f(\theta_{i}'', \theta_{-i}), \theta_{i}'')$$
%	\item $i$'s preference between $f(\theta_{i}', \theta_{-i})$ and $f(\theta_{i}'', \theta_{-i})$ should flip when his type changes from $\theta_i'$ to $\theta_i''$.
%	\item Obviously a necessary condition for DSIC. Can show it's also sufficient, meaning in the end Preference Reversal is equivalent to $f$ being DSIC.
%\end{itemize}
%\end{frame}

% Now let's talk about whether sum of utilities is a good measure of welfare...

%\section{Social Choice Functions}
%
%
%\begin{frame}{Detour: Social Choice Theory}
%\begin{itemize}
%	\item Sum of utilities is just one measure of welfare -- others are available.
%	\item Further: utilities $u_i$ are nice for exploring intrapersonal trade-offs when making decisions;
%	\item not so good for interpersonal comparisons -- how to measure relative preference intensity?
%	\item What do?
%	\item Social Choice Theory (\& Welfare Economics) deal with aggregating individual preferences into social preference.
%\end{itemize}
%\end{frame}
%
%
%\begin{frame}{Social Choice: Axiomatic Approach}
%\begin{itemize}
%	\item If cardinal utilities bad -- can work with ordinal preference relations $\succsim_i$.
%	\item Can impose axioms on how \structure{individual preferences} $\succsim_i$ should map into \structure{social preference} relation $\succsim$ (and/or corresponding social choice function $f$).
%	\item Possible reasonable axioms:
%\end{itemize}
%\begin{description}
%	\item[(A1)] \structure{Domain}: any collection of individual preferences $\left(\succsim_1, ..., \succsim_N \right)$ can be aggregated into $\succsim$.
%	\item[(A2)] \structure{Unanimity}: if $a \succsim_i b$ for all $i$ then $a \succsim b$.
%	\item[(A3)] \structure{Independence of Irrelevant Alternatives}: if $\succsim_i$ and $\succsim'_i$ rank alternatives $a$ and $b$ the same for all $i$ then so should $\succsim$ and $\succsim'$.
%\end{description}
%\end{frame}
%
%
%\begin{frame}{Social Choice: Axiomatic Approach}
%\begin{block}{Arrow's Theorem}
%	With more than three alternatives, if $\succsim$ satisfies (A1)-(A3) then it is dictatorial, i.e. $\exists i: a \succsim b \Leftrightarrow a \succsim_i b$.
%\end{block}
%\begin{block}{Proof}
%	\href{https://link.springer.com/article/10.1007/s00199-004-0556-7}{Geanakoplos, J. (2005). Three brief proofs of Arrow's impossibility theorem. Economic Theory, 26(1), 211-215.}
%	\vspace{8em}
%\end{block}
%\end{frame}
%
%
%\begin{frame}{Social Choice}
%\begin{itemize}
%	\item See Geanakoplos' paper for [slightly] more details on Arrow's Thm, and MWG ch.21 for more details on Social Choice theory.
%	\item Lesson: aggregating preferences is a difficult problem in itself.
%	\item We won't be dealing with this problem in this class, from now on just take $f$ as given.
%\end{itemize}
%\end{frame}
%
%%TODO 2020: talk about median voter thm with single-peaked preferences here



\end{document}