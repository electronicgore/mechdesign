%%% License: Creative Commons Attribution Share Alike 4.0 (see https://creativecommons.org/licenses/by-sa/4.0/)

\documentclass[english,10pt
%,handout
,aspectratio=169
]{beamer}
%%% License: Creative Commons Attribution Share Alike 4.0 (see https://creativecommons.org/licenses/by-sa/4.0/)

\DeclareGraphicsExtensions{.eps, .pdf,.png,.jpg,.mps,}
\usetheme{reMedian}
\usepackage{parskip}
\makeatother

\renewcommand{\baselinestretch}{1.1} 

\usepackage{amsmath, amssymb, amsfonts, amsthm}
\usepackage{enumerate}
%\usepackage{enumitem}
\usepackage{hyperref}
\usepackage{url}
\usepackage{bbm}
\usepackage{color}

\usepackage{tikz}
\usepackage{tikzscale}
%\newcommand*\circled[1]{\tikz[baseline=(char.base)]{
%		\node[shape=circle,draw, inner sep=-20pt] (char) {#1};}}
%\usetikzlibrary{automata,positioning}
%\usetikzlibrary{decorations.pathreplacing}
\usepackage{pgfplots}
\usepgfplotslibrary{fillbetween}
\usepackage{graphicx}

\usepackage{setspace}
%\thinmuskip=1mu
%\medmuskip=1mu 
%\thickmuskip=1mu 


%\usecolortheme{default}
\usepackage{verbatim}
\usepackage[normalem]{ulem}

\usepackage{apptools}
\AtAppendix{
	\setbeamertemplate{frame numbering}[none]
}
\usepackage{natbib}


% red strikeout
\newcommand\soutred{\bgroup\markoverwith
	{\textcolor{red}{\rule[0.55ex]{2pt}{0.8pt}}}\ULon}



%% To use LyX frames from old version:
%\def\lyxframeend{} % In case there is a superfluous frame end
%\long\def\lyxframe#1{\@lyxframe#1\@lyxframestop}%
%\def\@lyxframe{\@ifnextchar<{\@@lyxframe}{\@@lyxframe<*>}}%
%\def\@@lyxframe<#1>{\@ifnextchar[{\@@@lyxframe<#1>}{\@@@lyxframe<#1>[]}}
%\def\@@@lyxframe<#1>[{\@ifnextchar<{\@@@@@lyxframe<#1>[}{\@@@@lyxframe<#1>[<*>][}}
%\def\@@@@@lyxframe<#1>[#2]{\@ifnextchar[{\@@@@lyxframe<#1>[#2]}{\@@@@lyxframe<#1>[#2][]}}
%\long\def\@@@@lyxframe<#1>[#2][#3]#4\@lyxframestop#5\lyxframeend{%
%	\frame<#1>[#2][#3]{\frametitle{#4}#5}}


\title{Mechanism Design}

\subtitle{0: Introduction}

\author{Egor Starkov}

\date{K{\o}benhavns Unversitet \\
	Fall 2022}


\begin{document}
	\AtBeginSection[]{
		\frame<beamer>{
			\frametitle{This slide deck:}
			\tableofcontents[currentsection,currentsubsection]
	}}
	\frame[plain]{\titlepage}






\section{Logistics}

\begin{frame}{Hi}
	\begin{itemize}
		\item Egor Starkov
		\item Contact: \texttt{egor.starkov@econ.ku.dk} or absalon inbox
		\item Research interests: information economics, dynamic games, communication
		\item Office: 26.1.13
		\item Office hours: Tue, 14-15
		\item Questions: email/absalon, before/after class
	\end{itemize}
\end{frame}


\begin{frame}{Logistics}
	\begin{itemize}
		\item Weekly lectures (except Fall break -- week \#42)
		\begin{itemize}
			\item Tue, 15:15-18:00, CSS 2.0.18 (but check timetable for room changes)
			\item talk later about what happens in those
		\end{itemize}
		
		\pause
		\item Mandatory assignment:
		\begin{itemize}
			\item ``Make your own problem'' -- find an interesting real-world setting and analyze it using the machinery from the class
			\item groups of up to 4 are allowed
			\item deadline: some time in November?
		\end{itemize}
		
		\pause
		\item Final exam:
		\begin{itemize}
			\item 24hrs take home (individual, no groups)
			\item Convert problems from text to math; solve them and explain intuition
		\end{itemize}
		
		\pause
		\item Research module for PhD students: contact me if you would like to do it
	\end{itemize}
\end{frame}


\begin{frame}{Covid-specific stuff}
	\begin{itemize}
		\item No online streams/recordings are currently planned, back to stone age
	\end{itemize}
\end{frame}


\begin{frame}{Materials}
	\begin{itemize}
		\item Intended workflow: 
		\begin{enumerate}
			\item We start a topic in class
			\item \uline{You watch video-lectures at home} during the week (and/or read textbook, slides, papers)
			\item We go through the same material \emph{very quickly} in class and discuss any questions you have
			\item We solve some problems in class
			\item You will have more problems to practice at home
		\end{enumerate}
		\pause
		\item Suggestion: organize into groups. Watch lectures in groups. Discuss problems in groups.
		\pause
		\item In class I use whiteboard+slides for ``lecture'' stuff. Slides on absalon include some of the whiteboard parts.
		\begin{itemize}
			\item I'll try to upload slides in advance, but they might be edited and updated afterwards
		\end{itemize}
		The problems will be whiteboard-only; with solutions (mostly) uploaded afterwards.
	\end{itemize}
\end{frame}


\begin{frame}{References: textbooks}
	This course is a compilation of many books, papers, courses; does not follow any single one too closely. Below are some books that might help (see the reading list on Absalon for full references). Note that notation will be different across books and the class!
	\begin{itemize}%[\noitemsep]
		\item \textbf{Narahari}: Probably your best bet. Hard to find in print, but you have online access through the library (see Absalon).
		\item \textbf{Diamantaras}: Another good textbook, but seems very hard to find.
		\item \textbf{B\"{o}rgers}: I used this as default in previous years, but it's quite hardcore and hard to follow. (Easy to find though.)
		\item \textbf{MWG}: A microeconomic bible. Very good, very clear, but has the smallest coverage for our course.
		\item \textbf{RS}: Relevant for two lectures on matching. Some material is contained in Narahari and Diamantaras. Nice reference if you are into matching.
	\end{itemize}

\end{frame}


\begin{frame}{References: other}
	\begin{itemize}
		\item I will sometimes refer to individual papers and surveys for results outside of textbooks.
		\item Some of these are completely optional 
		\item Some I expect you to know (but try to explain well enough in the slides).
		\item See the reading list on Absalon for details (will likely be updated during the course).
		%\item survey papers:
		%\begin{description}
		%	\item[B\&V] Bergemann, Dirk, and Juuso Välimäki. ``Dynamic mechanism design: An introduction.'' Journal of Economic Literature 57.2 (2019): 235-74.
		%	\item[B\&M] Bergemann, Dirk, and Stephen Morris. ``Information design: A unified perspective.'' Journal of Economic Literature 57.1 (2019): 44-95. 
		%\end{description}
	\end{itemize}
\end{frame}


%\begin{frame}{References}
%	In the end:
%	\begin{itemize}
	%		\item I suggest getting B{\"o}rgers' textbook. It is not necessary, but might be useful.
	%		\item Diamantaras or MWG are okay alternatives.
	%		\item I do \textbf{not} recommend getting Roth \& Sotomayor textbook unless you are really interested in the topic of matching. Promise the slides will be sufficient for that part of the class.
	%	\end{itemize}
%\end{frame}





\section{What is mechanism design?}

\begin{frame}{What is Economics?}
	\begin{columns}
		\begin{column}{0.6\linewidth}
			{\setstretch{1.3}
				\begin{itemize}
					\item What is \structure<1>{the big question of Economics}?
					\pause
					\item My take: it is about the \alert{optimal allocation of resources}.
					\begin{itemize}
						\item How to trade things from people who have them to people who value them (in terms of other things they offer in exchange)?
						\item How to allocate the right people to the right jobs?
						\item How to allocate education, healthcare, congestion, ...?
						% Educators' resources are limited and heterogeneous, how to allocate students to schools?
					\end{itemize}
					\item (Also about disagreements about what ``optimal'' means.)
				\end{itemize}
			}
		\end{column}
		\begin{column}{0.4\linewidth}
			\pause[1]
			\includegraphics<handout:0|1>[width=\linewidth]{pics/M0/alloc}
			\includegraphics<handout:0|2->[width=\linewidth, trim=245 0 35 0, clip]{pics/M0/market2}
		\end{column}
	\end{columns}
\end{frame}


\begin{frame}{What is the problem?}
	\begin{columns}
		\begin{column}{0.6\linewidth}
			{\setstretch{1.3}
				\begin{itemize}
					\item Isn't this solved?
					\begin{itemize}
						\item You've seen in Micro I that \alert<1>{markets} are efficient! (First Welfare Theorem)
					\end{itemize}
					\pause
					\item Well, not really...
					\begin{itemize}
						\item Many cases where FWT breaks: externalities, monopoly power, \structure<3>{asymmetric information}...
					\end{itemize}
					\pause
					\item Also, efficiency might not be the designer's objective
				\end{itemize}
			}
		\end{column}
		\begin{column}{0.4\linewidth}
			\pause[1]
			\includegraphics<handout:0>[width=\linewidth]{pics/M0/market}
		\end{column}
	\end{columns}
\end{frame}


\begin{frame}{What is Game Theory?}
\begin{center}
	Economic agents interact with each other.
	\pause
	
	$\Downarrow$
	
	What is the outcome? 
	
	How is it shaped by environment?
\end{center}
\end{frame}


\begin{frame}{What is Mechanism Design?}
\begin{center}
	\pause[2] 
	How to shape the environment to achieve it?
	
	$\Uparrow$
	
	\pause[1]
	There is some desirable outcome.
\end{center}
\end{frame}


%\begin{frame}{Example 1}
%\begin{columns}
%	\begin{column}{0.6\linewidth}
%		{\setstretch{1.3}\\
%			A country needs to elect a new president. How to organize the election?
%			\begin{itemize}
%				\item How many candidates, how many rounds, etc?
%				\item Turnout to physical voting is typically low. Should you allow online voting?
%				\item Should you allow buying votes (voting with money)?
%			\end{itemize}
%		}
%	\end{column}
%	\begin{column}{0.4\linewidth}
%		\pause[1]
%		\includegraphics<handout:0>[width=\linewidth]{pics/M0/debate}
%	\end{column}
%\end{columns}
%\end{frame}


\begin{frame}{Example 1}
\begin{columns}
	\begin{column}{0.45\linewidth}
		{\setstretch{1.3}\\
			Suppose EU wants to connect its high-speed rail networks. Who should pay for the connections? 
			\begin{itemize}
				\item Countries that own the land might not be willing to spend the money.
				\item Those with a lot of HSR feel they have already invested enough while others are free-riding.
			\end{itemize}
		}
	\end{column}
	\begin{column}{0.55\linewidth}
		\pause[1]
		\includegraphics<handout:0>[width=\linewidth]{pics/M0/rail}
	\end{column}
\end{columns}
\end{frame}


\begin{frame}{Example 2}
	\begin{columns}
		\begin{column}{0.6\linewidth}
			{\setstretch{1.3}\\
				You want to sell an apartment. What is the best way to do so?
				\begin{itemize}
					\item Market conditions change; potential buyers come and go. How long should you wait?
					\item Run an auction or bargain with each buyer?
					\item Reveal all house issues truthfully or let buyers find them?
				\end{itemize}
			}
		\end{column}
		\begin{column}{0.4\linewidth}
			\pause[1]
			\includegraphics<handout:0>[width=\linewidth]{pics/M0/8tallet}
		\end{column}
	\end{columns}
\end{frame}


\begin{frame}{Example 3}
\begin{columns}
	\begin{column}{0.6\linewidth}
		{\setstretch{1.3}\\
			Is it possible to design an effective decentralized currency? (Think cryptocurrency)
			\begin{itemize}
				\item Where does money come from and how to avoid counterfeits?
				\item If money is digital, how to avoid double spending?
				\item If someone needs to process transactions, how to design their rewards?
			\end{itemize}
		}
	\end{column}
	\begin{column}{0.4\linewidth}
		\pause[1]
		\includegraphics<handout:0>[width=\linewidth]{pics/M0/coins}
	\end{column}
\end{columns}
\end{frame}


\begin{frame}{Who plays the players?}
	\begin{columns}
		\begin{column}{0.6\linewidth}
			{\setstretch{1.3}
				\begin{itemize}
					\pause
					\item In principle, we can see the ``designer'' as yet another player in a game
					\begin{itemize}
						\item Government/regulator
						\item Organizer of an interaction
					\end{itemize}
					\item Moves first and sets the rules for other players
					\item Problem: designer has a \emph{very large} action set
					\begin{itemize}
						\item Specific tools are needed to solve the designer's problem
					\end{itemize}
				\end{itemize}
			}
		\end{column}
		\begin{column}{0.4\linewidth}
			\pause[1]
			\includegraphics<handout:0>[width=\linewidth]{pics/M0/housewins}
		\end{column}
	\end{columns}
\end{frame}


\begin{frame}{This course}
	What can you expect?
	\begin{itemize}
		\item overview of main results over past 40 years
		\item not that much from the frontier
		\item plenty of math! (brace yourselves) (also see math review on absalon)
		\item intuition and economics behind the models
		\item models are abstract but are applicable to a \textbf{lot of} areas (industrial organization, political economy, taxation, auctions\ldots{})
		\item Finally, I hope to give you some practice in economic \textbf{modelling}.
		%\item see course description for more details on what exactly you learn
	\end{itemize}
\end{frame}


\begin{frame}{Related courses at KU}
\begin{itemize}
	\item Contract theory, Auctions
	\item Economics of organization, Corporate Finance, Industrial organization
	\item Other courses in which applications of mechanism design appear:
	\begin{itemize}
		\item Public finance/Taxation
		\item Political economy
		\item Monetary
		\item Labor
		\item \ldots{}
	\end{itemize}
\end{itemize}
\end{frame}



\section{First Taste}

\begin{frame}{Deeper dive in Example 1}
\begin{columns}
	\begin{column}{0.6\linewidth}
		{\setstretch{1.3}\\
			\begin{itemize}
				\item Let's have a closer look at this problem and try to see \structure{how we can formalize it} and \alert{how we could solve it}.
				
				\item Consider a specific connection: from France to Italy via Monaco.
			\end{itemize}
		}
	\end{column}
	\begin{column}{0.4\linewidth}
		\pause[1]
		\includegraphics<handout:0>[width=\linewidth]{pics/M0/rail}
	\end{column}
\end{columns}
\end{frame}


\begin{frame}{Problem Formulation 1}
\begin{columns}
	\begin{column}{0.6\linewidth}
		{\setstretch{1.3}
			\begin{enumerate}
				\item What is the question we are answering?
				\pause
				\visible<handout:0>{
					\begin{enumerate}
						\item Should the connection be built?
						\item Who should pay for the construction?
						\item Who gets to control the usage (and collect revenues)?
					\end{enumerate}
				}
				\pause
				\item What kind of knowledge do we need to answer these questions?
				
				\item And importantly, who possesses this knowledge?
			\end{enumerate}
		}
	\end{column}
	\begin{column}{0.4\linewidth}
		\pause[1]
		\includegraphics<handout:0>[width=\linewidth]{pics/M0/rail2}
	\end{column}
\end{columns}
\end{frame}


\begin{frame}{Problem Formulation 2}
	Try to proceed as in game theory, and start by identifying \textbf<1>{players}, \textbf<1>{actions}, and \textbf<1>{payoffs}.
	\begin{itemize}
		\pause 
		\item \textbf{Players}: \visible<handout:0>{France, Italy at least (+EU as the designer). Can include Monaco and EU (as the funder/claimant). Can separate governments/companies that service rails from companies that operate rolling stock.}
		
		\pause
		\item \textbf{Actions}: \visible<handout:0>{this is tricky. No players' actions, but we have ``outcomes'': whether to build the railway (assume the route is inflexible), how the split the cost, how to split the revenues. Plus we can make direct payments.}
		
		\pause
		\item \textbf{Payoffs}: \visible<handout:0>{depend on outcomes, but also on the total cost of the project, on how much revenue can be collected, how much is generated in indirect benefits...}
	\end{itemize}
	You can already see that we can model this in many different ways, depending on what you consider relevant. What follows is one example, but by no means a definitive solution.
\end{frame}


\begin{frame}{Problem Formulation 3}
Try to proceed as in game theory, and start by identifying {players}, {actions}, and {payoffs}.
\begin{itemize}
	\item \textbf{Players}: \alert{$i \in \{F, I, E\}$} for France, Italy, EU, respectively
	
	\item \sout{Actions} \textbf{Outcomes}: \alert{$k_1 \in \{0,1\}$} whether construction occurs; \alert{$k_{2,i} \in [0,1]$} cost share of $i$; \alert{$k_{3,i} \in [0,1]$} revenue share of $i$; \alert{$t_i \in \mathbb{R}$} extra payment from $i$ to the EU/construction project (can be negative)
	
	\item \textbf{Payoffs}: 
	\begin{align*}
		u_i(k,t,C,R) &= k_1 \cdot \left( B_i + R k_{3,i} - C k_{2,i} \right) - t_i \quad \text{ for } i \in \{F,I\};
		\\
		u_E(k,t,C,R) &= k_1 \cdot \left( B_E + R k_{3,E} - C k_{2,E} \right) + t_I + t_F \quad \text{ for EU as the residual claimant; }
	\end{align*}
	where $C$ is the total cost of, $R$ is the total revenue from, and $B_i$ is the private benefit to $i$ from the rail connection.
\end{itemize}
\end{frame}


\begin{frame}{Problem Formulation 4}
	\begin{itemize}[<+->]
		\item But does EU as the designer simply maximize $u_E$?
		\begin{itemize}
			\item More reasonable that it cares about the member states, so the problem is \alert<2>{$\max_{k,t} \{ u_F + u_I + u_E \}$}
			\item equivalent to $\max_{k,t} \{ B_F + B_I + B_E + R - C \}$
		\end{itemize}
		\item If we knew $C, R, B_I, B_F, B_E$, then the problem is trivial: build the rail if and only if $B_F + B_I + B_E + R - C > 0$.
		\item But do we (the EU-designer) know these? Or do the players (the countries) have private information about these?
		\begin{itemize}
			\item $B_I, B_F$ are likely players' private info; they probably also have private knowledge about components of $C$ and $R$. Suppose so -- what do we do?
			\item Use payments as incentives to reveal info!
			\item ``If you report high private benefit $B_i$, the rail is more likely to be built, but you'll have to pay more for it.''
		\end{itemize}
	\end{itemize}
\end{frame}


\begin{frame}{Zoom out}
	\begin{itemize}
		\item This is effectively the kind of problem we will be thinking about, at least in the first half of the course:
		\begin{exampleblock}{MD problem}
			The designer has some preference over outcomes, which depends on the players' private information. How can the principal elicit this information and select the optimal outcome? 
		\end{exampleblock}
		
		\item \textbf{Assumptions} that we will mostly stick to:
		\begin{itemize}
			\item Players' \alert{private information} is the main concern.
			\item In particular, players' \structure{compliance} is not an issue; designer has the power to enforce their deicisions.
			\item We will often assume that transfers/payments can be used to provide incentives -- but not always.
		\end{itemize}
	\end{itemize}
\end{frame}


\begin{frame}{Takeaways}
	\begin{enumerate}
		\item Game-theory framework is almost enough for us to work with, but it needs modifications to be useful.
		
		\item Even after setting the problem up, it is not immediate how to approach it with conventional Economic tools -- could use some more specialized solution methods.
	\end{enumerate}
\end{frame}



\begin{frame}{For next week}
	\begin{enumerate}
		\item Split into study groups (let me know if you are looking for a group)
		\item Read math review notes on absalon
		\item Watch lectures 2.1 (`What is a mechanism?') and 2.2 (`Dominant strategy implementation') on youtube.
		\item Or read Narahari ch.14-16
	\end{enumerate}
\end{frame}


\end{document}