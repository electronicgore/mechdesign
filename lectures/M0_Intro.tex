%%% License: Creative Commons Attribution Share Alike 4.0 (see https://creativecommons.org/licenses/by-sa/4.0/)

\documentclass[english,10pt
%,handout
,aspectratio=169
]{beamer}
%%% License: Creative Commons Attribution Share Alike 4.0 (see https://creativecommons.org/licenses/by-sa/4.0/)

\DeclareGraphicsExtensions{.eps, .pdf,.png,.jpg,.mps,}
\usetheme{reMedian}
\usepackage{parskip}
\makeatother

\renewcommand{\baselinestretch}{1.1} 

\usepackage{amsmath, amssymb, amsfonts, amsthm}
\usepackage{enumerate}
%\usepackage{enumitem}
\usepackage{hyperref}
\usepackage{url}
\usepackage{bbm}
\usepackage{color}

\usepackage{tikz}
\usepackage{tikzscale}
%\newcommand*\circled[1]{\tikz[baseline=(char.base)]{
%		\node[shape=circle,draw, inner sep=-20pt] (char) {#1};}}
%\usetikzlibrary{automata,positioning}
%\usetikzlibrary{decorations.pathreplacing}
\usepackage{pgfplots}
\usepgfplotslibrary{fillbetween}
\usepackage{graphicx}

\usepackage{setspace}
%\thinmuskip=1mu
%\medmuskip=1mu 
%\thickmuskip=1mu 


%\usecolortheme{default}
\usepackage{verbatim}
\usepackage[normalem]{ulem}

\usepackage{apptools}
\AtAppendix{
	\setbeamertemplate{frame numbering}[none]
}
\usepackage{natbib}


% red strikeout
\newcommand\soutred{\bgroup\markoverwith
	{\textcolor{red}{\rule[0.55ex]{2pt}{0.8pt}}}\ULon}



%% To use LyX frames from old version:
%\def\lyxframeend{} % In case there is a superfluous frame end
%\long\def\lyxframe#1{\@lyxframe#1\@lyxframestop}%
%\def\@lyxframe{\@ifnextchar<{\@@lyxframe}{\@@lyxframe<*>}}%
%\def\@@lyxframe<#1>{\@ifnextchar[{\@@@lyxframe<#1>}{\@@@lyxframe<#1>[]}}
%\def\@@@lyxframe<#1>[{\@ifnextchar<{\@@@@@lyxframe<#1>[}{\@@@@lyxframe<#1>[<*>][}}
%\def\@@@@@lyxframe<#1>[#2]{\@ifnextchar[{\@@@@lyxframe<#1>[#2]}{\@@@@lyxframe<#1>[#2][]}}
%\long\def\@@@@lyxframe<#1>[#2][#3]#4\@lyxframestop#5\lyxframeend{%
%	\frame<#1>[#2][#3]{\frametitle{#4}#5}}


\title{Mechanism Design}

\subtitle{0: Introduction}

\author{Egor Starkov}

\date{K{\o}benhavns Unversitet \\
	Fall 2022}


\begin{document}
	\AtBeginSection[]{
		\frame<beamer>{
			\frametitle{This slide deck:}
			\tableofcontents[currentsection,currentsubsection]
	}}
	\frame[plain]{\titlepage}






\section{Logistics}

\begin{frame}{Hi}
	\begin{itemize}
		\item Egor Starkov
		\item Contact: \texttt{egor.starkov@econ.ku.dk} or absalon inbox
		\item Research interests: information economics, dynamic games, communication
		\item Office: 26.1.13
		\item Office hours: Tue, 14-15
		\item Questions: email/absalon, before/after class
	\end{itemize}
\end{frame}


\begin{frame}{Logistics}
	\begin{itemize}
		\item Weekly lectures (except Fall break -- week \#42)
		\begin{itemize}
			\item Tue, 15:15-18:00, CSS 2.0.18 (but check timetable for room changes)
		\end{itemize}
		
		\pause
		\item Mandatory assignment:
		\begin{itemize}
			\item ``Make your own problem'' -- find an interesting real-world setting and analyze it using the machinery from the class
			\item groups of up to 4 are allowed
			\item deadline: some time in November?
		\end{itemize}
		
		\pause
		\item Final exam:
		\begin{itemize}
			\item 24hrs take home (individual, no groups)
			\item Convert problems from text to math; solve them and explain intuition
		\end{itemize}
		
		\pause
		\item Research module for PhD students: contact me if you would like to do it
	\end{itemize}
\end{frame}


\begin{frame}{Covid-specific stuff}
	\begin{itemize}
		\item No online streams/recordings are currently planned, back to stone age
	\end{itemize}
\end{frame}


\begin{frame}{Materials}
	\begin{itemize}
		\item Intended workflow: 
		\begin{enumerate}
			\item We start a topic in class
			\item \uline{You watch video-lectures at home} during the week (+read textbook, slides, whatever's necessary)
			\item We go through the material \emph{very quickly} in class and discuss any questions you have
			\item We solve some problems in class
			\item You will have more problems to practice at home
		\end{enumerate}
		\pause
		\item Suggestion: organize into groups. Watch lectures in groups. Discuss problems in groups.
		\pause
		\item In class I use whiteboard+slides. Slides on absalon include some of the whiteboard parts.
		\begin{itemize}
			\item I'll try to upload slides in advance, but they might be edited and updated afterwards
		\end{itemize}
	\end{itemize}
\end{frame}


\begin{frame}{References: textbooks}
	This course is a compilation of many books, papers, courses; does not follow any single one too closely. But here are some books that might help (see the reading list on Absalon for full references)
	\begin{itemize}%[\noitemsep]
		\item \textbf{Narahari}: Probably your best bet. Hard to find in print, but you have online access through the library (see Absalon).
		\item \textbf{Diamantaras}: Another good textbook, but seems very hard to find.
		\item \textbf{B\"{o}rgers}: I used this as default in previous years, but it's quite hardcore and hard to follow. (Easy to find though.)
		\item \textbf{MWG}: A microeconomic bible. Very good, very clear, but has the smallest coverage for our course.
		\item \textbf{RS}: Relevant for two lectures on matching. Some material is contained in Narahari and Diamantaras. Nice reference if you are into matching.
	\end{itemize}
\end{frame}


\begin{frame}{References: other}
	\begin{itemize}
		\item I will sometimes refer to individual papers and surveys for results outside of textbooks.
		\item Some of these are completely optional 
		\item Some I expect you to know (but try to explain well enough in the slides).
		\item See the reading list on Absalon for details (will likely be updated during the course).
		%\item survey papers:
		%\begin{description}
		%	\item[B\&V] Bergemann, Dirk, and Juuso Välimäki. ``Dynamic mechanism design: An introduction.'' Journal of Economic Literature 57.2 (2019): 235-74.
		%	\item[B\&M] Bergemann, Dirk, and Stephen Morris. ``Information design: A unified perspective.'' Journal of Economic Literature 57.1 (2019): 44-95. 
		%\end{description}
	\end{itemize}
\end{frame}


%\begin{frame}{References}
%	In the end:
%	\begin{itemize}
	%		\item I suggest getting B{\"o}rgers' textbook. It is not necessary, but might be useful.
	%		\item Diamantaras or MWG are okay alternatives.
	%		\item I do \textbf{not} recommend getting Roth \& Sotomayor textbook unless you are really interested in the topic of matching. Promise the slides will be sufficient for that part of the class.
	%	\end{itemize}
%\end{frame}





\section{What is mechanism design?}

\begin{frame}{What is Economics?}
	\begin{columns}
		\begin{column}{0.6\linewidth}
			{\setstretch{1.3}
				\begin{itemize}
					\item What is \structure<1>{the big question of Economics}?
					\pause
					\item My take: it is about the \alert{optimal allocation of resources}.
					\begin{itemize}
						\item How to trade things from people who have them to people who value them (in terms of other things they offer in exchange)?
						\item How to allocate the right people to the right jobs?
						\item How to allocate education, healthcare, congestion, ...?
						% Educators' resources are limited and heterogeneous, how to allocate students to schools?
					\end{itemize}
					\item (Also about disagreements about what ``optimal'' means.)
				\end{itemize}
			}
		\end{column}
		\begin{column}{0.4\linewidth}
			\pause[1]
			\includegraphics<handout:0|1>[width=\linewidth]{pics/M0/alloc}
			\includegraphics<handout:0|2->[width=\linewidth, trim=245 0 35 0, clip]{pics/M0/market2}
		\end{column}
	\end{columns}
\end{frame}


\begin{frame}{What is the problem?}
	\begin{columns}
		\begin{column}{0.6\linewidth}
			{\setstretch{1.3}
				\begin{itemize}
					\item Isn't this solved?
					\begin{itemize}
						\item You've seen in Micro I that \alert<1>{markets} are efficient! (First Welfare Theorem)
					\end{itemize}
					\pause
					\item Well, not really...
					\begin{itemize}
						\item Many cases where FWT breaks: externalities, monopoly power, \structure<3>{asymmetric information}...
						\pause
						\item Also, efficiency might not be the objective
					\end{itemize}
				\end{itemize}
			}
		\end{column}
		\begin{column}{0.4\linewidth}
			\pause[1]
			\includegraphics<handout:0>[width=\linewidth]{pics/M0/market}
		\end{column}
	\end{columns}
\end{frame}


\begin{frame}{What is Game Theory?}
\begin{center}
	Economic agents interact with each other.
	\pause
	
	$\Downarrow$
	
	What is the outcome? 
	
	How is it shaped by environment?
\end{center}
\end{frame}


\begin{frame}{What is Mechanism Design?}
\begin{center}
	\pause[2] 
	How to shape the environment to achieve it?
	
	$\Uparrow$
	
	\pause[1]
	There is some desirable outcome.
\end{center}
\end{frame}


%\begin{frame}{Example 1}
%\begin{columns}
%	\begin{column}{0.6\linewidth}
%		{\setstretch{1.3}\\
%			A country needs to elect a new president. How to organize the election?
%			\begin{itemize}
%				\item How many candidates, how many rounds, etc?
%				\item Turnout to physical voting is typically low. Should you allow online voting?
%				\item Should you allow buying votes (voting with money)?
%			\end{itemize}
%		}
%	\end{column}
%	\begin{column}{0.4\linewidth}
%		\pause[1]
%		\includegraphics<handout:0>[width=\linewidth]{pics/M0/debate}
%	\end{column}
%\end{columns}
%\end{frame}


\begin{frame}{Example 1}
\begin{columns}
	\begin{column}{0.45\linewidth}
		{\setstretch{1.3}\\
			Suppose EU wants to connect its high-speed rail networks. Who should pay for the connections? 
			\begin{itemize}
				\item Countries that own the land might not be willing to spend the money.
				\item Those with a lot of HSR feel they have already invested enough while others are free-riding.
			\end{itemize}
		}
	\end{column}
	\begin{column}{0.55\linewidth}
		\pause[1]
		\includegraphics<handout:0>[width=\linewidth]{pics/M0/rail}
	\end{column}
\end{columns}
\end{frame}


\begin{frame}{Example 2}
	\begin{columns}
		\begin{column}{0.6\linewidth}
			{\setstretch{1.3}\\
				You want to sell an apartment. What is the best way to do so?
				\begin{itemize}
					\item Market conditions change; potential buyers come and go. How long should you wait?
					\item Run an auction or bargain with each buyer?
					\item Reveal all house issues truthfully or let buyers find them?
				\end{itemize}
			}
		\end{column}
		\begin{column}{0.4\linewidth}
			\pause[1]
			\includegraphics<handout:0>[width=\linewidth]{pics/M0/8tallet}
		\end{column}
	\end{columns}
\end{frame}


\begin{frame}{Example 3}
\begin{columns}
	\begin{column}{0.6\linewidth}
		{\setstretch{1.3}\\
			Is it possible to design an effective decentralized currency? (Think cryptocurrency)
			\begin{itemize}
				\item Where does money come from and how to avoid counterfeits?
				\item If money is digital, how to avoid double spending?
				\item If someone needs to process transactions, how to design their rewards?
			\end{itemize}
		}
	\end{column}
	\begin{column}{0.4\linewidth}
		\pause[1]
		\includegraphics<handout:0>[width=\linewidth]{pics/M0/coins}
	\end{column}
\end{columns}
\end{frame}


\begin{frame}{Who plays the players?}
	\begin{columns}
		\begin{column}{0.6\linewidth}
			{\setstretch{1.3}
				\begin{itemize}
					\pause
					\item In principle, we can see the ``designer'' as yet another player in a game
					\begin{itemize}
						\item Government/regulator
						\item Organizer of an interaction
					\end{itemize}
					\item Moves first and sets the rules for other players
					\item Problem: designer has a \emph{very large} action set
					\begin{itemize}
						\item Specific tools are needed to solve the designer's problem
					\end{itemize}
				\end{itemize}
			}
		\end{column}
		\begin{column}{0.4\linewidth}
			\pause[1]
			\includegraphics<handout:0>[width=\linewidth]{pics/M0/housewins}
		\end{column}
	\end{columns}
\end{frame}


\begin{frame}{This course}
	What can you expect?
	\begin{itemize}
		\item overview of main results over past 40 years
		\item not that much from the frontier
		\item plenty of math! (brace yourselves) (also see math review on absalon)
		\item intuition and economics behind the models
		\item models are abstract but are applicable to a \textbf{lot of} areas (industrial organization, political economy, taxation, auctions\ldots{})
		%\item see course description for more details on what exactly you learn
	\end{itemize}
\end{frame}


\begin{frame}{Related courses at KU}
\begin{itemize}
	\item Contract theory, Auctions
	\item Economics of organization, Corporate Finance, Industrial organization
	\item Other courses in which applications of mechanism design appear:
	\begin{itemize}
		\item Public finance/Taxation
		\item Political economy
		\item Monetary
		\item Labor
		\item \ldots{}
	\end{itemize}
\end{itemize}
\end{frame}







\begin{frame}{For next week}
	\begin{enumerate}
		\item Watch lectures 2.1 (`What is a mechanism?') and 2.2 (`Dominant strategy implementation')
	\end{enumerate}
\end{frame}


\end{document}