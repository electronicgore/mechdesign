%%% License: Creative Commons Attribution Share Alike 4.0 (see https://creativecommons.org/licenses/by-sa/4.0/)

\documentclass[english,10pt
,aspectratio=169
%,handout
%%%%%%%,notes
]{beamer}
%%% License: Creative Commons Attribution Share Alike 4.0 (see https://creativecommons.org/licenses/by-sa/4.0/)

\DeclareGraphicsExtensions{.eps, .pdf,.png,.jpg,.mps,}
\usetheme{reMedian}
\usepackage{parskip}
\makeatother

\renewcommand{\baselinestretch}{1.1} 

\usepackage{amsmath, amssymb, amsfonts, amsthm}
\usepackage{enumerate}
%\usepackage{enumitem}
\usepackage{hyperref}
\usepackage{url}
\usepackage{bbm}
\usepackage{color}

\usepackage{tikz}
\usepackage{tikzscale}
\newcommand*\circled[1]{\tikz[baseline=(char.base)]{
		\node[shape=circle,draw, inner sep=-20pt] (char) {#1};}}
\usetikzlibrary{automata,positioning}
\usetikzlibrary{decorations.pathreplacing}
\usepackage{pgfplots}
\usepgfplotslibrary{fillbetween}
\usepackage{graphicx}

\usepackage{setspace}
\thinmuskip=1mu
\medmuskip=1mu 
\thickmuskip=1mu 


\usecolortheme{default}
\usepackage{verbatim}
\usepackage[normalem]{ulem}

\usepackage{apptools}
\AtAppendix{
	\setbeamertemplate{frame numbering}[none]
}
\usepackage{natbib}


% red strikeout
\newcommand\soutred{\bgroup\markoverwith
	{\textcolor{red}{\rule[0.55ex]{2pt}{0.8pt}}}\ULon}



% To use LyX frames from old version:
\def\lyxframeend{} % In case there is a superfluous frame end
\long\def\lyxframe#1{\@lyxframe#1\@lyxframestop}%
\def\@lyxframe{\@ifnextchar<{\@@lyxframe}{\@@lyxframe<*>}}%
\def\@@lyxframe<#1>{\@ifnextchar[{\@@@lyxframe<#1>}{\@@@lyxframe<#1>[]}}
\def\@@@lyxframe<#1>[{\@ifnextchar<{\@@@@@lyxframe<#1>[}{\@@@@lyxframe<#1>[<*>][}}
\def\@@@@@lyxframe<#1>[#2]{\@ifnextchar[{\@@@@lyxframe<#1>[#2]}{\@@@@lyxframe<#1>[#2][]}}
\long\def\@@@@lyxframe<#1>[#2][#3]#4\@lyxframestop#5\lyxframeend{%
	\frame<#1>[#2][#3]{\frametitle{#4}#5}}


\title{Mechanism Design}

\subtitle{3: Optimal (revenue-maximizing) mechanisms}

\author{Egor Starkov}

\date{K{\o}benhavns Unversitet \\
	Fall 2021}


\begin{document}
	\AtBeginSection[]{
		\frame{
			\frametitle{This slide deck:}
			\tableofcontents[currentsection,currentsubsection]
	}}
	\frame[plain]{\titlepage}



\begin{frame}{Done so far}
\begin{itemize}
	\item Introduced basic notions and criteria:
	\begin{itemize}
		\item s.c.f., mechanism, implementation and implementability (DSIC, BIC),
		\item efficiency, individual rationality, budget balance.
	\end{itemize}
	\pause
	\item Covered some fundamental results in Mechanism Design:
	\begin{itemize}
		\item revelation principle (pretty universal),
		\item payoff/revenue equivalence (Euclidean model, slightly generalizable),
		\item necessary conditions for implementability (weak preference reversal, monotonicity)
	\end{itemize}
	\pause
	\item Learned to implement the efficient s.c.f.:
	\begin{itemize}
		\item DSIC: VCG;
		\item BIC: AGV, gVCG.
	\end{itemize}
\end{itemize}
\end{frame}


\begin{frame}{Today}
\begin{itemize}
	\item Finished with implementing efficient s.c.f.s
	\item Take a break from implementing arbitrary s.c.f.s
	\item Today will look at revenue maximization.
	\begin{itemize}
		\item Revenue-maximizing mechanisms called ``optimal'' in the literature (meaning optimal for the designer), after Myerson's ``optimal mechanism''.
	\end{itemize}
\end{itemize}
\end{frame}


\section{Two types (Monopolistic Screening)}

\begin{frame}{Setting 1: one buyer, discrete type}
\begin{itemize}
	\item Starting simple (\alert{Monopolistic Screening} / Second-Degree Price Discrimination).
	\item Seller-designer can set quantities $k$ and prices $t$ for product, has production costs $c(k) = k^2 (=-v_0(k))$.
	\begin{itemize}
		\item As usual, designer has no private information. ``Informed principal'' is a difficult problem.
	\end{itemize}
	\item There is one buyer with \structure{valuation $\theta \in \{ L,H \}$}, private info. Prior probabilities are $\phi(H) = \phi$, $\phi(L)=1-\phi$.
	\item Buyer's preferences Euclidean: \structure{$u_b(x,\theta) = \theta k - t$}
	\begin{itemize}
		\item Is this a Euclidean model? % No, types discrete. Even if it was - can't use known mechs because different objective.
	\end{itemize}
\end{itemize}
\end{frame}


\begin{frame}{Monopolistic Screening}
\begin{itemize}
	\item As usual, look at DRM $\Gamma = (\Theta, (k,t))$. Notation-wise, let $k_\theta \equiv k(\theta)$ and $t_\theta \equiv t(\theta)$.
	\item Seller's problem (contrary to before, we can now choose $k$ in addition to $t$.)
	% Problem is now more difficult because now choose k, not just t
\end{itemize}

\begin{columns}
	\begin{column}{0.5\linewidth}
		\begin{align*}
			\max_{(k,t)} & \left\{ \phi (t_H - k_H^2) + (1-\phi) (t_L - k_L^2) \right\}
			\\ \text{s.t. } & (IC_H):\quad \theta_H k_H - t_H \geq \theta_H k_L - t_L
			\\ & (IC_L):\quad \theta_L k_H - t_H \leq \theta_L k_L - t_L
			\\ & (IR_H):\quad \theta_H k_H - t_H \geq 0
			\\ & (IR_L):\quad \theta_L k_L - t_L \geq 0
		\end{align*}
	\end{column}
	\begin{column}{0.5\linewidth}
		\begin{enumerate}[<+->]
			\item show $IC_H$ and $IR_L$ imply $IR_H$;
			\item show $IC_H$ and $IC_L$ imply $k_H \geq k_L$;
			\item show $k_H \geq k_L$ and binding $IC_H$ imply $IC_L$;
			\item show $IC_H$ and $IR_L$ bind;
			\item solve for optimal $(k,t)$.
		\end{enumerate}
	\end{column}
\end{columns}



% 410-3 2019 ps7 q2
\end{frame}


\begin{frame}{Monopolistic Screening: lessons}
Lessons:
	\begin{enumerate}[<+->]
		\item offering a menu may be optimal to extract value from buyers;
		\item explains weird non-linear prices you can often encounter; %add photo?
		\item quantity is distorted downward for low type
		\item high type gets information rent (pays below valuation);
		\item IR must bind for at least some type.
	\end{enumerate}
\end{frame}


\section{Interval of types (Optimal Mechanism)}

\begin{frame}{Setting 2: one buyer, interval of types}
\begin{itemize}
	\item Designer/seller has one indivisible item for sale. Chooses menu including probability of sale $k(\theta) \in [0,1]$ and payment $t(\theta)$ given report $\theta$, no costs for simplicity.
	\begin{itemize}
		\item Nothing changes from when $k$ was quantity, since everyone is risk-neutral.
	\end{itemize}
	\item Buyer has \structure{valuation $\theta \sim F[0,\bar{\theta}]$}, private info.
	\item Buyer's preferences Euclidean: \structure{$u_b = \theta k - t$}
\end{itemize}
\end{frame}


\begin{frame}{Optimal Mechanism}
Solution approach
\begin{enumerate}[<+->]
	\item Show: if $\theta' \leq \theta''$ then $k(\theta') \leq k(\theta'')$. (use the two ICs)
	\item Get the envelope representation $U_b(\theta) = U_b(0) + \int_0^\theta k(s) ds$ 
	\\ (BIC and DSIC ERPs yield the same expression, since there is only one player).
	\item Recall that $U_b(\theta) = \theta k(\theta) - t(\theta)$, so
	{\footnotesize
		\begin{align*}
			\mathbb{E} U_s = \mathbb{E}_\theta [ t(\theta) ] &
			= \mathbb{E}_\theta \left[ \theta k(\theta) - U_b(\theta) \right]
			\\ \visible<+->{ &= \mathbb{E}_\theta \left[ \theta k(\theta) - \int_0^\theta k(s) ds \right] - U_b(0) }
			\\ \visible<+->{ &= \int_0^{\bar{\theta}} \theta k(\theta) \phi(\theta) d\theta - \structure{\int_0^{\bar{\theta}} \int_0^\theta k(s) \phi(\theta) ds d\theta} - U_b(0) }
			\\ \visible<+->{ &= \int_0^{\bar{\theta}} \theta k(\theta) \phi(\theta) d\theta - \structure{\int_0^{\bar{\theta}} (1-F(\theta)) k(\theta) d\theta} - U_b(0) }
			\\ \visible<+->{ &= \int_0^{\bar{\theta}} k(\theta) \left( \theta - \frac{1-F(\theta)}{\phi(\theta)} \right) \phi(\theta) d\theta - U_b(0) }
	\end{align*}	}
\end{enumerate}
\end{frame}


\begin{frame}{Optimal Mechanism: integration by parts}
Integration by parts under the microscope:
\begin{align*}
	\int_0^{\bar{\theta}} \left( \int_0^\theta k(s) ds \right) \phi(\theta) d\theta &= \left. \left[F(\theta) \int_0^\theta k(s) ds \right] \right|_{\theta=0}^{\bar{\theta}} - \int_0^{\bar{\theta}} F(\theta) k(\theta) d\theta
	\\ &= \int_0^{\bar{\theta}} k(\theta) d\theta - \int_0^{\bar{\theta}} F(\theta) k(\theta) d\theta
	\\ &= \int_0^{\bar{\theta}} (1-F(\theta)) k(\theta) d\theta
\end{align*}
\end{frame}


\begin{frame}{Optimal Mechanism: pinning $U_b(0)$}
\begin{equation*}
	\mathbb{E} U_s = \mathbb{E}_\theta [ t(\theta) ] = \int_0^{\bar{\theta}} k(\theta) \left( \theta - \frac{1-F(\theta)}{\phi(\theta)} \right) \phi(\theta) d\theta \only<1>{- U_b(0)} %\only<2->{\soutred{- U_b(0)}} 
\end{equation*}
\begin{itemize}
	\item To choose: $\{k(\theta)\}_\theta$ and $U_b(0)$ (pins transfers).
	\pause
	\item What do with $U_b(0)$?
	\begin{itemize}
		\item Want to minimize since decreases revenue.
		\item Gotta be $U_b(0) \geq 0$ to satisfy IR for $\theta = 0$. Other types?
		\pause
		\item Recall $U_b(\theta) = U_b(0) + \int_0^\theta k(s) ds$ and $k(\theta) \geq 0$, so $U_b(\theta)\geq U_b(0)$ for all $\theta$, 
		\item hence $U_b(0) = 0$ is optimal (max revenue, all IR hold, IR binds for $\theta=0$).
	\end{itemize}
\end{itemize}
\end{frame}


\begin{frame}{Optimal Mechanism: optimal $k$}
\begin{equation*}
	\mathbb{E} U_s = \int_0^{\bar{\theta}} k(\theta) \left( \theta - \frac{1-F(\theta)}{\phi(\theta)} \right) \phi(\theta) d\theta
\end{equation*}
\begin{itemize}
	\item What do with $k$?
	\begin{itemize}[<+->]
		\item Define \structure{virtual surplus} $VS(\theta) := \theta - \frac{1-F(\theta)}{\phi(\theta)}$.
		\item Pointwise maximization: 
		$k(\theta) = \begin{cases}
			1 & \text{ if } VS(\theta) \geq 0;
			\\ 0 & \text{ if } VS(\theta) < 0;
		\end{cases}$
		\item Remember: $k(\theta)$ is only implementable if it is monotone! Sufficient condition: $VS(\theta)$ increasing in $\theta$.
		\item In the end, if $VS(\theta)$ is increasing in $\theta$, the optimal mechanism is given by $k(\theta)$ as above and $t(\theta)$ that can be computed from ERP.
	\end{itemize}
\end{itemize}
\end{frame}


\begin{frame}{Optimal Mechanism: Virtual Surplus}
\begin{itemize}
	\item What is virtual surplus?
	\begin{itemize}
		\item Ir reflects \structure{information rents} we have to pay to high types to incentivize them to reveal type honestly.
	\end{itemize}
	\item Sufficient for increasing $VS(\theta)$ is \structure{increasing hazard rate} $\frac{\phi(\theta)}{1-F(\theta)}$. \\ 
	The assumption we usually live with; need suitable distribution $F(\theta)$.
	\item What do if ``unlucky'' and $F(\theta)$ is such that VS is sometimes decreasing?
	\begin{itemize}
		\item ``\structure{Ironing}'': find monotone $k(\theta)$ that is ``closest'' to the unconstrained optimum.
		\item E.g. if VS is globally decreasing then some constant $k$ is optimal.
		\item There is a kind of a general approach to this, but it's difficult, see \href{https://onlinelibrary.wiley.com/doi/abs/10.3982/ECTA18312}{\uline{Kleiner, Moldovanu, \& Strack (2021)}}
	\end{itemize}
\end{itemize}
\end{frame}


\begin{frame}{Optimal Mechanism: Lessons}
\begin{itemize}
	\item Incentives are costly.
	\begin{itemize}
		\item If $\theta$ is an attractive type to imitate, have to distort $\theta$'s allocation $k(\theta)$ compared to first-best (full-info benchmark).
		\item (That's why $k(\bar{\theta})$ is not distorted.)
	\end{itemize}
	\item Even though gains from trade always present, optimal to commit to not sell to low types to charge high types more.
	\item Distribution $\phi$ matters: if more high types then focus on them and sell with lower probability to the low types.
	\item It will most of the time be optimal to have some cutoff rule: $k(\theta) = \mathbb{I}\{\theta > \hat{\theta}\}$ for some $\hat{\theta}$.
	\begin{itemize}
		\item Things become more interesting in multi-item case, see \href{https://www.sciencedirect.com/science/article/pii/S0022053107000348}{\uline{Manelli \& Vincent (2007)}}
	\end{itemize}
\end{itemize}
\end{frame}


\begin{frame}{Optimal Mechanism: Example}
\begin{itemize}[<+->]
	\item Derive the optimal mechanism when $\theta \sim U[0,1]$.
	\item $VS(\theta) = \theta - \frac{1-F(\theta)}{\phi(\theta)} = 2 \theta - 1$,
	\item hence optimal $k(\theta) = \mathbb{I} \{\theta \geq 0.5 \}$. Payments?
	\item $U_b (\theta) = U_b(0) + \int_0^\theta k(\theta) d\theta = \max \{\theta - 0.5,\,0\}$,
	\item so $t(\theta) = 0.5 \cdot \mathbb{I} \{\theta \geq 0.5 \}$. Fixed price is optimal!
	\item Actually cannot do much better than fixed price in this simple trade model.
	\begin{itemize}
		\item Maximizing $\int k(\theta) VS(\theta) d\theta$, linear in $k(\theta)$, so will typically have either a cutoff rule, or constant rule -- unless $VS(\theta)$ non-monotone.
		\item Consequence of Euclidean payoffs. More interesting results with non-linear payoffs.
	\end{itemize}
\end{itemize}
\end{frame}


%TODO: this could be a nice midterm/final problem if not covered in the slides.
%\subsection{Principal-agent problem}
%
%\begin{frame}{Detour (setting 2b): principal-agent problem}
%\begin{itemize}
%	\item Principal (employer) and agent (employee).
%	\item Agent chooses effort $e \in [0,1]$, has private type $\theta \in [0,1]$ about cost of effort \structure{$c(e,\theta) = \theta \cdot e$}.
%	\item Principal offers a menu of contracts that specify effort $e$ and wage $w$, maximizes profit \structure{$\pi = e-w$}.
%\end{itemize}
%\end{frame}
%
%
%\begin{frame}{Principal-agent}
%\begin{itemize}
%	\item Difference between optimal selling mechanism and optimal labor contract?
%	\item None, they are the same model (effort=allocation, wage=transfer).
%	%TODO: the pic
%\end{itemize}
%\end{frame}
%
%
%\begin{frame}{Principal-agent}
%\begin{itemize}
%	\item Proceed as in the optimal mechanism.
%\end{itemize}
%\end{frame}


\section{Many buyers (Optimal Auction)}

\begin{frame}{Setting 3: many buyers, interval of types}
\begin{itemize}
	\item Designer/seller has one indivisible item for sale. Chooses allocation $k(\theta) \in \varDelta(N)$ and payment profile $t(\theta) \in \mathbb{R}^N$ given report profile $\theta$.
	\item Buyers $i \in \{1,...,N\}$ have \structure{valuations $\theta_i \sim \text{i.i.d.} F[0,\bar{\theta}_i]$}, private info.
	\begin{itemize}
		\item Independence of $\theta_i$ is important, since we rely on revenue equivalence / ERP
	\end{itemize}
	\item Buyer's preferences Euclidean: \structure{$u_b = \theta_i k_i - t_i$}
	\item What is the optimal BIC mechanism that maximizes seller's expected profit?
\end{itemize}
\end{frame}


\begin{frame}{Optimal Auction}
\begin{itemize}
	\item We are effectively designing the optimal auction.
	\begin{itemize}
		\item Selling the good to the highest bidder is efficient (assuming higher-value buyers bid more),
		\item so all standard auction formats -- first-/second-price, Dutch, English -- are revenue-equivalent! (buyer with value zero gets zero)
	\end{itemize}
	\item To get more profit often have to depart from efficiency, e.g. by
	\begin{itemize}
		\item setting reservation price,
		\item discriminating buyers (even if they are ex ante identical!).
	\end{itemize}
\end{itemize}
\end{frame}


\begin{frame}{Optimal Auction}
\begin{itemize}
	\item From the perspective of the individual bidder, things are not much different from single-player model, just take expectations over $\theta_{-i}$:
	\begin{align*}
		\bar{t}_i(\theta_i) &\equiv \mathbb{E}_{\theta_{-i}} t_i(\theta_i,\theta_{-i})
		\\
		\bar{k}_i(\theta_i) &\equiv \mathbb{E}_{\theta_{-i}} k_i(\theta_i,\theta_{-i})
		\\
		\bar{U}_i(\theta_i) &\equiv \mathbb{E}_{\theta_{-i}} u_i(x(\theta_i,\theta_{-i}),\theta_i)
	\end{align*}
	\item Monotonicity: if $\theta_i' < \theta_i''$ then 
	$\bar{k}_i(\theta_i') \leq \bar{k}_i(\theta_i'').$
	\item Envelope representation: 
	$$\bar{U}_i(\theta_i) = \bar{U}_i(0) + \int_0^{\theta_i} \bar{k}_i(s) ds.$$
\end{itemize}
\end{frame}


\begin{frame}{Optimal Auction: Seller}
\begin{align*}
	\mathbb{E} U_s &= \mathbb{E}_\theta \left[ \sum_i t_i(\theta) \right]
	\\ &= \sum_i \mathbb{E}_{\theta} \left[ \theta_i k_i(\theta) - U_i(\theta) \right]
	\\ &= \sum_i \mathbb{E}_{\theta_i} \left[ \theta_i \bar{k}_i(\theta_i) - \bar{U}_i(\theta_i) \right]
	\\  &= ...
	\\ &= \sum_i \left[ \mathbb{E}_{\theta_i} \left[ \bar{k}_i(\theta_i) \left( \theta_i - \frac{1-F_i(\theta_i)}{\phi_i(\theta_i)} \right) \right] - \bar{U}_i(0)\right]
	\\ &= \sum_i \left[ \mathbb{E}_{\theta_i} \bar{k}_i(\theta_i) \alert{VS_i(\theta_i)} - \bar{U}_i(0) \right]
\end{align*}
\end{frame}


\begin{frame}{Optimal Auction}
\begin{itemize}
	\item As before, set $\bar{U}_i(0) = 0$.
	{\footnotesize
		\begin{align*}
			\mathbb{E} U_s &= \sum_i \mathbb{E}_{\theta_i} \bar{k}_i(\theta_i) VS_i(\theta_i)
			\\
			&= \mathbb{E}_{\theta} \sum_i k_i(\theta) VS_i(\theta)
		\end{align*}
	}
	\item Pointwise maximization: for any $\theta$, give the item to $i$ with the highest $VS_i(\theta)$:
	\begin{align*}
		k_i(\theta) = 
		\begin{cases}
			1 & \text{ if } i = \arg \max_j VS_j(\theta)
			\\
			0 & \text { otherwise}
		\end{cases}
	\end{align*}
	(break ties as you wish)
\end{itemize}
\end{frame}


\begin{frame}{Optimal Auction: Conclusions}
\begin{itemize}
	\item Naive (pointwise) solution works only if the resulting allocations satisfy monotonicity.
	\begin{itemize}
		\item If they don't: ???
		\item Ironing is even more difficult because of joint constraint on allocations: $\sum_i k_i (\theta) \leq 1$.
	\end{itemize}
	\item Allocations are inefficient:
	\begin{itemize}
		\item Inefficient withholding when $\theta_i > 0$ but $VS_i < 0$ (and $i \in \arg \max_j VS_j$).
		\item $VS_i$ depend on respective distr-ns of $\theta_i$'s -- asymmetric players are treated asymmetrically.
	\end{itemize}
	\item In \textbf{symmetric} case, the optimal auction can be implemented as one of standard formats (FPA, SPA, APA, Dutch, English) with reserve price.
\end{itemize}
\end{frame}


\begin{frame}{Optimal Contests}
\begin{itemize}
	\item Related topic: optimal contests.
	\begin{itemize}
		\item $N$ contestants exert effort, have private abilities.
		\item Designer's goal: maximize total effort (e.g. maximize the amount of science that competing reserach teams produce).
		\item How should designer choose size and number of prizes; winning rules etc? 
	\end{itemize}
	\item Will not cover in this class.
\end{itemize}
\end{frame}


\end{document}