%%% License: Creative Commons Attribution Share Alike 4.0 (see https://creativecommons.org/licenses/by-sa/4.0/)

\documentclass[english,10pt
,aspectratio=169
%,handout
%%%%%%,notes
]{beamer}
%%% License: Creative Commons Attribution Share Alike 4.0 (see https://creativecommons.org/licenses/by-sa/4.0/)

\DeclareGraphicsExtensions{.eps, .pdf,.png,.jpg,.mps,}
\usetheme{reMedian}
\usepackage{parskip}
\makeatother

\renewcommand{\baselinestretch}{1.1} 

\usepackage{amsmath, amssymb, amsfonts, amsthm}
\usepackage{enumerate}
%\usepackage{enumitem}
\usepackage{hyperref}
\usepackage{url}
\usepackage{bbm}
\usepackage{color}

\usepackage{tikz}
\usepackage{tikzscale}
\newcommand*\circled[1]{\tikz[baseline=(char.base)]{
		\node[shape=circle,draw, inner sep=-20pt] (char) {#1};}}
\usetikzlibrary{automata,positioning}
\usetikzlibrary{decorations.pathreplacing}
\usepackage{pgfplots}
\usepgfplotslibrary{fillbetween}
\usepackage{graphicx}

\usepackage{setspace}
\thinmuskip=1mu
\medmuskip=1mu 
\thickmuskip=1mu 


\usecolortheme{default}
\usepackage{verbatim}
\usepackage[normalem]{ulem}

\usepackage{apptools}
\AtAppendix{
	\setbeamertemplate{frame numbering}[none]
}
\usepackage{natbib}


% red strikeout
\newcommand\soutred{\bgroup\markoverwith
	{\textcolor{red}{\rule[0.55ex]{2pt}{0.8pt}}}\ULon}



% To use LyX frames from old version:
\def\lyxframeend{} % In case there is a superfluous frame end
\long\def\lyxframe#1{\@lyxframe#1\@lyxframestop}%
\def\@lyxframe{\@ifnextchar<{\@@lyxframe}{\@@lyxframe<*>}}%
\def\@@lyxframe<#1>{\@ifnextchar[{\@@@lyxframe<#1>}{\@@@lyxframe<#1>[]}}
\def\@@@lyxframe<#1>[{\@ifnextchar<{\@@@@@lyxframe<#1>[}{\@@@@lyxframe<#1>[<*>][}}
\def\@@@@@lyxframe<#1>[#2]{\@ifnextchar[{\@@@@lyxframe<#1>[#2]}{\@@@@lyxframe<#1>[#2][]}}
\long\def\@@@@lyxframe<#1>[#2][#3]#4\@lyxframestop#5\lyxframeend{%
	\frame<#1>[#2][#3]{\frametitle{#4}#5}}


\title{Mechanism Design}

\subtitle{2: Implementability of arbitrary s.c.f.s}

\author{Egor Starkov}

\date{K{\o}benhavns Unversitet \\
	Fall 2022}


\begin{document}
	\AtBeginSection[]{
		\frame{
			\frametitle{This slide deck:}
			\tableofcontents[currentsection,currentsubsection]
	}}
	\frame[plain]{\titlepage}



\section{Testing Implementability}

\begin{frame}{Testing Implementability}
	\begin{exampleblock}{}
		How can we check whether a given $f(\theta)$ is implementable?
	\end{exampleblock} 
	\begin{itemize}
		\item We have seen an answer for the Euclidean setting. What about more general settings?
	\end{itemize}
\end{frame}


%\begin{frame}{DSIC: Weak Preference Reversal Property}
%	\begin{itemize}
%		\item \textbf{Answer 1}: \structure{revelation principle}. (Applies to any \alert{general} setting.)
%		Construct a DRM and check all players' \alert{IC conditions}; s.c.f. is implementable iff it satisfies them:
%		$$ u_{i}(f(\theta_{i}', \theta_{-i}), \theta_{i}') \geq u_{i}(f(\theta_{i}'', \theta_{-i}), \theta_{i}') \text{ for all } i,\theta_i',\theta_i'',\theta_{-i}.$$
%		
%		\item Note that they imply the following \structure{preference reversal} condition: for all $i,\theta_i',\theta_i'',\theta_{-i}$,
%		\begin{align*}
%			u_{i}(f(\theta_{i}', \theta_{-i}), \theta_{i}') &\geq u_{i}(f(\theta_{i}'', \theta_{-i}), \theta_{i}'),
%			\\
%			u_{i}(f(\theta_{i}', \theta_{-i}), \theta_{i}'') &\leq u_{i}(f(\theta_{i}'', \theta_{-i}), \theta_{i}'').
%		\end{align*}
%		\item $i$'s preference between $f(\theta_{i}', \theta_{-i})$ and $f(\theta_{i}'', \theta_{-i})$ should flip when his type changes from $\theta_i'$ to $\theta_i''$ for $f$ to be DSIC. In other words:
%		\item ``To each their own'': different types should get their most preferred option among the available ones.
%	\end{itemize}
%\end{frame}


\begin{frame}{Monotonicity: Euclidean setting}
	\begin{itemize}
		\item \textbf{Reminder}: \structure{monotonicity} for Euclidean problems.
		\begin{itemize}
			\item Note: only the players' preferences are required to be linear for this to hold; the principal can have non-linear prefs. 
		\end{itemize}
		\item In a \alert{Euclidean} setting, $k(\theta)$ is implementable only if it is monotone.
		\item Turns out, this is a sharp characterization: \\
		if $k(\theta)$ is monotone, there exist transfers $t$ such that $\Gamma = (\Theta, (k,t))$ is DSIC.
		\begin{itemize}
			\item Monotonicity may require $k(\theta)$ to be either weakly increasing, or weakly decreasing, depending on the problem.
			\item To prove: use the relevant ERP to construct all transfers; can then show that the resulting mechanism is DSIC/BIC as needed.
		\end{itemize}
	\end{itemize}
\end{frame}


\begin{frame}{Monotonicity: quasilinear setting}
	Extension to \alert{quasilinear} setting:
	\begin{exampleblock}{Definition (weak monotonicity)}
		Allocation $k$ is \alert{weakly monotone} if for all $i,\theta_i',\theta_i'',\theta_{-i}$:
		\begin{equation*}
			v_{i}(k(\structure{\theta_{i}'}, \theta_{-i}), \structure{\theta_{i}'}) - 
			v_{i}(k(\alert{\theta_{i}''}, \theta_{-i}), \structure{\theta_{i}'}) 
			\geq 
			v_{i}(k(\structure{\theta_{i}'}, \theta_{-i}), \alert{\theta_{i}''}) - 
			v_{i}(k(\alert{\theta_{i}''}, \theta_{-i}), \alert{\theta_{i}''}) 
		\end{equation*}
	\end{exampleblock}
	\begin{theorem}[Necessity of weak monotonicity in qlin setting]
		In the quasilinear setting: if $k$ is DSIC then $k$ is weakly monotone.
	\end{theorem}
	So $k$ must be weakly monotone to be implementable.\\ 
	But weak monotonicity does not guarantee implementability. \\
	But we can strengthen this...
\end{frame}


\begin{frame}{Monotonicity: quasilinear setting}
	\begin{exampleblock}{Definition (cyclical monotonicity)}
		Allocation $k$ is \alert{cyclically monotone} if for all $i,\theta_{-i}$, and all sequences $(\theta_i^1,\theta_i^2,...,\theta_i^M)\in \Theta_i^M$ of arbitrary length $M$ s.t. $\theta_i^M=\theta_i^1$, the following holds:
		\begin{equation*}
			\sum_{m=1}^{M-1}
			\left[
			v_{i}(k(\alert{\theta_{i}^m}, \theta_{-i}), \alert{\theta_{i}^{m+1}}) - 
			v_{i}(k(\alert{\theta_{i}^{m}}, \theta_{-i}), \alert{\theta_{i}^m}) 
			\right] 
			\leq 0
		\end{equation*}
	\end{exampleblock}
	\begin{theorem}[Rochet, 1987]
		In a quasilinear setting: $k$ is DSIC if and only if $k$ is cyclically monotone.
	\end{theorem}
	\textbf{Note}: ``Weak monotonicity'' = ``cyclical monotonicity for $M=3$''.
	See B{\"o}rgers, ch.5.3-5.4 for proofs or references to proofs (for $N=1$). See rest of ch.5 for other kinds of monotonicity for quasilinear setting.
\end{frame}


\begin{frame}{Monotonicity: general setting (1)}
	\textbf{Without transfers}, interesting results come up... \\
	\textbf{Assume} $X$ is finite and $\Theta_i$ are rich enough to contain all possible preferences over $X$ for all $i$.
	\begin{exampleblock}{Definition (outcome monotonicity)}
		In a general setting, outcome $x$ is \alert{monotone} if for all $\theta',\theta'' \in \Theta$ the following holds:
		\begin{itemize}
			\item if for all $i$ and all $x' \in X$ s.t. $u_i(x(\theta'),\theta') \geq u_i(x',\theta')$ it holds that $u_i(x(\theta'),\theta'') \geq u_i(x',\theta'')$,
			\item then $x(\theta'')=x(\theta')$.
		\end{itemize}
	\end{exampleblock}
	In words, if under $\theta''$ everyone likes $x(\theta')$ more than under $\theta'$, then we give $x(\theta')$ under $\theta''$.
	\begin{theorem}[Necessity of monotonicity in general setting]
		In the general setting: if $x$ is DSIC and $x(\Theta)=X$ then $x$ is monotone.
	\end{theorem}
	(This is not THE interesting part yet. The next result is.)
\end{frame}


\begin{frame}{Monotonicity: general setting (2)}
	\begin{exampleblock}{Definition (dictatorial s.c.f.)}
		S.c.f. $f$ is \alert{dictatorial} if there exists $i \in N$ s.t. for all type profiles $\theta$:
		$$ f(\theta) \in \arg \max_x u_i(x,\theta_i)$$
	\end{exampleblock}
	\begin{theorem}[Gibbard-Satterthwaite, 1973,'75]
		In a general setting with $|X|\geq 3$: if $x(\Theta)=X$, then\\
		\centering
		$x$ is \alert{DSIC} if and only if $x$ is \alert{dictatorial}.
	\end{theorem}
	\begin{itemize}
		\item To clarify: $x(\Theta) \equiv \left\{ x \in X \mid \exists \theta: x(\theta) = x \right\}$.
		\item Note: if $x(\Theta) \subset X$, then only preferences over alternatives in $x(\Theta)$ are relevant, and we will still have a dictatorship on $x(\Theta)$. This is something we'll come back to later.
		\item GS theorem is the mechanism design version of Arrow's theorem from social choice.
	\end{itemize}
\end{frame}


\begin{frame}{Monotonicity: general setting (3)}
	\begin{itemize}
		\item The missing link between the two results above is this:
		\begin{theorem}[Monotonicity implies dictatorship]
			In a general setting with $|X|\geq 3$: if $x(\Theta)=X$, then\\
			\centering
			if $x$ is monotone then $x$ is dictatorial.
		\end{theorem}
		\item For a full proof of GS thm, see B{\"o}rgers, ch.8.2, or \href{http://dx.doi.org/10.1016/j.jmateco.2014.09.007}{\uline{Svensson and Reffgen (2014)}}
		\item All three results for the general setting hold with ordinal preferences ($\succsim_i$) too, they do not rely on cardinal utilities $u_i$.
		%\begin{itemize}
		%	\item If anything, the domain assumption ($x(\Theta)=X$) is easier to state with ordinal preferences.
		%\end{itemize}
		\item The GS thm is also extendable to infinite $X$.
	\end{itemize}
\end{frame}


\begin{frame}{Monotonicity: general setting (4)}
	\begin{itemize}
		\item It seems like GS thm is a strong negative result saying ``we can't implement anything unless it's dictatorial!''. But we obviously can: we've seen examples (like VCG). Where is the contradiction?
		\item The source of evil in GS thm is the domain assumption: we often know \emph{something} about some players' preferences, so ``all possible preferences over $X$'' may be too rich for $\Theta_i$.
		\begin{itemize}
			\item We have already seen that once we restrict preferences (e.g. to quasilinear -- assuming ``everyone always likes money'') then there exist non-dictatorial DSIC implementable s.c.f.s $f(\theta)$.
			\item Another common example is the single-peaked preferences. They also allow for non-dictatorial DSIC mechanisms. (B{\"o}rgers, Prop 8.6)
		\end{itemize}
	\end{itemize}
\end{frame}


\end{document}