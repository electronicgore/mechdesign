%%% License: Creative Commons Attribution Share Alike 4.0 (see https://creativecommons.org/licenses/by-sa/4.0/)

\documentclass[english,10pt
,aspectratio=169
%,handout
%%%,notes
]{beamer}
%%% License: Creative Commons Attribution Share Alike 4.0 (see https://creativecommons.org/licenses/by-sa/4.0/)

\DeclareGraphicsExtensions{.eps, .pdf,.png,.jpg,.mps,}
\usetheme{reMedian}
\usepackage{parskip}
\makeatother

\renewcommand{\baselinestretch}{1.1} 

\usepackage{amsmath, amssymb, amsfonts, amsthm}
\usepackage{enumerate}
%\usepackage{enumitem}
\usepackage{hyperref}
\usepackage{url}
\usepackage{bbm}
\usepackage{color}

\usepackage{tikz}
\usepackage{tikzscale}
%\newcommand*\circled[1]{\tikz[baseline=(char.base)]{
%		\node[shape=circle,draw, inner sep=-20pt] (char) {#1};}}
%\usetikzlibrary{automata,positioning}
%\usetikzlibrary{decorations.pathreplacing}
\usepackage{pgfplots}
\usepgfplotslibrary{fillbetween}
\usepackage{graphicx}

\usepackage{setspace}
%\thinmuskip=1mu
%\medmuskip=1mu 
%\thickmuskip=1mu 


%\usecolortheme{default}
\usepackage{verbatim}
\usepackage[normalem]{ulem}

\usepackage{apptools}
\AtAppendix{
	\setbeamertemplate{frame numbering}[none]
}
\usepackage{natbib}


% red strikeout
\newcommand\soutred{\bgroup\markoverwith
	{\textcolor{red}{\rule[0.55ex]{2pt}{0.8pt}}}\ULon}



%% To use LyX frames from old version:
%\def\lyxframeend{} % In case there is a superfluous frame end
%\long\def\lyxframe#1{\@lyxframe#1\@lyxframestop}%
%\def\@lyxframe{\@ifnextchar<{\@@lyxframe}{\@@lyxframe<*>}}%
%\def\@@lyxframe<#1>{\@ifnextchar[{\@@@lyxframe<#1>}{\@@@lyxframe<#1>[]}}
%\def\@@@lyxframe<#1>[{\@ifnextchar<{\@@@@@lyxframe<#1>[}{\@@@@lyxframe<#1>[<*>][}}
%\def\@@@@@lyxframe<#1>[#2]{\@ifnextchar[{\@@@@lyxframe<#1>[#2]}{\@@@@lyxframe<#1>[#2][]}}
%\long\def\@@@@lyxframe<#1>[#2][#3]#4\@lyxframestop#5\lyxframeend{%
%	\frame<#1>[#2][#3]{\frametitle{#4}#5}}


\title{Mechanism Design}

\subtitle{4: Correlated Information}

\author{Egor Starkov}

\date{K{\o}benhavns Unversitet \\
	Fall 2024}


\begin{document}
	\AtBeginSection[]{
		\frame{
			\frametitle{This slide deck:}
			\tableofcontents[currentsection,currentsubsection]
	}}
	\frame[plain]{\titlepage}



\begin{frame}{Independent vs Correlated Information}
\begin{itemize}
	\item So far we always assumed types $\theta_i$ \structure{independently} distributed.
	\begin{itemize}
		\item Not an innocuous assumption.
		\item Very important for results regarding BIC mechanisms (DSIC are not affected).
	\end{itemize}
	\item What if types are \alert{correlated}? Does it hurt or help the designer?
	\begin{itemize}
		\item If $\theta_i$ are correlated then each $i$ has some information about all $\theta_j$.
		\item Can use it to cross-check reports of $\theta_j$.
	\end{itemize}
\end{itemize}
\end{frame}


\section{Perfect correlation}

\begin{frame}{Perfect Correlation}
\begin{itemize}
	\item Consider a modified \structure{quasilinear} setting.
	\item $N$ agents with \structure{perfectly correlated} types: $\theta_i = \omega$ $\forall i$.
	\item Designer does not know $\omega \in \Omega$, only knows the distribution $F(\omega)$ (as usual).
	\item Designer wants to implement some allocation $k(\omega)$.
	\item \textbf{Example:} a common-value auction, where all buyers have the same valuation for the item, but the seller is unaware of this valuation. (E.g., your parents selling your collection of rare pokemon cards on ebay.)
\end{itemize}
\end{frame}
\note{
	Recall that types have dual role as describing
	\begin{itemize}
		\item own preferences
		\item info about others' preferences.
	\end{itemize}
	If we interpret this as trade setting, with types as valuations: differences between this problem and uncorr types is in info about others' prefs, but not in own prefs.
}


\begin{frame}{Perfect Correlation}
\begin{itemize}
	\item Consider the following direct mechanism:
	\begin{itemize}
		\item \structure{If} all agents' reports \structure{agree} ($\hat{\theta}_i=\hat{\theta}_j=\hat{\omega}$ for all $i,j$) then implement $k(\hat{\omega})$ with zero transfers.
		\item \alert{Otherwise} implement any $k \in K$ and set \alert{$t_i = +\infty$} for all $i$.
		\item (If the desired s.c.f. $k(\theta)$ prescribes an allocation to mismatching types, can use that instead.)
	\end{itemize}
	\item This mechanism truthfully implements $k(\omega)$ (BIC? DSIC?).
	\begin{itemize}
		\item Never profitable to deviate alone (then must pay infinity).
		\item There are also many other equilibria apart from truthful one...
		\item[Q:] In truthful eqm, would $t_i = +\infty$ after disagreement interfere with IR? (if we care about IR)
	\end{itemize}
\end{itemize}
\end{frame}


\section{Imperfect correlation}

\begin{frame}{Imperfect correlation: setup}
\begin{itemize}
	\item Still a modified \structure{quasilinear world}.
	\item $N$ agents with private types $\theta_i \in \Theta_i$ ($\Theta_i$ finite).
	\item Types have some joint distribution: $(\theta_1,...,\theta_N) \sim \phi(\Theta)$ (p.d.f.).
	\item Player $i$'s beliefs about $\theta_{-i}$ are derived by Bayes' rule:
	$$\phi(\theta_{-i}|\theta_i) = \frac{\phi(\theta_{-i},\theta_i)}{\sum_{\theta_{-i}' \in \Theta_{-i}} \phi(\theta_{-i}',\theta_i)} $$
	\item Designer only knows the distribution $\phi$, wants to implement some allocation $k(\theta)$.
\end{itemize}
\end{frame}


\begin{frame}{Imperfect correlation: a simple example}
\begin{example}[2x2]
	\begin{center}
		\begin{tabular}{c | c | c |}
			 & H 				& L					\\ \hline
			H	& $\frac{1}{6}$ 	& $\frac{1}{3}$ 	\\ \hline
			L	& $\frac{1}{3}$ 	& $\frac{1}{6}$		\\ \hline
		\end{tabular}
	\end{center}
	\begin{itemize}
		\item 2 agents, 2 types each: $\theta_i \in \{H,L\}$, joint distribution $\phi(\theta_1,\theta_2)$ in the table above.
		\item Each agent thinks $\theta_j = \theta_i$ is twice less likely than otherwise. 
	\end{itemize}
\end{example}
\end{frame}


\begin{frame}{Imperfect correlation}
\begin{itemize}
	\item A similar idea can be used as with perfect correlation.
	\item Now $i$ does not know $\theta_j$ perfectly -- can't use this info to cross-verify.
	\item But can force $i$ to gamble on $\theta_j$.
	%\item Forget about the allocation rule $k$. Focus on gambles over other's types.
\end{itemize}
\end{frame}


\begin{frame}{Cremer-McLean condition}
\begin{itemize}
	\item Write your beliefs $\phi(\theta_{-i}|\theta_i)$ as a vector $\tilde\phi(\theta_i)$ (for each $\theta_{-i}\in\Theta_{-i}$ one entry in the vector).
	\item Cremer-McLean condition: no such vector $\tilde\phi(\theta_i)$ is a linear combination of the other vectors in $\{\tilde\phi(\theta_i'):\;\theta_i\neq\theta_i'\in\Theta_i\}$
\end{itemize}

\begin{definition}[CM condition]
	The distribution $\phi$ satisfies the \structure{CM condition} if there are no agent $i$ with type $\theta_i\in\Theta_i$ and weights $\lambda_i: \Theta_i\setminus\{\theta_i\} \to \mathbb{R}_+$ such that
	\begin{equation*}
		\phi(\theta _{-i}|\theta _i)=\sum_{\theta _i'\in\Theta_i\setminus\{\theta_i\}}\lambda_i(\theta_i')\phi(\theta_{-i}|\theta_i')\qquad\text{for all }\theta_{-i}\in\Theta_{-i}.
	\end{equation*}
\end{definition}
\end{frame}


\begin{frame}{Cremer-McLean condition}
\begin{itemize}
	%TODO: maybe add a note that this is equivalent to the prior distr-n matrix having full rank IF we want CM cond in a 2-player game for both players. In a 2x3 it can still hold for the two-type player. In >2-plr game prior distr-n is no longer a matrix.
	\item Stack vectors $\tilde\phi(\theta_i)$ for each type $\theta_i$ into a matrix.
	\item The condition holds if this matrix has full rank.
	\item In particular, every type $\theta_i$ must have its own distinct belief about the distribution of $\theta_{-i}$ (but the condition is stronger than this).
	\item Does the condition hold in the 2x2 example?
	\item How about in the following:
	\begin{example}[2x3]
		\begin{center}
			\begin{tabular}{c | c | c | c |}
					& H 				& M				& L					\\ \hline
				H	& $\frac{1}{12}$ 	& $\frac{1}{4}$ & $\frac{1}{6}$ 	\\ \hline
				L	& $\frac{1}{6}$ 	& $\frac{1}{4}$ & $\frac{1}{12}$		\\ \hline
			\end{tabular}
		\end{center}
	\end{example}
\end{itemize}
\end{frame}


\begin{frame}{Cremer-McLean result}
\begin{theorem}[Cremer-McLean]
	If $\phi$ satisfies the CM condition, then for any mechanism $(k,t)$ there is a direct mechanism $(k,t')$ such that
	\begin{itemize}
		\item $(k,t')$ is \alert{BIC},
		\item both mechanisms have the same allocation $k$,
		\item both mechanisms have the same interim expected payoffs: $\forall i,\theta_i$,
	\end{itemize}
	\begin{equation*}
		%\sum_{\theta _{-i}\in\Theta _{-i}}t_i(\theta _i,\theta _{-i})\phi(\theta _{-i}|\theta _i)=  \sum_{\theta _{-i}\in\Theta _{-i}}t_i'(\theta _i,\theta _{-i})\phi(\theta _{-i}|\theta _i)
		\mathbb{E}_{\theta_{-i}} \left[ t_i(\theta) | \theta_i \right] = \mathbb{E}_{\theta_{-i}} \left[ t_i'(\theta) | \theta_i \right]
	\end{equation*}
\end{theorem}
\begin{itemize}
	\item Holds for \structure{any} mechanism $(k,t)$ -- IC not required!
	\item Meaning \structure{any} allocation rule \structure{$k$ is BIC} if $\phi$ satisfies CM condition!
\end{itemize}
\end{frame}


\begin{frame}{Proof Idea}
Forget about the allocation \(k\) and think about how to elicit private information truthfully.
\begin{example}[Information elicitation]
	\begin{itemize}
		\item State of the world $\omega \in \Omega$; expert knows true distribution $\pi$ of states (but not the state); designer knows nothing.
		\item How to extract information about $\pi$ from the expert?
		\item Consider the following scheme:
		\begin{itemize}
			\item the expert announces a probability distribution $\hat{\pi}$;
			\item when state $\omega$ realizes, the expert is paid $\log(\hat{\pi}(\omega))$
		\end{itemize}
		{\footnotesize
			\begin{align*}
				\text{Expert's problem: } & \max_{\hat{\pi}}\sum_{\omega\in\Omega} \pi(\omega) log(\hat{\pi}(\omega))
				\\
				&\text{subject to } \sum_{\omega\in\Omega} \hat{\pi}(\omega)=1.
			\end{align*}
			\vspace{-1ex}
		}
		\item Solve using Lagrange method to see that truthful reporting is optimal.
	\end{itemize}
\end{example}
\end{frame}


\begin{frame}{Proof Idea}
	\begin{block}{Proof idea}
		\begin{itemize}
			\item In our mechanism design problem, consider transfers
			$$t'_i(\theta_i,\theta_{-i}) = t_i(\theta_i,\theta_{-i}) + C_{i,1} - C_{i,2} \log (\phi(\theta_{-i}|\theta_i)).$$
			\item If $C_{i,2}$ is large enough, $i$'s incentives are dominated by the need to report $\phi(\theta_{-i}|\theta_i)$ correctly (rather than desire to get best $k(\theta)$).
			\item So set $C_{i,2}$ large, then use $C_{i,1}$ to adjust the averages as required and voila. ``$\square $''
		\end{itemize}
	\end{block}
\begin{itemize}
	\item The above is \alert{not a complete proof}, since we would actually want $C_{i,1}$ to depend on $\theta_i$ to get $\mathbb{E}_{\theta_{-i}} \left[ t_i(\theta) | \theta_i \right] = \mathbb{E}_{\theta_{-i}} \left[ t_i'(\theta) | \theta_i \right]$ $\forall i,\theta_i$. Otherwise we only have $\mathbb{E}_{\theta} \left[ t_i(\theta) \right] = \mathbb{E}_{\theta} \left[ t_i'(\theta) \right]$ $\forall i$.
	\begin{itemize}
		\item But making $C_{i,1}(\theta_i)$ depend on $\theta_i$ affects reporting incentives...
	\end{itemize}
	\item For full proof (and construction) see B{\"o}rgers ch6.4 or \href{https://www.jstor.org/stable/1913096}{\uline{Cremer \& McLean (1988)}}.
\end{itemize}
\end{frame}


\begin{frame}{Conclusion}
\begin{itemize}
	\item Cremer-McLean result is a VERY powerful tool to implement literally anything under correlated information.
	\item \textbf{Issues:}
	\item Strong-ish condition on $\phi$ --
	\begin{itemize}
		\item this method cannot extract private information which does not affect $i$'s belief about $\theta_{-i}$.
		\item But we can use it as a first step to extract some info, then proceed as before.
	\end{itemize}
	\item With weak correlation $C_{i,2}$ can be HUGE,
	\begin{itemize}
		\item leading to extremely large (positive and negative) $t_i(\theta)$.
		\item Not good if want \emph{ex post} IR, \emph{ex post} BB, and/or \emph{limited liability} ($t_i(\theta)\leq 0$ -- reasonable requirement in some settings, close to ex post IR).
	\end{itemize}
\end{itemize}
\end{frame}





\begin{frame}{Correlated information: Conclusion}
\begin{itemize}
	\item \structure{Correlated information} can be \alert{\textbf{very} easily exploited}
	\begin{itemize}
		\item Make the players snitch about each other's types! %TODO
		\item We'll see an example later where if players' preferences conflict, we can set them against each other (divide et impera!).
	\end{itemize}
	\item Of course there are always issues that can invalidate our analysis:
	\begin{itemize}
		\item Risk aversion, collusion, competition, limited liability, budget constraints...
		\item Designer may not know the exact beliefs that every type has. See \href{https://doi.org/10.1016/j.jet.2020.105088}{\uline{Lopomo, Rigotti, and Shannon (2020)}} for analysis of this case and references to papers dealing with the above cases.
		\item B{\"o}rgers ch.10 takes a more general approach to designer's uncertainty about players' beliefs (``Robust Mechanism Design'')
	\end{itemize}
\end{itemize}
\end{frame}


\end{document}