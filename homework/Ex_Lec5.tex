%%% License: Creative Commons Attribution Share Alike 4.0 (see https://creativecommons.org/licenses/by-sa/4.0/)


%%%%%%%%%%%%%%%%%%%%%%%%%%%%%%%%%%%%%%%%%

%----------------------------------------------------------------------------------------
%	PACKAGES AND OTHER DOCUMENT CONFIGURATIONS
%----------------------------------------------------------------------------------------

\documentclass{article}

\usepackage{amssymb}
\usepackage{enumitem}
\usepackage[usenames,dvipsnames]{color}
\usepackage{fancyhdr} % Required for custom headers
\usepackage{lastpage} % Required to determine the last page for the footer
\usepackage{extramarks} % Required for headers and footers
\usepackage[usenames,dvipsnames]{color} % Required for custom colors
\usepackage{graphicx} % Required to insert images
\usepackage{listings} % Required for insertion of code
\usepackage{courier} % Required for the courier font
\usepackage[table]{xcolor}
\usepackage{amsfonts,amsmath,amsthm,parskip,setspace,url}
\usepackage[section]{placeins}
\usepackage[a4paper]{geometry}
\usepackage[USenglish]{babel}
\usepackage[utf8]{inputenc}


% Margins
\topmargin=-0.45in
\evensidemargin=0in
\oddsidemargin=0in
\textwidth=6.5in
\textheight=9.0in
\headsep=0.6in

\linespread{1.1} % Line spacing

%----------------------------------------------------------------------------------------
%	DOCUMENT STRUCTURE COMMANDS
%	Skip this unless you know what you're doing
%----------------------------------------------------------------------------------------

% Header and footer for when a page split occurs within a problem environment
\newcommand{\enterProblemHeader}[1]{
\nobreak\extramarks{#1}{#1 continued on next page\ldots}\nobreak
\nobreak\extramarks{#1 (continued)}{#1 continued on next page\ldots}\nobreak
}

% Header and footer for when a page split occurs between problem environments
\newcommand{\exitProblemHeader}[1]{
\nobreak\extramarks{#1 (continued)}{#1 continued on next page\ldots}\nobreak
\nobreak\extramarks{#1}{}\nobreak
}

\setcounter{secnumdepth}{0} % Removes default section numbers
\newcounter{homeworkProblemCounter} % Creates a counter to keep track of the number of problems

\newcommand{\homeworkProblemName}{}
\newenvironment{ex}[1][Problem \arabic{homeworkProblemCounter}]{ % Makes a new environment called homeworkProblem which takes 1 argument (custom name) but the default is "Problem #"
\stepcounter{homeworkProblemCounter} % Increase counter for number of problems
\renewcommand{\homeworkProblemName}{#1} % Assign \homeworkProblemName the name of the problem
\section{\homeworkProblemName} % Make a section in the document with the custom problem count
%\enterProblemHeader{\homeworkProblemName} % Header and footer within the environment
}{
%\exitProblemHeader{\homeworkProblemName} % Header and footer after the environment
}

\newcommand{\problemAnswer}[1]{ % Defines the problem answer command with the content as the only argument
\noindent\framebox[\columnwidth][c]{\begin{minipage}{0.98\columnwidth}#1\end{minipage}} % Makes the box around the problem answer and puts the content inside
}

\newcommand{\homeworkSectionName}{}
\newenvironment{homeworkSection}[1]{ % New environment for sections within homework problems, takes 1 argument - the name of the section
\renewcommand{\homeworkSectionName}{#1} % Assign \homeworkSectionName to the name of the section from the environment argument
\subsection{\homeworkSectionName} % Make a subsection with the custom name of the subsection
%\enterProblemHeader{\homeworkProblemName\ [\homeworkSectionName]} % Header and footer within the environment
}{
%\enterProblemHeader{\homeworkProblemName} % Header and footer after the environment
}

\newif\ifsolutions

%----------------------------------------------------------------------------------------
%----------------------------------------------------------------------------------------
%----------------------------------------------------------------------------------------
% Set up the header and footer
\pagestyle{fancy}
\lhead[c]{\textbf{{\color[rgb]{.5,0,0} K{\o}benhavns\\Universitet }} \firstxmark} % Top left header
\chead{\textbf{{\color[rgb]{.5,0,0} \Class }}\\ \hmwkTitle  } % Top center head
\rhead{\instructor \\ \theprofessor} % Top right header
\lfoot{\lastxmark} % Bottom left footer
\cfoot{} % Bottom center footer
\rfoot{Page\ \thepage\ of\ \protect\pageref{LastPage}} % Bottom right footer
\renewcommand\headrulewidth{0.4pt} % Size of the header rule
\renewcommand\footrulewidth{0.4pt} % Size of the footer rule

\setlength\parindent{0pt} % Removes all indentation from paragraphs







%----------------------------------------------------------------------------------------
%	NAME AND CLASS SECTION
%----------------------------------------------------------------------------------------

\newcommand{\hmwkTitle}{Exercises: Lec 5} % Assignment title
\newcommand{\Class}{Mechanism Design} % Course/class
\newcommand{\instructor}{Fall 2019} % TA
\newcommand{\theprofessor}{Prof. Egor Starkov} % Professor

%\theoremstyle{definition} \newtheorem{ex}{\textbf{\Large{Exercise & #}\\}}
\setlength{\parskip}{0 pt}




















%%%%%%%%%%%%%%%%%%%%%%%%%%%%%%%%%%%%%%%%%%%%%%%%%%%%%%%%%%%%%%%%%%%%%%%%%%%%%%%%%%%%%%
\solutionsfalse
%\solutionstrue
%%%%%%%%%%%%%%%%%%%%%%%%%%%%%%%%%%%%%%%%%%%%%%%%%%%%%%%%%%%%%%%%%%%%%%%%%%%%%%%%%%%%%%


\begin{document}
	
These exercises are for your own practice and are not to be handed in. Some exercises are open ended in that they may not have a unique correct answer. If you think there is a typo in the problem, attempt to amend it and proceed as best you can.

%%------------------------------------------------------------------------------------------------

\begin{ex}[Review Questions]
	\begin{itemize}
		\item Consider the results and mechanisms we have seen in the course so far. Which of them rely on the independence of players' types?
		\item How can the designer use correlation of players' types to elicit their private information?
		\item What assumption on the distribution of types is needed to for Cremer-McLean transfers to be effective at eliciting information?
		\item How can perfectly correlated information be elicited from the agents without the use of transfers?
	\end{itemize}
	
\end{ex}



%%------------------------------------------------------------------------------------------------

\begin{ex}
	Consider the 2x2 example from lecture. There are two players, $i=1,2$. Each of them has one of two types, $\theta_i \in \{H,L\}$. The joint distribution of types is given by
	\begin{center}
		\begin{tabular}{c | c | c |}
			& H 				& L					\\ \hline
			H	& $\frac{1}{6}$ 	& $\frac{1}{3}$ 	\\ \hline
			L	& $\frac{1}{3}$ 	& $\frac{1}{6}$		\\ \hline
		\end{tabular}
	\end{center}
	Both players have quasilinear utilities.
	
	First explore the problem of information elicitation (without the need to support any underlying allocation). 
	\begin{enumerate}
		\item Compute the players' interim beliefs $\phi(\theta_{-i} | \theta_i)$.
		\item Compute the truth-revealing transfers $\hat{t}_i(\theta) = -\ln(\phi(\theta_{-i} | \theta_i))$.
		\item Verify that a direct mechanism in which each player reports their type $\theta_i$ and pays $\hat{t}_i(\theta)$ is BIC (when coupled with some constant allocation rule $k$).
	\end{enumerate}

	Now suppose that the society of these two individuals chooses whether to adopt a new bank holiday called ``National Equality Day'', so the ``real outcome'' is $k \in \{1,0\}$ (where $1$ means bank holiday and $0$ means none). The holiday should only be adopted if all citizens are, in fact, equal, i.e. the desired allocation is $\tilde{k}(\theta) = \mathbb{I} \{\theta_1 = \theta_2\}$. Each citizen receives utility 1 if the holiday is adopted and 0 otherwise.
	\begin{enumerate}[resume]
		\item Is $\tilde{k}$ efficient?
		\item Show that $\tilde{k}$ cannot be sustained without transfers, i.e. that a mechanism $(\tilde{k},t)$ with $t(\theta)=0$ for all $\theta$ is not BIC.
		\item Consider transfers $t_i(\theta_i,\theta_{-i}) = C_1 + C_2 \hat{t} (\theta_i,\theta_{-i})$. Derive conditions on values of $C_1,C_2$ for which a direct mechanism $(\tilde{k},t)$ is BIC.
		\item Derive conditions on values of $C_1,C_2$ for which a direct mechanism $(\tilde{k},t)$ is interim IR and ex ante BB.
		\item Give an example of a BIC, interim IR and ex ante BB mechanism that implements $\tilde{k}$.
	\end{enumerate}

\ifsolutions
\section*{Solution}
...
\fi
\end{ex}



%%------------------------------------------------------------------------------------------------

\begin{ex}
	Construct a perfectly revealing equilibrium in the Battaglini(2003) model with agents' biases $b_1 = (1, -3)$ and $b_2 = (-4, 2)$. (Describe agents' reporting strategies as functions of state and the principal's decision strategy as a function of reports.)
	
	\ifsolutions
	\section*{Solution}
	...
	\fi
\end{ex}



%%%------------------------------------------------------------------------------------------------
%
%\begin{ex}
%	Consider the optimal auction problem in which there are two bidders whose valuations $v_i \in [0,1]$ are uniformly distributed but perfectly correlated, i.e. $v_1 = v_2$ with probability 1.  Construct a DSIC mechanism with the following properties:
%	\begin{enumerate}
%		\item Bidder 2 wins the object regardless of the type profile.
%		\item Both bidders earn zero utility at every type profile.
%	\end{enumerate}
%	
%	\ifsolutions
%	\section*{Solution}
%	Consider the following allocation:
%	\[q_1(v_1,v_2)=0\]
%	\[q_2(v_1,v_2)=\left\{\begin{array}{cc} 0 & \text{ if }\quad v_2\neq v_1\\
%	1 & \quad\text{otherwise}\quad
%	
%	\end{array}
%	\right.\]
%	\[t_1(v_1,v_2)=0\]
%	\[t_2(v_1,v_2)=\left\{\begin{array}{cc} 0 & \text{if}\quad v_2\neq v_1\\
%	v_1 & \quad\text{otherwise}\quad\end{array}\right.\]
%	Note that for bidder 1 it is weakly dominant to say his valuation: he never gets the good so he might as well say $v_1$. For player 2, we have exactly the same situation: given the message sent by agent 1, he is indifferent between saying the truth and not, so telling the truth is weakly dominant. Agent 2 gets the good always and both bidders have 0 utility at every profile.
%	\fi
%\end{ex}

\end{document}
