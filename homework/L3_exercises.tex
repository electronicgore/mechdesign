%%% License: Creative Commons Attribution Share Alike 4.0 (see https://creativecommons.org/licenses/by-sa/4.0/)


%%%%%%%%%%%%%%%%%%%%%%%%%%%%%%%%%%%%%%%%%

%----------------------------------------------------------------------------------------
%	PACKAGES AND OTHER DOCUMENT CONFIGURATIONS
%----------------------------------------------------------------------------------------

\documentclass[a4paper]{article}

\usepackage{amssymb}
\usepackage{enumerate}
\usepackage[usenames,dvipsnames]{color}
\usepackage{fancyhdr} % Required for custom headers
\usepackage{lastpage} % Required to determine the last page for the footer
\usepackage{extramarks} % Required for headers and footers
\usepackage[usenames,dvipsnames]{color} % Required for custom colors
\usepackage{graphicx} % Required to insert images
\usepackage{listings} % Required for insertion of code
\usepackage{courier} % Required for the courier font
\usepackage[table]{xcolor}
\usepackage{amsfonts,amsmath,amsthm,parskip,setspace,url}
\usepackage[section]{placeins}
\usepackage[a4paper]{geometry}
\usepackage[USenglish]{babel}
\usepackage[utf8]{inputenc}


% Margins
\topmargin=-0.45in
\evensidemargin=0in
\oddsidemargin=0in
\textwidth=6.5in
\textheight=9.0in
\headsep=0.6in

\linespread{1.1} % Line spacing



%----------------------------------------------------------------------------------------
%   FORMATTING
%----------------------------------------------------------------------------------------
% Set up the header and footer
\pagestyle{fancy}
\lhead[c]{\textbf{{\color[rgb]{.5,0,0} K{\o}benhavns\\Universitet }}} % Top left header
\chead{\textbf{{\color[rgb]{.5,0,0} \Class }}\\ \hmwkTitle  } % Top center head
\rhead{\instructor \\ \theprofessor} % Top right header
\lfoot{\lastxmark} % Bottom left footer
\cfoot{} % Bottom center footer
\rfoot{Page\ \thepage\ of\ \protect\pageref{LastPage}} % Bottom right footer
\renewcommand\headrulewidth{0.4pt} % Size of the header rule
\renewcommand\footrulewidth{0.4pt} % Size of the footer rule


% Other formatting stuff
%\setlength\parindent{12pt}
\setlength{\parskip}{5 pt}
%\theoremstyle{definition} \newtheorem{ex}{\textbf{\Large{Exercise & #}\\}}
\usepackage{titlesec}
\titleformat{\section}[hang]{\normalfont\bfseries\Large}{Problem \thesection:}{0.5em}{}




%----------------------------------------------------------------------------------------
%	NAME AND CLASS SECTION
%----------------------------------------------------------------------------------------
\newcommand{\hmwkTitle}{Exercises for Lecture 3 (M2)} % Assignment title
\newcommand{\Class}{Mechanism Design} % Course/class
\newcommand{\instructor}{Fall 2023} % TA
\newcommand{\theprofessor}{Prof. Egor Starkov} % Professor




%----------------------------------------------------------------------------------------
%   SOLUTIONS
%----------------------------------------------------------------------------------------
\newif\ifsolutions
\solutionstrue




\begin{document}

\begin{center}
		\LARGE\textbf{Exercises for Lecture 3 (M2):\\ Efficient Mechanisms}
\end{center}



\section{Combinatorial Auction}
% Bichler, example 7.3.1

There are two items, $l=A,B$ (e.g., two adjacent plots of land), being auctioned off among two bidders, $i=1,2$. Each bidder's valuations for the two items and the bundle consisting of both items are $(v_{i,A}, v_{i,B}, v_{i,AB}) \sim F_i$, distributed independently across bidders. Their value of not getting either item is zero, and their utility functions are quasilinear in payments. 

The auctioneer runs a VCG auction to allocate the items. In this auction, each bidder reports a bid for every combination of items. The resulting allocation is determined so as to maximize the bidders' welfare according to their reported valuations, and the payments are determined using the VCG payment rule.

The bidders' \emph{realized} valuations are described by Table \ref{tab:combac} (note that these valuations are still bidders' private information).
\begin{table}[h]
	\begin{center}
		\begin{tabular}{l|c|c|c}
			& $\{A\}$ & $\{B\}$ & $\{A,B\}$  \\ \hline
			Bidder 1 & $20$ & $11$ & $33$ \\
			Bidder 2 & $14$ & $14$ & $29$
		\end{tabular}
	\end{center}
	\caption{Realized valuations for the combinatorial auction problem}
	\label{tab:combac}
\end{table}

Calculate the resulting allocation and payments.

\ifsolutions
\section*{Solution}
	First, note that the VCG mechanism implements the efficient allocation in dominant strategies. This means there exists an equilibrium of the auction in which bidders report all their valuations truthfully. In what follows, we look at this equilibrium.
	
	Given that all valuations for all items are positive (including marginal valuations within the bundle), the efficient allocation must be non-wasteful, i.e., both items will be allocated to one of the bidders. If we let $(k_1,k_2)$ denote the bundles of items that the bidders obtain, then we can write all four of such non-wasteful allocations as: $(\{A\},\{B\})$, $(\{B\},\{A\})$, $(\{A,B\},\{\varnothing\})$, and $(\{\varnothing\},\{A,B\})$. The social welfare associated with each of the four is, respectively, $34$, $25$, $33$, and $29$. Welfare is maximized by awarding item $A$ to bidder 1 and item $B$ to bidder 2, hence this is the allocation selected by the VCG mechanism given bidders' reported types are as in Table \ref{tab:combac}.
	
	To calculate the VCG payments, we need to calculate $k^*(\theta_{-i})$ for both $i$, which is the efficient allocation ignoring bidder $i$ -- in our problem it would just maximize the other bidder's utility (since the designer has no utility function). It is straightforward that given the valuations in Table \ref{tab:combac}, utility of either bidder is maximized by awarding them both items. Hence the VCG payments given the realized valuations are given by
	\begin{align*}
		t_1^{VCG}(\theta) = -v_2(k^*(\theta),\theta_2) + v_2(k^*(\theta_2),\theta_2) = -14+29 = 15 \text{ from bidder 1,}
		\\
		t_2^{VCG}(\theta) = -v_1(k^*(\theta),\theta_1) + v_1(k^*(\theta_1),\theta_1) = -20+33 = 13 \text{ from bidder 2.}
	\end{align*}
	%These can be considered as e.g. B1 paying their bid 20, but getting a Vickrey rebate =34-29=5 bc B1's participation increases the total surplus by 5. Same for B2: pay 14, get a rebate =34-33=1, pay 13 in the end.
\fi 



\section{Efficient public good provision}
	
	This is a more standard problem on VCG mechanism. A version of this problem with numbers has been done in lecture (Moon base example), now do it with letters.
	
	There is a society of $N$ people. They must collectively decide whether to implement a public project (e.g., build a bridge, or pass a tax reform). Let $k \in \{0,1\}$ denote the outcome of this decision: $k=1$ if project is implemented, $k=0$ otherwise. Every person $i$ has some private valuation $\theta_i \in \mathbb{R}$ for the project, positive or negative. Preferences are quasilinear, so $i$'s utility can be written as
	$$
		u_i(x,\theta) = \theta_i k(\theta) - t_i(\theta).
	$$
	Here $x=(k,t)$ stands for some direct mechanism which prescribes outcome $k(\theta)$ and payment profile $t(\theta)$ given profile of reports $\theta$.
	
	\begin{enumerate}
		\item Calculate the efficient allocation rule $k^*(\theta)$.
		\item Calculate the ``efficient-ignoring-$i$'' allocation rule $k^{-i}(\theta_{-i})$.
		\item Calculate VCG transfers $t^{VCG}(\theta)$ that support $k^*$.
		\item Is the mechanism ex post IR?
		\item Is the mechanism ex post budget balanced?
	\end{enumerate}
	
\ifsolutions
\section*{Solution}
	\begin{enumerate}
		\item The efficient allocation $k^*(\theta)$ maximizes $\sum_{i=1}^N \theta_i k(\theta)$, hence 
		$$
			k^*(\theta) = 
			\begin{cases}
				1 & \text{ if } \sum_i \theta_i > 0
				\\
				0 & \text{ if } \sum_i \theta_i \leq 0
			\end{cases}
		$$
		For simplicity, break the tie towards not implementing the project when the society is indifferent.
		
		\item Similarly, $k^{-i}(\theta_{-i})$ maximizes $\sum_{j \neq i} \theta_j k(\theta)$, so 
		$$
		k^{-i}(\theta_{-i}) = 
		\begin{cases}
		1 & \text{ if } \sum_{j \neq i} \theta_j > 0
		\\
		0 & \text{ if } \sum_{j \neq i} \theta_j \leq 0
		\end{cases}
		$$
		
		\item VCG transfers are given by
		\begin{align*}
			t_{i}^{VCG}(\theta) &= -\left(\sum_{j\neq i} \theta_j k^*(\theta_i, \theta_{-i}) \right) + \sum_{j\neq i} \theta_j k^{-i}(\theta_{-i})
			\\
			&= -\left(\sum_{j\neq i} \theta_j \right) \mathbb{I}\left\{ \sum_j \theta_j > 0 \right\} +\left( \sum_{j\neq i} \theta_j \right) \mathbb{I}\left\{ \sum_{j \neq i} \theta_j > 0 \right\}
			\\
			&= \left(\sum_{j\neq i} \theta_j \right) \left[ \mathbb{I}\left\{ \sum_{j \neq i} \theta_j > 0 \right\} - \mathbb{I}\left\{ \sum_j \theta_j > 0 \right\} \right],
		\end{align*}
		where $\mathbb{I}\{\cdot\}$ is an indicator function (takes value $1$ if its argument is \texttt{True} and $0$ if it is \texttt{False}). In words, citizen $i$ only has a non-trivial transfer if they are pivotal in social decision:
		\begin{itemize}
			\item $i$ must pay $\left(\sum_{j\neq i} \theta_j \right)$ to the mechanism if the project is not implemented ($k^*(\theta)=0$), but it would have been implemented without $i$ ($k^*(\theta_{-i})=1$). This can happen if $\theta_i < 0$.
			
			\item $i$ receives a payment of $\left(\sum_{j\neq i} \theta_j \right)$ from the mechanism if the project is implemented ($k^*(\theta)=1$), but it would not have been implemented without $i$ ($k^*(\theta_{-i})=0$).
		\end{itemize}
		
		\item Let $U_i(\theta) \equiv u_i((k^*(\theta),t^{VCG}(\theta)),\theta)$. Bruteforcing over possible combinations of $k^*(\theta)$, $k^{-i}(\theta_{-i})$, and the sign of $\theta_i$, would lead us to the observation that if $\theta_i \leq 0$, $\sum_{j \neq i} \theta_j > 0$, and $\sum_j \theta_j > 0$, then $U_i(\theta) < 0$. Person $i$ does not have to pay anything ($t^{VCG}_i(\theta)=0$) in that case, but the project is implemented and $\theta_i < 0$. If such a type profile occurs, $i$ would wish they did not sign up for the mechanism -- so the mechanism is not ex post IR.
		
		%NOTE: this conclusion regarding BB changes if there is some cost C associated with the public good
		\item If you try to write out the whole surplus and analyze it, you will get a big and difficult-to-analyze expression. A better way to proceed is to analyze $t_i^{VCG}(\theta)$ for each $i$ individually. In particular, $i$'s transfer can be negative only in one of the two following cases:
		\begin{enumerate}
			\item $\sum_{j\neq i} \theta_j < 0$, $\mathbb{I}\left\{ \sum_{j \neq i} \theta_j > 0 \right\}=1$, and $\mathbb{I}\left\{ \sum_j \theta_j > 0 \right\}=0$;
			\item $\sum_{j\neq i} \theta_j > 0$, $\mathbb{I}\left\{ \sum_{j \neq i} \theta_j > 0 \right\}=0$, and $\mathbb{I}\left\{ \sum_j \theta_j > 0 \right\}=1$.
		\end{enumerate}
		You can see that the first two requirements in either case contradict one another. Therefore, $t_i^{VCG}(\theta) \geq 0$ for all $i$ and $\theta$, and thus the VCG mechanism is ex post budget balanced in this problem (but not ``exactly budget balanced'').
	\end{enumerate}
\fi



\section{Collusive mechanism in Cournot duopoly}
%TODO: add story from here: https://arstechnica.com/?p=1890378 and https://arstechnica.com/?p=1892303
Consider a Cournot duopoly with a inverse demand function $P(Q)=1-Q$ where $Q$ is aggregate quantity. Suppose that each firm $i=1,2$ has constant marginal cost $\theta_i$. This marginal cost is drawn uniformly from $[0,\frac{1}{2}]$. The realizations for the two firms are independent. Suppose that the firms observe their cost level, but not their rival’s cost level prior to choosing their quantity.  

\smallskip
Imagine that the two firms are able to collude by committing to a collusive “mechanism” whose outcomes are assignments of a quantity and a transfer payment to each of the two firms as a function of the announcements by the two firms of their cost type.  Let $(k_i(\theta),t_i(\theta))$ denote the output level and transfer assigned to firm $i$ if the announced profile of types (costs) is $\theta = (\theta_1, \theta_2)$.

\medskip
\begin{enumerate}
	\item We can use the VCG mechanism to implement the \emph{profit}-maximizing production decisions in dominant strategies.
	Explain why. (Given that collusion is not usually perceived as an efficient outcome.)
	\item Derive the VCG mechanism for this setting.
	\begin{enumerate}
		\item Find the profit-maximizing output profile $k^*(\theta)$.
		\item Find the output profile $k^{-i}(\theta_{-i})$ that maximizes profit of firm $-i$ given its type $\theta_{-i}$.
		\item Find the VCG transfers and describe the VCG mechanism.
	\end{enumerate}
	\item \label{part:irbb} Argue why the firms would want the collusive mechanism to be or not be budget balanced and/or individually rational for the participants. If it should, argue which notions of IR and BB are the most reasonable to demand in this setting.
	\item Is the VCG mechanism budget balanced? Is it individually rational assuming firms' outside options are zero (i.e., each firm's choice is between participating in the agreement and leaving the industry)?
	\item Now suppose instead that either firm can reject the mechanism's prescription once it has been announced (at ex post stage), in which case firms go back to playing Cournot outcome. In which cases -- i.e., for which realizations of $(\theta_1,\theta_2)$ -- would a firm want to back out of the agreement? Give formal conditions and explain them the best you can.
	
	(Assume that firms are not strategic about this contingency when making their reports to the mechanism, so truthful reporting is still an equilibrium.)
	%\item Suppose that the firms asked you to design a mechanism that is exactly ex ante BB and interim IR (because of or despite your response in part \ref{part:irbb}), and at least BIC. The firms have also revealed their true outside options to be $\underline{U}_i (\theta_i)=\frac{1-\theta_i}{4}$ from not participating in the mechanism. Derive the gVCG mechanism to either fulfill the firms' request, or demonstrate its infeasibility. (Your answer should indicate which of the two it is.)
\end{enumerate}

\ifsolutions
\section*{Solution}
\begin{enumerate}
	\item The objective is maximizing the sum of utilities of mechanism participants (i.e., firms), as it is in the efficient mechanisms. Consumers are not participating in the mechanism, so VCG does not account for their well-being.
	\item Given cost profile $\theta$, the profit-maximizing output profile is given by
	\begin{align*}
		k^*(\theta) = \arg \max_{(k_1,k_2)} \left\{ (k_1 + k_2) \cdot (1 - k_1 - k_2) - k_1 \theta_1 - k_2 \theta_2 \right\}.
	\end{align*}
	Attempting the usual approach yields two first-order conditions that are impossible to satisfy simultaneously (unless $\theta_1 = \theta_2)$.
	%\begin{align*}
	%	1 - 2(k_1^*(\theta) + k_2^*(\theta)) - \theta_1 = 0;
	%	\\
	%	1 - 2(k_1^*(\theta) + k_2^*(\theta)) - \theta_2 = 0.
	%\end{align*}
	However, one can split the problem into two: finding optimal aggregate output $Q = k_1 + k_2$ and allocating it among the two firms. Starting with the latter, it is easy to see that to maximize the sum of profits, it is best to let the firm with the lowest cost $\theta_i$ produce $k_i = Q$ and the other firm produce $k_j = 0$.	
	The optimal $Q$ is then found as a maximizer of $Q\cdot (1-Q) - \theta_i Q$, so in the end,
	\begin{align*}
		k_i^*(\theta) =
		\begin{cases}
			\frac{1-\theta_i}{2} & \text{ if } \theta_i < \theta_j;
			\\
			\frac{1-\theta_i}{4} & \text{ if } \theta_i = \theta_j;
			\\
			0 & \text{ otherwise.}
		\end{cases}
	\end{align*}
	(any other split in case of a tie works as well).
	
	The optimal output $k^{-i}(\theta_{-i})$ that maximizes the sum of profits of mechanism participants except for $i$ -- i.e., maximizes profit of firm $j\neq i$, -- is given by $k^{-i}_i = 0$ for firm $i$ and the monopoly output $k^{-i}_j = \frac{1-\theta_j}{2}$ for firm $j$.
	
	The VCG transfer can then be computed using the standard formula to be
	\begin{align*}
		t_i^{VCG}(\theta) =
		\begin{cases}
			\frac{(1-\theta_j)^2}{4} & \text{ if } \theta_i < \theta_j;
			\\
			\frac{(1-\theta_j)^2}{8}  & \text{ if } \theta_i = \theta_j;
			\\
			0 & \text{ otherwise.}
		\end{cases}
	\end{align*}
	(To clarify: $t_i$ is the payment firm $i$ makes to the mechanism, and it does not include market profits.)
	
	The VCG mechanism is then a direct mechanism characterized by a pair $(k^*,t^{VCG})$.
	
	\item 
	The collusive agreement should be beneficial for both firms, so individual rationality, at least in the interim sense, is definitely a desired property. Whether it should be strengthened to ex post IR depends on whether firms can make some kind of enforceable binding agreement -- if yes then interim IR is fine, otherwise ex post IR is needed to make sure that each firm complies with the mechanism's prescriptions when they are revealed. Enforceability of contracts is usually guaranteed by courts and the legal system -- but collusion is illegal, so going to court is not an option in our case, thus ex post IR is likely a more reasonable requirement.\footnote{A curious scheme, however, was proposed by Francesco Squintani in as-of-yet unpublished manuscript. In this scheme the firms make a collusive agreement like the mechanism we describe, and in addition sign a perfectly legal contract that is very costly for both firms. On equilibrium path both firms follow the informal agreement and ignore the formal contract. If one firm deviates, the other goes to court seeking damages from the deviant for not delivering on the formal contract.}
	
	Budget balance is less obvious. It would be nice if the mechanism was exactly budget balanced, of course. However, it is not the end of the world if it was not. If the mechanism runs a surplus, the firms can invest the resulting money in a joint venture or try to find some other way to put it to use without distorting incentives. If the mechanism is expected to run a deficit, when signing the agreement the firms could make lumpsum contributions to a trust fund which would cover the deficit. In the end, the case can be made in favor of either option.
	
	\item The VCG mechanism always runs a budget surplus: $t_1(\theta)+t_2(\theta) = \frac{(1-\max \theta_i)^2}{4}$, so it is weakly but not exactly budget balanced, ex post and, thus, ex ante.
	
	It is also ex post IR. Firm $i$'s profit under this collusive mechanism is
	\begin{align*}
		\pi_i^{VCG}(\theta) =
		\begin{cases}
			\frac{(1-\theta_i)^2}{4} - \frac{(1-\theta_j)^2}{4} >0 & \text{ if } \theta_i < \theta_j;
			\\
			\frac{(1-\theta_i)^2}{8} - \frac{(1-\theta_j)^2}{8} =0 & \text{ if } \theta_i = \theta_j;
			\\
			0 & \text{ otherwise.}
		\end{cases}
	\end{align*}
	
	\item Firm $i$'s profit under Cournot competition is $\pi_i^C (\theta) = \frac{(1 +\theta_j - 2\theta_i)^2}{9}$. This is assuming that costs are commonly known after being announced by the mechanism. This profit is greater than $\pi_i^{VCG}(\theta)$ if:
	\begin{itemize}
		\item either $\theta_i > \theta_j$,
		\item or $\theta_i < \theta_j$ and $4+13\theta_j^2 - 10 \theta_j - 16 \theta_j \theta_i + 2 \theta_i - 25 \theta_i^2 > 0$.
	\end{itemize}
	The former condition is straightforward: if $i$ has the higher marginal cost then it is told to not produce by the mechanism, which together with zero transfer implies zero profit. Cournot competition, on the other hand, almost always yields positive profit, and is hence a more appealing option.\footnote{The only exception is the case $\theta_i=\frac{1}{2}$, $\theta_j = 0$.}
	Therefore, at least one firm would almost always want to back out of the agreement.
	
	To get a complete picture, it is also interesting to explore the second condition above. It holds if $\theta_i$ is low enough (to see this, either plot it using your favorite software, or figure out which part of the saddle we are looking at with $\theta \in [0,\frac{1}{2}]^2$). I.e., if a firm's costs are very low then it is better off defeating the competitor in a fair fight than paying him off in a mechanism. 
	
	In the end, mechanism participation is only optimal for $i$ if its costs are high but the opponent's costs are even higher.
\end{enumerate}
\fi

%%-----------------------------------------------------------------------------------------------------

\end{document}
