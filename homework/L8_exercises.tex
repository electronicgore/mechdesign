%%% License: Creative Commons Attribution Share Alike 4.0 (see https://creativecommons.org/licenses/by-sa/4.0/)


%%%%%%%%%%%%%%%%%%%%%%%%%%%%%%%%%%%%%%%%%

%----------------------------------------------------------------------------------------
%	PACKAGES AND OTHER DOCUMENT CONFIGURATIONS
%----------------------------------------------------------------------------------------

\documentclass[a4paper]{article}

\usepackage{amssymb}
\usepackage{enumitem}
\usepackage[usenames,dvipsnames]{color}
\usepackage{fancyhdr} % Required for custom headers
\usepackage{lastpage} % Required to determine the last page for the footer
\usepackage{extramarks} % Required for headers and footers
\usepackage[usenames,dvipsnames]{color} % Required for custom colors
\usepackage{graphicx} % Required to insert images
\usepackage{listings} % Required for insertion of code
\usepackage{courier} % Required for the courier font
\usepackage[table]{xcolor}
\usepackage{amsfonts,amsmath,amsthm,parskip,setspace,url}
\usepackage[section]{placeins}
\usepackage[a4paper]{geometry}
\usepackage[USenglish]{babel}
\usepackage[utf8]{inputenc}
\usepackage{tikz}


% Margins
\topmargin=-0.45in
\evensidemargin=0in
\oddsidemargin=0in
\textwidth=6.5in
\textheight=9.0in
\headsep=0.6in

\linespread{1.1} % Line spacing



%----------------------------------------------------------------------------------------
%   FORMATTING
%----------------------------------------------------------------------------------------
% Set up the header and footer
\pagestyle{fancy}
\lhead[c]{\textbf{{\color[rgb]{.5,0,0} K{\o}benhavns\\Universitet }}} % Top left header
\chead{\textbf{{\color[rgb]{.5,0,0} \Class }}\\ \hmwkTitle  } % Top center head
\rhead{\instructor \\ \theprofessor} % Top right header
\lfoot{\lastxmark} % Bottom left footer
\cfoot{} % Bottom center footer
\rfoot{Page\ \thepage\ of\ \protect\pageref{LastPage}} % Bottom right footer
\renewcommand\headrulewidth{0.4pt} % Size of the header rule
\renewcommand\footrulewidth{0.4pt} % Size of the footer rule


% Other formatting stuff
%\setlength\parindent{12pt}
\setlength{\parskip}{5 pt}
%\theoremstyle{definition} \newtheorem{ex}{\textbf{\Large{Exercise & #}\\}}
\usepackage{titlesec}
\titleformat{\section}[hang]{\normalfont\bfseries\Large}{Problem \thesection:}{0.5em}{}




%----------------------------------------------------------------------------------------
%	NAME AND CLASS SECTION
%----------------------------------------------------------------------------------------
\newcommand{\hmwkTitle}{Exercises for Lecture 8 (M4)} % Assignment title
\newcommand{\Class}{Mechanism Design} % Course/class
\newcommand{\instructor}{Fall 2021} % TA
\newcommand{\theprofessor}{Prof. Egor Starkov} % Professor




%----------------------------------------------------------------------------------------
%   SOLUTIONS
%----------------------------------------------------------------------------------------
\newif\ifsolutions
%\solutionstrue




\begin{document}

\begin{center}
		\LARGE\textbf{Exercises for Lecture 8 (M4):\\ Correlated information.}
\end{center}



\section{Meeting the investor}
	%final2020-2 q1
	A team of two entrepreneurs $i=1,2$ approaches a venture investor $i=0$ with a request to fund their newest business idea. Suppose the real value of the idea is $\omega \sim U[-\infty, \infty]$.\footnote{This is called an ``improper prior'' -- the prior belief about $\omega$ is not a proper probability distribution, but the posterior belief resulting from updating it via Bayes' rule after an informative event would be a proper distribution.}
	Each entrepreneur $i=1,2$ estimates this value at $\theta_i = \omega + \epsilon_i$, where $\epsilon_i \sim i.i.d.U[-0.5,0.5]$. (Take unit of measurement to be millions of dollars.)
	
	\begin{enumerate}
		\item Design a mechanism that would allow the investor to perfectly learn both entrepreneurs' estimates $\theta_i$. In this mechanism, both would independently report their $\theta_i$ to the mechanism, and the mechanism would prescribe report-contingent transfers $t_i(\theta_1,\theta_2)$ from each entrepreneur $i=1,2$ to the investor. Assume that the two entrepreneurs cannot coordinate their reporting strategies. Derive the transfer rules that implement truthful reporting (and show that it is indeed optimal for both entrepreneurs to report truthfully under this transfer rule).
		
		\item How could your mechanism be implemented in the real world? I.e., is it reasonable to ask entrepreneurs to pay for a meeting with an investor? If not, how else could you induce the desired transfers?
	\end{enumerate}

\ifsolutions
\section*{Solution}
	\begin{enumerate}
		\item One way to proceed is to use the truth-revealing transfers we introduced while talking about the Cremer-McLean mechanism. Even though we only introduced this approach for settings with finite type spaces, it would be applicable in this problem -- but one would need to verify that the incentive compatibility constraints still hold.
		
		However, there is a simpler solution. Since $\theta_i \in [ \omega-0.5, \omega+0.5]$, it is also true that $\omega|\theta_i \in [\theta_i-0.5, \theta_i+0.5]$ and thus $\theta_j|\theta_i \in [\theta_i-1, \theta_i+1]$. In words, each entrepreneur $i$ knows for sure that their colleague's valuation $\theta_j$ is within $1$ of their own, because estimation errors $\epsilon_i,\epsilon_j$ are at most $0.5$ in absolute value. This means that we can use the following transfers in our mechanism:
		\begin{align*}
			t_i(\theta_1,\theta_2) = 
			\begin{cases}
				0 & \text{ if } \theta_j \in [\theta_i-1,\theta_i+1];
				\\
				\infty & \text{ if } \theta_j \notin [\theta_i-1,\theta_i+1].
			\end{cases}
		\end{align*}
		
		I.e., both $i$ and $j$ must pay infinity (or any other sufficiently large amount) to the mechanism in case their reports differ by more than $1$ [million dollars]. To see that this transfer rule induces truthtelling, consider $i$'s expected transfer given some own report $\hat{\theta}_i$ and under the assumption that $j$ reports truthfully (which is what happens in the desired equilibrium):
		\begin{align*}
			\mathbb{E} [t_i(\hat{\theta}_i,\theta_j) | \theta_i] &= 0 \cdot \mathbb{P} \{ |\theta_j-\hat{\theta}_i| \leq 1 | \theta_i \} + \infty \cdot \mathbb{P} \{ |\theta_j-\hat{\theta}_i| > 1 | \theta_i \}.
		\end{align*}
		The latter probability is zero (and so $i$ pays zero rather than infinity) if and only if $\hat{\theta}_i = \theta_i$, hence truthtelling is optimal for $i$ for any $\theta_i$.
		
		\item The payment itself can be non-monetary in principle -- e.g., by means of time and effort required to secure an appointment. However, the mechanism does not only require payment to meet the investor, but also requires that this payment is contingent upon the outcome of the meeting, which means that the aforementioned tools can not be easily used. 
		
		One way to interpret transfers is to say that they represent the reputation that these entrepreneurs will acquire in the venture capital circles after the meeting. If they are caught lying ($|\theta_1 - \theta_2|>1$), the investor will make sure to spread the news about it, and the entrepreneurs will have a harder time securing investment for this or other projects from other sources. 
		(It is not immediate whether the investor can commit to such behavior ex ante to incentivize agents -- which is crucial for the mechanism to be credible, -- but you can argue that the designer takes personal offense in having to waste her time listening to a bad pitch, and thus finds it individuall optimal to punish bad proposers.)
	\end{enumerate}
\fi 



\section{Optimal Auction with Correlated Values}
	%410-3ps6 q14 (bank 2015)
	Consider the optimal auction problem in which there are two bidders whose valuations $v_i \in [0,1]$ are uniformly distributed but 
	perfectly correlated, i.e. $v_1 = v_2$ with probability 1.  Construct a DSIC mechanism with the following properties:
	\begin{enumerate}
		\item Bidder 2 wins the object regardless of the type profile.
		\item Both bidders earn zero utility at every type profile.
	\end{enumerate}

\ifsolutions
\section*{Solution}
	Consider the following allocation:
	\[k_1(v_1,v_2)=0\]
	\[k_2(v_1,v_2)=\left\{\begin{array}{cc} 0 & \text{ if }\quad v_2\neq v_1\\
		1 & \quad\text{otherwise}\quad
		
	\end{array}
	\right.\]
	\[t_1(v_1,v_2)=0\]
	\[t_2(v_1,v_2)=\left\{\begin{array}{cc} 0 & \text{if}\quad v_2\neq v_1\\
		v_1 & \quad\text{otherwise}\quad\end{array}\right.\]
	Note that for bidder 1 it is weakly dominant to say her valuation: she never gets the good so she might as well say $v_1$. For player 2, we have exactly the same situation: given the message sent by agent 1, she is indifferent between saying the truth and not, so telling the truth is weakly dominant. Agent 2 gets the good always and both bidders have 0 utility at every type profile.
\fi 



\section{Cremer-McLean}
	%from 2019 hw
	%Consider the 2x2 example from lecture. 
	There are two players, $i=1,2$. Each of them has one of two types, $\theta_i \in \{H,L\}$. The joint distribution of types is given by $\phi(\theta_1,\theta_2)$ as follows:
	\begin{center}
		\begin{tabular}{c | c | c |}
			& H 				& L					\\ \hline
			H	& $\frac{1}{6}$ 	& $\frac{1}{3}$ 	\\ \hline
			L	& $\frac{1}{3}$ 	& $\frac{1}{6}$		\\ \hline
		\end{tabular}
	\end{center}
	Both players have quasilinear utilities.
	
	First explore the problem of information elicitation (without the need to support any underlying allocation). 
	\begin{enumerate}
		\item Compute the players' interim beliefs $\phi(\theta_{-i} | \theta_i)$.
		\item Compute the truth-revealing transfers $\hat{t}_i(\theta) = -\ln(\phi(\theta_{-i} | \theta_i))$.
		\item Verify that a direct mechanism in which each player reports their type $\theta_i$ and pays $\hat{t}_i(\theta)$ is BIC (when coupled with some constant allocation rule $k$).
	\end{enumerate}
	
	Now suppose that the society of these two individuals chooses whether to adopt a new bank holiday called ``National Equality Day'', so the ``real outcome'' is $k \in \{1,0\}$ (where $1$ means bank holiday and $0$ means none). The holiday should only be adopted if all citizens are, in fact, equal, i.e. the desired allocation is $\tilde{k}(\theta) = \mathbb{I} \{\theta_1 = \theta_2\}$. Each citizen receives utility 1 if the holiday is adopted and 0 otherwise.
	\begin{enumerate}[resume]
		\item Is $\tilde{k}$ efficient?
		\item Show that $\tilde{k}$ cannot be sustained without transfers, i.e. that a mechanism $(\tilde{k},t)$ with $t(\theta)=0$ for all $\theta$ is not BIC.
		\item Consider transfers $t_i(\theta_i,\theta_{-i}) = C_1 + C_2 \hat{t} (\theta_i,\theta_{-i})$. Derive conditions on values of $C_1,C_2$ for which a direct mechanism $(\tilde{k},t)$ is BIC.
		\item Derive conditions on values of $C_1,C_2$ for which a direct mechanism $(\tilde{k},t)$ is interim IR and ex ante BB.
		\item Give an example of a BIC, interim IR and ex ante BB mechanism that implements $\tilde{k}$.
	\end{enumerate}

\ifsolutions
\section*{Solution}
	
	\subsubsection*{1. Compute the players' interim beliefs $\phi(\theta_{-i}|\theta_{i})$}
	Players form their beliefs using Bayes' rule:
	\begin{equation}
		\phi(\theta_{-i}|\theta_{i})=\frac{\phi(\theta_{-i},\theta_{i})}{\sum_{\theta'_{-i}\in\Theta_{-i}}\phi(\theta'_{-i},\theta_{i})}
	\end{equation}
	This implies that: 
	\begin{align}
		\phi(\theta_{-i}=H|\theta_{i}=H) &= \frac{\phi(\theta_{-i}=H,\theta_{i}=H)}{\phi(\theta_{-i}=H,\theta_{i}=H)+\phi(\theta_{-i}=L,\theta_{i}=H)} \nonumber \\ 
		&= \frac{1/6}{1/6 + 1/3} \nonumber \\ 
		&= 1/3\\
		\phi(\theta_{-i}=L|\theta_{i}=H)&=2/3\\
		\phi(\theta_{-i}=H|\theta_{i}=L)&=2/3\\
		\phi(\theta_{-i}=L|\theta_{i}=L)&=1/3
	\end{align}
	
	
	\subsubsection*{2. Compute the truth-revealing transfers $\hat{t}_{i}(\theta)=-ln(\phi(\theta_{-i}|\theta_{i}))$}
	Inserting the values for $\phi(\theta_{-i}|\theta_{i})$, we find that:
	\begin{align}
		\hat{t}_{i}(\theta_{i}=H,\theta_{-i}=H)&=-ln(\phi(\theta_{-i=H}|\theta_{i=H})) \nonumber \\ 
		&= -ln(1/3) = ln(3)\\
		\hat{t}_{i}(\theta_{i}=H,\theta_{-i}=L)&=-ln(2/3)=ln(3)-ln(2)\\
		\hat{t}_{i}(\theta_{i}=L,\theta_{-i}=H)&=-ln(2/3)=ln(3)-ln(2)\\
		\hat{t}_{i}(\theta_{i}=L,\theta_{-i}=L)&=-ln(1/3)=ln(3)
	\end{align}
	
	
	\subsubsection*{3. Verify that a direct mechanism in which each player reports their type $\theta_{i}$ and pays $\hat{t}_{i}(\theta)$ is BIC (when
		coupled with some constant allocation rule $k$)}
	Given quasilinear utilites and the mechanism described, the players' utilities are given by:
	\begin{equation}
		u_{i}(\hat{\theta}_{i},\theta_{-i}|\theta_{i})=V_{i}(k,\theta_{i}) - \hat{t}_{i}(\hat{\theta}_{i},\theta_{-i})
	\end{equation}
	Where $V_{i}(k,\theta_{i})$ is the utility that player type $\theta_{i}$ receives from allocation $k$ and where $\hat{\theta}_{i}$ is the type, which player $i$ reports to the designer.\\\\
	We also know, that for the mechanism to be Bayesian Incentive Compatible (BIC), it must hold that for all $i$, $\theta_{i}$ and reported $\hat{\theta}_{i}\in\Theta_i$,
	\begin{equation}
		E_{\theta_{-i}}[u_{i}(\theta_{i},\theta_{-i}|\theta_{i})|\theta_{i}] \geq E_{\theta_{-i}}[u_{i}(\hat{\theta}_{i},\theta_{-i}|\theta_{i})|\theta_{i}]
	\end{equation}
	For player type $\theta_{i}=H$ in our example, this requires that
	\begin{align}
		\phi(\theta_{-i}=H|\theta_{i}=H)(V_{i}(k,\theta_{i}=H)-\hat{t}_{i}(\hat{\theta}_{i}=H,\theta_{-i}=H) + \nonumber \\\phi(\theta_{-i}=L|\theta_{i}=H)(V_{i}(k,\theta_{i}=H)-\hat{t}_{i}(\hat{\theta}_{i}=H,\theta_{-i}=L) \geq \nonumber\\
		\phi(\theta_{-i}=H|\theta_{i}=H)(V_{i}(k,\theta_{i}=H)-\hat{t}_{i}(\hat{\theta}_{i}=L,\theta_{-i}=H) + \nonumber \\\phi(\theta_{-i}=L|\theta_{i}=H)(V_{i}(k,\theta_{i}=H)-\hat{t}_{i}(\hat{\theta}_{i}=L,\theta_{-i}=L)\\
		\rightarrow \frac{1}{3}(V_{i}(k,\theta_{i}=H)-ln(3))+\frac{2}{3}(V_{i}(k,\theta_{i}=H)-ln(3)+ln(2)) \geq \nonumber\\ \frac{2}{3}(V_{i}(k,\theta_{i}=H)-ln(3))+\frac{1}{3}(V_{i}(k,\theta_{i}=H)-ln(3)+ln(2))\\
		\rightarrow \frac{2}{3}ln(2) \geq \frac{1}{3}ln(2)
	\end{align}
	The requirement for $\theta_{i}=H$ is therefore fulfilled. Likewise, for player type $\theta_{i}=L$, BIC requires that
	\begin{align}
		\phi(\theta_{-i}=H|\theta_{i}=L)(V_{i}(k,\theta_{i}=L)-\hat{t}_{i}(\hat{\theta}_{i}=L,\theta_{-i}=H) + \nonumber \\\phi(\theta_{-i}=L|\theta_{i}=L)(V_{i}(k,\theta_{i}=L)-\hat{t}_{i}(\hat{\theta}_{i}=L,\theta_{-i}=L) \geq \nonumber\\
		\phi(\theta_{-i}=H|\theta_{i}=L)(V_{i}(k,\theta_{i}=L)-\hat{t}_{i}(\hat{\theta}_{i}=H,\theta_{-i}=H) + \nonumber \\\phi(\theta_{-i}=L|\theta_{i}=L)(V_{i}(k,\theta_{i}=L)-\hat{t}_{i}(\hat{\theta}_{i}=H,\theta_{-i}=L)\\
		\rightarrow \frac{2}{3}(V_{i}(k,\theta_{i}=H)-ln(3)+ln(2))+\frac{1}{3}(V_{i}(k,\theta_{i}=H)-ln(3)) \geq \nonumber\\ \frac{1}{3}(V_{i}(k,\theta_{i}=L)-ln(3)+ln(2))+\frac{2}{3}(V_{i}(k,\theta_{i}=L)-ln(3))\\
		\rightarrow \frac{2}{3}ln(2) \geq \frac{1}{3}ln(2)
	\end{align}
	Since this also holds, we have verified that a direct mechanism in which each player reports their type $\theta_{i}$ and pays $\hat{t}_{i}(\theta)$ is BIC (when
	coupled with some constant allocation rule $k$).
	
	
	\subsubsection*{4. Is $\Tilde{k}$ efficient?}
	From the perspective of the two agents, the allocation $\Tilde{k}$ is not efficient. If both have the same type, $\Tilde{k}$ is optimal for the players, but if they are not equal, the players would be better off if the holiday was still adopted.
	
	
	\subsubsection*{5. Show that $\Tilde{k}$ cannot be sustained without transfers, i.e. that a mechanism $(\Tilde{k},t)$ with $t(\theta)=0$ for all $\theta$ is not BIC}
	Given the mechanism described, the players' utility is given by
	\begin{equation}
		u_{i}(\hat{\theta}_{i}, \theta_{-i}|\theta_{i})=\mathbb{I}\{\hat{\theta}_{i}=\theta_{-i}\}
	\end{equation}
	For the mechanism to be BIC, neither player type should be incentivized to misreport their type. For player type $\theta_{i}=H$, it must therefore hold that
	\begin{align}
		\phi(\theta_{-i}=H|\theta_{i}=H)\cdot 1 + \phi(\theta_{-i}=L|\theta_{i}=H)\cdot 0 \geq \nonumber\\
		\phi(\theta_{-i}=H|\theta_{i}=H)\cdot 0 + \phi(\theta_{-i}=L|\theta_{i}=H)\cdot 1\\
		\rightarrow \frac{1}{3} \geq \frac{2}{3}
	\end{align}
	This requirement is not fulfilled, which implies that the mechanism is not BIC. Since the player type $\theta_{i}=H$ expects that the other player is twice as likely to be type $\theta_{i}=L$, they are incentivized to misreport their type, as it would increase the chance that the holiday is adopted. Player type $\theta_{i}=L$ is likewise incentivized to report that they are player type $\theta_{i}=H$.
	
	
	\subsubsection*{6. Consider transfers $t_{i}(\theta_{i},\theta_{-i})=C_{1}+C_{2}\hat{t}_{i}(\theta_{i},\theta_{-i})$. Derive conditions on values of $C_{1}$, $C_{2}$ for which a direct mechanism $(\Tilde{k},t)$ is BIC}
	Given this transfer-scheme, the players' utilities are
	\begin{align}
		u_{i}(\hat{\theta}_{i}, \theta_{-i}|\theta_{i})=\mathbb{I}\{\hat{\theta}_{i}=\theta_{-i}\}-t_{i}(\hat{\theta}_{i},\theta_{-i}) \nonumber\\
		=\mathbb{I}\{\hat{\theta}_{i}=\theta_{-i}\}-C_{1}-C_{2}\hat{t}_{i}(\hat{\theta}_{i},\theta_{-i})
	\end{align}
	For player type $\theta_{i}=H$, BIC therefore requires that:
	\begin{align}
		\phi(\theta_{-i}=H|\theta_{i}=H)(1-C_{1}-C_{2}\hat{t}_{i}(\hat{\theta}_{i}=H,\theta_{-i}=H))+\nonumber\\
		\phi(\theta_{-i}=L|\theta_{i}=H)(0-C_{1}-C_{2}\hat{t}_{i}(\hat{\theta}_{i}=H,\theta_{-i}=L)) \geq \nonumber\\ \phi(\theta_{-i}=H|\theta_{i}=H)(0-C_{1}-C_{2}\hat{t}_{i}(\hat{\theta}_{i}=L,\theta_{-i}=H)) + \nonumber\\ \phi(\theta_{-i}=L|\theta_{i}=H)(1-C_{1}-C_{2}\hat{t}_{i}(\hat{\theta}_{i}=L,\theta_{-i}=L))
		\\
		\rightarrow 
		\frac{1}{3}(1-C_{1}-C_{2}ln(3))+\frac{2}{3}(0-C_{1}-C_{2}ln(3)+C_{2}ln(2)) \geq \nonumber\\ 
		\frac{2}{3}(1-C_{1}-C_{2}ln(3)) + \frac{1}{3}(0-C_{1}-C_{2}ln(3)+C_{2}ln(2))
		\\
		\rightarrow
		\frac{1}{3}-C_{1}-C_{2}\left(ln(3)+\frac{2}{3}ln(2)\right) \geq  \frac{2}{3}-C_{1}-C_{2}\left(ln(3)+\frac{1}{3}ln(2)\right)
		\\
		\frac{1}{3}C_{2}ln(2) \geq \frac{1}{3}\\
		C_{2}\geq \frac{1}{ln(2)}, C_{1}\in\mathbb{R}
	\end{align}
	Given symmetry, the same condition is required for the mechanism to tbe BIC for player type $\theta_{i}=L$.
	
	
	\subsubsection*{7. Derive conditions on values of $C_1,C_2$ for which a direct mechanism $(\Tilde{k},t)$ is interim Individually Rational (IR) and ex ante Budget Balanced (BB)}
	For the mechanism to be individually rational for a player, the expected utility for the player from participating must be at least as high as their outside utility, which in this case is zero. As such, the mechanism is interim IR for player type $\theta_{i}=H$, as long as 
	\begin{align}
		\phi(\theta_{-i}=H|\theta_{i}=H)(1-C_{1}-C_{2}\hat{t}_{i}(\hat{\theta}_{i}=H,\theta_{-i}=H))+\nonumber\\
		\phi(\theta_{-i}=L|\theta_{i}=H)(0-C_{1}-C_{2}\hat{t}_{i}(\hat{\theta}_{i}=H,\theta_{-i}=L)) \geq 0\\
		\rightarrow
		\frac{1}{3}(1-C_{1}-C_{2}ln(3))+\frac{2}{3}(0-C_{1}-C_{2}ln(3)+C_{2}ln(2)) \geq 0\\
		\rightarrow
		C_1\leq\frac{1}{3}-C_2(ln(3)-\frac{2}{3}ln(2))
	\end{align}
	Once again, the same condition holds true for player type $\theta_{i}=L$.\\\\
	For the mechanism to be ex ante Budget Balanced, the expected sum of payments must be non-negative, which implies that:
	\begin{align}
		\phi(\theta_{-i}=H|\theta_{i}=H)t_i(\theta_{i}=H,\theta_{-i}=H)+\nonumber\\
		\phi(\theta_{-i}=L|\theta_{i}=H)t_i(\theta_{i}=H,\theta_{-i}=L)+\nonumber\\
		\phi(\theta_{-i}=H|\theta_{i}=L)t_i(\theta_{i}=L,\theta_{-i}=H)+\nonumber\\
		\phi(\theta_{-i}=L|\theta_{i}=L)t_i(\theta_{i}=L,\theta_{-i}=L)\geq 0\\
		\rightarrow
		\frac{1}{6}(C_1+C_2ln(3))+\frac{1}{3}(C_1+C_2ln(3)-C_2ln(2))+\nonumber\\
		\frac{1}{3}(C_1+C_2ln(3)-C_2ln(2))+\frac{1}{6}(C_1+C_2ln(3))\geq 0\\
		\rightarrow
		C_1+C_2ln(3)-\frac{2}{3}C_2ln(2)\geq 0\\
		\rightarrow
		C_1\geq -C_2(ln(3)-\frac{2}{3}ln(2))
	\end{align}
	For the mechanism to be interim IR and ex ante BB, $C_1$ and $C_2$ must satisfy the condition:
	\begin{equation}
		1/3-C_2(ln(3)-\frac{2}{3}ln(2))\geq C_1 \geq -C_2(ln(3)-\frac{2}{3}ln(2))
	\end{equation}
	
	
	\subsubsection*{8. Give an example of a BIC, interim IR and ex ante BB mechanism that implements $\Tilde{k}$}
	We choose an arbitrary value $C_2=3$, which satisfies the BIC-condition $C_2\geq \frac{1}{ln(2)}$. We then choose $C_1$, such that the mechanism is exactly ex ante BB. This implies that 
	\begin{align}
		C_1=-3(ln(3)-\frac{2}{3}ln(2))\nonumber\\
		=ln(\frac{4}{27})
	\end{align}
	The direct revelation mechanism $(\tilde{k},t)$, where transfer-scheme $t$ is given by $t_i(\theta_i,\theta_{-i})=ln(\frac{4}{27})-3ln(\phi(\theta_{-i}|\theta_{i}))$ is therefore BIC, interim IR and ex ante BB and it implements the desired allocation $\Tilde{k}$.

\fi



\section{Battaglini}
	%from 2019 hw
	
	Construct a perfectly revealing equilibrium in the Battaglini model with agents' biases $b_1 = (1, -3)$ and $b_2 = (-4, 2)$. (Describe agents' reporting strategies as functions of state and the principal's decision strategy as a function of reports.)

\ifsolutions
\section*{Solution}
	
	We consider an example of the Battaglini model with one principal, two agents and a two-dimensional outcome, where the two agents $i\in\{1,2\}$ have biases $b_1=(1,-3)$ and $b_2=(-4,2)$ respectively. The timing of the interaction is as follows:
	\begin{enumerate}
		\item Nature determines the state of the world $w=(w_1,w_2)\in\mathbb{R}^2$, which is observed by both agents, but not the principal
		\item Having observed $w$, agents $1$ and $2$ simultaneously send messages\\ ($m_1(w),m_2(w))\in\mathbb{R}^2$ to the principal about the observed state
		\item Having observed messages $(m_1(w),m_2(w))$, the principal chooses the two-dimensional action $a=(m_1(w),m_2(w))\in\mathbb{R}^2$
	\end{enumerate}
	All players have squared-Euclidean-distance preferences, which means that they wish to minimize the distance between the principle's chosen action $a$ and their respective bliss points. The principal's bliss point is simply the true state $w$, while the agents' bliss points are the sum of their respective biases $(b_1,b_2)$ and the true state $w$. We can therefore express their utilities as:
	\begin{itemize}
		\item Principle: $u_p(a,w)=-(||a-w||)^2$
		\item Agent $i$: $u_i(a,w)=-(||a-(w+b_i)||)^2$
	\end{itemize}
	where $||x||=\sqrt{x_1^2+x_2^2}$, for $x=(x_1,x_2)\in\mathbb{R}^2$. The agents biases $b_i$ are commonly known. This implies that the principal chooses the equilibrium action such that $a^*(m_1(w),m_2(w))=E[w|(m_1(w),m_2(w))]$ and with this in mind, agent $i$ sends message $\hat{m}_i(w)$ in order to maximize their utility $u_i(a^*,w)=-(||a^*(\hat{m}_i(w),m_{-i}(w))-(w+b_i)||)^2$\\\\
	The principal is able to design how the agents communicate to the principal about the observed state. In order to find a fully revealing equilibrium, the goal of the principal is to design communication strategies such that the principal is able to identify the state from the two messages and such that it is optimal for the agents to tell the truth. The Battaglini model suggests that the principal should restrict the agents' messaging, such that instead of simply reporting $w$, a given agent must report a line which is orthogonal to the other player's bias. The implication is that there is a second stage where the player then reports the true state $w$ on the line reported by the other player. However, in practice, this second stage is not necessary for the fully revealing equilibrium, since the reported lines are sufficient in order to locate $w$.\\\\
	We can construct this formally by considering the basis $(c_1,c_2)$ where $c_1$ and $c_2$ are vectors which are orthogonal to the biases $b_1$ and $b_2$ respectively. As long as $b_1$ and $b_2$ are linearly independent, then so are $c_1$ and $c_2$, which implies that the true state $w$ can be uniquely represented as a linear combination of $c_1$ and $c_2$:
	\begin{equation}
		w=o^1 \cdot c_1 + o^2 \cdot c_2 \nonumber
	\end{equation}
	where $(o^1,o^2)$ are a given pair of coordinates. The principal imposes on agent $i$ that they must report the value of $o^j$. After receiving the reports, the principal then chooses $a(o^1,o^2)=o^1 \cdot c_1 + o^2 \cdot c_2$. Given that the other player reports truthfully, Battaglini shows that it is optimal for a player to also report truthfully, which in turn ensures that the true state is chosen by the principal. This implies that the mechanism provides a fully revealing equlibrium.\\\\
	In our example, where $b_1=(1,-3)$ and $b_2=(-4,2)$, we use the formula $c_i=\frac{1}{||b_i||}
	\begin{bmatrix}
		0 & 1\\
		-1 & 0
	\end{bmatrix}b_i$ to calculate the vectors $(c_1,c_2)$ with unit length:
	\begin{align}
		c_1&=\bigg(\frac{-3}{\sqrt{10}},\frac{-1}{\sqrt{10}}\bigg) \nonumber\\
		c_2&=\bigg(\frac{1}{\sqrt{5}},\frac{2}{\sqrt{5}}\bigg) \nonumber
	\end{align}
	This implies that $w$ can be represented as 
	\begin{align}
		w&=o^1 \cdot \bigg(\frac{-3}{\sqrt{10}},\frac{-1}{\sqrt{10}}\bigg) + o^2 \cdot \bigg(\frac{1}{\sqrt{5}},\frac{2}{\sqrt{5}}\bigg)\nonumber\\
		\rightarrow w_1&=o^1(\frac{-3}{\sqrt{10}})+o^2(\frac{1}{\sqrt{5}})\ \ \text{and}\ \ \nonumber
		w_2=o^1(\frac{-1}{\sqrt{10}})+o^2(\frac{2}{\sqrt{5}}) \nonumber
	\end{align}
	Solving for $o^1$ and $o^2$, we find that
	\begin{equation}
		o^1=-\sqrt{10}(w_1-2w_2) \ \ \text{and} \ \ o^2=\frac{\sqrt{5}}{2}(3w_2-w_1)\nonumber
	\end{equation}
	Given a realized state $w=(w_1,w_2)$, these values of $o^1$ and $o^2$ are the equilibrium reporting strategies of the two agents in our example, which imply that the principal chooses the true state $a=w$.

\fi


\end{document}
