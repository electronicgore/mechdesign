%%% License: Creative Commons Attribution Share Alike 4.0 (see https://creativecommons.org/licenses/by-sa/4.0/)


%%%%%%%%%%%%%%%%%%%%%%%%%%%%%%%%%%%%%%%%%

%----------------------------------------------------------------------------------------
%	PACKAGES AND OTHER DOCUMENT CONFIGURATIONS
%----------------------------------------------------------------------------------------

\documentclass{article}

\usepackage{amssymb}
\usepackage{enumerate}
\usepackage[usenames,dvipsnames]{color}
\usepackage{fancyhdr} % Required for custom headers
\usepackage{lastpage} % Required to determine the last page for the footer
\usepackage{extramarks} % Required for headers and footers
\usepackage[usenames,dvipsnames]{color} % Required for custom colors
\usepackage{graphicx} % Required to insert images
\usepackage{listings} % Required for insertion of code
\usepackage{courier} % Required for the courier font
\usepackage[table]{xcolor}
\usepackage{amsfonts,amsmath,amsthm,parskip,setspace,url}
\usepackage[section]{placeins}
\usepackage[a4paper]{geometry}
\usepackage[USenglish]{babel}
\usepackage[utf8]{inputenc}


% Margins
\topmargin=-0.45in
\evensidemargin=0in
\oddsidemargin=0in
\textwidth=6.5in
\textheight=9.0in
\headsep=0.6in

\linespread{1.1} % Line spacing

%----------------------------------------------------------------------------------------
%	DOCUMENT STRUCTURE COMMANDS
%	Skip this unless you know what you're doing
%----------------------------------------------------------------------------------------

% Header and footer for when a page split occurs within a problem environment
\newcommand{\enterProblemHeader}[1]{
\nobreak\extramarks{#1}{#1 continued on next page\ldots}\nobreak
\nobreak\extramarks{#1 (continued)}{#1 continued on next page\ldots}\nobreak
}

% Header and footer for when a page split occurs between problem environments
\newcommand{\exitProblemHeader}[1]{
\nobreak\extramarks{#1 (continued)}{#1 continued on next page\ldots}\nobreak
\nobreak\extramarks{#1}{}\nobreak
}

\setcounter{secnumdepth}{0} % Removes default section numbers
\newcounter{homeworkProblemCounter} % Creates a counter to keep track of the number of problems

\newcommand{\homeworkProblemName}{}
\newenvironment{ex}[1][Problem \arabic{homeworkProblemCounter}]{ % Makes a new environment called homeworkProblem which takes 1 argument (custom name) but the default is "Problem #"
\stepcounter{homeworkProblemCounter} % Increase counter for number of problems
\renewcommand{\homeworkProblemName}{#1} % Assign \homeworkProblemName the name of the problem
\section{\homeworkProblemName} % Make a section in the document with the custom problem count
%\enterProblemHeader{\homeworkProblemName} % Header and footer within the environment
}{
%\exitProblemHeader{\homeworkProblemName} % Header and footer after the environment
}

\newcommand{\problemAnswer}[1]{ % Defines the problem answer command with the content as the only argument
\noindent\framebox[\columnwidth][c]{\begin{minipage}{0.98\columnwidth}#1\end{minipage}} % Makes the box around the problem answer and puts the content inside
}

\newcommand{\homeworkSectionName}{}
\newenvironment{homeworkSection}[1]{ % New environment for sections within homework problems, takes 1 argument - the name of the section
\renewcommand{\homeworkSectionName}{#1} % Assign \homeworkSectionName to the name of the section from the environment argument
\subsection{\homeworkSectionName} % Make a subsection with the custom name of the subsection
%\enterProblemHeader{\homeworkProblemName\ [\homeworkSectionName]} % Header and footer within the environment
}{
%\enterProblemHeader{\homeworkProblemName} % Header and footer after the environment
}

\newif\ifsolutions

%----------------------------------------------------------------------------------------
%----------------------------------------------------------------------------------------
%----------------------------------------------------------------------------------------
% Set up the header and footer
\pagestyle{fancy}
\lhead[c]{\textbf{{\color[rgb]{.5,0,0} K{\o}benhavns\\Universitet }} \firstxmark} % Top left header
\chead{\textbf{{\color[rgb]{.5,0,0} \Class }}\\ \hmwkTitle  } % Top center head
\rhead{\instructor \\ \theprofessor} % Top right header
\lfoot{\lastxmark} % Bottom left footer
\cfoot{} % Bottom center footer
\rfoot{Page\ \thepage\ of\ \protect\pageref{LastPage}} % Bottom right footer
\renewcommand\headrulewidth{0.4pt} % Size of the header rule
\renewcommand\footrulewidth{0.4pt} % Size of the footer rule

\setlength\parindent{12pt}







%----------------------------------------------------------------------------------------
%	NAME AND CLASS SECTION
%----------------------------------------------------------------------------------------

\newcommand{\hmwkTitle}{Exercises: Lec 1-2a} % Assignment title
\newcommand{\Class}{Mechanism Design} % Course/class
\newcommand{\instructor}{Fall 2019} % TA
\newcommand{\theprofessor}{Prof. Egor Starkov} % Professor

%\theoremstyle{definition} \newtheorem{ex}{\textbf{\Large{Exercise & #}\\}}
\setlength{\parskip}{5 pt}




















%%%%%%%%%%%%%%%%%%%%%%%%%%%%%%%%%%%%%%%%%%%%%%%%%%%%%%%%%%%%%%%%%%%%%%%%%%%%%%%%%%%%%%
%\solutionsfalse
\solutionstrue
%%%%%%%%%%%%%%%%%%%%%%%%%%%%%%%%%%%%%%%%%%%%%%%%%%%%%%%%%%%%%%%%%%%%%%%%%%%%%%%%%%%%%%


\begin{document}
	
	
	\ifsolutions
	\begin{center}
		
		{\Huge Problem Set for Lectures 1-2(a)
			(with Solutions)}
	\end{center}
	
	%\emph{The solutions below brought to you by Hannah Barth and Inga Reymann. Where necessary, I also present older solutions available to me (instead of or in addition to those by Hannah and Inga). Notes in cursive added by Egor.}
	\fi
	
These exercises are for your own practice and are not to be handed in. Some exercises are open ended in that they may not have a unique correct answer. If you think there is a typo in the problem, attempt to amend it and proceed as best you can.

%\begin{ex}[Second-price auction]
%	\emph{(From lecture.)} A second-price auction is described as following. There are $N$ players with private valuations $\theta_i \in \Theta_i \subseteq \mathbb{R}_+$ for some item that is to be allocated. All players simultaneously submit their bids $b_i \in \mathbb{R}_+$ to the auctioneer. Player $i$ with the highest bid gets the item and pays $t_i = \max_{j \neq i} b_j$ to the auctioneed, so his (player $i$'s) utility is $\theta_i - t_i$. All other players get nothing and pay nothing, so their utility is zero.
%	\begin{enumerate}
%		\item Show that truthful bidding $b_i = \theta_i$ is a weakly dominant strategy for player $i$. 
%		\begin{itemize}
%			\item \emph{Hint: (try to think about the problem on your own before reading this hint)} you need to show that truthful bidding is dominant, i.e. no other strategy can ever be strictly better independently of how others bid.  Suppose that the highest bid among other players was $b_j > \theta_i$. Would it have been [strictly] better for $i$ to bid something other than $b_i=\theta_i$? What if $b_j < \theta_i$? Congratulations, you have just solved the problem.
%		\end{itemize}
%		\item Is the outcome produced by equilibrium in these dominant strategies efficient?
%		\item Argue that there exists a direct mechanism that implements the efficient allocation of the item in dominant strategies.
%		\begin{itemize}
%			\item \emph{Hint:} the game above \emph{is} your mechanism. Write it down using the notation for mechanisms we introduced in the lecture. Show that it is efficient and DSIC. This is an exercise in the tautology of formal math, not in deep thinking, there is no catch.
%		\end{itemize}
%	\end{enumerate}
%	
%	\ifsolutions
%	\section*{Solution}
%	If any
%	\fi
%\end{ex}


%%------------------------------------------------------------------------------------------------

\begin{ex}[Review Questions]
	\begin{itemize}
		\item What is a social choice function?
		\item What is a mechanism?
		\item What is a direct mechanism?
		\item What does truthful implementation mean?
	\end{itemize}
	
\end{ex}



%%------------------------------------------------------------------------------------------------

\begin{ex}[P2: Social Choice]
	Consider a social choice problem between two alternatives in $X = \{l, r\}$ (could be a ``left-leaning policy'' and a ``right-leaning policy'').
	\begin{enumerate}
		\item What \emph{ordinal} preferences (rankings $\succsim_i$) can an individual have over $X$?
		\item Consider a society of two citizens, $i=1,2$ that can have any of the preferences you identified above.
		\begin{enumerate}
			\item Give an example of a social preference $\succsim$ that satisfies all three axioms from slide 31.
			\item Is it dictatorial? (Note that Arrow's theorem does not apply.)
			\item Do there exist any other social preferences that satisfy the same requirements? Are any/all of them dictatorial?
		\end{enumerate}
		\item Now consider a society of three citizens, $i=1,2,3$. Give an example of a social preference that satisfies the three axioms from slide 31 and is not dictatorial.
	\end{enumerate}
	
	\ifsolutions
	\subsection*{Solution}
	
	Currently missing
	\fi
\end{ex}



%%------------------------------------------------------------------------------------------------

\begin{ex}[P3: Voting]
	Consider the voting process we had for the lecture format. Does it implement the efficient outcome (in dominant strategies)? Write the setting down as a mechanism design problem and provide a reasonably formal proof if yes, or give a counterexample if no.
	
	\ifsolutions
	\subsection*{Solution}
	The voting was related to the choice between three lecture formats, with two, one, or no breaks during the three-academic-hour lecture. All students had to assign an integer number between $-3$ and $3$ to each of the three alternatives; the alternative with the highest sum of votes was adopted.
	
	We now describe this setting as a mechanism design problem and show that it does not implement the efficient outcome.
	
	The set of alternatives is $X = \{0,1,2\}$. The players are indexed by $i = 1,2, ..., N$. Each player receives utility $u_i(x) \in \mathbb{R}$ if alternative $x \in X$ is chosen. 
	
	The goal is to check whether a given mechanism $\Gamma = (S, g)$ implements the efficient allocation 
	$$f^* = \arg \max_x \left\{ \sum_{i=1}^N u_i(x) \right\}$$ 
	for any profile of utilities $u_i$. The mechanism is given by strategy sets $S_i = \{-3, -2, -1, 0, 1, 2, 3\}^X$ for each player $i$ and the outcome function $g(s) = \arg \max_{x \in X} \left\{ \sum_{i=1}^N s_i(x) \right\}$.
	
	There are many ways to break this mechanism. One way is by assuming that each player's utility from any alternative can be arbitrarily high or low, i.e. $u_i(x) \in (-\infty, \infty)$.\footnote{The same argument could be made, using slightly more effort, if utilities have full support on a finite interval, e.g., $u_i(x) \in [0,1]$.} 
	Then for any players' strategy (where $i$'s strategy $\sigma_i$ maps $i$'s utilities $u_i$ into votes $s_i$) we can find a utility profile -- or, as we shall call this, type profile -- for which the outcome of the game is not efficient. Note that this argument does not even touch upon the strategy profile being an equilibrium -- it merely says that the strategy space of the game is not rich enough to accomodate efficiency for all possible utilities.
	
	In this solution, however, we will follow another path, and show that even when $u_i(x) \in \{-3, -2, -1, 0, 1, 2, 3\}$, truthful voting (defined as choosing $s_i(x) = u_i(x)$) is not a dominant strategy equilibrium of the game. To do this we need to find a single utility profile for which one of the players would like to deviate ex post -- i.e., after learning other players' votes, this player would like to change their votes. In particular, let $N=2$, $u_2 = (0,0,2)$ and $u_1 = (1,0,0)$. Assuming that player $2$ votes truthfully, if player $1$ votes truthfully then $x=2$ is chosen by the mechanism. If, however, player $1$ votes $s_1 = (3,0,0)$ then $x=1$ is chosen, which is better for player $1$ given her true preferences. Therefore, truthful reporting is not a dominant strategy equilibrium of the game. No other strategy profile yields the efficient outcome, so the mechanism does not implement the efficient outcome in dominant strategies.
	\fi
\end{ex}



%%------------------------------------------------------------------------------------------------

\begin{ex}[P4: Weak Preference Reversal Property]
	Prove that if s.c.f. $f$ satisfies the weak preference reversal property from slide 51 then $f$ is DSIC.
	
	\ifsolutions
	\subsection*{Solution (Hannah Barth and Inga Reymann)}
	
	\textbf{Weak preference reversal property:}
	Consider an agent $i \in N$ and let $\theta_i', \theta_i'' \in \Theta_i$ be any two possible types. Given $\theta_{-i} \in \Theta_{-i}$, we say that the social choice function f satisfies the weak preference reversal property for agent $i$ and for types $\theta_i', \theta_i''$ if the following inequalities are satisfied: 
	
	$$u_i(f(\theta_i', \theta_{-i}), \theta_i') \geq\ u_i(f(\theta_i'', \theta_{-i}), \theta_i')$$
	
	$$u_i(f(\theta_i', \theta_{-i}),\theta_i'') \leq\ u_i(f(\theta_i'', \theta_{-i}), \theta_i '')$$
	
	\textbf{Revelation principle:}
	A mechanism implements a social choice function $f$ in dominant strategies if there is a strategy profile $(s_1^*,...,s_N^*)$ that constitutes a dominant-strategy equilibrium of the mechanism and is such that 
	$$g(s_1^*(\theta_1),...s_N^*(\theta_N))=f(\theta) \quad \forall \quad \theta_i\in \Theta_i \quad \text{and} \quad \forall i$$
	
	The fact that $(s_1^*,...,s_N^*)$ is a dominant-strategy equilibrium implies that
	$$u_i(g(s_i^*(\theta_i'),s_{-i}^*(\theta_{-i})),\theta_i')\geq u_i(g(s_i^*(\theta_i''),s^*_{-i}(\theta_{-i})),\theta_i') \quad \forall \quad \theta_i',\theta_i''\in \Theta$$
	And because $g(s^*(\theta))=f(\theta)$, it holds that:
	$$u_i(f(\theta',\theta_{-i}),\theta_i')\geq u_i(f(\theta_i'',\theta_{-i}),\theta_i')$$
	which is equivalent to the statement of the weak preference reversal property.
	\fi
\end{ex}



%%------------------------------------------------------------------------------------------------

\begin{ex}[P5: Judicial Design]
	A suspect is in custody, accused of murder.  If he goes to trial he will either be convicted or acquitted. If he
is convicted he will be sent to prison for life giving him a payoff of $-1$.  If he is acquitted he goes free and has a
payoff of $0$.  The district attorney can offer plea bargains: allowing the defendant to plead guilty in return for a
lighter sentence.  In
particular, for any $r\in (0,1)$, the DA can offer a reduced sentence which, if accepted, would give the defendant
a payoff of $-r.$

	
	The defendant is privately informed about his chances for acquittal at trial:  $\theta\in [0,1]$ is the defendant's privately
known probability of acquittal.  If the defendant does not enter into a plea bargain with the DA he will go to trial and
be convicted with probability $1 - \theta$.

	
	Consider the mechanism design problem where the DA is the principal and the defendant is the agent.  A social choice
function is a mapping $f:[0,1] \rightarrow \left\{ \text{trial} \right\} \cup (0,1)$ where $f(\theta) = \text{trial}$ means
that type $\theta$ will go to trial and $f(\theta) = r \in (0,1)$ means that type $\theta$ accepts a plea bargain
giving him a sentence with payoff $-r$.
DA thinks $\theta$ has full support on $[0,1]$.
	
	\begin{enumerate}

		\item Write down the inequalities that characterize whether some given social choice function $f$ is incentive-compatible
for the defendant.

		\item What is the set of all incentive-compatible social choice functions?
You can proceed in the following steps:
		\begin{itemize}
			\item Show that in any IC $f$ at most one plea bargain $r$ is available.
			\item Show that $f$ must be of cutoff type, with the suspect taking the plea if $\theta < \bar{\theta}$ and going to court otherwise.
			\item Find the value of $r$ that makes the cutoff s.c.f. $f$ incentive compatible given some cutoff type $\bar{\theta}$.
			\item Combine all of the above to characterize the set of implementable $f$.
		\end{itemize}
	\end{enumerate}


	\ifsolutions
	\subsection*{Solution}
	By going to trial a defendant of type $\theta$ receives (expected) utility of $-(1-\theta)$, while from accepting a plea bargain his utility is $-r$. This implies that the IC constraint is $f(\theta)=r \Rightarrow -r\geq -(1-\theta) ~\&~ \forall \theta' \neq \theta ~ f(\theta') \geq f(\theta)$
	

	We will characterize the set of IC social choice functions by a series of claims. 
	%For simplicity we will initially assume that $\theta$ has full support on $[0,1]$, and then correct for other cases.

	\begin{itemize}

		\item[claim 1] $f(\bullet)$ has at most one value on the real line.
		
		Proof: if $f(\theta_1)<f(\theta_2) ~\theta_1,\theta_2 \in [0,1]$ then a defendant of type $\theta_2$ gains higher utility by declaring $\theta_1$ (as $-f(\theta_1)>-f(\theta_2)$. This implies the mechanism is not IC for $\theta_2$.

		\item[claim 2] $f(\bullet)$ has a cutoff at some $\bar{\theta}$. i.e. $f(\theta)= \begin{cases} r & \text{if } \theta<\bar{\theta}
		\\
		T & \text{if } \theta\geq\bar{\theta} \end{cases}$ (value at $\bar{\theta}$ is not unique)\\

		Proof: assume $\theta'>\theta , f(\theta)=T, f(\theta')=r$. By IC for $\theta$ we know that $-r\leq -(1-\theta)$. However as $-(1-\theta')>-(1-\theta)$ this implies that $-(1-\theta')>-r$ and we don't have IC for $\theta'$.
		
		\item[claim 3] $u(-f(\bar{\theta}),\bar{\theta})\geq -(1-\bar{\theta})$
		
		This follows immediately from IC for type $\bar{\theta}$.
		
		\item[claim 4] $r= 1-\bar{\theta}$
		
		Proof: $r\leq 1-\bar{\theta}$ follows directly from the last claim, while $r\geq 1-\bar{\theta}$ follows from IC of type $\bar{\theta}+\epsilon$. If type $\bar{\theta}$ were strictly better of by accepting the plea bargain, by continuity and monotonicity of benefit of trial, type $\bar{\theta}+\epsilon$ would also strictly prefer the plea bargain contradicting IC for that type.

	\end{itemize}

	These four claims  imply that for any $(r,\bar{\theta})$ s.t. $r= 1-\bar{\theta}$ the social choice function

	\[ f(\theta)= \begin{cases} r & \text{if } \theta<\bar{\theta} \\

	T & \text{if } \theta\geq\bar{\theta} \end{cases} \]

	is incentive compatible.
	
	%As long as the support of types has measure one, the arguments above hold. However, if there is a interval $(a,b)$ with a zero probability, then when we set $\bar{\theta}=a$ any $r\in [a-(b-a),a]$ can be used. This is the case as claim four no longer holds, and we just need to get IC for type $b$. The same argument also holds for closed or half closed intervals.
	\fi
\end{ex}



%%------------------------------------------------------------------------------------------------

%\begin{ex}[Budget balance]
%	Prove that if a mechanism is ex post budget balanced then it is also ex ante budget balanced.
%	
%	\ifsolutions
%	\section*{Solution}
%	If any
%	\fi
%\end{ex}



%%------------------------------------------------------------------------------------------------

%\begin{ex}[Bilateral trade]
%	\begin{enumerate}
%		\item Verify VCG transfers in bilateral trade setting (slides 68-69).
%		\item Is there an efficient, DSIC, ex post IR, ex post BB mechanism? (Hint: Start from VCG and invoke revenue equivalence.)
%	\end{enumerate}
%	
%	
%	\ifsolutions
%	\section*{Solution}
%	If any
%	\fi
%\end{ex}



%%------------------------------------------------------------------------------------------------

\end{document}
