%%% License: Creative Commons Attribution Share Alike 4.0 (see https://creativecommons.org/licenses/by-sa/4.0/)


%%%%%%%%%%%%%%%%%%%%%%%%%%%%%%%%%%%%%%%%%

%----------------------------------------------------------------------------------------
%	PACKAGES AND OTHER DOCUMENT CONFIGURATIONS
%----------------------------------------------------------------------------------------

\documentclass[a4paper]{article}

\usepackage{amssymb}
\usepackage{enumerate}
\usepackage[usenames,dvipsnames]{color}
\usepackage{fancyhdr} % Required for custom headers
\usepackage{lastpage} % Required to determine the last page for the footer
\usepackage{extramarks} % Required for headers and footers
\usepackage[usenames,dvipsnames]{color} % Required for custom colors
\usepackage{graphicx} % Required to insert images
\usepackage{listings} % Required for insertion of code
\usepackage{courier} % Required for the courier font
\usepackage[table]{xcolor}
\usepackage{amsfonts,amsmath,amsthm,parskip,setspace,url}
\usepackage[section]{placeins}
\usepackage[a4paper]{geometry}
\usepackage[USenglish]{babel}
\usepackage[utf8]{inputenc}


% Margins
\topmargin=-0.45in
\evensidemargin=0in
\oddsidemargin=0in
\textwidth=6.5in
\textheight=9.0in
\headsep=0.6in

\linespread{1.1} % Line spacing



%----------------------------------------------------------------------------------------
%   FORMATTING
%----------------------------------------------------------------------------------------
% Set up the header and footer
\pagestyle{fancy}
\lhead[c]{\textbf{{\color[rgb]{.5,0,0} K{\o}benhavns\\Universitet }}} % Top left header
\chead{\textbf{{\color[rgb]{.5,0,0} \Class }}\\ \hmwkTitle  } % Top center head
\rhead{\instructor \\ \theprofessor} % Top right header
\lfoot{\lastxmark} % Bottom left footer
\cfoot{} % Bottom center footer
\rfoot{Page\ \thepage\ of\ \protect\pageref{LastPage}} % Bottom right footer
\renewcommand\headrulewidth{0.4pt} % Size of the header rule
\renewcommand\footrulewidth{0.4pt} % Size of the footer rule


% Other formatting stuff
%\setlength\parindent{12pt}
\setlength{\parskip}{5 pt}
%\theoremstyle{definition} \newtheorem{ex}{\textbf{\Large{Exercise & #}\\}}
\usepackage{titlesec}
\titleformat{\section}[hang]{\normalfont\bfseries\Large}{Problem \thesection:}{0.5em}{}




%----------------------------------------------------------------------------------------
%	NAME AND CLASS SECTION
%----------------------------------------------------------------------------------------
\newcommand{\hmwkTitle}{Exercises after Lecture 4 (M1)} % Assignment title
\newcommand{\Class}{Mechanism Design} % Course/class
\newcommand{\instructor}{Fall 2020} % TA
\newcommand{\theprofessor}{Prof. Egor Starkov} % Professor




%----------------------------------------------------------------------------------------
%   SOLUTIONS
%----------------------------------------------------------------------------------------
\newif\ifsolutions
%\solutionstrue




\begin{document}

\begin{center}
		\LARGE\textbf{Exercises after Lecture 4 (M1):\\ IR and BB efficient mechanisms}
\end{center}



\section{Efficient public good provision (continued)}

	Consider problem 2 from the previous problem set.
	
	Assume now that valuations $\theta_i$ are constrained to some interval $[-\hat{\theta},\hat{\theta}]$ for all $i$, and that the public project has some known social cost $c > 0$.
	
	\begin{enumerate}
		\item Derive the gVCG transfers.
		\item Derive the AGV transfers. 
		\emph{Look the expression up in the slides; we will talk about what it means next week.}
	\end{enumerate}
	
\ifsolutions
\section*{Solution}

\fi



\section{Payoff equivalence in BIC mechanisms}

	Prove the payoff equivalence result for BIC mechanisms using the analog of the argument we had for DSIC mechanisms (i.e., by showing monotonicity first).
	
\ifsolutions
\section*{Solution}

\fi



\section{Myerson-Satterthwaite theorem}

Derive the gVCG transfers for the bilateral trade problem. Show that the resulting mechanism is not ex ante budget balanced (even in the weak sense, not just that it is not ``exactly BB'').

\ifsolutions
\section*{Solution}

\fi
	


\section{Auction with non-trivial seller valuation}
	A seller, $i=1$, possesses a single indivisible object for which there are two potential buyers. Each buyer $i \in \{2,3\}$ has value $v_i$ for the good and the seller has an opportunity cost $c$ from selling the good.  Utility is quasi-linear in money, so if buyer $i$ purchases the good at price $p$, his final utility is $v_i-p$, and the seller's utility is $p-c$. Each agent has zero utility if he does not trade and zero is therefore also the reservation utility of each agent.
	
	Each $v_i$ is drawn independently from the same distribution $F$ which has full support on $[0,\bar v]$, and $c$ is drawn independently from a distribution $G$ which has support $[0,\bar c]$.  Assume that $F$ and $G$ satisfy all of the conditions necessary for the revenue
	equivalence theorem and our characterization results in class.
	
	\medskip
	A mechanism consists of two collections of functions, $q(c, v_2, v_3)$ and $t(c, v_2, v_3)$, where $q(c, v_2, v_3)$ is a probability distribution that prescribes the probabilities that the good will be allocated to each of the three agents, and $t(c, v_2, v_3)$ gives the list of transfers paid to each of the three agents.
	Say that a mechanism is \emph{feasible} if it is Bayesian incentive compatible, interim individually rational, and ex ante exactly budget-balanced.
	
	\begin{enumerate}
		\item What is the efficient allocation rule?
		
		\item Assume $\bar c = \bar v$.  Show that there does not exist a feasible mechanism that implements the efficient allocation rule. \emph{Hint: use gVCG and its properties.}
		
		\item Now assume $\bar v > \bar c$.  Show that the following is a sufficient condition for the existence of a fesible mechanism that implements the efficient allocation rule:
		$$ \mathbb{E}(\min\{v_2, v_3\}) \geq \bar c. $$
		\item Again assume $\bar{v} > \bar{c}$, but now suppose that there are $N$ potential buyers with values drawn independently from $F$.  Prove that for any $F$ and $G$ there is a $\bar N$ such that whenever $N > \bar{N}$  there exists a feasible mechanism that
		implements the efficient allocation rule.
	\end{enumerate}
	
	
\ifsolutions
\section*{Solution}
	This solution uses the following notation: $a \vee b \equiv \max\{a,b\}$; $a \wedge b \equiv \min \{a,b\}$.
	\begin{enumerate}%[\label=(\alph{enumi})]
		\item The efficient allocations must satisfy
		\begin{align*}
			q(c, v_2, v_3) = 
			\begin{cases}
				(1,0,0) &\text{ if } c > v_2, v_3, \\
				(0,1,0) &\text{ if } v_2 > c,v_3, \\
				(0,0,1) &\text{ if } v_3 > c,v_2.
			\end{cases}
		\end{align*}
		Ties can be broken arbitrarily.
		
		\item Let's use the Krishna--Perry theorem. The least charitable types are $0$ for the buyers and $\bar{c}$ for the seller (I leave it to you to verify this). The gVCG mechanism has the following transfers:
		%
		\begin{itemize}
			
			\item when no trade occurs, there are no transfers
			
			\item when trade occurs between the seller and buyer $i$, $i$ pays $v_{-i} \vee c$, the seller receives $v_i \wedge \bar{c}$, and $-i$ has transfer zero.
			%\footnote{This solution uses the following notation: $a \vee b \equiv \max \{a,b\}$ and $a \wedge b \equiv \min \{a,b\}$.}
			
		\end{itemize}
		
		%When $c=0$ and $v_i > v_{-i}$, trade occurs, with the buyer paying $v_{-i} < \bar{v} =\bar{c}$ and the seller receiving $\bar{c}$. So there is a budget deficit.
		If $\bar{v}=\bar{c}$ then $v_i \wedge \bar{c} = v_i$, so this mechanism yields weakly negative revenue: %assuming wlog that $v_i > v_{-i}$, we get
		\begin{align*}
			\sum_i t_i (\theta) = \begin{cases}
				0 &\text{ if } c > v_2, v_3 \\
				v_3 \vee c - v_2 < 0 &\text{ if } v_2 > c,v_3 \\
				v_2 \vee c - v_3 < 0 &\text{ if } v_3 > c,v_2.
			\end{cases}
		\end{align*}
		
		Since gVCG mechanism has maximal revenue among efficient, BIC and IIR mechanisms, this implies that any other mechanism that is efficient, BIC, and interim IR will run an expected deficit. If there is no ex ante BB mechanism then there is no ex post BB mechanism either (deficit in expectation means that there must be deficit for at least some type realizations). Therefore, no feasible mechanism exists in this case.
		
		
		\item Let's use the same mechanism. The budget surplus is zero conditional on no trade, and
		%
		\begin{equation*}
			[ v_2 \wedge v_3 ] \vee c - [ v_2 \vee v_3 ] \wedge \bar{c}
			\geq [ v_2 \wedge v_3 ] - \bar{c} 
		\end{equation*}
		%
		conditional on trade. So the expected budget surplus is bounded below by
		%
		\begin{equation*}
			\mathbb{E}( [ v_2 \wedge v_3 ] - \bar{c} | v_2 \vee v_3 \geq c )
			\geq \mathbb{E}( [ v_2 \wedge v_3 ] - \bar{c} )
			\geq 0 .
		\end{equation*}
		
		So the gVCG mechanism runs an expected budget surplus, and we can make the mechanism exactly ex ante budget balanced by distributing the expected revenue among players.
		
		
		\item By analogous reasoning, the expected budget surplus is bounded below by
		\begin{equation*}
			\mathbb{E}\left( v^{(2)} - \bar{c} \right),
		\end{equation*}
		where $v^{(2)}$ is the second order statistic (the second-highest value). Since $v \in (\bar{c},\bar{v})$ with positive probability under $F$, the expected value $\mathbb{E} v^{(2)}$ can be made arbitrarily close to $\bar{v}$ by taking $N$ large enough; in particular, $N$ can be chosen large enough to make it higher than $\bar{c}$.
	\end{enumerate}
\fi




%%-----------------------------------------------------------------------------------------------------

\end{document}
