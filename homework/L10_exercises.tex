%%% License: Creative Commons Attribution Share Alike 4.0 (see https://creativecommons.org/licenses/by-sa/4.0/)


%%%%%%%%%%%%%%%%%%%%%%%%%%%%%%%%%%%%%%%%%

%----------------------------------------------------------------------------------------
%	PACKAGES AND OTHER DOCUMENT CONFIGURATIONS
%----------------------------------------------------------------------------------------

\documentclass[a4paper]{article}

\usepackage{amssymb}
\usepackage{enumitem}
\usepackage[usenames,dvipsnames]{color}
\usepackage{fancyhdr} % Required for custom headers
\usepackage{lastpage} % Required to determine the last page for the footer
\usepackage{extramarks} % Required for headers and footers
\usepackage[usenames,dvipsnames]{color} % Required for custom colors
\usepackage{graphicx} % Required to insert images
\usepackage{listings} % Required for insertion of code
\usepackage{courier} % Required for the courier font
\usepackage[table]{xcolor}
\usepackage{amsfonts,amsmath,amsthm,parskip,setspace,url}
\usepackage[section]{placeins}
\usepackage[a4paper]{geometry}
\usepackage[USenglish]{babel}
\usepackage[utf8]{inputenc}
\usepackage{tikz}


% Margins
\topmargin=-0.45in
\evensidemargin=0in
\oddsidemargin=0in
\textwidth=6.5in
\textheight=9.0in
\headsep=0.6in

\linespread{1.1} % Line spacing



%----------------------------------------------------------------------------------------
%   FORMATTING
%----------------------------------------------------------------------------------------
% Set up the header and footer
\pagestyle{fancy}
\lhead[c]{\textbf{{\color[rgb]{.5,0,0} K{\o}benhavns\\Universitet }}} % Top left header
\chead{\textbf{{\color[rgb]{.5,0,0} \Class }}\\ \hmwkTitle  } % Top center head
\rhead{\instructor \\ \theprofessor} % Top right header
\lfoot{\lastxmark} % Bottom left footer
\cfoot{} % Bottom center footer
\rfoot{Page\ \thepage\ of\ \protect\pageref{LastPage}} % Bottom right footer
\renewcommand\headrulewidth{0.4pt} % Size of the header rule
\renewcommand\footrulewidth{0.4pt} % Size of the footer rule


% Other formatting stuff
%\setlength\parindent{12pt}
\setlength{\parskip}{5 pt}
%\theoremstyle{definition} \newtheorem{ex}{\textbf{\Large{Exercise & #}\\}}
\usepackage{titlesec}
\titleformat{\section}[hang]{\normalfont\bfseries\Large}{Problem \thesection:}{0.5em}{}




%----------------------------------------------------------------------------------------
%	NAME AND CLASS SECTION
%----------------------------------------------------------------------------------------
\newcommand{\hmwkTitle}{Exercises after Lecture 10 (M7)} % Assignment title
\newcommand{\Class}{Mechanism Design} % Course/class
\newcommand{\instructor}{Fall 2024} % TA
\newcommand{\theprofessor}{Prof. Egor Starkov} % Professor




%----------------------------------------------------------------------------------------
%   SOLUTIONS
%----------------------------------------------------------------------------------------
\newif\ifsolutions
\solutionstrue




\begin{document}

\begin{center}
		\LARGE\textbf{Exercises after Lecture 10 (M7):\\ Matching models.}
\end{center}



\section{Solve your own problem}
	%TODO: in part 4 ask to identify a deviation if it exists, not just give y/n answer
	
	This problem is meant to demonstrate the power of DA algorithm, which finds a stable matching in \emph{any} marriage market. 
	
	Consider a market with four men and four women. Come up with arbitrary preferences for all players (i.e., a ranking for each player of all players on the other side of the market and the option to stay single).
	\begin{enumerate}
		\item Find a stable matching generated by men-proposing DA algorithm.
		\item Find a stable matching generated by women-proposing DA algorithm.
		\item Are there any other stable matchings?
		\item Suppose a men-proposing DA algorithm is run. Is there a profitable deviation for any of the women -- i.e., can any woman misreport her preferences to the mechanism to improve her matching? If yes, show it; if not, explain why.
		
		(\emph{Hint: such a deviation exists if and only if you have more than one stable matching, which happens if and only if the outcomes of W-DA and M-DA algorithms are different.})
	\end{enumerate}
	
	
\ifsolutions
\section*{Solution}
	We consider a marriage market with four men, denoted
	$M=\left\{ m_{1},m_{2},m_{3},m_{4}\right\} $, and four women, denoted
	$W=\left\{ w_{1},w_{2},w_{3},w_{4}\right\} $. We assume the following
	arbitrary ordinal prefererences of men over women (and the option
	of remaining single, denoted by the name of the player himself) and
	of women over men (and the same option):
	\begin{align*}
		m_{1}:w_{1}\succ_{m_{1}}w_{2}\succ_{m_{1}}m_{1}\succ_{m_{1}}w_{3}\succ_{m_{1}}w_{4}\;\;\; & \;\;\;w_{1}:m_{4}\succ_{w_{1}}m_{3}\succ_{w_{1}}w_{1}\succ_{w_{1}}m_{2}\succ_{w_{1}}m_{1}\\
		m_{2}:w_{4}\succ_{m_{2}}w_{1}\succ_{m_{2}}w_{2}\succ_{m_{2}}m_{2}\succ_{m_{2}}w_{3}\;\;\; & \;\;\;w_{2}:m_{3}\succ_{w_{2}}m_{4}\succ_{w_{2}}w_{2}\succ_{w_{2}}m_{2}\succ_{w_{2}}m_{1}\\
		m_{3}:w_{3}\succ_{m_{3}}w_{4}\succ_{m_{3}}w_{1}\succ_{m_{3}}w_{2}\succ_{m_{3}}m_{3}\;\;\; & \;\;\;w_{3}:m_{1}\succ_{w_{3}}m_{4}\succ_{w_{3}}m_{3}\succ_{w_{3}}w_{3}\succ_{w_{3}}m_{2}\\
		m_{4}:m_{4}\succ_{m_{4}}w_{3}\succ_{m_{4}}w_{4}\succ_{m_{4}}w_{1}\succ_{m_{4}}w_{2}\;\;\; & \;\;\;w_{4}:m_{4}\succ_{w_{4}}m_{2}\succ_{w_{4}}m_{3}\succ_{w_{4}}w_{4}\succ_{w_{4}}m_{1}.
	\end{align*}
	
	\textbf{(1)} We proceed to find the matching $\mu_{MDA}:M\cup W\rightarrow M\cup W$
	generated by the men-proposing deferred acceptance algorithm. At stage
	0, all men propose to their most preferred partner (or simply opt
	to remain single). Thus, $m_{1}$ proposes to $w_{1}$, $m_{2}$ proposes
	to $w_{4}$, $m_{3}$ proposes to $w_{3}$ and $m_{4}$ opts to remain
	single. The woman $w_{1}$ has one offer, but prefers remaining single,
	so she rejects $m_{1}$, while $w_{3}$ and $w_{4}$ also have one
	offer, which they hold on to. At stage 1, all men have outstanding
	offers (or have retired from the marriage market) except $m_{1}$,
	who proposes to $w_{2}$. She also prefers remaining single to marrying
	$m_{1}$, so she rejects his offer. At stage 2, $m_{1}$ is still
	the only man without an outstanding offer. He prefers remaining single
	to marrying either of the remaining women. Therefore, matching is
	finalised at this stage. The resulting matching is
	\[
	\mu_{MDA}=\left(\left(m_{1},m_{1}\right),\left(m_{2},w_{4}\right),\left(m_{3},w_{3}\right),\left(m_{4},m_{4}\right),\left(w_{1},w_{1}\right),\left(w_{2},w_{2}\right)\right).
	\]
	
	\textbf{(2)} We now find the matching $\mu_{WDA}:M\cup W\rightarrow M\cup W$
	generated by the women-proposing deferred acceptance algorithm. At
	stage 0, all women propose to their most preferred partner. Thus,
	$w_{1}$ proposes to $m_{4}$, $w_{2}$ proposes to $m_{3}$, $w_{3}$
	proposes to $m_{1}$ and $w_{4}$ proposes to $m_{4}$. The men $m_{3}$
	and $m_{1}$ have one offer each, however $m_{1}$ prefers remaining
	single so he rejects the offer from $w_{3}$, while $m_{3}$ holds
	on to the offer from $w_{2}$. The man $m_{4}$ has two offers. However,
	he prefers remaining single and rejects both. At stage 1, $w_{2}$
	has an outstanding offer, so she does nothing. The woman $w_{1}$
	proposes to $m_{3}$, $w_{3}$ proposes to $m_{4}$ and $w_{4}$ proposes
	to $m_{2}$. The latter now has one offer, which he holds on to, while
	$m_{4}$ still prefers remaining single, so that he rejects the offer
	from $w_{3}$. Finally, $m_{3}$ has two offers, one from $w_{1}$
	and one from $w_{2}$. He prefers $w_{1}$ and holds on to her offer.
	Accordingly, he rejects $w_{2}$. At stage 2, $w_{1}$ and $w_{4}$
	have outstanding offers, so they do nothing. The woman $w_{2}$ prefers
	remaining single over her other options. She therefore exits the marriage
	market. The woman $w_{3}$ proposes to $m_{3}$ who now has two offers.
	He prefers the offer from $w_{3}$ and rejects $w_{1}$. At stage
	3, $w_{1}$ is the only woman without outstanding offers. She prefers
	remaining single to her remaining options, so no new offers are made
	and matching is finalised. The resulting matching is 
	\[
	\mu_{WDA}=\left(\left(w_{1},w_{1}\right),\left(w_{2},w_{2}\right),\left(w_{3},m_{3}\right),\left(w_{4},m_{2}\right),\left(m_{1},m_{1}\right)\left(m_{4},m_{4}\right)\right).
	\]
	
	\textbf{(3)} First, we see that the set of singles is the same in both matchings ($\mu_{WDA}$ and $\mu_{MDA}$), as expected. Then, we observe that the two matchings are in fact identical, i.e. $\mu_{WDA}=\mu_{MDA}$. Because of this, this is the unique stable matching.
	
	\textbf{(4)} We only have one stable matching, therefore if
	a men-proposing deferred acceptance algorithm is run, there is no
	profitable deviation for any of the women. 
\fi



\section{College admissions}
	%TODO: part 2 of this question needs to be more explicit -- does it allude to C-DA being C-optimal [among stable] or to DA outcomes being pareto-optimal among all players?
	
	This problem demonstrates how marriage model can be extended to allow many-to-one matchings, which turns it into a ``college admissions model''.
	
	There is a market with four students $S = \{s_1, ..., s_4\}$ and three colleges $C= \{c_1, c_2, c_3\}$. College $c_1$ can admit two students (its \emph{quota} is $q_1=2$); the remaining two colleges can admit one student each ($q_2=q_3=1$). Players' preferences (ordinal rankings, written best to worst) are given by
	\begin{align*}
		\succ_{s_1}: &\ c_3, c_1, c_2	&	\succ_{c_1}: &\ s_1, s_2, s_3, s_4
		\\
		\succ_{s_2}: &\ c_2, c_1, c_3	&	\succ_{c_2}: &\ s_1, s_2, s_3, s_4
		\\
		\succ_{s_3}: &\ c_1, c_3, c_2	&	\succ_{c_3}: &\ s_3, s_1, s_2, s_4
		\\
		\succ_{s_4}: &\ c_1, c_2, c_3
	\end{align*}
	
	Your goal is to find a stable matching in this problem. The only difference from the marriage model we considered in class is that college $c_1$ can admit \emph{two} students. The trick is to represent the two available spots in $c_1$ as two independent players which have the same preferences over students and which rank equally against other colleges among the students.
	
	In particular, consider instead a market with the same four students but now four colleges $C' = \{c_{1.1}, c_{1.2}, c_2, c_3\}$ (each with quota $q_i=1$, as in the marriage model), and preferences are given by 
	\begin{align*}
		\succ_{s_1}: &\ c_3, c_{1.1}, c_{1.2}, c_2	&	\succ_{c_{1.1}}: &\ s_1, s_2, s_3, s_4
		\\
		\succ_{s_2}: &\ c_2, c_{1.1}, c_{1.2}, c_3	&	\succ_{c_{1.2}}: &\ s_1, s_2, s_3, s_4
		\\
		\succ_{s_3}: &\ c_{1.1}, c_{1.2}, c_3, c_2	&	\succ_{c_2}: &\ s_1, s_2, s_3, s_4
		\\
		\succ_{s_4}: &\ c_{1.1}, c_{1.2}, c_2, c_3	&	\succ_{c_3}: &\ s_3, s_1, s_2, s_4
	\end{align*}
	
	\begin{enumerate}
		\item Use the college-proposing DA algorithm to find a stable matching.
		\item Matching $\mu$ generated by the C-DA algorithm is $C'$-optimal. However, there is another matching $\mu' = \{ (c_1, s_2, s_4), (c_2, s_1), (c_3, s_3) \}$ that is strictly preferred to $\mu$ by all colleges in $C$. How can you explain this contradiction?
	\end{enumerate}


\ifsolutions
\section*{Solution}
	% See proof of Thm 5.10 in RS.
	%\begin{enumerate}
	%	\item $\mu = \{ (c_1, s_3, s_4), (c_2, s_2), (c_3, s_1) \}$.
	%	\item Matching $\mu'$ is not stable (we defined $X$-optimality as being the best for $X$ among \emph{stable matchings}).
	%\end{enumerate}
	
	\textbf{(1)} We want to find the matching $\mu$ generated
	by the college-proposing deferred acceptance algorithm, where the
	two available slots in $c_{1}$ are represented as two separate colleges
	$c_{1,1}$ and $c_{1,2}$.
	\begin{itemize}
		\item  At stage 0, all colleges make an offer
		of admission to their preferred students: $c_{1,1}$, $c_{1,2}$ and
		$c_{2}$ all make offers to $s_{1}$ while $c_{3}$ makes an offer
		to $s_{3}$. The latter has got one offer, and he holds on to it.
		The student $s_{1}$ prefers the offer from $c_{1,1}$, so he rejects
		the offers from $c_{1,2}$ and $c_{2}$.
		\item  At stage 1, $c_{3}$ and
		$c_{1,1}$ have outstanding offers, so they do nothing, while $c_{1,2}$
		and $c_{2}$ both make offers to $s_{2}$, who prefers the offer from
		$c_{2}$. In consequence, she rejects the offer from $c_{1,2}$.
		\item At stage 2, $c_{1,1}$, $c_{2}$ and $c_{3}$ all have outstanding offers,
		so they do nothing. The college $c_{1,2}$ makes and offer to $s_{3}$
		who prefers this offer to the one from $c_{3}$. In consequence, he
		rejects the latter college and holds on to the offer from $c_{1,2}$.
		\item A stage 3, $c_{3}$ is the only college without outstanding offers.
		It makes an offer to $s_{1}$, who prefers this to the offer from
		$c_{1,1}$. 
		\item At stage 4, the latter is the only college without outstanding
		offers. It makes an offer of admission to $s_{2}$, who prefers what
		she already had and rejects the offer. A stage 4, $c_{1,1}$ makes
		and offer to $s_{3}$ who prefers this to the offer from $c_{1,2}$.
		He therefore rejects the latter offer and accepts the one from $c_{1,1}$.
		\item Now, at stage 5, $c_{1,2}$ is the only college without outstanding
		offers. It makes and offer to $s_{4}$, who now has one offer. She
		holds on to it. 
		\item At stage 6, all colleges have outstnading offers,
		so no new offers are made. Therefore, matching is finalised. The resulting
		matching is
		\[
		\mu=\left(\left(c_{1,1},s_{3}\right),\left(c_{1,2},s_{4}\right),\left(c_{2},s_{2}\right),\left(c_{3},s_{1}\right)\right).
		\]
	\end{itemize}
	
	\textbf{(2)} %It is in fact surprising that there is some matching that all colleges prefer to the $C'$-weakly optimal matching $\mu$. However, $\mu$ is only $C'$-weakly optimal in the sense that there can be no matching that awards \emph{all} colleges a better student in \emph{every} position than it gets in $\mu$. There can be matchings however, that like $\mu'$ award some spots better students, and other spots the same students as in $\mu$.
	C-DA matching $\mu$ is strongly Pareto-optimal for colleges among \textbf{stable} matchings. The suggested matching $\mu'$ is not stable.
\fi



\section{Book giveaway}
% final2021_2. matching, open + monotonicity + corr.info
Djul has defended his Ph.D. and found a job. He looks back at the small library of books that he has assembled during his studies and decides that he does not need them as much any more. Therefore, he decides to give the books away to fellow Ph.D. students. Suppose there are $b \in \{1,...,B\}$ books and $i \in \{1,...,N\}$ interested students. Since $N > B$, Djul decides that it would be fair to limit the giveaway to one book per person. Let $\theta_{i,b}$ denote the valuation of student $i$ for book $b$ (privately known by student $i$). Assume that all students are economists who act in pure self-interest.

\begin{enumerate}
	\item Given that Ph.D. students are poor,\footnote{The story is taking place in the U.S.} and Djul himself now has a well-paying job, he would prefer to give the books away for free. Propose a mechanism that Djul could use to allocate the books among fellow students for free and in a way that would be Pareto optimal. 
	
	\item Suppose now that $N=6$, $B=4$, and the realized valuations are as given in Table \ref{table:books}. Calculate the allocation produced by your mechanism from part 1.
	\begin{table}[h]
		\begin{center}
			\begin{tabular}{| c || c | c | c | c | c | c |}
				\hline
				$\theta_{i,b}$ & $i=1$ & $i=2$ & $i=3$ & $i=4$ & $i=5$ & $i=6$
				\\ \hline\hline
				$b=1$ & $4$ & $4$ & $1$ & $8$ & $9$ & $9$
				\\ \hline
				$b=2$ & $0$ & $2$ & $4$ & $9$ & $5$ & $3$
				\\ \hline
				$b=3$ & $9$ & $5$ & $5$ & $2$ & $6$ & $4$
				\\ \hline
				$b=4$ & $7$ & $6$ & $0$ & $7$ & $2$ & $6$
				\\ \hline
			\end{tabular}
			\caption{Preferences for the Book Giveaway problem}
			\label{table:books}
		\end{center}
	\end{table}
	
	\item Does there exist a mechanism that allocates the books without transfers efficiently (i.e., in a welfare-maximizing way)? If yes: present a mechanism. If not: explain why.
	
	\item Djul has run your mechanism from part 1 and messaged people regarding who got which book, but lost his phone with all the notes and messages before actually giving any books away. He thus cannot remember which book was promised to which student. Each student, however, knows which book they were promised. How can Djul recover the promised allocation without running the whole mechanism again? (Propose a mechanism that relies on students' reports of the books they were promised and explain why it works.)
\end{enumerate}



\ifsolutions
\subsection*{Solution}
\begin{enumerate}
	\item There are a few alternatives. Djul could use the deferred acceptance algorithm with students proposing in a random order (analogous to random serial dictatorship in social choice). This could effectively be implemented via a ``first come-first serve'' rule. Since in this situation every student gets their most preferred book (out of those that were not most preferred by preceding students), there is no scope for Pareto-improving exchanges, hence the resulting book allocation is Pareto-optimal. Note, however, that it need not be welfare maximizing. E.g., let there be two students, two books, $\theta_{1,b} = (10,9)$ and $\theta_{2,b} = (10,1)$, and student $i=1$ gets to choose first. Then student $1$ gets book $1$ and student $2$ gets book $2$, which yields welfare $11$, but they could \emph{trade}, rather than just exchange, books (with student $2$ paying student $1$ any amount in $[1,9]$), to arrive at an allocation that yields welfare $19$.
	
	The same issue arises if we try to use the Top Trading Cycles algorithm with any arbitrary initial allocation -- the resulting allocation would be Pareto-efficient for the same reason, but not necessarily welfare-maximizing for the same reason.
	
	\item Take the DA algorithm, in which students select books in the order of their indices. Then student $1$ picks book $3$, student $2$ picks book $4$, student $3$ picks book $2$ (since $3$ was taken), and student $4$ picks book $1$.
	
	\item The arguments in part 1 suggest that the standard matching algorithms are not efficient. While we would typically resort to VCG to implement an efficient allocation, it is not an option in this case since the goal is to avoid payments. Using non-monetary transfers like time or effort would, strictly speaking, fulfill the goal (hence VCG would be an acceptable answer if a non-monetary implementation of transfers is specified), but it defeats the spirit of the problem, since the intent is to not impose extra burden on the students.
	
	At the same time, without transfers the allocation can not be implemented, which is easy to see from the IC conditions. The example given in part 1 with realized types $\theta_{1,b} = (10,9)$ and $\theta_{2,b} = (10,1)$ shows that DRM $(k^*,t=0)$ is not DSIC, since under this realized type profile $k^*$ prescribes that student $1$ gets book $2$, but they would prefer to misreport their valuation vector as, e.g., $\hat{\theta}_{1,b} =(11,1)$ in order to get book $1$, which they prefer more. The fact that DRM $(k^*,t=0)$ is not BIC follows from the same example if we assume that $\theta_{2,b} = (10,1)$ is the only type possible for player $2$.
	
	\item Consider the following mechanism: if every book is claimed by exactly one student, implement the reported allocation, otherwise burn all books in a book-burning van. This mechanism has an equilibrium in which all reports are truthful. To see this, note that no student $i$ has an incentive to report the book they like less than the one they were assigned (or report no book), since this cannot result in $i$ getting a better book. On the other hand, if $i$ reports a more preferred book $b$ than the one they were assigned, Pareto-optimality implies that this book is claimed by some other student $j$, who reports truthfully in equilibrium -- hence book $b$ is claimed by both $i$ and $j$, which triggers the burn clause in the mechanism, and neither of them gets any book. This outcome is worse for $i$ than getting the initially allocated book, hence this deviation is not profitable either. So none of the available deviations is profitable for $i$, and $i$ was arbitary, hence truthtelling is an equilibrium of the game induced by this mechanism.
\end{enumerate}
\fi



\section{Teams and sponsors}
% stable matchings, final 2023-1
There are four racing teams without sponsors in Formula 0: Abarth,
Bentley, Caterpillar, and Dacia (referred to respectively as A, B, C,
and D). There are three sponsors looking to invest in racing teams in
exchange for exposure and marketing activities: XFT Technologies,
Yggdrasil Finance, and Zero Energy (X, Y, and Z, respectively).
The sponsorship contracts in Formula 0 are standard in terms of sums and
responsibilities, but different teams and sponsors may prefer different
partners, due to differences in values and vision.

Both teams and sponsors are very secretive about their preferences, but
it is known that everyone has a very clear ranking over potential
partners (and each wants exactly one partner). Further, the following 
information has been revealed through interviews and leaks:
\begin{itemize}
	\item all three sponsors and all four teams want to find a partner more
	than to stay independent/uninvolved;
	\item sponsor Y is by far the most preferred option for teams B and D;
	\item sponsor Y itself favors B over the other teams;
	\item sponsor Z, conversely, said they had no talks with B or D, and
	among the other two teams, it is generally understood that they favor a
	deal with C;
	\item team C ranks Z over X, but it is not clear where Y stands in this
	ranking;
	\item sponsor X suffered from an internal leak, suggesting that their
	marketing team ranked C over D over B over A.
\end{itemize}
You are a correspondent for an autosport website, and your editor asked
you to write an article ``This is how the sponsorships will play out in
F0''. The information is incomplete, but your editor insists that it
allows to learn at least something about how the final matching will
look like.

\begin{enumerate}
	\item You decided to start by speculating. Complete the players'
	rankings (in any way that is consistent with the information provided) and find a
	stable matching in this market.
	
	\item Your editor is not happy with speculation and says you should not make
	any assumptions about preferences except for those given above. What can
	you answer to the following questions about the stable matching in this
	market, based only on the information given above:
	\begin{enumerate}
		\item Whom will Y sponsor?
		\item Who will sponsor team C?
		\item Who will sponsor team A?
	\end{enumerate}
	(Explain how you obtained your answers.)
\end{enumerate}


\ifsolutions
\subsection*{Solution}
\begin{enumerate}
	\item For part 1, the student should present a complete strict ranking for every player. The simplest way to then find a stable matching is to run a deferred acceptance algorithm. If so, the student should mention which side of the market proposes and which accepts, but the step-by-step solution of the algorithm is not required, and neither is a proof of why the resulting matching is stable, only the matching itself has to be presented. If the student does not use the DA algorithm, then it should be explained why the matching is stable. 
	
	Part 1 can be skipped if the student finds and characterizes the unique stable matching in part 2.
	
	\item We cannot run the DA algorithm to find a stable matching since we do not know the players' preferences perfectly, and hence cannot properly simulate the offers and the acceptance decisions. However, we can find a stable matching in a series of claims as follows:
	\begin{itemize}
		\item Y must sponsor B, since for any other matching, $(Y,B)$ would be a blocking pair (since both players are each other's top choices).
		\item Z must sponsor C by the same logic (once we remove Y and B from the market).
		\item The only remaining sponsor is then X, who will match with team D. This is because C is not available and D is X's second-best choice, while D prefers to match with X over staying unsponsored like all teams do (and Y,Z are unavailable by the arguments above).
		\item Finally, A is the only team left in the market with no sponsors, so A is unsponsored.
	\end{itemize}
	We conclude that the unique stable matching in the market is $\big\{ (A,\varnothing), (B,Y), (C,Z), (D,X) \big\}$.
\end{enumerate}
\fi


\end{document}
