%%% License: Creative Commons Attribution Share Alike 4.0 (see https://creativecommons.org/licenses/by-sa/4.0/)


%%%%%%%%%%%%%%%%%%%%%%%%%%%%%%%%%%%%%%%%%

%----------------------------------------------------------------------------------------
%	PACKAGES AND OTHER DOCUMENT CONFIGURATIONS
%----------------------------------------------------------------------------------------

\documentclass{article}

\usepackage{amssymb}
\usepackage{enumitem}
\usepackage[usenames,dvipsnames]{color}
\usepackage{fancyhdr} % Required for custom headers
\usepackage{lastpage} % Required to determine the last page for the footer
\usepackage{extramarks} % Required for headers and footers
\usepackage[usenames,dvipsnames]{color} % Required for custom colors
\usepackage{graphicx} % Required to insert images
\usepackage{listings} % Required for insertion of code
\usepackage{courier} % Required for the courier font
\usepackage[table]{xcolor}
\usepackage{amsfonts,amsmath,amsthm,parskip,setspace,url}
\usepackage[section]{placeins}
\usepackage[a4paper]{geometry}
\usepackage[USenglish]{babel}
\usepackage[utf8]{inputenc}


% Margins
\topmargin=-0.45in
\evensidemargin=0in
\oddsidemargin=0in
\textwidth=6.5in
\textheight=9.0in
\headsep=0.6in

\linespread{1.1} % Line spacing

%----------------------------------------------------------------------------------------
%	DOCUMENT STRUCTURE COMMANDS
%	Skip this unless you know what you're doing
%----------------------------------------------------------------------------------------

% Header and footer for when a page split occurs within a problem environment
\newcommand{\enterProblemHeader}[1]{
\nobreak\extramarks{#1}{#1 continued on next page\ldots}\nobreak
\nobreak\extramarks{#1 (continued)}{#1 continued on next page\ldots}\nobreak
}

% Header and footer for when a page split occurs between problem environments
\newcommand{\exitProblemHeader}[1]{
\nobreak\extramarks{#1 (continued)}{#1 continued on next page\ldots}\nobreak
\nobreak\extramarks{#1}{}\nobreak
}

\setcounter{secnumdepth}{0} % Removes default section numbers
\newcounter{homeworkProblemCounter} % Creates a counter to keep track of the number of problems

\newcommand{\homeworkProblemName}{}
\newenvironment{ex}[1][Problem \arabic{homeworkProblemCounter}]{ % Makes a new environment called homeworkProblem which takes 1 argument (custom name) but the default is "Problem #"
\stepcounter{homeworkProblemCounter} % Increase counter for number of problems
\renewcommand{\homeworkProblemName}{#1} % Assign \homeworkProblemName the name of the problem
\section{\homeworkProblemName} % Make a section in the document with the custom problem count
%\enterProblemHeader{\homeworkProblemName} % Header and footer within the environment
}{
%\exitProblemHeader{\homeworkProblemName} % Header and footer after the environment
}

\newcommand{\problemAnswer}[1]{ % Defines the problem answer command with the content as the only argument
\noindent\framebox[\columnwidth][c]{\begin{minipage}{0.98\columnwidth}#1\end{minipage}} % Makes the box around the problem answer and puts the content inside
}

\newcommand{\homeworkSectionName}{}
\newenvironment{homeworkSection}[1]{ % New environment for sections within homework problems, takes 1 argument - the name of the section
\renewcommand{\homeworkSectionName}{#1} % Assign \homeworkSectionName to the name of the section from the environment argument
\subsection{\homeworkSectionName} % Make a subsection with the custom name of the subsection
%\enterProblemHeader{\homeworkProblemName\ [\homeworkSectionName]} % Header and footer within the environment
}{
%\enterProblemHeader{\homeworkProblemName} % Header and footer after the environment
}

\newif\ifsolutions

%----------------------------------------------------------------------------------------
%----------------------------------------------------------------------------------------
%----------------------------------------------------------------------------------------
% Set up the header and footer
\pagestyle{fancy}
\lhead[c]{\textbf{{\color[rgb]{.5,0,0} K{\o}benhavns\\Universitet }} \firstxmark} % Top left header
\chead{\textbf{{\color[rgb]{.5,0,0} \Class }}\\ \hmwkTitle  } % Top center head
\rhead{\instructor \\ \theprofessor} % Top right header
\lfoot{\lastxmark} % Bottom left footer
\cfoot{} % Bottom center footer
\rfoot{Page\ \thepage\ of\ \protect\pageref{LastPage}} % Bottom right footer
\renewcommand\headrulewidth{0.4pt} % Size of the header rule
\renewcommand\footrulewidth{0.4pt} % Size of the footer rule

\setlength\parindent{12pt} % Removes all indentation from paragraphs







%----------------------------------------------------------------------------------------
%	NAME AND CLASS SECTION
%----------------------------------------------------------------------------------------

\newcommand{\hmwkTitle}{Exercises: Lec 9} % Assignment title
\newcommand{\Class}{Mechanism Design} % Course/class
\newcommand{\instructor}{Fall 2019} % TA
\newcommand{\theprofessor}{Prof. Egor Starkov} % Professor

%\theoremstyle{definition} \newtheorem{ex}{\textbf{\Large{Exercise & #}\\}}
\setlength{\parskip}{5 pt}




















%%%%%%%%%%%%%%%%%%%%%%%%%%%%%%%%%%%%%%%%%%%%%%%%%%%%%%%%%%%%%%%%%%%%%%%%%%%%%%%%%%%%%%
\solutionsfalse
%\solutionstrue
%%%%%%%%%%%%%%%%%%%%%%%%%%%%%%%%%%%%%%%%%%%%%%%%%%%%%%%%%%%%%%%%%%%%%%%%%%%%%%%%%%%%%%


\begin{document}
	
These exercises are for your own practice and are not to be handed in. Some exercises are open ended in that they may not have a unique correct answer. If you think there is a typo in the problem, attempt to amend it and proceed as best you can.

%%------------------------------------------------------------------------------------------------

\begin{ex}[Review Questions]
	\begin{itemize}
		\item What are the literal and the metaphorical interpretations of information design?
		\item What are the two approaches to solving information design problems?
		\item What is the geometric intuition behind the concave closure representing the maximal payoff of the designer?
	\end{itemize}
	
\end{ex}



%%------------------------------------------------------------------------------------------------

\begin{ex}[P2: Information Design]
	Consider the following information design problem. There are two possible states, $\omega \in \{L,R\}$, the common prior belief that the state is $R$ is $\phi_0 = \mathbb{P}(\omega = R) = 1/2$. There is one player (receiver) and two actions $a \in \{u,d\}$ available to him. The receiver's payoffs as a function of state are given by the function $v(a,\omega)$, which is defined as
	\begin{center}
		\begin{tabular}{c | c | c |}
			$v(a,\omega)$ 		& $\omega = L$ 	& $\omega = R$ \\ \hline
			$a=u$	& $3$ 	& $0$	\\ \hline
			$a=d$	& $0$ 	& $1$	\\ \hline
		\end{tabular}
	\end{center}
	There is a designer who (before getting to observe $\omega$) designs an experiment that will send a message to the receiver, which may be informative about the true state $\omega$. The designer's payoff coincides with that of the receiver, with one exception: the designer receives a bribe of $4$ if action $a=d$ is chosen in state $\omega=L$. In other words, the designer's payoff function $\pi(a,\omega)$ is given by
	\begin{center}
		\begin{tabular}{c | c | c |}
			$\pi(a,\omega)$ 		& $\omega = L$ 	& $\omega = R$ \\ \hline
			$a=u$	& $3$ 	& $0$	\\ \hline
			$a=d$	& $4$ 	& $1$	\\ \hline
		\end{tabular}
	\end{center}
	
	\begin{enumerate}
		\item Derive the receiver's optimal action rule $\hat{a}(\phi)$, which maximizes his expected payoff, as a function of $\phi$, his posterior belief about the state after observing message $m$ generated by the experiment ($\phi = \mathbb{P} (\omega=R | m)$).
		\item Derive and plot the designer's payoff function $\Pi(\phi) \equiv \mathbb{E}_{\phi(\omega)} \left[\pi (\hat{a}(\phi), \omega)\right]$ as a function of the receiver's posterior $\phi$.
		\item Derive and plot (on the same graph) the concave closure $\Pi^* (\phi)$ of the receiver's payoff function $\Pi(\phi)$.
		\item By looking at the plots of $\Pi(\phi)$ and $\Pi^* (\phi)$ and recalling that $\phi_0 = 1/2$, answer the following: what is the set of posteriors $\{\phi_1, \phi_2, ...\}$ induced by the optimal experiment (the one that maximizes the designer's expected payoff)? What is the designer's payoff from the optimal experiment?
		\item Use the ``correlated equilibria approach'' to find the optimal experiment. In particular, find a decision rule $\sigma: \{L,R\} \to \varDelta(\{u,d\})$ (so $\sigma(u|\omega)+\sigma(d|\omega)=1$ for any $\omega$) which maximizes the designer's expected payoff as given by
		\begin{equation*}
			\pi^* (\sigma) \equiv  \sum_{a,\omega} \pi(a,\omega) \sigma (a | \omega) \phi(\omega)
		\end{equation*}
		subject to the obedience constraint: for any $a,a' \in \{u,d\}$,
		\begin{align*}
		\sum_{\omega} v (a, \omega) \sigma (a | \omega) \phi(\omega) 
		\geq \sum_{\omega} v_i (a', \omega) \sigma (a | \omega) \phi(\omega) .
		\end{align*} 
	\end{enumerate}
	

	\ifsolutions
	\subsection*{Solution}
	
	\fi
\end{ex}



%%------------------------------------------------------------------------------------------------




\end{document}
