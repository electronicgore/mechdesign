%%% License: Creative Commons Attribution Share Alike 4.0 (see https://creativecommons.org/licenses/by-sa/4.0/)


%%%%%%%%%%%%%%%%%%%%%%%%%%%%%%%%%%%%%%%%%

%----------------------------------------------------------------------------------------
%	PACKAGES AND OTHER DOCUMENT CONFIGURATIONS
%----------------------------------------------------------------------------------------

\documentclass[a4paper]{article}

\usepackage{amssymb}
%\usepackage{enumerate}
\usepackage[usenames,dvipsnames]{color}
\usepackage{fancyhdr} % Required for custom headers
\usepackage{lastpage} % Required to determine the last page for the footer
\usepackage{extramarks} % Required for headers and footers
\usepackage[usenames,dvipsnames]{color} % Required for custom colors
\usepackage{graphicx} % Required to insert images
\usepackage{listings} % Required for insertion of code
\usepackage{courier} % Required for the courier font
\usepackage[table]{xcolor}
\usepackage{amsfonts,amsmath,amsthm,parskip,setspace}
\usepackage[section]{placeins}
\usepackage[a4paper]{geometry}
\usepackage[USenglish]{babel}
\usepackage[utf8]{inputenc}
\usepackage{tikz}
\usepackage{hyperref}
\usepackage[hyphenbreaks]{breakurl}
\usepackage[]{url}
\usepackage[shortlabels]{enumitem}
\usepackage{framed}
\usepackage{pdfpages}


% Margins
\topmargin=-0.45in
\evensidemargin=0in
\oddsidemargin=0in
\textwidth=6.5in
\textheight=9.0in
\headsep=0.6in

\linespread{1.1} % Line spacing



%----------------------------------------------------------------------------------------
%   FORMATTING
%----------------------------------------------------------------------------------------
% Set up the header and footer
\pagestyle{fancy}
\lhead[c]{\textbf{{\color[rgb]{.5,0,0} K{\o}benhavns\\Universitet }}} % Top left header
\chead{\textbf{{\color[rgb]{.5,0,0} \Class }}\\ \hmwkTitle  } % Top center head
\rhead{\instructor \\ \theprofessor} % Top right header
\lfoot{\lastxmark} % Bottom left footer
\cfoot{} % Bottom center footer
\rfoot{Page\ \thepage\ of\ \protect\pageref{LastPage}} % Bottom right footer
\renewcommand\headrulewidth{0.4pt} % Size of the header rule
\renewcommand\footrulewidth{0.4pt} % Size of the footer rule


% Other formatting stuff
%\setlength\parindent{12pt}
\setlength{\parskip}{5 pt}
%\theoremstyle{definition} \newtheorem{ex}{\textbf{\Large{Exercise & #}\\}}
\usepackage{titlesec}
\titleformat{\section}[hang]{\normalfont\bfseries\Large}{Problem \thesection:}{0.5em}{}




%----------------------------------------------------------------------------------------
%	NAME AND CLASS SECTION
%----------------------------------------------------------------------------------------
\newcommand{\hmwkTitle}{Re-Exam} % Assignment title
\newcommand{\Class}{Mechanism Design} % Course/class
\newcommand{\instructor}{Fall 2020} % TA
\newcommand{\theprofessor}{Prof. Egor Starkov} % Professor




%----------------------------------------------------------------------------------------
%   SOLUTIONS
%----------------------------------------------------------------------------------------
\newif\ifsolutions
\solutionstrue




\begin{document}

{\ifsolutions \else
	\includepdf{MDreexam_frontpage20.pdf}
\fi}

\begin{center}
		\LARGE\textbf{Final re-exam {\ifsolutions solutions \fi}}
\end{center}


Write up your responses to questions below and submit them to Digital Exam. Be concise, but show your work and explain your answers. The deadline to submit the responses is Feb 14, 10:00 AM. No cooperation with other students is permitted.

Some questions are open ended in that they may not have a unique correct answer. It is recommended that you look through all problems before beginning to solve them. You are allowed to refer to textbooks, lecture notes, slides, problem sets etc for results and proofs contained therein.




\section{Meeting the investor}
	%corr, intuit, easy

	A team of two entrepreneurs $i=1,2$ approaches a venture investor $i=0$ with a request to fund their newest business idea. Your goal is to take on the role of the investor and to extract maximal profit for yourself from these naive youngsters.
	
	First, you must learn everything they know. Suppose the real value of the idea is $\omega \sim U[-\infty, \infty]$.\footnote{This is called an ``improper prior'' -- the prior belief about $\omega$ is not a proper probability distribution, but the posterior belief resulting from updating it via Bayes' rule after an informative event would be a proper distribution.}
	Each entrepreneur $i=1,2$ estimates this value at $\theta_i = \omega + \epsilon_i$, where $\epsilon_i \sim i.i.d.U[-0.5,0.5]$. (Take unit of measurement to be millions of dollars.)
	
	\begin{enumerate}
		\item Design a mechanism that would allow the investor to perfectly learn both entrepreneurs' estimates $\theta_i$. In this mechanism, both would independently report their $\theta_i$ to the mechanism, and the mechanism would prescribe report-contingent transfers $t_i(\theta_1,\theta_2)$ from each entrepreneur $i=1,2$ to the investor. Assume that the two entrepreneurs cannot coordinate their reporting strategies. Derive the transfer rules that implement truthful reporting (and show that it is indeed optimal for both entrepreneurs to report truthfully under this transfer rule).
		
		\item How could your mechanism be implemented in the real world? I.e., is it reasonable to ask entrepreneurs to pay for a meeting with an investor? If not, how else could you induce the desired transfers?
	\end{enumerate}


\ifsolutions
\subsection*{Solution}
	This problem tests the following learning outcomes:
	\begin{framed}
		\underline{Knowledge}:
		\begin{itemize}[$\circ$]
			\item {Account for the fundamental ideas behind and approaches to mechanism design.}
			\item {Define main trade-offs arising in information extraction problems.}
			\item {Identify the limitations of existing approaches to mechanism design.}
			\item \textcolor{gray}{Explain and discuss key theoretical concepts from academic articles, as well as discuss their interpretation.}
		\end{itemize}
		\underline{Skills}:
		\begin{itemize}[$\circ$]
			\item {Set up policy, trade, and management issues as mechanism design problems.}
			\item {Propose mechanisms that induce the desired outcomes in various environments.}
			\item {Analyze the drawbacks of existing institutions and suggest alternatives or identify reasons why none are available.}
		\end{itemize}
		\underline{Competencies}:
		\begin{itemize}[$\circ$]
			\item {Apply the most relevant theoretical apparatus to analyze case-based problems.}
			\item {Use the analytical framework of mechanism design in discussions of the real-world institutions, proposed policies, and market strategies.}
		\end{itemize}
	\end{framed}
	
	The solutions is as follows:
	\begin{enumerate}
		\item One way to proceed is to use the truth-revealing transfers we introduced while talking about the Cremer-McLean mechanism. Even though we only introduced this approach for settings with finite type spaces, it would be applicable in this problem -- but one would need to verify that the incentive compatibility constraints still hold.
		
		However, there is a simpler solution. Since $\theta_i \in [ \omega-0.5, \omega+0.5]$, it is also true that $\omega|\theta_i \in [\theta_i-0.5, \theta_i+0.5]$ and thus $\theta_j|\theta_i \in [\theta_i-1, \theta_i+1]$. In words, each entrepreneur $i$ knows for sure that their colleague's valuation $\theta_j$ is within $1$ of their own, because estimation errors $\epsilon_i,\epsilon_j$ are at most $0.5$ in absolute value. This means that we can use the following transfers in our mechanism:
		\begin{align*}
			t_i(\theta_1,\theta_2) = 
			\begin{cases}
				0 & \text{ if } \theta_j \in [\theta_i-1,\theta_i+1];
				\\
				\infty & \text{ if } \theta_j \notin [\theta_i-1,\theta_i+1].
			\end{cases}
		\end{align*}
		
		I.e., both $i$ and $j$ must pay infinity (or any other sufficiently large amount) to the mechanism in case their reports differ by more than $1$ [million dollars]. To see that this transfer rule induces truthtelling, consider $i$'s expected transfer given some own report $\hat{\theta}_i$ and under the assumption that $j$ reports truthfully (which is what happens in the desired equilibrium):
		\begin{align*}
			\mathbb{E} [t_i(\hat{\theta}_i,\theta_j) | \theta_i] &= 0 \cdot \mathbb{P} \{ |\theta_j-\hat{\theta}_i| \leq 1 | \theta_i \} + \infty \cdot \mathbb{P} \{ |\theta_j-\hat{\theta}_i| > 1 | \theta_i \}.
		\end{align*}
		The latter probability is zero (and so $i$ pays zero rather than infinity) if and only if $\hat{\theta}_i = \theta_i$, hence truthtelling is optimal for $i$ for any $\theta_i$.
		
		\item The payment itself can be non-monetary in principle -- e.g., by means of time and effort required to secure an appointment. However, the mechanism does not only require payment to meet the investor, but also requires that this payment is contingent upon the outcome of the meeting, which means that the aforementioned tools can not be easily used. 
		
		One way to interpret transfers is to say that they represent the reputation that these entrepreneurs will acquire in the venture capital circles after the meeting. If they are caught lying ($|\theta_1 - \theta_2|>1$), the investor will make sure to spread the news about it, and the entrepreneurs will have a harder time securing investment for this or other projects from other sources. 
		(It is not immediate whether the investor can commit to such behavior ex ante to incentivize agents -- which is crucial for the mechanism to be credible, -- but you can argue that the designer takes personal offense in having to waste their time listening to a bad pitch, and thus finds it incentive compatible to punish bad proposers.)
	\end{enumerate}
\fi



\section{Securing funding}
	% Problem VERY difficult if solved properly
	
	Continue the setting of Problem 1, but now assume that only one of the two entrepreneurs attends the meeting and speaks on behalf of the team.
	
	The entrepreneur privately knows the true valuation of the project they are proposing, $\theta \sim U[0,1]$. They are approaching the investor asking to fund the project, which costs $c$; the cost $c$ is commonly known by all players. The investor/designer would like to design an optimal funding contract, which yields the maximal expected profit under the constraints that it must be optimal for the entrepreneur to report $\theta$ truthfully, and that they must find it optimal to sign the contract as opposed to pursuing an outside option (approaching another, less interested investor), which yields a payoff of $\theta/2$.
	
	Define an allocation rule as a pair of functions $(k(\theta),l(\theta))$, where $k(\theta) \in [0,1]$ denotes the probability that the project is funded, and $l(\theta)$ denotes the share of profits of the final product claimed by the investor. Let transfer $t(\theta)$ represent the fixed repayment that the entrepreneur will have to pay back to the investor regardless of the project outcome. The resulting utility functions of the entrepreneur and the investor respectively are given by
	\begin{align*}
		u_e((k,l,t),\theta) &= k \cdot ((1-l)\theta - t)
		\\
		u_i((k,l,t),\theta) &= k \cdot (l \theta + t - c).
	\end{align*}
	
	Derive the optimal mechanism/contract for the investor. Explain this contract intuitively (does it resemble any real-world contracts?) How does your answer depend on $c$? 
	
	\emph{Hint: derive the appropriate versions of monotonicity of $(k,l)$ and of the envelope representation of payoffs. Use ERP to derive the optimal rule $(k,l)$ and verify that it satisfies monotonicity (or find the second-best rule if it does not). Use ERP and the IR constraint to derive the transfer rule that supports your chosen allocation rule.}


\ifsolutions
\subsection*{Solution}
	This problem tests the following learning outcomes:
	\begin{framed}
		\underline{Knowledge}:
		\begin{itemize}[$\circ$]
			\item {Account for the fundamental ideas behind and approaches to mechanism design.}
			\item {Define main trade-offs arising in information extraction problems.}
			\item \textcolor{gray}{Identify the limitations of existing approaches to mechanism design.}
			\item \textcolor{gray}{Explain and discuss key theoretical concepts from academic articles, as well as discuss their interpretation.}
		\end{itemize}
		\underline{Skills}:
		\begin{itemize}[$\circ$]
			\item {Set up policy, trade, and management issues as mechanism design problems.}
			\item {Propose mechanisms that induce the desired outcomes in various environments.}
			\item \textcolor{gray}{Analyze the drawbacks of existing institutions and suggest alternatives or identify reasons why none are available.}
		\end{itemize}
		\underline{Competencies}:
		\begin{itemize}[$\circ$]
			\item {Apply the most relevant theoretical apparatus to analyze case-based problems.}
			\item {Use the analytical framework of mechanism design in discussions of the real-world institutions, proposed policies, and market strategies.}
		\end{itemize}
	\end{framed}
	
	\textbf{NOTE}: the original solution of this problem made a number of mistakes that made it easier. The actual solution of this problem (presented below) is much more involved than was initially anticipated.
	
	Further, the problem is written in such a way that the entrepreneurs' outside option must be covered even if the project is not funded.\footnote{While possibly counterintuitive, think of this as mutually exclusive alternatives: the entrepreneurs can either try to convince this investor to fund their project, or they can secure an alternative contract netting them $\theta/2$, but the latter must be done immediately, and this option is lost if they waste time trying to convice the investor in the problem.}
	This is only possible if $k(\theta)t(\theta) \geq \theta/2$ even for such $\theta$ that $k(\theta)=l(\theta)=0$. Therefore, for such types we either need to allow payments $p(\theta) \equiv k(\theta)t(\theta)$ that are not conditional on funding (which would mean altering the problem conditions), or consider $\varepsilon$-optimal mechanisms, where $k(\theta) = \varepsilon$, and $t(\theta)$ is of the order $\frac{1}{\varepsilon}$ (but no exactly optimal mechanisms would exist, since they would require $\varepsilon \to 0$). This solution takes the former approach and instead of deriving $t(\theta)$, we shall restrict attention to unconditional/expected payments $k(\theta)t(\theta)$.
	
	\begin{enumerate}
		\item \emph{Monotonicity and ERP.} Consider two types $\theta > \hat{\theta}$. Using $U_e(\theta)$ to denote the expected utility that type $\theta$ of the entrepreneur team receives from truthtelling in the mechanism, the IC constraint for type $\theta$ to not be willing to report $\hat{\theta}$ instead can be written as:
		\begin{align*}
			U_e(\theta) \equiv k(\theta) \cdot \left( (1-l(\theta)) \theta - t(\theta) \right)
			& \geq 
			k(\hat{\theta}) \cdot \left( (1-l(\hat{\theta})) \theta - t(\hat{\theta}) \right)
			\\
			& = U_e(\hat{\theta}) + k(\hat{\theta}) (1-l(\hat{\theta})) (\theta -\hat{\theta})
			\\
			\Leftrightarrow \frac{U_e(\theta)-U_e(\hat{\theta})}{\theta-\hat{\theta}} & \geq k(\hat{\theta}) (1-l(\hat{\theta})).
		\end{align*}
		Doing the same for the reverse IC constraint (for $\hat{\theta}$ to not be willing to report $\theta$), we get
		\begin{align}
			k({\theta}) (1-l({\theta})) \geq \Leftrightarrow \frac{U_e(\theta)-U_e(\hat{\theta})}{\theta-\hat{\theta}} \geq k(\hat{\theta}) (1-l(\hat{\theta})).
			\label{eq:ERP_kl}
		\end{align}
		I.e., $k(\theta)(1-l(\theta))$ must be weakly increasing in $\theta$ to be implementable. We will refer to this as the monotonicity of the allocation rule.
		
		By taking limit of \eqref{eq:ERP_kl} as $\hat{\theta}\to \theta$, we also get that $U'(\theta) = k(\theta) (1-l(\theta))$ for almost all $\theta$. Invoking the fundamental theorem of calculus, we can then conclude that for any $\theta$ and $\bar{\theta}$,
		\begin{align}
			U_e(\theta) = U_e(\bar{\theta}) + \int_{\bar{\theta}}^\theta k(s) (1-l(s)) ds.
			\label{eq:ERP_kl3}
		\end{align}
		The crux is in selecting the right anchor type $\bar{\theta}$. The idea is to select it in such a way as to know what the utility level $U_e(\bar{\theta})$ is. The usual way to do so is to select (by guessing) some type for which IR binds, since then $U_e(\bar{\theta}) = \bar{U}_e(\bar{\theta})$. At least one such type should exist in the optimal mechanism, since otherwise all IR constraints are slack, in which case we can increase all $k(\theta)t(\theta)$ by $\epsilon$ and increase profit. Guessing this type is, however, difficult, in this problem, hence let us leave the anchor type $\bar{\theta}$ undefined for now (but we are assuming that this is the type for which the IR constraint binds).
		
		\item \emph{Expected revenue.} Using \eqref{eq:ERP_kl3} and the definition of $U_e(\theta)$, we get
		\begin{align}
			U_e(\theta) &= k(\theta) \cdot \left( (1-l(\theta)) \theta - t(\theta) \right) = U_e(\bar{\theta}) + \int_{\bar{\theta}}^\theta k(s) (1-l(s)) ds
			\nonumber
			\\
			\Leftrightarrow
			k(\theta) t(\theta) &= k(\theta) (1-l(\theta)) \theta - U_e(\bar{\theta}) - \int_{\bar{\theta}}^\theta k(s) (1-l(s)) ds
			\label{eq:kt}
			\\
			\Rightarrow
			\mathbb{E}_\theta [k(\theta) t(\theta)] &= \int_0^1 k(\theta) (1-l(\theta)) \theta d\theta - \int_0^1 \int_{\bar{\theta}}^\theta k(s) (1-l(s)) ds d\theta - U_e(\bar{\theta}).
			\label{eq:Ekt}
		\end{align}
		(where we have already plugged in $\phi(\theta)=1\Leftrightarrow F(\theta)=1$ in the third line).
		Focus on the double integral. Expanding the outer integral into two according to whether $\theta \gtrless \bar{\theta}$, and then using integration by parts on each term, we get
		\begin{align}
			\int_0^1 \int_{\bar{\theta}}^\theta k(s) (1-l(s)) ds d\theta
			&= -\int_0^{\bar{\theta}} \int^{\bar{\theta}}_\theta k(s) (1-l(s)) ds d\theta + \int_{\bar{\theta}}^1 \int_{\bar{\theta}}^\theta k(s) (1-l(s)) ds d\theta
			\nonumber
			\\
			&= -\int_0^{\bar{\theta}} \theta k(\theta) (1-l(\theta)) d\theta + \int_{\bar{\theta}}^1 (1-\theta) k(\theta) (1-l(\theta)) d\theta.
			\label{eq:IBP}
		\end{align}
		Plug \eqref{eq:IBP} into \eqref{eq:Ekt} to obtain
		\begin{align*}
			\mathbb{E}_\theta [k(\theta) t(\theta)] = \int_0^{\bar{\theta}} 2\theta k(\theta) (1-l(\theta)) d\theta + \int_{\bar{\theta}}^1 (2\theta-1) k(\theta) (1-l(\theta)) d\theta - U_e(\bar{\theta}).
		\end{align*}
	
		Therefore, the investor's expected profit in any IC mechanism will be given by the following function of the allocation rule:
		\begin{align}
			\mathbb{E}_\theta[U_i(\theta)] &= \mathbb{E}_\theta \left[ k(\theta) \cdot (l(\theta) \theta + t(\theta) - c) \right] 
			\nonumber
			\\
			&= \int_0^{\bar{\theta}} k(\theta) \left[ 2\theta(1-l(\theta)) + l(\theta) \theta - c  \right] d\theta + \int_{\bar{\theta}}^1 k(\theta) \left[ (2\theta-1)(1-l(\theta)) + l(\theta) \theta - c  \right] d\theta - U_e(\bar{\theta})
			\nonumber
			\\
			&= \int_0^{\bar{\theta}} k(\theta) \left[ 2\theta - \theta l(\theta) - c  \right] d\theta + \int_{\bar{\theta}}^1 k(\theta) \left[ 2\theta-1 + (1-\theta) l(\theta) - c  \right] d\theta - U_e(\bar{\theta}).
			\label{eq:EUi}
		\end{align}
	
		\item \emph{Allocation rule: attempt 1.} 
		Now we attempt the standard pointwise optimization, where for every $\theta$ we select the optimal allocations $k(\theta)$ and $l(\theta)$ (that maximize the resp. parts of \eqref{eq:EUi}). This yields the following:
		\begin{align*}
			\text{ if } \theta < \bar{\theta}: & (k(\theta),l(\theta)) = 
			\begin{cases}
				(0,\cdot) & \text{ if } \theta < \frac{c}{2};
				\\
				(1,0) & \text{ if } \theta \geq \frac{c}{2};
			\end{cases}
			\\
			\text{ if } \theta \geq \bar{\theta}: & (k(\theta),l(\theta)) = 
			\begin{cases}
				(0,\cdot) & \text{ if } \theta < c;
				\\
				(1,1) & \text{ if } \theta \geq c.
			\end{cases}
		\end{align*}
		(The optimal $l(\theta)$ is indeterminate when $k(\theta)=0$.)
		In the end, there are three different cases, depending on how $\bar{\theta}$ compares to $c/2$ and $c$.
		\begin{enumerate}
			\item If $\bar{\theta} < c/2$ then the implied optimal allocation is $k(\theta)=l(\theta) = \mathbb{I}\{\theta \geq c \}$ and from ERP we can get the optimal payment rule $k(\theta)t(\theta) = -U_e(\bar{\theta})$ for all $\theta$. But then the type for whom IR would be the most binding is $\bar{\theta}=1$, which contradicts $\bar{\theta} < c/2$ (unless $c>2$, which is the trivial case).
			
			\item If $c/2 \leq \bar{\theta} < c$ then the optimal mechanism prescribes $(k,l) = (1,0)$ if $\theta \in (c/2,\bar{\theta})$, $(k,l)=(1,1)$ if $\theta > c$, and $k=0$ otherwise. This allocation rule violates monotonicity, hence there does not exist a transfer rule that implements it.
			
			\item If $\bar{\theta} \geq c$ then the optimal mechanism prescribes $(k,l) = (1,0)$ if $\theta \in (c/2,\bar{\theta})$, $(k,l)=(1,1)$ if $\theta > \bar{\theta}$, and $k=0$ otherwise. This allocation rule also violates monotonicity.
		\end{enumerate}
		The above implies that $\bar{\theta} \geq c/2$ and that to obtain the optimal \emph{implementable} allocation, we must resort to other methods.
		
		\item \emph{Allocation rule: attempt 2.}
		The problem that we are now trying to solve is $\max_{k,l} \mathbb{E} U_i$ subject to IR conditions and monotonicity. 
		The objective function \eqref{eq:EUi} can be rewritten as
		\small
		\begin{align}
			\mathbb{E}_\theta[U_i(\theta)] 
			% &= \int_0^{\bar{\theta}} k(\theta) \left[ 2\theta - \theta l(\theta) - c  \right] d\theta + \int_{\bar{\theta}}^1 k(\theta) \left[ 2\theta-1 + (1-\theta) l(\theta) - c  \right] d\theta - U_e(\bar{\theta})
			%\nonumber
			%\\
			&= \int_0^{\bar{\theta}} \left[ \theta k(\theta)(1-l(\theta)) + k(\theta) (\theta - c) \right] d\theta + \int_{\bar{\theta}}^1 \left[ (\theta-1) k(\theta)(1-l(\theta)) + k(\theta) (\theta - c) \right] d\theta - \frac{\bar{\theta}}{2}.
			\label{eq:EUi2}
		\end{align}
		\normalsize
		where we rearranged the terms and used the fact $U_e(\bar{\theta}) = \frac{\bar{\theta}}{2}$. On the other hand, the IR conditions look as follows:
		\begin{align}
			k(\theta)(1-l(\theta)) \theta - k(\theta)t(\theta) &\geq \frac{\theta}{2}
			\nonumber
			\\
			\Leftrightarrow
			U_e(\bar{\theta}) + \int_{\bar{\theta}}^{\theta} k(s)(1-l(s)) ds & \geq \frac{\theta}{2}
			\nonumber
			\\
			\Leftrightarrow
			\int_{\bar{\theta}}^{\theta} k(s)(1-l(s)) ds & \geq \frac{\theta-\bar{\theta}}{2}
			\label{eq:IR}
		\end{align}
		where the second line invoked \eqref{eq:kt}, and the third line used $ U_e(\bar{\theta}) = \frac{\bar{\theta}}{2}$.
		
		Therefore, our goal is to find such functions $k(\theta),l(\theta)$ that maximize \eqref{eq:EUi2} subject to \eqref{eq:IR} being satisfied for all $\theta$ and subject to $k(\theta)(1-l(\theta))$ being weakly increasing in $\theta$. From looking attentively at \eqref{eq:EUi2} and \eqref{eq:IR}, we can make the following observations:
		\begin{itemize}
			\item For $\theta < \bar{\theta}$ we would like to set $k(\theta)(1-l(\theta))$ as large as possible (to maximize the objective). The IR constraint amounts to $\mathbb{E}\left[ k(s)(1-l(s)) \mid s \in [\theta,\bar{\theta}] \right] \leq 1/2$, which, by taking the limit $\theta \to \bar{\theta}$, implies that $k(\bar{\theta})(1-l(\bar{\theta})) \leq 1/2$. Note also that by monotonicity we have that $k(\theta)(1-l(\theta)) \leq k(\bar{\theta})(1-l(\bar{\theta}))$.
			\item For $\theta \geq \bar{\theta}$, by the mirror logic, the objective function dictates that we would like to minimize $k(\theta)(1-l(\theta))$, while the IR constraints and monotonicity imply that $k(\theta)(1-l(\theta)) \geq k(\bar{\theta})(1-l(\bar{\theta})) \geq 1/2$.
			\item In both cases above, we would like to maximize $k(\theta)$ when $\theta \geq c$ and minimize it otherwise, to maximize the objective. 
		\end{itemize}
		From the considerations above it looks like we want to set $k(\theta)=0$ for some very low $\theta$ (due to the last bullet point), and set $k(\theta)(1-l(\theta)) = 1/2$ for all other $\theta$, with $(k(\theta),l(\theta))=(1/2,0)$ if $\theta < c$ and $(k(\theta),l(\theta))=(1,1/2)$ if $\theta \geq c$. I do not, however, have a formal proof that this is indeed the optimal contract. Nonetheless, taking this structure for granted, we can optimize over the cutoffs between different cases and obtain the (possibly) optimal contract:
		\begin{align*}
			(k(\theta),l(\theta),k(\theta)t(\theta)) = \begin{cases}
				\left(0,\cdot,-\frac{c}{2}\right) & \text{ if } \theta \in \left[0,\frac{c}{2} \right),
				\\
				\left(\frac{1}{2},0,0\right) & \text{ if } \theta \in \left[\frac{c}{2},c \right),
				\\
				\left(1,\frac{1}{2},0\right) & \text{ if } \theta \in \left[c,1 \right].
			\end{cases}
		\end{align*}
		In this contract, the very worst projects are not funded but the entrepreneurs receive a ``participation award'' meant to incentivize them to participate in the mechanism. For average projects paying this participation award with money becomes too expensive, so instead the project is funded with some probability and all money is left to the entrepreneurs. The investor loses money on all such projects. However, if a project is truly worthy, it is funded for sure, and the profits are split equally between the entrepreneurs and the investor -- i.e., investor becomes a shareholder in the project.
		
		The weirdness of the contract for $\theta < c$ is explained by us trying to satisfy the implicit requirement of the mechanism being IR for all types. This, however, is not particularly justified in the story, since the investor is only interested in attracting entrepreneurs with profitable projects. So we can relax the requirement of mechanism being ``IR for all types'' to it being ``IR for all $\theta$ s.t. $k(\theta)>0$. In this case the mechanism will be the same for $\theta \geq c$, and we will have $(k(\theta),l(\theta),k(\theta)t(\theta)) = (0,\cdot,0)$ for $\theta < c$.
	\end{enumerate}
\fi

	
	
	
\section{Efficient dissolution}
	%gVCG, 2018 final, p3
	
	Fast forward a few years now.
	
	Suppose that our two entrepreneurs jointly own an asset (e.g., a facility). Under this ownership structure, each agent can use the facility half of the time. The value of using the facility full time for agent $i$ is $\theta_i \in [0, 1]$. These values are agents’ private information, and are independent draws from the uniform distribution. A mechanism designer considers making one of the	agents a single owner of the facility, that is, she considers three alternatives: making agent $i = 1, 2$ a single owner, and leaving the facility owned jointly. If agent i becomes the single owner, she can use the facility full time, but because of changing the ownership structure, she will be unable to use the facility for some initial period; in this case, her utility from using the facility is $\theta_i - 1/4$, and the other agent obtains zero utility. Under the joint ownership, the utility from using the facility is $\theta_i/2$ for both $i = 1,2$. Note that joint ownership is also the outside option here, because each agent can reject any change in ownership.
	
	\begin{enumerate}
		\item What is the set of allocations $K$?
		\item What is the efficient allocation rule $k^*(\theta)$?
		\item Find the transfer rule $t(\theta)$, such that the mechanism $(k^*,t)$ is Bayesian incentive compatible (BIC), interim individually rational (IIR), and maximizes the designer’s expected revenue over all efficient mechanisms with these properties.
	\end{enumerate}
	
\ifsolutions
\subsection*{Solution}
	This problem tests the following learning outcomes:
	\begin{framed}
		\underline{Knowledge}:
		\begin{itemize}[$\circ$]
			\item {Account for the fundamental ideas behind and approaches to mechanism design.}
			\item {Define main trade-offs arising in information extraction problems.}
			\item \textcolor{gray}{Identify the limitations of existing approaches to mechanism design.}
			\item \textcolor{gray}{Explain and discuss key theoretical concepts from academic articles, as well as discuss their interpretation.}
		\end{itemize}
		\underline{Skills}:
		\begin{itemize}[$\circ$]
			\item \textcolor{gray}{Set up policy, trade, and management issues as mechanism design problems.}
			\item {Propose mechanisms that induce the desired outcomes in various environments.}
			\item \textcolor{gray}{Analyze the drawbacks of existing institutions and suggest alternatives or identify reasons why none are available.}
		\end{itemize}
		\underline{Competencies}:
		\begin{itemize}[$\circ$]
			\item {Apply the most relevant theoretical apparatus to analyze case-based problems.}
			\item \textcolor{gray}{Use the analytical framework of mechanism design in discussions of the real-world institutions, proposed policies, and market strategies.}
		\end{itemize}
	\end{framed}
	
	The solution is presented below.
	\begin{enumerate}
		\item The set of alternatives $K=\{0,1,2\}$: joint ownership, sole ownership by 1, and sole ownership by 2.
		
		\item The sum of players' real utilities is
		%
		\begin{equation*}
			v_1(k,\theta)+v_2(k,\theta)
			= 
			\begin{cases}
				( \theta_1 + \theta_2 ) / 2	&\text{ if } k=0, \\
				\theta_1 - 1/4				&\text{ if } k=1, \\
				\theta_2 - 1/4				&\text{ if } k=2 .
			\end{cases}
		\end{equation*}
		%
		So the efficient (deterministic) allocation rules $k : [0,1]^2 \to \{0,1,2\}$ are those that satisfy
		%
		\begin{equation*}
			k(\theta)
			= 
			\begin{cases}
				2 &\text{ if } \theta_1-\theta_2 \in [-1,-1/2) ,
				\\
				0 &\text{ if } \theta_1-\theta_2 \in (-1/2,1/2) ,
				\\
				1 &\text{ if } \theta_1-\theta_2 \in (1/2,1] ,
			\end{cases}
		\end{equation*}
		%
		and choose either $2$ or $0$ when $\theta_1-\theta_2=-1/2$ and either $0$ or $1$ when $\theta_1-\theta_2=1/2$. (So there are technically four efficient allocations, but they differ only in how they break ties.)
		
		\item By a theorem from class, the mechanism with this property is the Krishna-Perry generalized VCG mechanism. First, we find the least charitable types of each player. Applying the definition:
		\begin{align*}
			\tilde{\theta}_i &= \arg \min_{\theta_i} \mathbb{E}_{\theta_{-i}} \left[ v_1(k,\theta)+v_2(k,\theta) - \underline{U}_i(\theta_i) \right]
		\end{align*}
		where $\underline{U}_i(\theta_i) = \theta/2$ is the utility $i$ receives from the outside option. Denote the expression inside the expectation as $S(\theta) \equiv v_1(k,\theta)+v_2(k,\theta) - \underline{U}_i(\theta_i)$; it then evaluates to 
		\begin{equation*}
			S(\theta)
			= 
			\begin{cases}
				\theta_{-i} - \theta_{i}/2 - 1/4 &\text{ if } \theta_i-\theta_{-i} \in [-1,-1/2) ,
				\\
				\theta_{-i}/2 &\text{ if } \theta_i-\theta_{-i} \in (-1/2,1/2) ,
				\\
				\theta_i/2 - 1/4 &\text{ if } \theta_i-\theta_{-i} \in (1/2,1] ,
			\end{cases}
		\end{equation*}
		
		
		
		To compute the expected surplus, we separate two cases: if $\theta_{i}\geq1/2$,
		$$S(\theta)=
		\begin{cases}
			\theta_{-i}/2 
			&\text{if $\theta_{-i} \in [\theta_i-1/2, 1]$} \\
			\theta_{i} / 2-1/4
			&\text{if $\theta_{-i} \in [0,\theta_i-1/2]$},
		\end{cases}$$
		and if $\theta_i<1/2$
		%$$S(\theta)=
		%\begin{cases}
		%	\theta_{-i} - 1/2
		%	&\text{if $\theta_{-i} \in [\theta_i+1/2, 1]$} \\
		%	\theta_{-i} - \theta_{i}/2 - 1/4
		%	&\text{if $\theta_{-i} \in [0,\theta_i+1/2]$}.
		%\end{cases}$$
		$$S(\theta)=
		\begin{cases}
			\theta_{-i} - \theta_{i}/2 - 1/4
			&\text{if $\theta_{-i} \in [\theta_i+1/2, 1]$} \\
			\theta_{-i}/2
			&\text{if $\theta_{-i} \in [0,\theta_i+1/2]$}.
		\end{cases}$$
		
		For the first case, we have
		\begin{equation*}
			\mathbb{E}_{\theta_{-i}}[S(\theta_i,\theta_{-i})] = \theta_i^2/4 - \theta_i/4 + 5/16
		\end{equation*}
		and for the second case, we have 
		\begin{equation*}
			\mathbb{E}_{\theta_{-i}}[S(\theta_i,\theta_{-i})] = \theta_i^2/4 - 3\theta_i/4 + 9/16
			%- \theta_i^2/2 + 1/8.
		\end{equation*}
		Minimizing both expressions over $\theta_i$ on their respective domains implies that $\tilde{\theta}_i=1/2$ is the least charitable type for either player $i$.
		
		
		Now plugging these LCTs into the definition of VCG transfers, we get
		
		\begin{equation*}
			t^{gVCG}_i(\theta)
			= 
			\begin{cases}
				- \left( \theta_{-i}/2 - 1/4 \right)
				&\text{if $\theta_i-\theta_{-i} \in [-1,-1/2)$} \\
				0
				&\text{if $\theta_i-\theta_{-i} \in [-1/2,1/2]$} \\
				\theta_{-i}/2
				&\text{if $\theta_i-\theta_{-i} \in (1/2,1]$} .
			\end{cases}
		\end{equation*}
		%
		That is: when $-i$ gets the asset, $i$ \emph{receives} the utility gain $\theta_{-i}/2 - 1/4$ of $-i$. When the asset is shared, no transfer is made. When $i$ gets the asset, $i$ \emph{pays} the utility loss $\theta_{-i}/2$ of $-i$.
	\end{enumerate}
\fi



\end{document}
