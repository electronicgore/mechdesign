%%% License: Creative Commons Attribution Share Alike 4.0 (see https://creativecommons.org/licenses/by-sa/4.0/)


%%%%%%%%%%%%%%%%%%%%%%%%%%%%%%%%%%%%%%%%%

%----------------------------------------------------------------------------------------
%	PACKAGES AND OTHER DOCUMENT CONFIGURATIONS
%----------------------------------------------------------------------------------------

\documentclass[a4paper]{article}

\usepackage[utf8x]{inputenc}
\usepackage{amssymb}
\usepackage{enumitem}
\usepackage[usenames,dvipsnames]{color}
\usepackage{fancyhdr} % Required for custom headers
\usepackage{lastpage} % Required to determine the last page for the footer
\usepackage{extramarks} % Required for headers and footers
\usepackage[usenames,dvipsnames]{color} % Required for custom colors
\usepackage{graphicx} % Required to insert images
\usepackage{listings} % Required for insertion of code
\usepackage{courier} % Required for the courier font
\usepackage[table]{xcolor}
\usepackage{amsfonts,amsmath,amsthm,parskip,setspace,url}
\usepackage[section]{placeins}
\usepackage[a4paper]{geometry}
\usepackage[USenglish]{babel}
\usepackage{tikz}


% Margins
\topmargin=-0.45in
\evensidemargin=0in
\oddsidemargin=0in
\textwidth=6.5in
\textheight=9.0in
\headsep=0.6in

\linespread{1.1} % Line spacing



%----------------------------------------------------------------------------------------
%   FORMATTING
%----------------------------------------------------------------------------------------
% Set up the header and footer
\pagestyle{fancy}
\lhead[c]{\textbf{{\color[rgb]{.5,0,0} K{\o}benhavns\\Universitet }}} % Top left header
\chead{\textbf{{\color[rgb]{.5,0,0} \Class }}\\ \hmwkTitle  } % Top center head
\rhead{\instructor \\ \theprofessor} % Top right header
\lfoot{\lastxmark} % Bottom left footer
\cfoot{} % Bottom center footer
\rfoot{Page\ \thepage\ of\ \protect\pageref{LastPage}} % Bottom right footer
\renewcommand\headrulewidth{0.4pt} % Size of the header rule
\renewcommand\footrulewidth{0.4pt} % Size of the footer rule


% Other formatting stuff
%\setlength\parindent{12pt}
\setlength{\parskip}{5 pt}
%\theoremstyle{definition} \newtheorem{ex}{\textbf{\Large{Exercise & #}\\}}
\usepackage{titlesec}
\titleformat{\section}[hang]{\normalfont\bfseries\Large}{Problem \thesection:}{0.5em}{}




%----------------------------------------------------------------------------------------
%	NAME AND CLASS SECTION
%----------------------------------------------------------------------------------------
\newcommand{\hmwkTitle}{Exercises after Lecture 11 (M5)} % Assignment title
\newcommand{\Class}{Mechanism Design} % Course/class
\newcommand{\instructor}{Fall 2021} % TA
\newcommand{\theprofessor}{Prof. Egor Starkov} % Professor




%----------------------------------------------------------------------------------------
%   SOLUTIONS
%----------------------------------------------------------------------------------------
\newif\ifsolutions
%\solutionstrue




\begin{document}

\begin{center}
		\LARGE\textbf{Exercises after Lecture 11 (M5):\\ Asymptotically inefficient dynamic mechanisms.}
\end{center}



\section{Parenting}
	
	This problem asks you to solve a simplified version of the model presented in Li, Powell, and Matouschek (2017). 
	For a change, we will consider a story that is different from theirs. Suppose that parents (the designer) are debating with their teenage kid (the player/agent) regarding the kid's possible career paths. In particular, in every period $t \in \{0,1,\dots,\infty\}$ the kid can take up one of three jobs, $a \in \{l,m,r\}$, which are as follows:
	\begin{itemize}
		\item $a=l$ means beginning/continuing on track to becoming a lawyer, whether it is studying or taking up appropriate jobs. This yields payoff $B$ to the parents and $b$ to the kid.
		
		\item $a=m$ means working at a local McDonald's™, which yields payoff $0$ to both parents and the kid. 
		
		\item $a=r$ means playing rock music. This yields payoff $B$ to the kid, who really enjoys playing rock music, but only yields payoff $b$ to the parents, who would prefer the kid to become a lawyer, but agree that rock music is better than working at McDonald's™.
	\end{itemize}
	The assumption in the above is that $B>b>0$.
	
	Being a good lawyer requires inspiration, which may or may not be present in any given period. Let the state $\theta_t \in \{0,1\}$ denote whether the kid has lawyer's inspiration in period $t$. If $\theta_t=0$ then $a=l$ is not a feasible choice -- i.e., it can not be chosen in period $t$.\footnote{You can think that choosing $a_t=0$ when $\theta_t=0$ yields utility $-\infty$ to both the kid and the parents. This is a bit too dramatic, but is good enough for the example.} The common belief of the parents and the kid is that $\theta_t$ is i.i.d. across periods, and $\mathbb{P}\{\theta_t=1\} = \phi$. Once period $t$ arrives, the kid privately observes $\theta_t$ before making a decision, but the parents do not observe $\theta_t$, at $t$ or afterwards.
	
	The parents can put a hard veto on whether their kid becomes a rock musician in any given period, $s_t \in \{0,1\}$. I.e., they can exclude $a=r$ from the choice set. If this happens ($s_t=1$), the kid then faces a choice between $a_t \in \{l,m\}$ if $\theta_t=1$ and $a_t = m$ (no choice) if $\theta_t=0$, while if the parents give the kid the freedom to choose ($s_t=0$), then the choice is between $a_t \in \{l,m,r\}$ if $\theta_t=1$ and $a_t \in \{m,r\}$ if $\theta_t=0$.
	
	Assume that both the parents and the kid are forward-looking and discount the future using discount factor $\delta \in (0,1)$.\footnote{One util tomorrow is worth $\delta$ utils today.} Your goal is to design an optimal veto strategy for the parents. 
	
	\begin{enumerate}
		\item Consider the parents' strategy of never imposing the veto power ($s_t=0$ for all $t$). What is the kid's optimal strategy then? (Note that a strategy must specify an action for every history -- i.e., for every period $t$, every realization of $\theta_t$, given every possible history of past states and actions.) Calculate the parents' expected discounted lifetime payoff $V_0^{free}$ from the kid following this optimal strategy. Calculate the kid's expected discounted lifetime payoff $V_1^{free}$.
		
		\emph{Note: we think of these payoffs as being estimated before the first state $\theta_0$ is revealed to the kid.}
		
		\item Consider the parents' strategy of always restricting the kid ($s_t=1$ for all $t$). Answer the same questions as in part 1: what is the kid's optimal strategy? What is the parents' payoff $V_0^{veto}$? What is the kid's payoff $V_1^{veto}$?
	\end{enumerate}
	
	Consider now the following strategy for the parents. In the first period, $t=0$, they give the kid the freedom of choice ($s_0=0$). If $a_0=l$, the parents never control the kid again $s_t=0$ for all $t \geq 1$). If $a_0=r$, the parents will always control the kid ($s_t=1$ for all $t \geq 1$). The intent is that this will incentivize the kid to choose $a_0=l$ if $\theta_0=1$. Assume also that parents can commit to such ``conditional veto'' strategy at $t=0$.
	
	\begin{enumerate}[resume]
		\item Derive the IC condition for the kid (for it to be optimal for them to choose $a_0=l$ if $\theta_0=1$ under the conditional veto strategy).
		
		\item Calculate the parents' expected utility from adopting the conditional veto strategy (assuming the kid's IC condition holds).
		
		\item Assume the following parameter values: $\phi=1/2$, $\delta=1/3$, $B=3$, $b=2$. Verify that the kid's IC condition holds.
		
		\item Given the parameter values from part 5: is this strategy better for the parents than laissez-faire (no veto ever) and/or permanent restriction, from parts 1 and 2 of this problem respectively?
		
		\item Given the parameter values from part 5: argue to the best of your ability what the parents' optimal strategy is. (If they can do better than conditional veto, explain how. If they can not, explain why.)
	\end{enumerate}
		
		
	
	
\ifsolutions
\section*{Solution}
	\begin{enumerate}
		\item Since in all $t$, restrictions imposed in periods after $t$ do not depend on period-$t$ choice $a_t$, the kid will simply choose $a_t$ that yields the largest instant payoff for them -- i.e., $a_t=r$ for all $t$ and $\theta_t$ and all past histories. Then
		\begin{align*}
			V_0^{free} &= b + \delta b + \delta^2 b + \dots 
			\\
			&= \frac{1}{1-\delta}b,
			\\
			\text{and }V_1^{free} &= \frac{1}{1-\delta} B.
		\end{align*}
		
		\item By the same logic, the kid will always the select the myopically best action among the available ones, which is $a_t = l$ if $\theta_t = 1$ and $a_t = m$ if $\theta_t = 0$. The expected payoffs are
		\begin{align*}
			V_0^{veto} &= (\phi B + (1-\phi) 0) + \delta \cdot (...) + \dots
			\\
			&= \frac{1}{1-\delta} \phi B,
			\\
			\text{and }V_1^{veto} &= \frac{1}{1-\delta} \phi b.
		\end{align*}
		
		\item We are looking at period $t=0$ and realized state $\theta_0 = 1$. The kid's choice is between playing $a_0 = l$ and receiving continuation value $V_1^{free}$ on the one hand and playing $a_0 = r$ and receiving continuation value $V_1^{veto}$ on the other. (Note that the latter is strictly better than playing $a_0=m$.) So the IC condition is:
		\begin{align}
			b + \delta V_1^{free} &\geq B + \delta V_1^{veto}
			\nonumber
			\\
			\iff (1-\delta) b + \delta B & \geq (1-\delta) B + \delta \phi b
			\nonumber
			\\
			\iff (1-\delta-\delta \phi) b & \geq (1-2\delta) B.
			\label{eq:kidIC}
		\end{align}
		
		\item The parents' expected utility is:
		\begin{align*}
			V_0^{cond} &= \phi \left( B + \delta V_0^{free} \right)+ (1-\phi) \left( b + \delta V_0^{veto} \right) 
			\\
			&= \frac{1}{1-\delta} \left[ \phi \left( (1-\delta)B + \delta b \right)+ (1-\phi) \left( (1-\delta)b + \delta \phi B \right) \right]
			\\
			&= \frac{1}{1-\delta} \left[ (1-\delta \phi) \phi B + (1-\phi - \delta + 2 \phi \delta) b \right]
		\end{align*}
	
		\item Plugging the numbers into \eqref{eq:kidIC}, we get:
		\begin{align*}
			\left(1-\frac{1}{3} - \frac{1}{3} \frac{1}{2} \right) 2 & \geq \left( 1 - 2\frac{1}{3} \right) 3
			\\
			\iff 1 & \geq 1,
		\end{align*}
		which is satisfied.
		
		\item Plugging the numbers into the three expressions, we get
		\begin{align*}
			V_0^{cond} &= \frac{27}{8} = 3.375
			\\
			V_0^{free} &= 3
			\\
			V_0^{veto} &= \frac{9}{4} = 2.25.
		\end{align*}
		Conditional veto is indeed better than the other two options.
		
		\item Conditional veto is the best the parents can do. They would prefer the kid to choose $a_t=l$ when $\theta_t=1$, but this is not myopically optimal for the kid -- hence this choice must be driven by variance in parents' continuation mechanism. Note, however, that \emph{free} and \emph{veto} are the most extreme policies -- i.e., the parents cannot cannot promise the kid any utility level higher than $V_1^{free}$ and lower than $V_1^{veto}$. As we saw in part 4, this maximal variance is barely sufficient to induce the desired behavior in a single period -- meaning that after that one period, no more incentives can be provided, because the kid is promised either eternal reward, or eternal punishment. It is also trivial that inducing the desired behavior in period $0$ is better for the parents than delaying this to some later period, and playing some other delegation strategy until then -- there are simply no reasons to delay the desired outcome.
	\end{enumerate}
\fi



\end{document}
