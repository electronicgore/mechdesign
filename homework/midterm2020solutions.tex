%%% License: Creative Commons Attribution Share Alike 4.0 (see https://creativecommons.org/licenses/by-sa/4.0/)


%%%%%%%%%%%%%%%%%%%%%%%%%%%%%%%%%%%%%%%%%

%----------------------------------------------------------------------------------------
%	PACKAGES AND OTHER DOCUMENT CONFIGURATIONS
%----------------------------------------------------------------------------------------

\documentclass[a4paper]{article}

\usepackage{amssymb}
%\usepackage{enumerate}
\usepackage[usenames,dvipsnames]{color}
\usepackage{fancyhdr} % Required for custom headers
\usepackage{lastpage} % Required to determine the last page for the footer
\usepackage{extramarks} % Required for headers and footers
\usepackage[usenames,dvipsnames]{color} % Required for custom colors
\usepackage{graphicx} % Required to insert images
\usepackage{listings} % Required for insertion of code
\usepackage{courier} % Required for the courier font
\usepackage[table]{xcolor}
\usepackage{amsfonts,amsmath,amsthm,parskip,setspace}
\usepackage[section]{placeins}
\usepackage[a4paper]{geometry}
\usepackage[USenglish]{babel}
\usepackage[utf8]{inputenc}
\usepackage{tikz}
\usepackage{hyperref}
\usepackage[hyphenbreaks]{breakurl}
\usepackage[]{url}
\usepackage{enumitem}


% Margins
\topmargin=-0.45in
\evensidemargin=0in
\oddsidemargin=0in
\textwidth=6.5in
\textheight=9.0in
\headsep=0.6in

\linespread{1.1} % Line spacing



%----------------------------------------------------------------------------------------
%   FORMATTING
%----------------------------------------------------------------------------------------
% Set up the header and footer
\pagestyle{fancy}
\lhead[c]{\textbf{{\color[rgb]{.5,0,0} K{\o}benhavns\\Universitet }}} % Top left header
\chead{\textbf{{\color[rgb]{.5,0,0} \Class }}\\ \hmwkTitle  } % Top center head
\rhead{\instructor \\ \theprofessor} % Top right header
\lfoot{\lastxmark} % Bottom left footer
\cfoot{} % Bottom center footer
\rfoot{Page\ \thepage\ of\ \protect\pageref{LastPage}} % Bottom right footer
\renewcommand\headrulewidth{0.4pt} % Size of the header rule
\renewcommand\footrulewidth{0.4pt} % Size of the footer rule


% Other formatting stuff
%\setlength\parindent{12pt}
\setlength{\parskip}{5 pt}
%\theoremstyle{definition} \newtheorem{ex}{\textbf{\Large{Exercise & #}\\}}
\usepackage{titlesec}
\titleformat{\section}[hang]{\normalfont\bfseries\Large}{Problem \thesection:}{0.5em}{}




%----------------------------------------------------------------------------------------
%	NAME AND CLASS SECTION
%----------------------------------------------------------------------------------------
\newcommand{\hmwkTitle}{Midterm} % Assignment title
\newcommand{\Class}{Mechanism Design} % Course/class
\newcommand{\instructor}{Fall 2020} % TA
\newcommand{\theprofessor}{Prof. Egor Starkov} % Professor




%----------------------------------------------------------------------------------------
%   SOLUTIONS
%----------------------------------------------------------------------------------------
\newif\ifsolutions
\solutionstrue




\begin{document}

\begin{center}
		\LARGE\textbf{Midterm assignment}
\end{center}



You have until 17:00 on Oct 29 to submit your responses on absalon. Group submissions are allowed and encouraged (no more than 5 people per group). If something in the assignment is too ambiguous or you think something is incorrect, email me.


\section{S{\o}ndre campus}

	There are currently talks at KU about moving the Faculty of Social Sciences from the Kommunehospitalet that we occupy now to S{\o}ndre campus, where some other faculties are currently located.\footnote{News article from Uniavisen (in Danish): \url{https://tinyurl.com/y4uwrefe}.}
	The costs and benefits of such a move are currently being evaluated. Some, however, see this whole discussion as a bargaining maneuver in the upcoming negotiations with the firm that owns the Kommunehospitalet and leases it to the university -- a credible threat of leaving may help the university bargain a better lease rate.
	
	Your mission is to frame this choice of whether SAMF should move as a mechanism design problem. The goal of the mechanism is to extract the information about costs and benefits of the potential move from the relevant parties. In particular, answer the following questions within this setting:
	
	\begin{enumerate}
		\item Who is the designer?
		\item What is the outcome in this setting? (Do we have access to transfers? Is the set of allocations $k$ given by simply $K = \{\text{move},\text{no move}\}$ or is it more multifaceted?)
		\item Who are the players?
		\item What information do the players have that is relevant to determining the optimal outcome/allocation?
		\item How would you model the players' utility functions? (Give a concrete example.)
		\item What criteria or conditions should the mechanism satisfy? 
		\item What would be the desirable outcome/allocation rule that you would want to implement with such a mechanism? How can you check whether this rule is, in fact, implementable?
		\item If you allowed for transfers: how would you proceed with designing transfers that support the chosen allocation rule? (You do not need to actually derive the transfers.)
		\item How would your mechanism work in the real world, in terms of organization and logistics?\footnote{Example: ``all faculty, staff, and students must post a note on the door of their office which would contain their \texttt{report of something}; a dedicated person will walk around and enter responses in an excel sheet, which will then be used to determine the outcome''.}
	\end{enumerate}

	NOTE: treat this as a real-life assignment from the university officials. Your goal is to give the best possible answer to the question they ask, NOT to frame the problem in the simplest way possible. That said, you should still be realistic and try to set up the problem in a way that would be tractable and doable given the resources available to a committee responsible for this decision.

\ifsolutions
\section*{Solution}
	
	This is an open-ended question, so many answers are possible. Below is one example.
	\begin{enumerate}
		\item The designer is the university leadership (rector), possibly proxied by a ``committee on moving''.
		
		\item It feels somewhat strange to include monetary transfers in this problem. While it may be fine to pay small amounts to students and faculty for completing a survey or pay departments to compile a report on a given issue, making these payments contingent on responses to provide incentives is the weird part, which will likely not be well accepted. Therefore, an outcome is simply an allocation.
		
		An allocation, however, is much richer than just a binary decision. In case of a move we would also need to decide how to allocate the spaces on S{\o}ndre campus between the departments and faculties, whether to build new lecture halls (or force the students and faculty to commute to CSS or N{\o}rre campus for classes), whether and how to merge the duplicating departments, etc. 
		
		\item The set of players should include anyone who possesses information relevant to the outcome, and has preferences regarding the outcome, which could prevent them from communicating this information to us truthfully. In this problem, this includes:
		\begin{itemize}[noitemsep]
			\item Students, faculty, and staff, who all have private valuations for the move.
			\item Future students, for the same reason. We obviously cannot include future students in our mechanism because we do not know who they are, so the next best option is to let the current students speak on their behalf.
			\item The university's building administration, which knows how much room capacity SAMF requires, how much capacity is available on S{\o}ndre campus, how much the exploitation of these rooms costs on both campuses, and how much it would cost to build more lecture halls on S{\o}ndre campus. It is not immediate that there exists any conflict preventing this information from being openly communicated, but there could be some. E.g., I believe that at the moment, these administrations exist separately on different campuses, so they may be opposed to merging or, conversely, they may actively prefer the merge.
			\item Some university departments (e.g., IT) can better estimate the cost of the actual process of moving. Potential conficts of interest are as above.
			\item The firm which owns the Kommunehospitalet and leases it to the university -- it likely has some understanding of what its outside option is in case the university leaves, and the company would have to find new tenants for this property. Note that it makes a lot of sense to allow monetary transfers with this particular player.
			\item ...
		\end{itemize}
		
		\item See above.
		
		\item For concreteness, let us suppose from this point onwards that the costs of the move can be evaluated by internal departments without any conflict and need for a mechanism. Then our set of players is narrowed down to two groups: stakeholders (students, faculty, and staff) and the propertyowner firm. The firm's utility can be modelled as:
		\begin{align*}
			u_f(x,\theta) = \begin{cases}
				-t_f & \text{ if no move};
				\\
				\theta_f & \text{ if move},
			\end{cases}
		\end{align*}
		where again the firm's type $\theta_f$ is its outside option relative to continuing the current agreement, and $t_f$ is the negative of the change in lease that KU pays for the Kommunehospitalet. The firm does not care about the details of the move if it happens.
		
		With stakeholders it is a little more difficult, since the details of the allocation now matter. What we can do is assume that the allocation can be split into a number of aspects $l \in \{1,...,L\}$ and each $k_l$ can be represented as a number. E.g., one aspect is ``will all classes be held in the same place after the move'', another is ``if yes, will this place be S{\o}ndre campus'', another is ``will there be place for a student bar'', etc. If we take this approach, then stakeholder $i$'s type can be represented as a vector of valuations for every aspect $\theta_i = (\theta_{i,1}, \theta_{i,2}, ..., \theta_{i,L})$, and then the utility can be approximated as $u_i(x,\theta) = \sum_{l=1}^L k_l \theta_{i,l}$.
		
		\item We obviously want the mechanism to be incentive compatible, at least in the Bayesian sense.\footnote{Large number of players is one excuse to switch from DSIC to BIC: heuristically, the more possible type profiles $\theta_{-i}$ other players collectively have, the less important every single individual case is for $i$'s expected utility at the point where they are making a decision.}
		Individual rationality is not an issue, once you realize that ``not responding to a survey'' is just a kind of response. While students and staff and faculty have an outside option of leaving KU and applying to another university, this is likely a costly option. You can, however, make a case for IR being a desirable condition to satisfy for future students (so they choose KU over other universities), as well as the firm that owns Kommunehospitalet (depending on how you model its utility function).
		Finally, since transfers are not a part of the problem (except when dealing with the firm), budget balance in not a particularly relevant requirement.
		
		\item The simplest answer: the desired allocation rule $k^d(\theta)$ should maximize the stakeholders' welfare, i.e., the sum of utilities net of the costs borne by the university. Note that this is not the efficient allocation according to the standard definition, since it ignores the firm's well-being. You can also present various social choice arguments saying that the sum of utilities is not the best measure to use here. E.g., depending on the university's priorities, you may want to assign different weights to students' well-being versus that of the faculty.
		
		After calculating this allocation rule, you can realize that this problem fits the general setting (but not the quasilinear or Euclidean, since we do not allow for transfers), so you can use weak monotonicity of the outcome function/allocation rule to test for implementability. (Again, weak monotonicity for the general setting, not the weak monotonicity for the Euclidean setting, even though $k$ is a vector of numbers!)
		
		\item I went with a mix, allowing transfers to/from the firm, but not the stakeholders. If the firm's utility entered the objective function, we could use the (g)VCG transfers to align the firm's objective with the derired objective. However, the firm's utility is not a part of the objective function, so the VCG does not work as is. There is no set recipe for this case, but you can employ the first principles to come up with a mechanism that may not necessarily be best, but is good enough. In particular, the firm must be unable to change its transfer without also altering the allocation rule. 
		
		The simplest (but not the only!) way to achieve this is to give the firm no power over rent. E.g., the university can determine its willingness to pay for the Kommunehospitalet given stakeholders' reports, and then present the firm with a take-it-or-leave-it offer.
		
		\item It is probably easiest to set up an online survey for the stakeholders, and no issues should arise there. Negotiations with the firm are a more subtle matter, since using a direct revelation mechanism may not be appealing for the firm if it does not believe in the university's power to commit to the mechanism, or if it thinks its report may be used against it in future negotiations. A take-it-or-leave-it-offer as suggested above solves this issue. 
	\end{enumerate}

\fi



\section{Divine intervention}

	The year is 854 AD. The place is Denmark. The reigning king Horik is challenged by his nephew Guttorm for the claim to the kingdom. Both know that a grand battle between them in inevitable, and both are praying to Odin and the rest of {\AE}sir to tilt the outcome of this battle in their favor. You are to assume the role of Odin and to decide the outcome of the battle.
	
	In particular, you are a mechanism designer dealing with two players $i=H,G$. Every player $i$ has a private type $\theta_i \sim U[0,1]$ ($\theta_H$ and $\theta_G$ are indepedent). An outcome is given by $x=(k,t)$, where $k=(k_H,k_G)$ is an allocation such that $k_i \in [-1,1]$ and $k_H + k_G \leq 0$, and $t$ is a vector of transfers. Players' payoffs are given by $u_i(x,\theta) = k_i \theta_i - t_i$. The outside options are normalized to zero.
	
	In our story, allocations represent who wins the battle: e.g., $(k_H,k_G)=(1,-1)$ means Horik wins while Guttorm loses and is killed in battle. Restriction $k_H + k_G \leq 0$ means that they cannot both win, but can both lose. Transfer $t_i$ need not mean money, but rather (the negative of) favor that the gods will show to $i$ outside of battle, in life or afterlife.
	Horik's and Guttorm's types represent their paltriness. Higher $\theta_i$ means that $i$ would very much prefer to become a king and is very afraid of death. Low $\theta_i$ means that $i$ values honorable death in combat almost as highly as reign over the realm. Finally, the ``outside option'' represents fleeing from the battle, retaining life but losing the kingdom. The level of ``favor'' $t_i$ in this case is normalized to zero (i.e., favor is measured relative to this scenario).
	
	For the first part of this problem, suppose that Odin \& the co-gods love a good battle (so do not want players to flee), but are otherwise benevolent and do not care about favor. They thus decide to use a generalized VCG mechanism to determine the outcome.
	
	\begin{enumerate}
		\item Find the efficient allocation rule $k^*(\theta) = \arg \max_k \{u_H(k,\theta) + u_G(k,\theta)\}$.
		\item Find the least charitable types $\tilde{\theta}_i$.
		\item Calculate the gVCG transfer rule $t^{gVCG}(\theta)$ that supports the efficient allocation rule.
	\end{enumerate}
	
	Suppose now instead that the gods are not benevolent, but try to minimize the amount of favor they owe to mortals -- i.e., to maximize $\mathbb{E}[t_H+t_G]$. That said, they still like to see a good battle, so ensuring that neither player decides to flee takes priority.
	
	\begin{enumerate}[resume]
		\item Find the optimal BIC mechanism that maximizes the expected ``revenue'' $\mathbb{E}_\theta[t_H(\theta)+t_G(\theta)]$ subject to the players' IR constraints.
	\end{enumerate}

\ifsolutions
\section*{Solution}

	\begin{enumerate}
		\item The efficient allocation is obviously (you may have a different tie-breaking
		rule)
		\begin{align*}
			k^{*}(\theta) & =\begin{cases}
				(1,-1) & \text{ if }\theta_{H}>\theta_{G};\\
				(0, 0) & \text{ if }\theta_{H}=\theta_{G};\\
				(-1,1) & \text{ if }\theta_{H}<\theta_{G}.
			\end{cases}
		\end{align*}
		
		\item The least charitable type is 
		\begin{align*}
			\tilde{\theta}_{i} & \equiv \arg\min_{\theta_{i}} \mathbb{E}_{\theta_{-i}} \left\{ \theta_{H} k^*_{H}(\theta) + \theta_{G} k^*_{G}(\theta) - 0 \right\} 
			\\
			& =\arg\min_{\theta_{i}}\mathbb{E}_{\theta_{j}} \left\{  \left(\theta_{i}-\theta_{j}\right) k_{i}^{*} (\theta) \right\} 
			\\
			& =0.5
		\end{align*}
	
		\item The gVCG transfer rule for $i$ is
		\begin{align*}
			t^{gVCG}_i(\theta_i,\theta_{-i}) &= \theta_j k_j^*(\tilde{\theta}_i,\theta_{j}) + \tilde{\theta}_i k_i^*(\tilde{\theta}_i,\theta_{j}) - \theta_j k_j^*({\theta}_i,\theta_{j}) - \underline{U}_i (\tilde{\theta}_i)
			\\
			&= |0.5-\theta_j| - \theta_j \cdot \text{sgn}(\theta_j - \theta_i)
			\\
			&= \begin{cases}
				|0.5-\theta_j| - \theta_j & \text{ if } \theta_i < \theta_j,
				\\
				|0.5-\theta_j| 			& \text{ if } \theta_j = \theta_i,
				\\
				|0.5-\theta_j| + \theta_j & \text{ if } \theta_i > \theta_j.
			\end{cases}
		\end{align*}
		I.e., if $\theta_j < 0.5$ then $i$ pays $0.5$ if he wins and $0.5-2\theta_j \gtrless 0$ if he loses; while if $\theta_j > 0.5$ then $i$ pays $2\theta_j - 0.5 > 0$ if he wins and receives $0.5$ if he loses.
		
		\item Invoke BIC ERP (we can use it since the problem fits the Euclidean setting):
		\begin{align}
			\bar{U}_i(\theta_i) &= \bar{U}_i(0) + \int_0^{\theta_i} \mathbb{E}_{\theta_{-i}} [k_i(s,\theta_{-i})] ds
		\end{align}
		and substitute the definition of the expected utility $\bar{U}_i(\theta_i)$ to get
		\begin{align}
			\mathbb{E}_{\theta_{-i}} [t_i(\theta_i,\theta_{-i})] 
			&= \theta_i \mathbb{E}_{\theta_{-i}} [k_i(\theta_i,\theta_{-i})] - \bar{U}_i(0) - \int_0^{\theta_i} \mathbb{E}_{\theta_{-i}} [k_i(s,\theta_{-i})] ds
			\\
			&= \int_0^1 \theta_i k_i(\theta_i,\theta_{-i}) \phi(\theta_{-i}) d\theta_{-i} - \int_0^1 \left( \int_0^{\theta_i} k_i(s,\theta_{-i}) ds \right) \phi(\theta_{-i}) d\theta_{-i} - \bar{U}_i(0).
		\end{align}
		The second line rewrites the expectations as integrals. Taking the expectation of both sides over $\theta_i$ as well and recalling that pdfs are $\phi(\theta_i)=1$ for $U[0,1]$ distribution, we get
		\begin{align}
			\mathbb{E}_{\theta} [t_i(\theta_i,\theta_{-i})] 
			&= \int_0^1 \int_0^1 \theta_i k_i(\theta_i,\theta_{-i}) d\theta_{-i} d\theta_{i} - \int_0^1 \int_0^1 \int_0^{\theta_i} k_i(s,\theta_{-i}) ds d\theta_{i} d\theta_{-i} - \bar{U}_i(0)
			\label{eq:opt4}
			\\
			&= \int_0^1 \int_0^1 \theta_i k_i(\theta_i,\theta_{-i}) d\theta_{-i} d\theta_{i} - \int_0^1 \int_0^1 (1-\theta_i) k_i(\theta_i,\theta_{-i}) d\theta_{i} d\theta_{-i} - \bar{U}_i(0)
			\label{eq:opt5}
			\\
			&= \int_0^1 \int_0^1 \left( \theta_i k_i(\theta) - (1-\theta_i) k_i(\theta) \right) d\theta_{i} d\theta_{-i} - \bar{U}_i(0)
			\label{eq:opt6}
			\\
			&= \int_0^1 \int_0^1 \left( (2\theta_i - 1) k_i(\theta) \right) d\theta_{i} d\theta_{-i} - \bar{U}_i(0).
			\label{eq:opt7}
		\end{align}
		In the above, from \eqref{eq:opt4} to \eqref{eq:opt5} we used integration by parts as in class, then combined the two integrals to get \eqref{eq:opt6} and simplified the resulting expression in \eqref{eq:opt7}. The total expected ``revenue'' (negative of the favor owed) from both players is then given by
		\begin{align}
			\mathbb{E}_\theta \left[ t_H(\theta)+t_G(\theta) \right] 
			&= \int_0^1 \int_0^1 \left[ (2\theta_H - 1) k_H(\theta) + (2\theta_G - 1) k_G(\theta) \right] d\theta_H d\theta_G - \bar{U}_G(0) - \bar{U}_H(0).
			\label{eq:opt8}
		\end{align}
	
		The goal is to maximize the expression above over $k$. To maximize the integral pointwise ($\theta$-by-$\theta$), we would want to set $k_i(\theta) = -1$ if $\theta_i < 0.5$ and $k_i(\theta) = 1$ if $\theta_i > 0.5$. The former is not a problem -- in particular, if both $\theta_H,\theta_G < 0.5$, then both Horik and Guttorm would die in battle.\footnote{According to legends, this was the actual outcome: ``A huge battle was fought which lasted for three days. King Horik I ``and the other kings'' were killed, as were Guttorm and a great many chiefs''. (\url{https://en.wikipedia.org/wiki/Horik_I\#Downfall})}
		The latter, however, is not possible whenever both $\theta_H,\theta_G>0.5$, since we have the constraint that $k_H+k_G \leq 0$ -- both cannot win at the same time. The constrained-optimal allocation that maximizes $(2\theta_H - 1) k_H(\theta) + (2\theta_G - 1) k_G(\theta)$ s.t. $k_H+k_G \leq 0$ would be giving the victory to the person with the higher $\theta_i$ when $\theta_H,\theta_G>0.5$. In the end, the allocation rule in the optimal mechanism is given by
		\begin{align*}
			k(\theta) & =\begin{cases}
				(1,-1) & \text{ if }\theta_{H}>\max\{\theta_{G},0.5\};\\
				(-1,1) & \text{ if }\theta_{G}>\max\{\theta_{H},0.5\};\\
				(-1,-1) & \text{ if }0.5\geq \max\{\theta_H,\theta_G\}.
			\end{cases}
		\end{align*}
		Monotonicity holds: $k_i(\theta_i,\theta_{-i})$ is weakly increasing in $\theta_i$ for all $\theta_{-i}$, hence $\mathbb{E}_{\theta_{-i}} [k_i(\theta_i,\theta_{-i})]$ is also weakly increasing in $\theta_i$, hence this allocation rule $k$ is indeed implementable in BNE. In fact, the former statement implies that $k$ is even implementable in dominant strategies.
		
		To calculate transfers, go back to ERP. You can actually use either BIC, or DSIC ERP to obtain transfers that will support $k$ in BNE or in dominant strategies, respectively. Let us take BIC ERP:
		\begin{align*}
			\mathbb{E}_{\theta_{-i}} [t_i(\theta_i,\theta_{-i})] 
			&= \theta_i \mathbb{E}_{\theta_{-i}} [k_i(\theta_i,\theta_{-i})] - \bar{U}_i(0) - \int_0^{\theta_i} \mathbb{E}_{\theta_{-i}} [k_i(s,\theta_{-i})] ds
			\\
			&= \begin{cases}
				\theta_i(-1) - \int_0^{\theta_i}(-1)ds - \bar{U}_i(0)	& \text{ if } \theta_i \leq 0.5;
				\\
				\theta_i (2\theta_i-1) - \int_0^{0.5}(-1)ds - \int_{0.5}^{\theta_i} (2s-1) ds - \bar{U}_i(0)	& \text{ if } \theta_i > 0.5;
			\end{cases}
			\\
			&= \begin{cases}
				- \bar{U}_i(0)	& \text{ if } \theta_i \leq 0.5;
				\\
				\theta_i^2 + \frac{1}{4} - \bar{U}_i(0)	& \text{ if } \theta_i > 0.5.
			\end{cases}
		\end{align*}
		This gives us the expected (over $\theta_{-i}$) transfer of $i$ given $\theta_i$, but how do can we get $t_i(\theta_i, \theta_{-i})$? A simple answer is: unless we care about ex post IR or BB, there is no real reason in BIC mechanisms to make $t_i$ contingent on $\theta_{-i}$. I.e., given $\theta_i$, for any $\theta_{-i}$ we can just set $i$'s transfer to be equal to\footnote{
			An alternative path is, as was mentioned previously, to use DSIC ERP to obtain transfers $t$ that support our $k$ in DS. If you did that, you would get
			$$t_i(\theta_i,\theta_{-i}) = 
			\begin{cases}
				- U_i(0,\theta_{-i})	& \text{ if } \theta_i \leq 0.5;
				\\
				2\max\{\theta_{-i}, 0.5\} - U_i(0,\theta_{-i})	& \text{ if } \theta_i > 0.5.
			\end{cases}$$
		} 
		$$t_i(\theta_i) = \mathbb{E}_{\theta_{-i}} [t_i(\theta_i,\theta_{-i})] = 
		\begin{cases}
			- \bar{U}_i(0)	& \text{ if } \theta_i \leq 0.5;
			\\
			\theta_i^2 + \frac{1}{4} - \bar{U}_i(0)	& \text{ if } \theta_i > 0.5.
		\end{cases}$$
		
		We are almost done. The only thing left is to pin down the utilities $\bar{U}_i(0)$, which would also pin down the transfers. According to \eqref{eq:opt8}, we want to set them as low as possible, but we also want to satisfy IR. Write out players' expected utility:
		\begin{align*}
			\bar{U}_i (\theta_i) &= \theta_i \mathbb{E}_{\theta_{-i}} [k_i(\theta_i,\theta_{-i})] - \mathbb{E}_{\theta_{-i}} [t_i(\theta_i,\theta_{-i})] 
			\\
			&= \begin{cases}
				-\theta_i + \bar{U}_i(0)	& \text{ if } \theta_i \leq 0.5;
				\\
				\theta_i^2 - \theta_i - \frac{1}{4} + \bar{U}_i(0)	& \text{ if } \theta_i > 0.5.
			\end{cases}
		\end{align*}
		It is minimized at $\theta_i=0.5$, meaning that if the IR constraint holds for type $\theta_i=0.5$ then it will also hold for all other types.\footnote{While this coincides with the least charitable type in this problem, this is purely by coincidence.}
		Setting the IR constraint to bind for that type ($\bar{U}_i(0.5)=0$), we obtain $\bar{U}_i(0) = 0.5$, so in the end, the transfers that support the favor-minimizing allocation $k$ in BNE are given by
		$$t_i(\theta_i) = 
		\begin{cases}
			- 0.5	& \text{ if } \theta_i \leq 0.5;
			\\
			\theta_i^2 - \frac{1}{4}	& \text{ if } \theta_i > 0.5.
		\end{cases}$$
	\end{enumerate}

\fi












%%-----------------------------------------------------------------------------------------------------

\end{document}
