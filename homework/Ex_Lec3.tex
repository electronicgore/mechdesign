%%% License: Creative Commons Attribution Share Alike 4.0 (see https://creativecommons.org/licenses/by-sa/4.0/)


%%%%%%%%%%%%%%%%%%%%%%%%%%%%%%%%%%%%%%%%%

%----------------------------------------------------------------------------------------
%	PACKAGES AND OTHER DOCUMENT CONFIGURATIONS
%----------------------------------------------------------------------------------------

\documentclass{article}

\usepackage{amssymb}
\usepackage{enumerate}
\usepackage[usenames,dvipsnames]{color}
\usepackage{fancyhdr} % Required for custom headers
\usepackage{lastpage} % Required to determine the last page for the footer
\usepackage{extramarks} % Required for headers and footers
\usepackage[usenames,dvipsnames]{color} % Required for custom colors
\usepackage{graphicx} % Required to insert images
\usepackage{listings} % Required for insertion of code
\usepackage{courier} % Required for the courier font
\usepackage[table]{xcolor}
\usepackage{amsfonts,amsmath,amsthm,parskip,setspace,url}
\usepackage[section]{placeins}
\usepackage[a4paper]{geometry}
\usepackage[USenglish]{babel}
\usepackage[utf8]{inputenc}


% Margins
\topmargin=-0.45in
\evensidemargin=0in
\oddsidemargin=0in
\textwidth=6.5in
\textheight=9.0in
\headsep=0.6in

\linespread{1.1} % Line spacing

%----------------------------------------------------------------------------------------
%	DOCUMENT STRUCTURE COMMANDS
%	Skip this unless you know what you're doing
%----------------------------------------------------------------------------------------

% Header and footer for when a page split occurs within a problem environment
\newcommand{\enterProblemHeader}[1]{
\nobreak\extramarks{#1}{#1 continued on next page\ldots}\nobreak
\nobreak\extramarks{#1 (continued)}{#1 continued on next page\ldots}\nobreak
}

% Header and footer for when a page split occurs between problem environments
\newcommand{\exitProblemHeader}[1]{
\nobreak\extramarks{#1 (continued)}{#1 continued on next page\ldots}\nobreak
\nobreak\extramarks{#1}{}\nobreak
}

\setcounter{secnumdepth}{0} % Removes default section numbers
\newcounter{homeworkProblemCounter} % Creates a counter to keep track of the number of problems

\newcommand{\homeworkProblemName}{}
\newenvironment{ex}[1][Problem \arabic{homeworkProblemCounter}]{ % Makes a new environment called homeworkProblem which takes 1 argument (custom name) but the default is "Problem #"
\stepcounter{homeworkProblemCounter} % Increase counter for number of problems
\renewcommand{\homeworkProblemName}{#1} % Assign \homeworkProblemName the name of the problem
\section{\homeworkProblemName} % Make a section in the document with the custom problem count
%\enterProblemHeader{\homeworkProblemName} % Header and footer within the environment
}{
%\exitProblemHeader{\homeworkProblemName} % Header and footer after the environment
}

\newcommand{\problemAnswer}[1]{ % Defines the problem answer command with the content as the only argument
\noindent\framebox[\columnwidth][c]{\begin{minipage}{0.98\columnwidth}#1\end{minipage}} % Makes the box around the problem answer and puts the content inside
}

\newcommand{\homeworkSectionName}{}
\newenvironment{homeworkSection}[1]{ % New environment for sections within homework problems, takes 1 argument - the name of the section
\renewcommand{\homeworkSectionName}{#1} % Assign \homeworkSectionName to the name of the section from the environment argument
\subsection{\homeworkSectionName} % Make a subsection with the custom name of the subsection
%\enterProblemHeader{\homeworkProblemName\ [\homeworkSectionName]} % Header and footer within the environment
}{
%\enterProblemHeader{\homeworkProblemName} % Header and footer after the environment
}

\newcommand{\join}{\vee}

\newcommand{\meet}{\wedge}

\newif\ifsolutions

%----------------------------------------------------------------------------------------
%----------------------------------------------------------------------------------------
%----------------------------------------------------------------------------------------
% Set up the header and footer
\pagestyle{fancy}
\lhead[c]{\textbf{{\color[rgb]{.5,0,0} K{\o}benhavns\\Universitet }} \firstxmark} % Top left header
\chead{\textbf{{\color[rgb]{.5,0,0} \Class }}\\ \hmwkTitle  } % Top center head
\rhead{\instructor \\ \theprofessor} % Top right header
\lfoot{\lastxmark} % Bottom left footer
\cfoot{} % Bottom center footer
\rfoot{Page\ \thepage\ of\ \protect\pageref{LastPage}} % Bottom right footer
\renewcommand\headrulewidth{0.4pt} % Size of the header rule
\renewcommand\footrulewidth{0.4pt} % Size of the footer rule

\setlength\parindent{0pt} % Removes all indentation from paragraphs







%----------------------------------------------------------------------------------------
%	NAME AND CLASS SECTION
%----------------------------------------------------------------------------------------

\newcommand{\hmwkTitle}{Exercises: Lec 3} % Assignment title
\newcommand{\Class}{Mechanism Design} % Course/class
\newcommand{\instructor}{Fall 2019} % TA
\newcommand{\theprofessor}{Prof. Egor Starkov} % Professor

%\theoremstyle{definition} \newtheorem{ex}{\textbf{\Large{Exercise & #}\\}}
\setlength{\parskip}{0 pt}




















%%%%%%%%%%%%%%%%%%%%%%%%%%%%%%%%%%%%%%%%%%%%%%%%%%%%%%%%%%%%%%%%%%%%%%%%%%%%%%%%%%%%%%
\solutionsfalse
%\solutionstrue
%%%%%%%%%%%%%%%%%%%%%%%%%%%%%%%%%%%%%%%%%%%%%%%%%%%%%%%%%%%%%%%%%%%%%%%%%%%%%%%%%%%%%%


\begin{document}
	
These exercises are for your own practice and are not to be handed in. Some exercises are open ended in that they may not have a unique correct answer. If you think there is a typo in the problem, attempt to amend it and proceed as best you can.

%%------------------------------------------------------------------------------------------------

\begin{ex}[Review Questions]
	\begin{itemize}
		\item What are the arguments for and against using Bayseian incentive compatibility as opposed to implementation in dominant strategies?
		\item What are the two mechanisms you can use to BIC-implement the efficient allocation? What is the difference between them?
		\item How would you proceed about implementing the efficient allocation with a BIC, IR, BB mechanism?
	\end{itemize}
	
\end{ex}



%%------------------------------------------------------------------------------------------------

\begin{ex}
	Figure out whether gVCG is DSIC or not.
	
	\ifsolutions
	\section*{Solution}
	
	\fi
\end{ex}



%%------------------------------------------------------------------------------------------------

\begin{ex}
	Prove the Myerson-Satterthwaite theorem by verifying that generalized VCG mechanism is not budget balanced in the bilateral trade setting.
	
	\ifsolutions
	\section*{Solution}
	
	\fi
\end{ex}



%%------------------------------------------------------------------------------------------------

\begin{ex}
	There is one indivisible item to be auctioned and two bidders, $i \in \{1,2\}$, with respective valuations $\theta_i$ which are independently and identically distributed, each uniformly on $[0,1]$.
	\begin{enumerate}
		\item What is the efficient allocation here?
		\item Describe the AGV mechanism that implements the efficient $k^*$ in this setting. Is it IR?
		\item Describe the gVCG mechanism (including the least charitable type) that implements the efficient $k^*$ in this setting. Is it BB?
	\end{enumerate}
	\ifsolutions
	\section*{Solution}
	
	\fi
\end{ex}



%%------------------------------------------------------------------------------------------------

\begin{ex}
	A seller possesses a single indivisible object for which there are two potential buyers.    Each buyer $i \in \{1,2\}$ has value
$v_i$ for the good and the seller has an opportunity cost $c$ from selling the good.  Utility is quasi-linear in money  so if buyer $i$ purchases the good at price $p$ his final utility is
$v_i-p$, and the seller's utility is $p-c$.
	Each agent has zero utility if he does not trade and zero is therefore also the reservation utility of each agent.

	Each $v_i$ is drawn independently from the same distribution $F$ which has support $[0,\bar v]$, and $c$ is drawn independently from a distribution $G$ which has support $[0,\bar c]$.  Assume that $F$ and $G$ satisfy all of the conditions necessary for the revenue
equivalence theorem and our characterization results in class.

	
	\medskip
	A mechanism consists of two functions $q(c, v_1, v_2)$ and $t(c, v_1, v_2)$ where, as a function of the agents' types, 
$q(c, v_1, v_2)$ is the allocation rule that gives the probabilities that the good will be allocated to each of the three agents and $t(c, v_1, v_2)$ gives the list of transfers paid to each of the three agents.
Say that a mechanism \emph{feasible} if it is Bayesian incentive compatible, interim individually rational, and ex-post exactly budget-balanced.

	
	\begin{enumerate}

		\item What is the efficient allocation rule?

		\item Assume $\bar c = \bar v$.  Show that there does not exist a feasible mechanism that implements the efficient allocation rule.

		\item Now assume $\bar v > \bar c$.  Show that the following is a sufficient condition for the existence of a fesible mechanism that
implements the efficient allocation rule.
		$$ \mathbb{E}(\min\{v_1, v_2\}) \geq \bar c. $$
		\item Again assume $\bar{v} > \bar{c}$, but now suppose that there are $N$ potential buyers with values drawn independently from $F$.  Prove that for any $F$ and $G$ there is a $\bar N$ such that whenever $N > \bar{N}$  there exists a feasible mechanism that
implements the efficient allocation rule.

	\end{enumerate}
	
	
	\ifsolutions
	\section*{Solution}
%	\begin{enumerate}[leftmargin=*,label=(\alph*)]

%		\item The efficient allocations must satisfy
%		\begin{align*}
%			q(c,v_1,v_2) = 
%			\begin{cases}
%				(1,0,0) &\text{ if } c > v_1,v_2 \\

%				(0,1,0) &\text{ if } v_1 > c,v_2 \\

%				(0,0,1) &\text{ if } v_2 > c,v_1.

%			\end{cases}
%		\end{align*}
%		Ties can be broken arbitrarily.

%		
%		\item Let's use the Krishna--Perry theorem. The least charitable types are $0$ for the buyers and $\bar{c}$ for the seller. The VCG mechanism with least charitable default types has the following transfers:

%		%

%		\begin{itemize}

%			
%			\item when no trade occurs, there are no transfers

%			
%			\item when trade occurs between the seller and buyer $i$, $i$ pays $v_{-i} \join c$, the seller receives $v_i \meet \bar{c}$, and $-i$ has transfer zero.

%			
%		\end{itemize}

%		
%		When $c=0$ and $v_i > v_{-i}$, trade occurs, with the buyer paying $v_{-i} < \bar{v} =\bar{c}$ and the seller receiving $\bar{c}$. So there is a budget deficit.

%		
%		Since the Krishna--Perry mechanism has maximal revenue among efficient, BIC and IIR mechanisms, this shows that no efficient mechanism is feasible.

%		
%		
%		\item Let's use the same mechanism. The budget surplus is zero conditional on no trade, and

%		%

%		\begin{equation*}

%		[ v_1 \meet v_2 ] \join c - [ v_1 \join v_2 ] \meet \bar{c}

%		\geq [ v_1 \meet v_2 ] - \bar{c} 

%		\end{equation*}

%		%

%		conditional on trade. So the expected budget surplus is bounded below by

%		%

%		\begin{equation*}

%		\E( [ v_1 \meet v_2 ] - \bar{c} | v_1 \join v_2 \geq c )

%		\geq \E( [ v_1 \meet v_2 ] - \bar{c} )

%		\geq 0 .

%		\end{equation*}

%		
%		So the Krishna--Perry mechanism runs an expected budget surplus. By a theorem from class, it follows that there is a feasible efficient mechanism.

%		
%		
%		\item By analogous reasoning, the expected budget surplus is bounded below by

%		%

%		\begin{equation*}

%		\E\left( v^{(2)} - \bar{c} \right) ,

%		\end{equation*}

%		%

%		where $v^{(2)}$ is the second order statistic (the second-highest value). Since $(\bar{c},\bar{v})$ has positive probability under $F$ ($F$ being absolutely continuous), this bound can be made arbitrarily close to $\bar{v} - \bar{c}$ by taking $N$ large enough; in particular, $N$ can be chosen large enough to make it non-negative.

%		
%	\end{enumerate}
	\fi
\end{ex}



%%------------------------------------------------------------------------------------------------

\end{document}
