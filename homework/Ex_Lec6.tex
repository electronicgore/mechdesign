%%% License: Creative Commons Attribution Share Alike 4.0 (see https://creativecommons.org/licenses/by-sa/4.0/)


%%%%%%%%%%%%%%%%%%%%%%%%%%%%%%%%%%%%%%%%%

%----------------------------------------------------------------------------------------
%	PACKAGES AND OTHER DOCUMENT CONFIGURATIONS
%----------------------------------------------------------------------------------------

\documentclass{article}

\usepackage{amssymb}
\usepackage{enumitem}
\usepackage[usenames,dvipsnames]{color}
\usepackage{fancyhdr} % Required for custom headers
\usepackage{lastpage} % Required to determine the last page for the footer
\usepackage{extramarks} % Required for headers and footers
\usepackage[usenames,dvipsnames]{color} % Required for custom colors
\usepackage{graphicx} % Required to insert images
\usepackage{listings} % Required for insertion of code
\usepackage{courier} % Required for the courier font
\usepackage[table]{xcolor}
\usepackage{amsfonts,amsmath,amsthm,parskip,setspace,url}
\usepackage[section]{placeins}
\usepackage[a4paper]{geometry}
\usepackage[USenglish]{babel}
\usepackage[utf8]{inputenc}



% Margins
\topmargin=-0.45in
\evensidemargin=0in
\oddsidemargin=0in
\textwidth=6.5in
\textheight=9.0in
\headsep=0.6in

\linespread{1.1} % Line spacing

%----------------------------------------------------------------------------------------
%	DOCUMENT STRUCTURE COMMANDS
%	Skip this unless you know what you're doing
%----------------------------------------------------------------------------------------

% Header and footer for when a page split occurs within a problem environment
\newcommand{\enterProblemHeader}[1]{
	\nobreak\extramarks{#1}{#1 continued on next page\ldots}\nobreak
	\nobreak\extramarks{#1 (continued)}{#1 continued on next page\ldots}\nobreak
}

% Header and footer for when a page split occurs between problem environments
\newcommand{\exitProblemHeader}[1]{
	\nobreak\extramarks{#1 (continued)}{#1 continued on next page\ldots}\nobreak
	\nobreak\extramarks{#1}{}\nobreak
}

\setcounter{secnumdepth}{0} % Removes default section numbers
\newcounter{homeworkProblemCounter} % Creates a counter to keep track of the number of problems

\newcommand{\homeworkProblemName}{}
\newenvironment{ex}[1][Problem \arabic{homeworkProblemCounter}]{ % Makes a new environment called homeworkProblem which takes 1 argument (custom name) but the default is "Problem #"
	\stepcounter{homeworkProblemCounter} % Increase counter for number of problems
	\renewcommand{\homeworkProblemName}{#1} % Assign \homeworkProblemName the name of the problem
	\section{\homeworkProblemName} % Make a section in the document with the custom problem count
	%\enterProblemHeader{\homeworkProblemName} % Header and footer within the environment
}{
	%\exitProblemHeader{\homeworkProblemName} % Header and footer after the environment
}

\newcommand{\problemAnswer}[1]{ % Defines the problem answer command with the content as the only argument
	\noindent\framebox[\columnwidth][c]{\begin{minipage}{0.98\columnwidth}#1\end{minipage}} % Makes the box around the problem answer and puts the content inside
}

\newcommand{\homeworkSectionName}{}
\newenvironment{homeworkSection}[1]{ % New environment for sections within homework problems, takes 1 argument - the name of the section
	\renewcommand{\homeworkSectionName}{#1} % Assign \homeworkSectionName to the name of the section from the environment argument
	\subsection{\homeworkSectionName} % Make a subsection with the custom name of the subsection
	%\enterProblemHeader{\homeworkProblemName\ [\homeworkSectionName]} % Header and footer within the environment
}{
	%\enterProblemHeader{\homeworkProblemName} % Header and footer after the environment
}

\newif\ifsolutions

%----------------------------------------------------------------------------------------
%----------------------------------------------------------------------------------------
%----------------------------------------------------------------------------------------
% Set up the header and footer
\pagestyle{fancy}
\lhead[c]{\textbf{{\color[rgb]{.5,0,0} K{\o}benhavns\\Universitet }} \firstxmark} % Top left header
\chead{\textbf{{\color[rgb]{.5,0,0} \Class }}\\ \hmwkTitle  } % Top center head
\rhead{\instructor \\ \theprofessor} % Top right header
\lfoot{\lastxmark} % Bottom left footer
\cfoot{} % Bottom center footer
\rfoot{Page\ \thepage\ of\ \protect\pageref{LastPage}} % Bottom right footer
\renewcommand\headrulewidth{0.4pt} % Size of the header rule
\renewcommand\footrulewidth{0.4pt} % Size of the footer rule

\setlength\parindent{12pt} % Removes all indentation from paragraphs







%----------------------------------------------------------------------------------------
%	NAME AND CLASS SECTION
%----------------------------------------------------------------------------------------

\newcommand{\hmwkTitle}{Exercises: Lec 6} % Assignment title
\newcommand{\Class}{Mechanism Design} % Course/class
\newcommand{\instructor}{Fall 2019} % TA
\newcommand{\theprofessor}{Prof. Egor Starkov} % Professor

%\theoremstyle{definition} \newtheorem{ex}{\textbf{\Large{Exercise & #}\\}}
\setlength{\parskip}{5 pt}




















%%%%%%%%%%%%%%%%%%%%%%%%%%%%%%%%%%%%%%%%%%%%%%%%%%%%%%%%%%%%%%%%%%%%%%%%%%%%%%%%%%%%%%
%\solutionsfalse
\solutionstrue
%%%%%%%%%%%%%%%%%%%%%%%%%%%%%%%%%%%%%%%%%%%%%%%%%%%%%%%%%%%%%%%%%%%%%%%%%%%%%%%%%%%%%%


\begin{document}
	
	
\ifsolutions
	\begin{center}
		
		{\Huge Problem Set for Lecture 6
			(with Solutions)}
	\end{center}
	
	\emph{The solution to Problem 2 below is brought to you by Christian Kastrup \& Nicolai Waldstr{\o}m. Notes in cursive added by Egor.}
\fi
	
These exercises are for your own practice and are not to be handed in. Some exercises are open ended in that they may not have a unique correct answer. If you think there is a typo in the problem, attempt to amend it and proceed as best you can.

%%------------------------------------------------------------------------------------------------

\begin{ex}[Review Questions]
	\begin{itemize}
		\item What are the possible dynamic linkages that might prevent us from splitting a dynamic mechanism design problem into a collection of independent static problems?
		\item How does the solution to the static problem of efficient implementation extend to the dynamic problem? What are the main challenges?
		\item In the optimal mechanism problems, why can we sometimes disregard the problem of extracting the information that the agent receives after the point of contracting and focus on the problem of extracting the information that he already has?
		%\item 
	\end{itemize}
	
\end{ex}



%%------------------------------------------------------------------------------------------------

\begin{ex}
	Consider a problem of climate change, modelled as follows. There are two generations (players) $i=1,2$ and two periods $t=1,2$; each player $i$ only exists in their respective period $t=i$.\footnote{The distinction between indices $i$ and $t$ is somewhat arbitrary in this problem. The problem tries to stick to the more appropriate of the two indices in all cases, but they are often interchangeable. The double index $i,t$ is collapsed to a single one wherever required.} 
	Each player has some type $\theta_i \sim \text{i.i.d.} U[0,1]$, and their utilities are given by:
	\begin{align*}
	u_1 (\theta_1,k_1) &= \theta_1 k_1 - k_1^2 - p_1
	\\
	u_2 (\theta_2,k_1,k_2) &= \theta_2 k_2 - (k_1+k_2)^2 - p_2,
	\end{align*}
	in periods when they exist, and zero in other periods.
	Here allocation $k_t \in \mathbb{R_+}$ represents the level of fossil fuels usage at time $t$, $\theta_i$ is the value that generation $i$ extracts from burning fossils at time $t=i$, and the quadratic term represents the global losses stemming from this usage. Note that this includes past usage: player $2$ suffers from usage $k_1$.
	
	Your goal is to derive the efficient mechanism. You can think of the designer as some international organization (e.g., the UN), and of transfers as some kind of globally agreed upon carbon tax. We will use the dynamic pivot mechanism.\footnote{This problem does not strictly fit into the general model we set up in class due to the explicit dependence of $u_2$ on $k_1$, but all same methods are still applicable. Further, you can omit dependence of all objects on $K_t$, since the set $K_t$ of feasible allocations does not change over time in this problem.}
	Follow closely the expressions on slide 172. Assume that there is no discounting ($\delta = 1$).
	\begin{enumerate}
		\item Write down the flow social surplus function for $t=2$. Write down the welfare function for $t=2$. Find the efficient allocation $k^*_2$ for the time-$2$ generation.
		\item Find the transfers $p^{DPM}_2$ for the time-$2$ generation as prescribed by the dynamic pivot mechanism. \emph{(Hint: player $2$ is the only player in existence from $t=2$ onwards. What externalities does he impose on other existing players? What externalities is he subject to? What does this imply for $p^{DPM}_2$?)}
		\item Write down the flow social surplus function for $t=1$. Write down the welfare function for $t=1$. (Keep in mind that it should account for the effect of $k_1$ on future generation.) Find the efficient allocation $k^*_1$ for the time-$1$ generation.
		\item Write down the marginal contributions of player $1$ in both periods. Find flow marginal contributions of player $1$. Find the optimal payments for player $1$.
		\item Verify that mechanism is ``online'', i.e., each player is only required to make payments in those periods in which they are alive.
		\emph{Note: as it turns out, if you solve the problem correctly, the mechanism is not, in fact, online -- see solutions.}
	\end{enumerate}
	
	\ifsolutions
	\subsection*{Solution (Christian \& Nicolai)}
	
	\emph{Note: this problem turned out more difficult than I expected, hence I allowed it to be solved with $k_t \in \mathbb{R}$ rather than $k_t \in \mathbb{R_+}$, i.e. ignoring the positivity constraints on $k_t$. Otherwise the problem is still solvable, but you have to keep track of many cases when computing welfare and payments. The allocations, however, are relatively simple, and are given by $k_2^* = (\frac{\theta_2}{2} - k_1)_+$ and $k_1^* = \frac{1}{2} \left( \sqrt{2 \theta_1} - 1 \right)_+$ respectively in that case. --Egor}
	
	\paragraph{1)}
	Since only generation 2 is alive at $t=2$ the flow social surplus function simply equals the utility function of generation 2, minus transfers:
	\begin{align*}
	w_{t=2}\left(\theta,k\right) = v_{2}(\theta_{2},k_{1},k_{2})=\theta_{2}k_{2}-(k_{1}+k_{2})^{2}.
	\end{align*}
	Furthermore, since the period 2 is also the last period of the game the welfare function at time 2 also equals the flow surplus at this time:
	\begin{align*}
	W_{t=2}\left(\theta,k\right)=\max_{k_{2}} w_{t=2}\left(\theta,k\right). 
	\end{align*}
	The derivative w.r.t. $k_2$ is:
	\begin{align*}
		\frac{\partial w_{t=2} \left( \theta,k \right) }{ \partial k_{2} } = \theta_{2} - 2 (k_{1}+k_{2}).
	\end{align*}
	Equating with zero and solving yields the optimal $k_2^*$:\footnote{If emissions are restricted to be positive, $k>0$, corner solutions are possible depending on the realizations of $\theta$. In the following we abstract from this and focus on inner solutions.}
	\begin{align*}
		\theta_{2}-2(k_{1}+k_{2}^{*})=0 \\
		\Leftrightarrow k_{2}^{*}=\frac{\theta_{2}-2k_{1}}{2} \\
		\Leftrightarrow k_{2}^{*}=\frac{1}{2}\theta_{2}-k_{1}
	\end{align*}
	
	
	\paragraph{2)}
	As player $2$ is the only player alive at $t=2$ he imposes no externalities on other players. As such, the prescribed payment $p_{2}^{DPM}$ is simply zero. To see this more formally, consider the marginal contribution of generation 2:
	\begin{align*}
	M_{i=2,t=2}(\theta,k)=W_{t=2}(\theta,k)-W_{t=2,-2}(\theta,k)
	\end{align*}
	Since generation 2 is the only one alive at t=2 it holds that $W_{t=2,-2}=0$. Therefore, the marginal contribution is simply equal to the welfare $W_{t=2}$. 
	The marginal contribution, $M_{2}$, is equal to the flow marginal contribution, $m_{2}$, since t=2 is the last time period. The payment at time 2 when applying the dynamic pivot mechanism is then given by:
	
	\begin{align*}
	p_2^*(\theta,k^*)&=v_{i=2}(\theta,k^*)-m_{i=2}(\theta,k^*) \\
	&= \theta_{2}k^*_{2}-(k^*_{1}+k^*_{2})^{2} -\left(\theta_{2}k_{2}^{*}-(k_{1}^{*}+k_{2}^{*})^{2}\right) \\
	&= 0 
	\end{align*}
	Where the latter equality holds as $v_{2}(\theta,k^*)=m_2(\theta,k^*)$
	
	
	\paragraph{3)}
	The flow social surplus equals the utility of player 1, as he is the only player alive at time $t=1$:
	\begin{align*}
	w_{t=1}\left(\theta_{1},k_{1}\right)=v_{1}(\theta_{1},k_{1})=\theta_{1}k_{1}-k_{1}^{2}.
	\end{align*}
	The welfare function differs, however, as it also takes into account future welfare, and evaluates this at the efficient allocation:
	\begin{align*}
	W_{t=1}\left(\theta,k\right)&=\max_{k_{1},k_{2}}\left\{ w_{t=1}\left(\theta_{1},k_{1}\right)+\mathbb{E}\left[W_{t=2}\left(\theta,k\right)\right]\right\}  \\
	&=\max_{k_{1},k_{2}}\left\{ \theta_{1}k_{1}-k_{1}^{2}+\mathbb{E}\left[\theta_{2}k_{2}-(k_{1}+k_{2})^{2}\right]\right\}, 
	\end{align*}
	where we apply that there is zero discounting of future welfare ($\delta=1$). 
	The efficient $k_1$ is found by maximizing welfare:
	\begin{align*}
	\frac{\partial W_{t=1}\left(\theta,k\right)}{\partial k_{1}}=\theta_{1}-2k_{1}^{*}-\mathbb{E}\left[2(k_{1}^{*}+k^{*}_{2})\right]=0 \\
	\Leftrightarrow\theta_{1}-2k_{1}^{*}-2k_{1}^{*}-2\mathbb{E}k^{*}_{2}=0 \\
	\Leftrightarrow k_{1}^{*}=\frac{\theta_{1}-2\mathbb{E}k^{*}_{2}}{4} \\
	\Leftrightarrow k_{1}^{*}=\frac{1}{4}\theta_{1}-\frac{1}{2}\mathbb{E}(k^{*}_{2})
	\end{align*}
	Using the optimal $k_2$ from above yields: 
	\begin{align*}
	k_{1}^{*}=\frac{1}{4}\theta_{1}-\frac{1}{2}\mathbb{E}(k_{2}^{*})=\frac{1}{4}\theta_{1}-\frac{1}{2}\mathbb{E}\left(\frac{1}{2}\theta_{2}-k_{1}\right) \\ 
	\Leftrightarrow k_{1}^{*}=\frac{1}{4}\theta_{1}-\frac{1}{8}+\frac{1}{2}k_{1}^{*} \\
	\Leftrightarrow k_{1}^{*}=\frac{1}{2}\theta_{1}-\frac{1}{4}
	\end{align*}
	
	
	
	\paragraph{4)}
	We start by considering period 2. Welfare is: 
	\begin{align*}
	W_{t=2}\left(\theta,k\right)=\theta_{2}k_{2}^{*}-(k_{1}^{*}+k_{2}^{*})^{2},
	\end{align*}
	where:
	\begin{align*}
	k_{2}^{*}=\frac{1}{2}\theta_{2}-k_{1}^{*}=\frac{1}{2}\theta_{2}-\left(\frac{1}{2}\theta_{1}-\frac{1}{4}\right)=\frac{1}{4}+\frac{1}{2}\left(\theta_{2}-\theta_{1}\right).
	\end{align*}
	Inserting this, welfare equals: 
	\begin{align*}
	W_{t=2}\left(\theta,k\right)=\theta_{2}k_{2}^{*}-(k_{1}^{*}+k_{2}^{*})^{2} \\
	=\theta_{2}\left(\frac{1}{4}+\frac{1}{2}\left(\theta_{2}-\theta_{1}\right)\right)-(\frac{1}{2}\theta_{1}-\frac{1}{4}+\frac{1}{4}+\frac{1}{2}\left(\theta_{2}-\theta_{1}\right))^{2} \\  
	=\theta_{2}\left(\frac{1}{4}+\frac{1}{2}\left(\theta_{2}-\theta_{1}\right)\right)-(\frac{1}{2}\theta_{2})^{2} \\
	=\frac{1}{4}\theta_{2}^2-\frac{1}{2}\theta_{2}\theta_{1}+\frac{1}{4}\theta_{2}
	\end{align*}
	In the absence of the negative externality imposed by player 1 welfare is:
	\begin{align*}
	W_{t=2,-1}\left(\theta,k\right)=\theta_{2}k_{2}^{*}\left(k_{1}=0\right)-\left(k_{2}^{*}\left(k_{1}=0\right)\right)^{2},
	\end{align*}
	where $k_{2}^{*}\left(k_{1}=0\right)=\frac{1}{2}\theta_{2}-0=\frac{1}{2}\theta_{2}$. Inserting yields: 
	\begin{align*}
	W_{t=2,-1}\left(\theta,k\right)=\theta_{2}k_{2}^{*}\left(k_{1}=0\right)-\left(k_{2}^{*}\left(k_{1}=0\right)\right)^{2} \\
	=\theta_{2}\frac{1}{2}\theta_{2}-\left(\frac{1}{2}\theta_{2}\right)^{2}=\frac{1}{4}\theta_{2}^{2}
	\end{align*}
	
	
	
	\noindent 
	The marginal contribution (which equals the flow marginal contribution because period 2 is the terminal period) is then: 
	\begin{align*}
	M_{i=1,t=2}=m_{i=1,t=2} = W_{t=2}\left(\theta,k\right)-W_{t=2,-1}\left(\theta,k\right) \\
	=\frac{1}{4}\theta_{2}^{2}-\frac{1}{2}\theta_{2}\theta_{1}+\frac{1}{4}\theta_{1}-\left[\frac{1}{4}\theta_{2}^{2}\right] \\
	=\frac{1}{4}\theta_{2}-\frac{1}{2}\theta_{2}\theta_{1}
	\end{align*}
	The optimal tax of generation 1 in period 2 is then given by:
	\begin{align*}
	p_{i=1,t=2} = v_{i=1,t=2} - m_{i=1,t=2} =  - \frac{1}{4}\theta_{2}+\frac{1}{2}\theta_{2}\theta_{1}
	\end{align*}
	Next, we move on to analyzing the payments in period 1. Welfare is given as the expression found above:
	\begin{align*}
	W_{t=1}\left(\theta,k\right)=\max_{k_{1},k_{2}}\left\{ \theta_{1}k_{1}-k_{1}^{2}+\mathbb{E}\left[\theta_{2}k_{2}-(k_{1}+k_{2})^{2}\right]\right\}
	\end{align*}
	And the optimal allocations were $k_1^*=\frac{1}{2}\theta_1-\frac{1}{4}$ and $k_2^*=\frac{1}{2}\theta_2-k_1$. By inserting this we get that welfare equals:
	\begin{align*}
	W_{t=1}(\theta,k)=  \theta_{1}k_{1}^*-(k_{1}^*) ^{2}+\mathbb{E}\left[\theta_{2}k_{2}^*-(k_{1}^*+k_{2}^*)^{2}\right]
	\\=  \theta_1(\frac{1}{2}\theta_1-\frac{1}{4})-(\frac{1}{2}\theta_1-\frac{1}{4}) ^{2}+\mathbb{E}\left[\theta_{2}k_{2}^*-(k_1^*+k_{2}^*)^{2}\right]
	\\=  \frac{1}{4}\theta_1^2-\frac{1}{16}+\mathbb{E}\left[\theta_{2}k_{2}^*-(k_1^*+k_{2}^*)^{2}\right]
	\\=  \frac{1}{4}\theta_1^2-\frac{1}{16}+\mathbb{E}\left[\theta_{2}k_{2}^*-(k_1^*+\frac{1}{2}\theta_2-k_1^*)^{2}\right]
	\\=  \frac{1}{4}\theta_1^2+\frac{1}{8}-\frac{1}{4}\theta_1
	\end{align*}
	The welfare when player 1 do not participate is:
	\begin{align*}
	W_{t=1,-1} = E[\theta_2k_2^*-(k_2^*)^2]=E[\frac{1}{2}\theta_2^2-\frac{1}{4}\theta_2^2]=\frac{1}{4}E[\theta_2^2]=\frac{1}{16}
	\end{align*}
	And the marginal gain is then:
	\begin{align*}
	M_{t=1,i=1}= W_{i=1,t=1}-W_{-i,t=1}=\frac{1}{4}\theta_1^2+\frac{1}{8}-\frac{1}{4}\theta_1 - \frac{1}{16} = \frac{1}{4}\theta_1^2-\frac{1}{4}\theta_1 +\frac{1}{16}
	\end{align*}
	To calculate the optimal tax we next have to decompose the marginal contribution into flow terms. The marginal contribution is decomposed as:
	\begin{align*}
	M_{i=1,t=1} = m_{i=1,t=1}+E[M_{i=1,t=2}]
	\end{align*}
	where $m_{i=1,t=1}$ is the flow marginal contribution, used to calculate the optimal tax. This can be calculated residually as:
	\begin{align*}
	m_{i=1,t=1}=M_{i=1,t=1}-E[M_{i=1,t=2}]
	\\=\frac{1}{4}\theta_1^2-\frac{1}{4}\theta_1+\frac{1}{16}-E[\frac{1}{4}\theta_2+\frac{1}{2}\theta_1\theta_2]
	\\=\frac{1}{4}\theta_1^2-\frac{1}{2}\theta_1-\frac{1}{16}
	\end{align*}
	The payment of player 1 is then given by:
	\begin{align*}
	p_{i=1,t=1} = \theta_1k_1^*-(k_1^*)^2 - [ \frac{1}{4}\theta_1^2-\frac{1}{2}\theta_1-\frac{1}{16}] \\
	=\theta_1(\frac{1}{2}\theta_1-\frac{1}{4})-(\frac{1}{2}\theta_1-\frac{1}{4})^2-[ \frac{1}{4}\theta_1^2-\frac{1}{2}\theta_1-\frac{1}{16}] \\
	=\frac{1}{2}\theta_1.
	\end{align*}
	
	
	
	\paragraph{5)}
	Since player 2 do not impose any externalities on other than himself, he does not have to make any payments. Player 1 does, however, have to make a payment in period 1, which we showed in subsection 4. We also showed that in period 2, where only player 2 is alive, the payment of player 1 should be:
	
	\begin{align*}
	p_{i=1,t=2}= \frac{1}{2}\theta_2\theta_1-\frac{1}{4}\theta_2
	\end{align*}
	Since $\theta_i$ are uniformly distributed we have that:
	\begin{align*}
	E[p_{i=1,t=2}]= E[\frac{1}{2}\theta_2\theta_1-\frac{1}{4}\theta_2]=\frac{1}{2}\cdot\frac{1}{2}\cdot\frac{1}{2}-\frac{1}{4}\cdot\frac{1}{2}=0 
	\end{align*}
	Thus, in expectation, none of the players will have to make transfers in periods where they are not alive.
	
	\emph{Note: the mechanism is indeed not online according to the definition in the problem.}
	
	\fi
\end{ex}



%%------------------------------------------------------------------------------------------------

\begin{ex}
Let us consider the problem of designing a revenue-maximizing (optimal) mechanism that we explored, and enrich it with minimal dynamics. In particular, suppose that the buyer receives some information about their value between the moments of contracting and transaction. How would the seller be able to exploit this? 

Formal setup of the model follows. A seller designs a profit-maximizing mechanism to sell a single indivisible item to one buyer. There are two periods: $t = 0,1$; the good is to be transferred at $t=1$ (and there are no allocation decisions to be made at $t=0$). At $t=0$ the buyer's type is given by $\theta_0 \sim F_0 [\underline{\theta},\bar{\theta}]$; at $t=1$ it changes to $\theta_1 \in [\underline{\theta},\bar{\theta}]$ according to some conditional distribution $F(\theta_1|\theta_0)$. Type $\theta_1$ represents the buyer's final valuation for the item, but it is not observed by either buyer or seller at $t=0$.

In a direct mechanism, at $t=0$ the buyer reports $\tilde{\theta}_0$, at $t=1$ the buyer reports $\tilde{\theta}_1$, after which the item is transferred to the buyer with probability $k(\tilde{\theta}_0,\tilde{\theta}_1)$, and the buyer pays $p(\tilde{\theta}_0,\tilde{\theta}_1)$ (irrespectively of whether they got the item). The buyer's payoff is then $u_b(\theta_1,k,p) = \theta_1 k - p$, and the seller's payoff is $u_s(\theta_1,k,p) = p$. (Neither player receives any payoff at $t=0$.)

Consider the relaxed problem in which the seller gets to learn all new \emph{information} that the buyer receives at $t=1$ (e.g., the seller observes $\varepsilon_1 = F(\theta_1|\theta_0)$). In what follows, you may use the expressions given in the slides without derivations.

\begin{enumerate}
	\item What is the social surplus (gains from trade) in this problem?
	\item Write down the virtual surplus as a function of $(\theta_0,\theta_1)$. How does it differ from that in the static model?
	\item What is the optimal allocation rule (assuming $VS(\theta_0,\theta_1)$ is increasing in both arguments)? How does it differ from that in the static model?
\end{enumerate} 

Suppose now in addition that $\theta_0 \sim U[0,1]$ and $\theta_1 = \theta_0 + \varepsilon$ with $\varepsilon \sim U[-1,1]$.

\begin{enumerate}[resume]
	\item Show that impulse response function $I_1 (\theta_1, \theta_0) = \frac{-\frac{\partial F(\theta_1|\theta_0)}{\partial \theta_0 }}{f(\theta_1 | \theta_0)}$ evaluates to $I_1 (\theta_1, \theta_0) = 2$.
	\item Show that virtual surplus evaluates to $VS(\theta_0, \theta_1)$. Derive the optimal allocation rule $k(\theta_0, \theta_1)$ which maximizes the expected profit given by $\mathbb{E}_{(\theta_0, \theta_1)} \left[ k(\theta_0, \theta_1) \cdot VS(\theta_0, \theta_1) \right]$.
	\item Discuss how the expected allocation $Ek(\theta_0) \equiv \mathbb{E}_{\theta_1} \left[ k^*(\theta_0, \theta_1) | \theta_0 \right]$ differs in this problem from the static case (where the buyer's valuation is given by $\theta_0 \sim U[0,1]$).
	\item The buyer's expected utility under the optimal mechanism, $U_0(\theta_0) \equiv \mathbb{E}_{\theta_1} \left[ u_b(\theta_1,k(\theta_0, \theta_1),p(\theta_0, \theta_1)) \mid \theta_0 \right]$, has the envelope representation $\frac{d U_0(\theta_0)}{d\theta_0} = \mathbb{E}_{\theta_1} \left[ I_1(\theta_1,\theta_0) \frac{\partial u_b(\theta_1,k,p)}{\partial \theta_1} \mid \theta_0 \right]$.
	Compute the expected utility $U_0(\theta_0)$.
	\item Use $U_0(\theta_0)$ to derive [expected] optimal payments $p(\theta_0)$. How do they compare to payments in the static model (where the buyer's valuation is given by $\theta_0 \sim U[0,1]$)?
	\item Assume now that $\varepsilon$ is not observable. Consider a mechanism $(k,p)$, where $k$ is the optimal allocation rule you derived, and $p$ is the payment rule you derived, assuming $p(\theta_0, \theta_1) = p(\theta_0)$ for all $\theta_1$ Would this mechanism be incentive compatible for the buyer?
\end{enumerate}

\ifsolutions
\subsection*{Solution}
See Bergemann and V{\"a}lim{\"a}ki (2019), section 5.1, for a partial solution.
\fi
\end{ex}




\end{document}