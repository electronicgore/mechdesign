%%% License: Creative Commons Attribution Share Alike 4.0 (see https://creativecommons.org/licenses/by-sa/4.0/)


%%%%%%%%%%%%%%%%%%%%%%%%%%%%%%%%%%%%%%%%%

%----------------------------------------------------------------------------------------
%	PACKAGES AND OTHER DOCUMENT CONFIGURATIONS
%----------------------------------------------------------------------------------------

\documentclass{article}

\usepackage{amssymb}
\usepackage{enumerate}
\usepackage[usenames,dvipsnames]{color}
\usepackage{fancyhdr} % Required for custom headers
\usepackage{lastpage} % Required to determine the last page for the footer
\usepackage{extramarks} % Required for headers and footers
\usepackage[usenames,dvipsnames]{color} % Required for custom colors
\usepackage{graphicx} % Required to insert images
\usepackage{listings} % Required for insertion of code
\usepackage{courier} % Required for the courier font
\usepackage[table]{xcolor}
\usepackage{amsfonts,amsmath,amsthm,parskip,setspace,url}
\usepackage[section]{placeins}
\usepackage[a4paper]{geometry}
\usepackage[USenglish]{babel}
\usepackage[utf8]{inputenc}
\usepackage{hyperref}


% Margins
\topmargin=-0.45in
\evensidemargin=0in
\oddsidemargin=0in
\textwidth=6.5in
\textheight=9.0in
\headsep=0.6in

\linespread{1.1} % Line spacing

%----------------------------------------------------------------------------------------
%	DOCUMENT STRUCTURE COMMANDS
%	Skip this unless you know what you're doing
%----------------------------------------------------------------------------------------

% Header and footer for when a page split occurs within a problem environment
\newcommand{\enterProblemHeader}[1]{
\nobreak\extramarks{#1}{#1 continued on next page\ldots}\nobreak
\nobreak\extramarks{#1 (continued)}{#1 continued on next page\ldots}\nobreak
}

% Header and footer for when a page split occurs between problem environments
\newcommand{\exitProblemHeader}[1]{
\nobreak\extramarks{#1 (continued)}{#1 continued on next page\ldots}\nobreak
\nobreak\extramarks{#1}{}\nobreak
}

\setcounter{secnumdepth}{0} % Removes default section numbers
\newcounter{homeworkProblemCounter} % Creates a counter to keep track of the number of problems

\newcommand{\homeworkProblemName}{}
\newenvironment{ex}[1][Problem \arabic{homeworkProblemCounter}]{ % Makes a new environment called homeworkProblem which takes 1 argument (custom name) but the default is ``Problem #''
\stepcounter{homeworkProblemCounter} % Increase counter for number of problems
\renewcommand{\homeworkProblemName}{#1} % Assign \homeworkProblemName the name of the problem
\section{\homeworkProblemName} % Make a section in the document with the custom problem count
\enterProblemHeader{\homeworkProblemName} % Header and footer within the environment
}{
\exitProblemHeader{\homeworkProblemName} % Header and footer after the environment
}

\newcommand{\problemAnswer}[1]{ % Defines the problem answer command with the content as the only argument
\noindent\framebox[\columnwidth][c]{\begin{minipage}{0.98\columnwidth}#1\end{minipage}} % Makes the box around the problem answer and puts the content inside
}

\newcommand{\homeworkSectionName}{}
\newenvironment{homeworkSection}[1]{ % New environment for sections within homework problems, takes 1 argument - the name of the section
\renewcommand{\homeworkSectionName}{#1} % Assign \homeworkSectionName to the name of the section from the environment argument
\subsection{\homeworkSectionName} % Make a subsection with the custom name of the subsection
\enterProblemHeader{\homeworkProblemName\ [\homeworkSectionName]} % Header and footer within the environment
}{
\enterProblemHeader{\homeworkProblemName} % Header and footer after the environment
}


%----------------------------------------------------------------------------------------
%----------------------------------------------------------------------------------------
%----------------------------------------------------------------------------------------
% Set up the header and footer
\pagestyle{fancy}
\lhead[c]{\textbf{{\color[rgb]{.5,0,0} K{\o}benhavns\\Universitet }} \firstxmark} % Top left header
\chead{\textbf{{\color[rgb]{.5,0,0} \Class }}\\ \hmwkTitle  } % Top center head
\rhead{\instructor \\ \theprofessor} % Top right header
\lfoot{\lastxmark} % Bottom left footer
\cfoot{} % Bottom center footer
\rfoot{Page\ \thepage\ of\ \protect\pageref{LastPage}} % Bottom right footer
\renewcommand\headrulewidth{0.4pt} % Size of the header rule
\renewcommand\footrulewidth{0.4pt} % Size of the footer rule

\setlength\parindent{0pt} % Removes all indentation from paragraphs







%----------------------------------------------------------------------------------------
%	NAME AND CLASS SECTION
%----------------------------------------------------------------------------------------

\newcommand{\hmwkTitle}{Reading List} % Assignment title
\newcommand{\Class}{Mechanism Design} % Course/class
\newcommand{\instructor}{Fall 2020} % TA
\newcommand{\theprofessor}{Prof. Egor Starkov} % Professor

%\theoremstyle{definition} \newtheorem{ex}{\textbf{\Large{Exercise & #}\\}}
\setlength{\parskip}{0 pt}




















%%%%%%%%%%%%%%%%%%%%%%%%%%%%%%%%%%%%%%%%%%%%%%%%%%%%%%%%%%%%%%%%%%%%%%%%%%%%%%%%%%%%%%


\begin{document}

\begin{center}
	{\huge Mechanism Design: Reading List}
\end{center}
\bigskip

The primary reference in each section usually contains the most complete treatment of the topic, possibly complemented by one or more other ``primary'' references. Sources listed as ``alternatives'' are substitutes, usually either imperfect ones (so give a narrower coverage), or less accessible ones (harder to read, harder to get the source). 

Items marked with a star indicate optional readings that may be briefly mentioned but not extensively covered in class.

General references:
\begin{description}
	\item[MWG] Mas-Colell, Andreu, Michael Dennis Whinston, and Jerry R. Green. Microeconomic theory. New York: Oxford university press, 1995. 
	\item[B{\"o}rgers] B{\"o}rgers, Tilman. An introduction to the theory of mechanism design. Oxford University Press, 2015.
	\item[Diamantaras] Diamantaras, Cardamone, Campbell, Deacle, and Delgado. A toolbox for economic design. Macmillan, 2009.
	\item[RS] Roth, Alvin E., and Marilda Sotomayor. Two-Sided Matching: A Study in Game-Theoretic Modeling and Analysis. Cambridge: Cambridge University Press. 1990.
\end{description}
\medskip

\subsection{Introduction and mechanism design problem (week 1)}
\begin{itemize}
	\item B{\"o}rgers ch.1.
	
	\textbf{Alternatives:}
	\begin{itemize}
		\item MWG 23.A-B.
		\item Diamantaras ch.1.
	\end{itemize}
\end{itemize}

\subsection{DSIC \& BIC mechanisms; Efficient implementation with transfers (weeks 2-4)}
\begin{itemize}
	\item B{\"o}rgers ch.: 7.1-7.3 (DSIC mechanisms), 6.1-6.3 (BIC mechanisms), 3 and 4 (examples).
	
	\textbf{Alternatives:}
	\begin{itemize}
		\item MWG 23.C-E.
		\item Diamantaras ch.2, 4.1-4.4, 4.6
	\end{itemize}
	\item Krishna, Vijay, and Motty Perry. ``Efficient Mechanism Design.'' (2000). Unpublished.
	
	\url{https://drive.google.com/file/d/0B9qyCPfbmExnbmE1OTk5OGJmQzA/view}
	
	\item (*) Krishna, Vijay, and Eliot Maenner. ``Convex Potentials with an Application to Mechanism Design.'' Econometrica 69.4 (2001): 1113-1119.\\
	\url{https://onlinelibrary.wiley.com/doi/abs/10.1111/1468-0262.00233}
\end{itemize}

\subsection{Arbitrary DSIC social choice rules (with and without transfers) (week 5)}
\begin{itemize}
	\item B{\"o}rgers ch.: 7.4 and 5.1-5.7 (transfers), 8 (no transfers)
	
	\textbf{Alternatives:}
	\begin{itemize}
		\item MWG 21, 23.C
		\item Diamantaras ch.2
	\end{itemize}
	\item (*) Geanakoplos, John. ``Three brief proofs of Arrow’s impossibility theorem.'' Economic Theory 26.1 (2005): 211-215. \url{https://link.springer.com/article/10.1007/s00199-004-0556-7}
	\item (*) Svensson, Lars-Gunnar, and Alexander Reffgen. ``The proof of the Gibbard–Satterthwaite theorem revisited.'' Journal of Mathematical Economics 55 (2014): 11-14.
	
	\url{https://www.sciencedirect.com/science/article/pii/S0304406814001177}
\end{itemize}

\subsection{Optimal mechanisms (weeks 6-7)}
\begin{itemize}
	\item B{\"o}rgers ch.: 6.1-6.3 (BIC mechanisms and revenue equivalence), 2-4 (actual optimal mechanisms, esp. 3.2).
	
	\textbf{Alternatives:}
	\begin{itemize}
		\item MWG 23.F.
		\item Diamantaras ch.4.5
	\end{itemize}
	\item (*) Manelli, Alejandro M., and Daniel R. Vincent. ``Multidimensional mechanism design: Revenue maximization and the multiple-good monopoly.'' Journal of Economic theory 137.1 (2007): 153-185. \url{https://www.sciencedirect.com/science/article/pii/S0022053107000348}
	\item (*) Olszewski, Wojciech, and Ron Siegel. ``Performance‐maximizing large contests.'' Theoretical Economics 15.1 (2020): 57-88.
	
	\url{https://econtheory.org/ojs/index.php/te/article/viewForthcomingFile/3588/24803/1}
\end{itemize}

\subsection{Correlated information (week 8)}
\begin{itemize}
	\item B{\"o}rgers ch.6.4.
	
	\textbf{Alternatives:}
	\begin{itemize}
		\item Cremer, Jacques, and Richard P. McLean. ``Full extraction of the surplus in Bayesian and dominant strategy auctions.'' Econometrica: Journal of the Econometric Society (1988): 1247-1257.
		
		\url{https://www.jstor.org/stable/1913096}
	\end{itemize}
	\item Battaglini, Marco. ``Multiple referrals and multidimensional cheap talk.'' Econometrica 70.4 (2002): 1379-1401. \url{https://onlinelibrary.wiley.com/doi/abs/10.1111/1468-0262.00336}
	
	\item (*) B{\"o}rgers ch.10.
\end{itemize}

\subsection{Dynamic mechanism design (weeks 9-10)}
\begin{itemize}
	\item Bergemann, Dirk, and Juuso Välimäki. ``Dynamic mechanism design: An introduction.'' Journal of Economic Literature 57.2 (2019): 235-74. 
	\url{https://www.aeaweb.org/articles?id=10.1257/jel.20180892}
	
	\textbf{Alternatives:}
	\begin{itemize}
		\item B{\"o}rgers ch.11.
	\end{itemize}
	\item Topics:
	\begin{itemize}
		\item (\emph{risk-aversion}) Thomas, Jonathan, and Tim Worrall. ``Income fluctuation and asymmetric information: An example of a repeated principal-agent problem.'' Journal of Economic Theory 51.2 (1990): 367-390. \url{https://www.sciencedirect.com/science/article/pii/002205319090023D}
		\item (\emph{no transfers}) Guo, Yingni, and Johannes Hörner. ``Dynamic mechanisms without money.'' Working paper (2018). 
		
		\url{http://yingniguo.com/wp-content/uploads/2018/07/Dynamic-Allocation-without-Money.pdf}
		\item (\emph{no transfers or commitment}) Li, Jin, Niko Matouschek, and Michael Powell. ``Power dynamics in organizations.'' American Economic Journal: Microeconomics 9.1 (2017): 217-41.
		
		\url{https://www.aeaweb.org/articles?id=10.1257/mic.20150138}
	\end{itemize}
	\item (*) Gershkov, Alex, and Benny Moldovanu. Dynamic Allocation and Pricing: A Mechanism Design Approach. MIT Press, 2014. 
\end{itemize}

\subsection{Matching (weeks 11-12)}
\begin{itemize}
	\item RS ch.2,4
	
	\textbf{Alternatives:}
	\begin{itemize}
		\item Diamantaras ch.9.
	\end{itemize}
	\item (*) RS ch.3,5-9
	\item (*) Liu, Q., Mailath, G. J., Postlewaite, A., \& Samuelson, L.  ``Stable matching with incomplete information.'' Econometrica 82.2 (2014): 541-587. \url{https://doi.org/10.3982/ECTA11183}
\end{itemize}

\subsection{Information design (weeks 13-14)}
\begin{itemize}
	\item Bergemann, Dirk, and Stephen Morris. ``Information design: A unified perspective.'' Journal of Economic Literature 57.1 (2019): 44-95. 
	
	\url{https://www.aeaweb.org/articles?id=10.1257/jel.20181489}
%	\item Kamenica, Emir. ``Bayesian persuasion and information design.'' Annual Review of Economics 11 (2018).
%	
%	\url{https://www.annualreviews.org/doi/abs/10.1146/annurev-economics-080218-025739}
	\item Kamenica, Emir, and Matthew Gentzkow. "Bayesian persuasion." American Economic Review 101.6 (2011): 2590-2615.
	
	\url{https://www.aeaweb.org/articles?id=10.1257/aer.101.6.2590}
\end{itemize}

\bigskip
The list may be expanded and updated as we progress through the course.


%%-----------------------------------------------------------------------------------------------------

\end{document}
